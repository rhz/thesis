% this needs package epigraph
% \setlength{\epigraphwidth}{.8\textwidth}
% \setlength{\epigraphrule}{0pt}
% \epigraph{
%   I regard as quite useless the reading of
%   large treatises of pure analysis:
%   too large a number of methods pass at once before the eyes.
%   It is in the works of applications that one must study them;
%   one judges their ability there and
%   one apprises the manner of making use of them.}{
%   --- Joseph Louis Lagrange}

% \begin{quotation}
%   \textit{
%     ``I regard as quite useless the reading of
%     large treatises of pure analysis:
%     too large a number of methods pass at once before the eyes.
%     It is in the works of applications that one must study them;
%     one judges their ability there and
%     one apprises the manner of making use of them.''} \\
%   \par\raggedleft--- Joseph Louis Lagrange
% \end{quotation}

% from https://hbfs.wordpress.com/2011/01/18/epigraphs-in-latex/
% epigraph with 3 params: width, text, author
\newcommand{\epigraph}[3]{
\vspace{1em}\hfill{}\begin{minipage}{#1}{\begin{spacing}{0.9}
\small\noindent\textit{#2}\end{spacing}
\vspace{1em}
\hfill{}{#3}}\vspace{2em}
\end{minipage}}
\epigraph{.8\textwidth}{
  I regard as quite useless the reading of
  large treatises of pure analysis:
  too large a number of methods pass at once before the eyes.
  It is in the works of applications that one must study them;
  one judges their ability there and
  one apprises the manner of making use of them.}{
  --- Joseph Louis Lagrange}

\noindent
In the history of natural sciences,
there has been two main approaches to describe dynamical systems:
\emph{Kinetics} and \emph{Thermodynamics}.
The former goes all the way back to
Newton's laws of motion~\citep{newton}.
% In particular, the second law of motion describes
% how these forces translate into the second derivative of
% the position of the particles over time, \ie their acceleration.
Broadly speaking, in the kinetic approach we describe the forces
acting on each particle at every point in time.
The net force determines how
the position of the particle changes over time.
The fact this net force can be decomposed into separate forces
that can be measured independently is very useful in practice,
\eg the gravitational force on an object does not depend on
the other forces acting on that object.
% In a way this is the most detailed description
% that we can have at the level of point particles.

% On the other hand,
In the thermodynamic approach
% the movement and transformation of particles
% is represented in the energy function of the system.
% is seen as a consequence of the minimisation of the energy of the system.
all information concerning the system and the forces acting on it
is contained in the energy function.
This approach first appeared in the work of
\citet{lagrange2} and \citet{hamilton}.
It endowed the description of a dynamical system in classical mechanics
with a remarkable conciseness, simplicity and elegance. % :
% just one function could describe the entirety of the system.
The kinetic description can then be derived from this energy function.
However the opposite is not true:
in general a kinetic description might not have an energy function
from which it can be derived in the classical framework.
This is because thermodynamics puts restrictions on possible processes.
The specifics of this correspondence have been described
in detail by . % FIXME: cite

Later on in the history of natural sciences,
% when chemistry started to become a quantitative science,
the dynamics of gas compression and expansion
and its relation to heat was established.
% also came in the two variants.
First came the Kinetic theory of gases by . % FIXME: cite
% First the Kinetic theory of gases was developed by .
This theory single-handledly gave birth to
the field of statistical mechanics. % TODO: check stat mech wiki page
% Is the kinetic theory of gases the one that has two sides?
% The equations that link energy, entropy, temperature, pressure,
% volume and so on are part of that theory?
Then the laws of thermodynamics came about
to describe the macro behaviour of systems
with an underlying energy function. % TODO: check this please
Similarly to classical mechanics,
in the thermodynamic approach to statistical mechanics
some processes are improbable and some outright impossible.

% Later still,
Half a century had passed since the publication of
the Kinetic theory of gases and 
a formal language to describe chemical reaction systems
% came to be by the hand of Petri
was invented by Petri. % FIXME: cite
% the description of dynamical reaction systems
% was also done in both ways.
This language, later called Petri nets,
would define a reaction as a transformation of
multisets of chemical species.
% What happens historically between the invention of Petri nets
% and an algorithm to simulate them?
% When can we start talking about the two approaches in Petri nets?
In this framework we find the two approaches too:
SSA and Metropolis-Hastings...

However, Petri nets as a language for describing chemical reactions
has limitations when we take into consideration
how chemistry actually works.
It's an abstraction that leaks because it doesn't fully cover
the underlying phenomena.
In the world of (non-radioactive) chemical reactions
atoms do not transform,
they just change their partners with whom they share their electrons.
It's all about atom binding and unbinding,
establishing connections and breaking them.

More recently,
a formal language to describe biochemical interactions
where molecules not just react but can also bind other molecules
non-covalently has been created by . % FIXME: cite

In this work, we are interested in the correspondence between
the kinetic and thermodynamic descriptions of graph rewriting.
Graph rewriting is ...


















%%% Local Variables:
%%% mode: latex
%%% TeX-master: "thesis"
%%% End:

