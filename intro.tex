% % this needs package epigraph
% \setlength{\epigraphwidth}{.8\textwidth}
% \setlength{\epigraphrule}{0pt}
% \epigraph{
%   I regard as quite useless the reading of
%   large treatises of pure analysis:
%   too large a number of methods pass at once before the eyes.
%   It is in the works of applications that one must study them;
%   one judges their ability there and
%   one apprises the manner of making use of them.}{
%   --- Joseph Louis Lagrange}

% \begin{quotation}
%   \textit{
%     ``I regard as quite useless the reading of
%     large treatises of pure analysis:
%     too large a number of methods pass at once before the eyes.
%     It is in the works of applications that one must study them;
%     one judges their ability there and
%     one apprises the manner of making use of them.''} \\
%   \par\raggedleft--- Joseph Louis Lagrange
% \end{quotation}

% from https://hbfs.wordpress.com/2011/01/18/epigraphs-in-latex/
% epigraph with 3 params: width, text, author
\newcommand{\epigraph}[3]{
\vspace{1em}\hfill{}\begin{minipage}{#1}{\begin{spacing}{0.9}
\small\noindent\textit{#2}\end{spacing}
\vspace{1em}
\hfill{}{#3}}\vspace{2em}
\end{minipage}}
\epigraph{.8\textwidth}{
  I regard as quite useless the reading of
  large treatises of pure analysis:
  too large a number of methods pass at once before the eyes.
  It is in the works of applications that one must study them;
  one judges their ability there and
  one apprises the manner of making use of them.}{
  --- Joseph Louis Lagrange}


%%% Local Variables:
%%% mode: latex
%%% TeX-master: "thesis"
%%% End:


% \noindent
In the history of natural sciences,
there has been two main approaches to describe dynamical systems,
which I call here
\emph{Kinetics} and \emph{Thermodynamics}.
Broadly speaking, in the kinetic approach
the system is described by the positions and velocities
(or rather, momenta)
of each particle.
This is a ``ground'' description in the sense that
it is as explicit and detailed as possible.
If a mechanical system has $N$ particles
then a state of the system is described by
a vector in $\RR^{2 \cdot 3 \cdot N}$.
This approach goes all the way back to
Newton's laws of motion~\citep{newton}.
% Regarding the role of forces in Newtonian mechanics:
% Interactions producing a change of momentum on a particle
% that can be measured independently do not interact between them
% and thus the resulting derivative of the momentum (force)
% is just the sum of the forces as measured independently.

In the thermodynamic approach
all information about the change of the system in time
is contained in the energy function.
Given a set of states $\statespace$ for the system,
an energy function $E: \statespace \to \RR$ maps a state to its energy.
In this way, a state of the system is described by a single scalar
regardless of how many particles it comprises.
Naturally, this approach endowed the description of
a dynamical system in classical mechanics
with a remarkable conciseness, simplicity and elegance.
It first appeared in the work of
\citet{lagrange2} and \citet{hamilton},
and has been subsequently used as the basis for most of modern physics.
The kinetic description can then be derived from this energy function.
However the converse is not true:
in general a kinetic description might not have an energy function
from which it can be derived. % in the classical framework.
% This is because thermodynamics puts restrictions on possible processes.
This is called the \emph{inverse problem} in classical mechanics
\citep{santilli} and it was first attended to by \citet{helmholtz}.
I call this approach `thermodynamic' not because of thermodynamics,
the science that deals with heat and temperature,
but because energy drives the time evolution
of the system under this approach.
% it gives an energy-based representation of dynamical systems.
Certainly, there are connections to thermodynamics
that will be highlighted as they arise.

% Later on in the history of natural sciences,
A century after Hamilton's paper
a formal language to describe chemical reaction systems
% came to be by the hand of Petri
was invented by Petri. % FIXME: cite
% the description of dynamical reaction systems
% was also done in both ways.
This language, later called Petri nets,
would define a reaction as a transformation of
multisets of chemical species.
% What happens historically between the invention of Petri nets
% and an algorithm to simulate them?
% When can we start talking about the two approaches in Petri nets?
In this framework we find the two approaches too:
SSA and Metropolis-Hastings...

However, Petri nets as a language for describing chemical reactions
has limitations when we take into consideration
how chemistry actually works.
It's an abstraction that leaks because it doesn't fully cover
the underlying phenomena.
In the world of (non-radioactive) chemical reactions
atoms do not transform,
they just change their partners with whom they share their electrons.
It's all about atom binding and unbinding,
establishing connections and breaking them.

More recently,
a formal language to describe biochemical interactions
where molecules not just react but can also bind other molecules
non-covalently has been created by . % FIXME: cite

In this work, we are interested in the correspondence between
the kinetic and thermodynamic descriptions of graph rewriting.
Graph rewriting is ...


















%%% Local Variables:
%%% mode: latex
%%% TeX-master: "thesis"
%%% End:

