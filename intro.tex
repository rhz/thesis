% % this needs package epigraph
% \setlength{\epigraphwidth}{.8\textwidth}
% \setlength{\epigraphrule}{0pt}
% \epigraph{
%   I regard as quite useless the reading of
%   large treatises of pure analysis:
%   too large a number of methods pass at once before the eyes.
%   It is in the works of applications that one must study them;
%   one judges their ability there and
%   one apprises the manner of making use of them.}{
%   --- Joseph Louis Lagrange}

% \begin{quotation}
%   \textit{
%     ``I regard as quite useless the reading of
%     large treatises of pure analysis:
%     too large a number of methods pass at once before the eyes.
%     It is in the works of applications that one must study them;
%     one judges their ability there and
%     one apprises the manner of making use of them.''} \\
%   \par\raggedleft--- Joseph Louis Lagrange
% \end{quotation}

% this needs package spacing
% from https://hbfs.wordpress.com/2011/01/18/epigraphs-in-latex/
% epigraph with 3 params: width, text, author
\newcommand{\epigraph}[3]{
\vspace{1em}\hfill{}\begin{minipage}{#1}{\begin{spacing}{0.9}
\small\noindent\textit{#2}\end{spacing}
\vspace{1em}
\hfill{}{#3}}\vspace{2em}
\end{minipage}}
\epigraph{.8\textwidth}{
  I regard as quite useless the reading of
  large treatises of pure analysis:
  too large a number of methods pass at once before the eyes.
  It is in the works of applications that one must study them;
  one judges their ability there and
  one apprises the manner of making use of them.}{
  --- Joseph Louis Lagrange}


%%% Local Variables:
%%% mode: latex
%%% TeX-master: "thesis"
%%% End:


% \noindent
\section{Historical background}

In the history of natural sciences,
there has been two main approaches to describe dynamical systems,
which I call here
\emph{kinetics} and \emph{thermodynamics}.
% This thesis is about the relationship between the two.
The first approach goes all the way back to
Newton's laws of motion~\citep{newton}.
Loosely speaking, it describes a system by
the positions and momenta of each particle.
A description of this type is as explicit and detailed as it can get
for the type of systems under consideration in classical mechanics,
\ie the movement of point particles.
In the case of a classical mechanical system that has $N$ particles,
a state of the system is specified by
a vector in $\RR^{2 \cdot 3 \cdot N}$.
In general, a kinetic description is the full description of the
dynamics of the system in terms of the velocities of its processes.
% Regarding the role of forces in Newtonian mechanics:
% Interactions producing a change of momentum on a particle
% that can be measured independently do not interact between them
% and thus the resulting derivative of the momentum (force)
% is just the sum of the forces as measured independently.

The thermodynamic approach, on the other hand,
gives a description based on an \emph{energy function}.
The energy function is defined on the states of the system
and assigns a real value to each state, its energy.
That is, a state is described by a single scalar
regardless of how many particles it comprises.
Naturally, this approach endowed the description of
a dynamical system in classical mechanics
with a remarkable conciseness, simplicity and elegance.
It first appeared in the work of
\citet{lagrange2} and \citet{hamilton},
and has been subsequently used as the basis for most of modern physics.
Once in possession of the energy function,
the kinetic description (\ie the equations of motion)
can be derived from it.
However the converse is not true:
in general a kinetic description might not have an energy function
from which it can be derived \citep{santilli}
because of non-conservative (dissipative) forces.
Obtaining an energy function from the equations of motion
is referred to as the \emph{inverse problem} % in classical mechanics
and it was first attended to by \citet{helmholtz}.
Both the direct and the inverse problem are the interest of this thesis
and we aim to answer these questions
in the context of biomolecular interaction networks.
% A more precise statement of the problem will have to wait
% until we have introduced all the relevant concepts.
% Note that this approach has been given the name `thermodynamic'
% not because of thermodynamics,
% the science that studies the dynamics of heat and temperature,
% but because of the protagonical role of the energy
% in driving the system's evolution.
% Certainly, there are connections to thermodynamics
% that will be highlighted as they arise.

Half a century after Hamilton's work
researchers like Maxwell, Boltzmann, and Gibbs
applied the ideas of classical mechanics to \emph{atoms}
in order to describe physical properties of matter like pressure,
the capacity to transfer heat, and others.
This body of work came to be known as \emph{statistical mechanics}
and was used to explain Brownian motion by \citet{einstein-brownian},
which after its experimental verification \citep{perrin}
settled the debate about the existence of atoms.
This work however did not attempt to explain
the chemical interactions and reactions that molecules undergo.
That would have to wait yet half a century
for the axiomatisation of probability theory by \citet{kolmogorov}
and the further developments by \citet{doob} and \citet{feller},
who, among others, established the theoretical framework
for continuous-time Markov chains (CTMCs).
Below you can find the definition of (time-homogeneous) CTMCs
and \qmatrices that will be used here.
% In particular, we work with time-homogeneous CTMCs.

\begin{definition}%[Infinitesimal generator]
  A (stable and conservative) \emph{\qmatrix} $\qm$
  on an at most countable set of states $\states$
  is an $\states \times \states$ matrix
  with elements $q_{ij} \in \RR$, $i,j \in \states$
  such that $0 \leqslant q_{ij} < \infty$ when $i \neq j$
  and $q_{ii} = - \sum_{j \neq i} q_{ij} > -\infty$.\footnote{
    If unstable $q_{ii}$ can be $-\infty$ and if non-conservative
    $q_{ii} \leqslant - \sum_{j \neq i} q_{ij}$.}
\end{definition}

The \qmatrix plays the role of
the time derivative of the transition probabilities at time $0$
and induces the evolution of a probabilistic state
according to the Kolmogorov backward equation,
\begin{equation}
  \label{eq:transition-function}
  \ddt P(t) = Q P(t), \quad P(0) = I
\end{equation}
where $P(t)$ is the $\states \times \states$ matrix
with elements $p_{ij}(t) \in \RR$ the probability that
we were in state $i$ at time $0$ and are in state $j$ at time $t$.
% transition probability from state $i$ to $j$ at time $t$.
When the \qmatrix is stable and conservative
there exists a unique minimal\footnote{
  % In the sense that
  If $P'(t)$ is any
  non-negative solution of \eqn{transition-function},
  then $p_{ij}(t) \leqslant p'_{ij}(t)$
  for all $i, j \in \states$ and $t \geqslant 0$.}
solution $P(t)$ of \eqn{transition-function} \citep{anderson}.
% Proved in Th 2.2.2, page 70, of Anderson's book
% Properties of transition functions and q-matrices:
% P(t) is stable if \ddt p_{ii}(t) is finite at t = 0 for all i. (p 4)
% P(t) is standard if \lim_{t \to 0} p_{ii} = 1 for all i. (p 6)
% Q is stable if q_{ii} > -\infty for all i. (p 9)
% Q is conservative if \sum_j q_{ij} = 0 for all i. (p 13)
We shall work with this type of \qmatrices
and assume there is a transition function $P(t)$
whenever we have a \qmatrix $\qm$ and vice versa.

Given a \pmf $\state(0)$ on $\states$ (seen as a row vector)
the state at $t = 0$,
the probability distribution $\state(t)$ after time $t$
is given by $\state(t) = \state(0) P(t)$.
% We can obtain the time derivative of this distribution
% from \eqn{transition-function}.
% % $s_i(t)$ that the Markov chain is in state $i$ at time $t$.
% In coordinate form, we have
% \begin{equation} % TODO: is this equation correct?
%   \label{eq:prob-deriv}
%   \ddt s_i(t) = \sum_{j \in \states} q_{ji} s_j(t)
% \end{equation}
We say the \qmatrix is \emph{irreducible}
if every state is reachable regardless of the initial state,
\ie $p_{ij}(t) > 0$ for all $i,j \in \states$
and some $t \geqslant 0$.

\begin{definition}[CTMC]%[continuous-time Markov chain]
  A \emph{continuous-time Markov chain} is a tuple
  $\tuple{\states, \state(0), \qm}$ with
  $\states$ an at most countable set of states,
  $\state(0)$ a \pmf on $\states$
  representing the initial state and
  $\qm$ the \qmatrix of the Markov chain.
  % The Markov chain can be presented as a time-indexed family
  % of random variables $X_t$.
\end{definition}

A few important properties of CTMCs
for the present work are given below.

\begin{definition}[detailed balance]
  A \qmatrix $\qm$ on $\states$
  is said to be \emph{time reversible} iff
  there is a \pmf $\ip$ on $\states$ such that
  \begin{equation}
    \label{eq:detailed-balance}
    \ip_i q_{ij} = \ip_j q_{ji}
  \end{equation}
  for all $i,j \in \states$.
  Then $\qm$ is said to have \emph{detailed balance}
  with respect to $\ip$.
\end{definition}

Detailed balance was first proposed,
in a slightly stronger form
that requires every path going from $i$ to $j$
to have an reverse path with which it is in equilibrium,
by \citet{wegscheider} in the context of chemical kinetics.
Its validity for other physical systems was argued by
\citet{lewis} and \citet{tolman}.
Tolman called the generalised principle
\emph{microscopic reversibility}.
% The stronger form requires every path going from $i$ to $j$
% to have an reverse path with which it is in equilibrium.

\begin{definition}
  A \pmf $\ip$ on $\states$ is
  \emph{invariant} for a \qmatrix $\qm$
  iff $\ip \qm = 0$, \ie
  \[ -\ip_i q_{ii} = \ip_i \sum_{j \neq i} q_{ij}
                  = \sum_{j \neq i} \ip_j q_{ji} \]
\end{definition}

That is to say, $\ip$ is invariant
whenever the action of $\qm$ on $\ip$ does not change $\ip$
or equivalently when $\ip$ is a fixpoint of $\qm$.
% The relationship between detailed balance and an invariant \pmf
% is established by the following lemma.

\begin{lemma}
  Suppose the \qmatrix $\qm$
  has detailed balance with respect to $\ip$.
  Then $\ip$ is invariant for $\qm$.
\end{lemma}
\begin{proof}
  From \eqn{detailed-balance} we obtain
  \[ \sum_{i \in \states} \ip_i q_{ij} =
     \sum_{i \in \states} \ip_j q_{ji} = -\ip_j q_{jj}, \]
  as $\sum_{i \in \states} q_{ji} = -q_{jj}$ for any fixed state $j$.
\end{proof}

Once an invariant \pmf is reached by the Markov chain
it stays there forever.
We would therefore like to know when an invariant
\pmf is realised by the Markov chain.

\begin{definition}[ergodicity]
  A \qmatrix $\qm$ is \emph{ergodic} when
  there is a \pmf $\ip$ on $\states$ such that
  for all $i,j \in \states$,
  \[ \lim_{t \to \infty} P_{ij}(t) = \ip_j \]
\end{definition}

This is equivalent to say that the Markov chain
will converge to the \pmf $\ip$
regardless of the initial state $\state(0)$.

\begin{lemma}
  Suppose the \qmatrix $\qm$ is irreducible
  and has an invariant \pmf $\ip$.
  Then $\qm$ is ergodic and converges to $\ip$.
\end{lemma}

The proof for this lemma can be found in part 2 of theorem 1.6
in chapter 5 of Anderson's book (\cite*[][pages 160--161]{anderson}).

CTMCs have a strong kinetic flavour as they describe
stochastic processes in terms of probability flows
happening at a certain rate. % (read velocities).
% They are the most explicit description in the stochastic world
% for discrete state, continuous time processes.
% and all approaches to describe these processes
% are interpreted in terms of them.
%
It is natural to wonder then how the thermodynamic approach
looks like in the stochastic world.
It turns out the energy function has a very clear interpretation
in this setting, namely, that of defining the probability $\ip_i$
that the system finds itself in state $i \in \states$ as follows
% \citep[page 72]{mcquarrie-stat-mech}.\footnote{
\citep[page 40]{mcquarrie-stat-mech}.\footnote{
  We express the energy in units of $1/kT$
  % (known as inverse temperature)
  to avoid writing this term explicitly.}
\begin{equation}
  \label{eq:energy}
  \ip_i = \frac{e^{-E(i)}}{\sum_{j \in \states} e^{-E(j)}}
\end{equation}
This is known as the \emph{Boltzmann distribution}.
Note that
(i) when given the probability distribution $\ip$
the energy function is defined uniquely
only up to an additive constant;\footnote{
  In other words, if we change the energy of each state by adding
  a fixed constant we obtain the same probability distribution $\ip$.}
(ii) by convention the sign of the energy is inverted
so lower energies represent more favourable states; and
(iii) in the case of detailed balance,
we obtain $e^{E(j)-E(i)} = q_{ji}/q_{ij}$
by combining \eqn{energy} and \eqn{detailed-balance}.

% The next question is
How do we construct a CTMC
from an energy function?
% What else do we need?
% Clearly, we need to know the state space $\states$.
% Also, unlike in classical mechanics,
% we would need to know which transitions between states are possible
% since there are no assumptions of continuity on $\states$.
The first formulation to shed light on this problem
was proposed by \citet{metropolis}.
The algorithm asks for an energy function and
an \emph{a priori} one-step transition probability matrix $A$
where each element $a_{ij}$ ($i,j \in \states$)
denotes the probability that
we choose to jump to state $j$ when we are at state $i$.
Hence $\sum_{j \in \states} a_{ij} = 1$ for any fixed $i$
and we write $a_{i-}$ for this probability distribution.
The $A$ matrix is assumed to be symmetric,
\ie $a_{ij} = a_{ji}$ for all $i,j \in \states$,
% This matrix plays the role of the \qmatrix in the
% discrete-time setting (\ie where time is indexed by the naturals)
% and each element $a_{ij}$ denotes the probability that
% we choose to jump to state $j$ when we are at state $i$.
although this is not strictly necessary
and the algorithm has been later generalised
to work under a weaker assumption ($a_{ij} = 0$ iff $a_{ji} = 0$)
by \citet{hastings}.

% The $A$ matrix asserts which transitions are possible.
% By symmetry, their reverse transitions are also possible.
Note that when addressing the direct problem for CTMCs
by using the Metropolis algorithm
we require an extra ingredient --- the $A$ matrix ---
which was not needed in classical mechanics.
% This extra ingredient was unnecesary
% when addressing the direct problem in classical mechanics
% because implicit assumptions of continuity on $\states$
% supply us with this information.
This is because in classical mechanics there are
implicit assumptions of continuity on $\states$
that supply this information.
The state space is $\RR^{2 \cdot 3}$
and, intuitively, an allowed transition in this continuous space
is a differential change in any direction,
\ie $dx$, $dy$, $dz$.
% In addition to the energy function and the state space,
% we would need to know which transitions are possible.
% This is unlike classical mechanics,
% where assumptions of continuity on $\states$
% make this unnecesary.
%
On the other hand,
the method that will be presented in \chp{direct}
does not ask for a priori transition probabilities
% (\ie an $A$ matrix)
but only which reversible transitions are possible at all.

The construction gives a \emph{discrete-time} Markov chain that
converges to the probability distribution $\ip$ in \eqn{energy}.
% For the sake of simplicity we present here
% only the original formulation.
The algorithm works as follows.
Given any state $i \in \states$ we pick a neighbour state $j$
at random according to the probability distribution $a_{i-}$.
We evaluate the energy function at $i$ and $j$
to compute $\Delta E = E(j)-E(i)$ and proceed with the transition
with probability $1$ if $\Delta E < 0$ and
probability $e^{-\Delta E}$ if $\Delta E > 0$.
Otherwise we stay at state $i$.
In both cases time (a natural number) is increased by 1.
We repeat for state $j$ if the transition was successful
and $i$ otherwise.

To see that $\ip$, as defined in \eqn{energy},
is the invariant probability distribution
of the discrete-time Markov chain
we show that it has (the discrete-time version of)
detailed balance with respect to $\ip$.
The probability $p_{ij}$ of jumping from $i$ to $j$ is
a combination of the a priori probability $a_{ij}$ and
the probability of accepting that transition,
which depends on $\Delta E$.
\[ p_{ij} = a_{ij}\; \min(1, e^{-\Delta E}) \]
By taking the ratio of $p_{ij}$ and $p_{ji}$ we obtain
\[ \frac{p_{ij}}{p_{ji}} =
   \frac{a_{ij}\; \min(1, e^{E(i)-E(j)})}{
         a_{ji}\; \min(1, e^{E(j)-E(i)})} =
   \frac{\min(1, e^{E(i)-E(j)})}{
         \min(1, e^{E(j)-E(i)})} \]
since $a_{ij} = a_{ji}$ by symmetry of $A$.
% of the a priori transition probability matrix.
Suppose $E(i)-E(j) > 0 > E(j)-E(i)$, then
\[ \frac{p_{ij}}{p_{ji}} = e^{E(i)-E(j)}
     = \frac{e^{-E(j)}}{e^{-E(i)}} = \frac{\ip_{j}}{\ip_{i}} \]
It is easy to see that when
$E(j)-E(i) > 0 > E(i)-E(j)$ we obtain the same equation.
Hence the discrete-time Markov chain has detailed balance
with respect to $\ip$. % as defined in \eqn{energy}.
Provided the a priori transition probability matrix $A$
makes it possible to reach any state from any other state,
the Markov chain will converge to $\ip$ as $t \to \infty$.

The Metropolis-Hastings algorithm can be generalised
to the continuous-time case \citep{diaconis}.
However, the algorithm require us to either
(i) compute the energy of all states to obtain the probabilities
$p_{ij}$ (or transition rates $q_{ij}$ in the continuous-time case),
or (ii) do rejection sampling, as outlined above.
Option (i) can be very time-consuming when $\states$ is large
% or the evaluation of the energy function is expensive.
or it's costly to evaluate the energy function.
Option (ii) can be inefficient when the rejection rate is high.
For these reasons we explore an alternative method in this thesis.
We partition the state space in regions of equal energy
and group transitions according to these regions.
This is made possible by assuming extra structure on $\states$
(to be introduced in \sct{kappa}).

Let us go back to the stochastic modelling of
chemical interactions mentioned above.
The theory of CTMCs allows one to frame
the dynamics of chemical reaction systems.
A stochastic approach to such systems
was pioneered by \citet{delbruck}
and has been common practice for decades
\citep{mcquarrie-stoch-kinetics}.
The physical conditions under which this approach is plausibly valid
has been argued by \citet{gillespie76}.
% The stochastic approach has provided insights
% in the understanding of living organisms. Posible refs:
% http://dx.doi.org/10.1371/journal.pbio.1000115
% http://science.sciencemag.org/content/297/5584/1183
% http://www.nature.com/nrg/journal/v6/n6/full/nrg1615.html
% http://science.sciencemag.org/content/320/5872/65
% http://www.alexeikurakin.org/text/akDGE2005.pdf

Since the number of molecules of a species
is a priori unbounded and thus $\states$ might be infinite,
one would like to have a way to express
these systems in a finite and simple form.
A language that could do this was designed by \citet{petri}.
% came to be in the work of \citet{petri}.
This language, later called \emph{Petri nets},
sees reactions as transformations of
multisets of chemical species.

\begin{definition}
  A \emph{multiset} $M$ over a set $X$ is a map from $X$ to
  the naturals assigning to each element $x \in X$
  the number of copies $M(x) \in \NN$ of that element
  in the multiset.
\end{definition}

There is a natural partial order $\leqslant$ on multisets over $X$.
We say $M \leqslant N$ if for each element $x \in X$,
$M(x) \leqslant N(x)$.
We write $\MM(X)$ for the set of all multisets over $X$.

\begin{definition}
  Given a set of species $\species$,
  a \emph{reaction} $r$ is a pair $\tuple{L,R}$
  with $L$ and $R$ multisets over $\species$.
  We refer to $L$ and $R$ as the left- and right-hand side of $r$.
  % and write $L \to R$ for the reaction.
\end{definition}

\begin{definition}%[PN]%[Petri net]
  A \emph{Petri net} is a pair $\tuple{\species, \reactions}$ of
  sets of species $\species$ and reactions $\reactions$.
\end{definition}

A state of a Petri net is a multiset over $\species$,
usually called a \emph{marking}.
A reaction can occur in a given state $M$ only if
its left-hand side $L \leqslant M$.

\begin{definition}
  A \emph{match} of the left-hand side $L$ of a reaction
  on a state $M$ is an injective function from $L$ to $M$
  that identifies each copy of species $s$ in $L$
  with a copy of $s$ in $M$.
\end{definition}

We write $\matches{L}{M}$ for the set of matches from $L$ to $M$.
From this definition we have that the number of matches
$\abs{\matches{L}{M}}$ from $L$ to $M$ is
\[ \abs{\matches{L}{M}} = \prod_{s \in \species} \binom{L(s)}{M(s)} \]
A reaction is said to be elementary iff its rate is
proportional to the number of matches of its left-hand side.
This is known as the \emph{law of mass action} in chemistry.
Here we consider only elementary reactions.

Petri nets can be given a stochastic interpretation
in terms of a CTMC.
Given a Petri net $\tuple{\species, \reactions}$,
an initial marking $M_0$ and
a rate map $k: \reactions \to \RR_{\geqslant 0}$
that assigns rates to reactions,
we construct a CTMC $\tuple{\states, \state(0), \qm}$ as follows.
\begin{align*}
  \states &= \MM(\species) \\
  \state(0)(x) &= \begin{cases}
    1 \quad\text{if } x = M_0 \\
    0 \quad\text{if } x \neq M_0
  \end{cases} \\
  q_{MN} &= \sum_{\substack{r \in \reactions\\r = \tuple{L,R}}}
    k(r) \; \nitoj_{MN}(L,R)
\end{align*}
with
\begin{equation*}
  \nitoj_{MN}(L,R) = \left\{\begin{array}{ll}
    \abs{\matches{L}{M}} & \text{if } M - L + R = N \\
    0 & \text{otherwise}
  \end{array}\right.
\end{equation*}

% Interestingly,
\citet{et2} have solved
the \emph{direct} and \emph{inverse} problem for Petri nets,
that is, they have shown the conditions
the set of reactions and rate map have to fulfil
for a Petri net to have an energy function
and what is the structure of said energy function.
% that is, the problem of constructing a Petri net
% from an energy function on $\states$ and vice versa.

Petri nets have limitations when we take into consideration
what happens inside molecules in a chemical reaction.
The chemical transformation taking place amounts to
a change in the way electrons are shared by atoms
resulting in a relocation of chemical bonds.
In other words, (non-radioactive) reactions are all about
the binding and unbinding of atoms,
how they establish connections and break them.
This is poorly captured by a conversion of species,
as it is modelled by Petri nets.
A consequence of this lack of a formal representation for
molecular bonds is that certain systems of chemical reactions
cannot be described in a finite way using Petri nets,
\eg unbounded polymerisation
(think of a molecular chain that can always attach new links).

Recently,
a formal language to describe biochemical interactions
using rewriting rules,
where molecules not just react but also can bind other molecules
has been proposed by \citet{kappa}.
In the next section we introduce this language, called Kappa,
keeping in mind that we want to address
the \emph{direct} and \emph{inverse} problem mentioned above,
namely, the problem of generating a set of rewriting rules
from an energy function and vice versa.


\section{Kappa}
\label{sec:kappa}

Kappa represents interactions among proteins,
nucleic acids and other biomolecules as
connections in a biomolecular network.
In these networks, nodes stand for the biomolecules
while connections represent transient molecular bonds
(\eg non-covalent interactions like hydrogen bonds).
This network is constantly changing as molecules
travel and interact with other molecules in a cell,
which is viewed as the constant destruction and creation
of the connections that make up the network.

Due to spatial constraints,
molecules can physically interact with
just so many other molecules at once.
Exactly how many will depend on multiple factors like
the size of the two interacting molecules and
the region where they come in contact.
These regions, known in molecular biology by the names of domains,
motifs or binding sites, are simply called \emph{sites} in Kappa.
Any such site can bind at most one other site at a time.
These sites belong to the nodes of the graph,
which Kappa calls \emph{agents}.
In the same way a molecule is of a certain species,
agents can be of different types.
These types also live in a network,
a static network which represents the ``network of possibilities''.
It tell us which sites a site \emph{can} bind
instead of what is actually bound to at a given moment.

To make these ideas formal we will use
the category-theoretical approach
introduced in the work of \citet{kappadpo}.
We first define the networks where types live and then
use them as a basis to construct the actual biomolecular networks.%
% build on them to construct the actual biomolecular networks.%
\footnote{
  Below we use the words graph and edge
  as synonyms for network and connection.}

\begin{definition}%[site graph]
  A \emph{site graph} $G$ consists of
  a finite set of agents $\agents_G$,
  a finite set of sites $\sites_G$,
  a map $\sitemap_G: \sites_G \to \agents_G$
  that assigns sites to agents
  and a symmetric edge relation $\edges_G$ on $\sites_G$.
\end{definition}

The pair $\sites_G$, $\edges_G$ form an undirected graph.
Clearly, the definition of site graphs does not impose
a bound on the number of connections a site can have.
Indeed there is no restriction at all so far.
This is the network where types live.

Sites not in the domain of $\edges_G$ are said to be \emph{free}.
One says $G$ is \emph{realisable} iff
(i) no site has an edge to itself and
(ii) sites have at most one incident edge.
Each realisable site graph represents a
(possibly partially specified\footnote{
  Below you can find the definition of a fully specified state,
  which we call a mixture.})
state in which our biomolecular network can be.
Note however that it contains no typing information.
We give a type to each agent and site in the graph
by assigning to it an agent and site in the type graph.
More precisely,
we need a map from a realisable site graph to a site graph.
Below we introduce such maps.

\begin{flushleft}
\begin{minipage}{.66\linewidth}
\begin{definition}
  A \emph{homomorphism} $h: G \to G'$ of site graphs is
  a pair of functions, $h_\sites: \sites_G \to \sites_{G'}$
  and $h_\agents: \agents_G \to \agents_{G'}$, such that
  for all $s,s' \in \sites_G$ we have
  (i) $h_\agents(\sitemap_G(s)) = \sitemap_{G'}(h_\sites(s))$
  and (ii) if $s \mathbin{\edges_G} s'$ then
  $h_\sites(s) \mathbin{\edges_{G'}} h_\sites(s')$.
\end{definition}
\end{minipage}
\begin{minipage}{.3\linewidth}
\begin{flushright}
  \begin{tikzpicture}
    \matrix (m) [matrix of math nodes,row sep=30pt,column sep=30pt] {
      \sites_G & \sites_{G'} \\
      \agents_G & \agents_{G'} \\};
    \draw[hom] (m-1-1) -- node[above] {$h_\sites$} (m-1-2);
    \draw[hom] (m-2-1) -- node[below] {$h_\agents$} (m-2-2);
    \draw[hom] (m-1-1) -- node[left] {$\sitemap_G$} (m-2-1);
    \draw[hom] (m-1-2) -- node[right] {$\sitemap_{G'}$} (m-2-2);
  \end{tikzpicture}
\end{flushright}
\end{minipage}
\end{flushleft}

Put simply, homomorphisms preserve site ownership and connections.
The diagram to the right is the corresponding
commutative diagram in the category $\Set$
of sets and total functions
to condition (i) in the definition.
We say the homomorphism $h: G \to C$ is
a \emph{contact map} over $C$ iff
(i) $G$ is realisable and
(ii) whenever $h_\sites(s_1) = h_\sites(s_2)$
and $\sitemap_G(s_1) = \sitemap_G(s_2)$, then $s_1 = s_2$.
Condition (ii) means that every agent in $G$ has at most
one copy of each site of its corresponding agent in $C$.
We refer to $C$ as the contact graph.
Contact maps act as the typing map mentioned above.
In particular,
$C$ specifies the types of agents that can exist in $G$,
the sites that they may possess,
and which of the $|\sites_C|(|\sites_C|+1)/2$
possible edge types are actually valid.
% |\sites_C|(|\sites_C|+1)/2 = |\sites_C|(|\sites_C|-1)/2+|\sites_C|,
% that is, each site can bind any other site (divided by 2 to
% compensate for counting each edge twice) or it can bind itself
% (these edges aren't counted twice).

Site graphs and homomorphisms form a category $\SG$.
The composition of two homomorphisms
$h_1: G_1 \to G_2$ and $h_2: G_2 \to G_3$
is a homomorphism $h: G_1 \to G_3$ with
$h_\sites = h_{2,\sites} \comp h_{1,\sites}$ and
$h_\agents = h_{2,\agents} \comp h_{1,\agents}$.
It is easy to see that composition defined in this way is associative.
The identity arrow $\id_G: G \to G$ in $\SG$ is defined
using the identity functions of the corresponding sets.

A homomorphism $\psi: G \to G'$ is an \emph{embedding} iff
(i) $\psi_\agents$ and $\psi_\sites$ are injective and
(ii) if $s$ is free in $G$, so is $\psi_\sites(s)$ in $G'$.
Injectivity of $\psi_\agents$ and $\psi_\sites$ implies that
whenever $\psi: G \to G'$ is an embedding and $G'$ is realisable
then $G$ is also
\begin{wrapfigure}[3]{r}{0.28\textwidth}
  \vspace{-1.5em}
  \begin{center}
    \begin{tikzpicture}
      \matrix (m) [matrix of math nodes,row sep=20pt,column sep=20pt] {
        G & & G' \\
        & C & \\};
      \draw[hom] (m-1-1) -- node[above] {$\psi$} (m-1-3);
      \draw[hom] (m-1-1) -- node[below left] {$g$} (m-2-2);
      \draw[hom] (m-1-3) -- node[below right] (g) {\phantom{$g$}}
                 (m-2-2);
      \node[anchor=south] at (g.south) {$g'$};
    \end{tikzpicture}
  \end{center}
\end{wrapfigure}
realisable.
An embedding $\psi: G \to G'$ between realisable site graphs
can be lifted to a morphism between contact maps $g: G \to C$
and $g': G' \to C$ iff the diagram on the right commutes in $\SG$.

Contact maps over $C$ and embeddings form a category $\rSGe_C$.
Composition and the identity arrow
are defined in a similar manner to $\SG$.
We write $\matches{g}{g'}$ for the set of embeddings
between $g$ and $g'$ in $\rSGe_C$.
We have a functor $\anon{\cdot}$
from $\rSGe_C$ to $\SG$ which forgets types.
In particular, if $g: G \to C$ is a contact map,
we write $\anon{g}$ for its domain $G$.

As an example, consider the site graph $T$ for a triangle.

\vspace{-.2cm}
\begin{minipage}{.5\textwidth}
  \begin{align*}
    \agents_T &= \set{1, 2, 3}, \quad
    \sites_T = \set{l_1, r_1, l_2, r_2, l_3, r_3}, \\
    \sitemap_T &= \setof{s_a \mapsto a}{
      s \in \set{l, r},\; a \in \agents_T}, \\
    \edges_T &= \set{(r_1, l_2), (l_2, r_1), (r_2, l_3),
      (l_3, r_2), (r_3, l_1), (l_1, r_3)}
  \end{align*}
\end{minipage}
\begin{minipage}{.32\textwidth}
  \vspace{.7cm}
  \begin{center}
    \begin{tikzpicture}[grphdiag]
      \nn[n5]{n1}{0,0}{1};
      \nn[n5]{n2}{0:1.4}{2};
      \nn[n5]{n3}{60:1.4}{3};
      \e{n1}{n2};
      \e{n2}{n3};
      \e{n1}{n3};
      \site{r1}{0:10pt};
      \site{l1}{60:10pt};
      \node at (86:15pt) {\scriptsize $l_1$};
      \node at (-26:15pt) {\scriptsize $r_1$};
      \begin{scope}[shift={(0:1.4)}]
        \site{r2}{180:10pt};
        \site{l2}{120:10pt};
        \node at (206:15pt) {\scriptsize $l_2$};
        \node at (94:15pt) {\scriptsize $r_2$};
      \end{scope}
      \begin{scope}[shift={(60:1.4)}]
        \site{r3}{-120:10pt};
        \site{l3}{-60:10pt};
        \node at (-146:15pt) {\scriptsize $r_3$};
        \node at (-34:15pt) {\scriptsize $l_3$};
      \end{scope}
    \end{tikzpicture}
  \end{center}
\end{minipage}
\newline
\vspace{.2cm}

Let us use $T$ as the contact graph
for a contact map $g: G \to T$ where

\vspace{.3cm}
\begin{center}
  $G = \;\;$
  \begin{tikzpicture}[grphdiag,baseline=-2.5,thick]
    \path[use as bounding box] (-0.8,0.4) rectangle (3.4,-0.42);
    \e{-0.6,0}{3.2,0};
    \begin{scope}
      \nn[n5]{x}{0,0}{$x$};
      \site{lx}{x.west};
      \site{rx}{x.east};
      \node[anchor=north east,yshift=2] at (180:.2) {\scriptsize $l_x$};
      \node[anchor=south west,yshift=-1] at (0:.2) {\scriptsize $r_x$};
    \end{scope}
    \begin{scope}[shift={(1.3,0)}]
      \nn[n5]{y}{0,0}{$y$};
      \site{ly}{y.west};
      \site{ry}{y.east};
      \node[anchor=north east,yshift=2] at (180:.2) {\scriptsize $l_y$};
      \node[anchor=south west,yshift=-1] at (0:.2) {\scriptsize $r_y$};
    \end{scope}
    \begin{scope}[shift={(2.6,0)}]
      \nn[n5]{z}{0,0}{$z$};
      \site{lz}{z.west};
      \site{rz}{z.east};
      \node[anchor=north east,yshift=2] at (180:.2) {\scriptsize $l_z$};
      \node[anchor=south west,yshift=-1] at (0:.2) {\scriptsize $r_z$};
    \end{scope}
  \end{tikzpicture}
\end{center}
\begin{alignat*}{3}
  \agents_G &= \set{x, y, z} &
  \sitemap_G &= \setof{s_a \mapsto a}{
    s \in \set{l, r},\; a \in \agents_G} \\
  \sites_G &= \set{l_x, r_x, l_y, r_y, l_z, r_z} \quad\quad &
  \edges_G &= \set{(r_x, l_y), (l_y, r_x), (r_y, l_z), (l_z, r_y)}
\end{alignat*}
and
\vspace{-.4cm}
\begin{align*}
  g_\agents &= \set{x \mapsto 1, y \mapsto 2, z \mapsto 3} \\
  g_\sites &= \setof{s_a \mapsto s_{a'}}{
    s \in \set{l, r},\; a \in \agents_G,\; a' = g_\agents(a)}
\end{align*}

Sites $l_x$ and $r_z$ in $G$ are free,
which we denote graphically by a stub coming out of the site.
$T$ and $G$ are realisable since no site is bound to itself or
bound to more than one other site.
Note however that the codomain of a contact map ($T$ in this case)
does not have to be realisable in general.

To ease the definition of concrete contact maps,
we colour agents according to their type
and annotate sites by their name in $C$.
The contact map $g: G \to T$ above can then be defined
more succinctly by the following drawing.
\vspace{.3cm}
\begin{center}
  \begin{tikzpicture}[grphdiag,thick]
    \e{-0.5,0}{2.7,0};
    \begin{scope}
      \n[n1]{x}{0,0};
      \site{lx}{x.west};
      \site{rx}{x.east};
      \node[anchor=north east,yshift=2] at (180:.2) {\scriptsize $l$};
      \node[anchor=south west,yshift=-1] at (0:.2) {\scriptsize $r$};
    \end{scope}
    \begin{scope}[shift={(1.1,0)}]
      \n[n2]{y}{0,0};
      \site{ly}{y.west};
      \site{ry}{y.east};
      \node[anchor=north east,yshift=2] at (180:.2) {\scriptsize $l$};
      \node[anchor=south west,yshift=-1] at (0:.2) {\scriptsize $r$};
    \end{scope}
    \begin{scope}[shift={(2.2,0)}]
      \n[n3]{z}{0,0};
      \site{lz}{z.west};
      \site{rz}{z.east};
      \node[anchor=north east,yshift=2] at (180:.2) {\scriptsize $l$};
      \node[anchor=south west,yshift=-1] at (0:.2) {\scriptsize $r$};
    \end{scope}
  \end{tikzpicture}
\end{center}
where we have assigned colours orange, blue and green to
agent types $1$, $2$, $3$ in $C$.
We have written $l$ and $r$ for sites $l_1,l_2,l_3$ and $r_1,r_2,r_3$
as the subscript can be deduced from the colour of the agent as well.

Whenever a contact map $g: G \to C$ specifies all sites
that its type $C$ permits for all its agents, that is,
if for all $a \in \agents_G$,
$h_\sites(\sitemap_G^{-1}(a)) = \sitemap_C^{-1}(h_\agents(a))$,
then we say $g$ is a \emph{mixture}.
We write $\MM(C)$ for the set of all mixtures in $\rSGe_C$.
In the above example, $g$ is a mixture.
What other mixtures are there that have $T$ as contact graph?
We can have chains of any length and closed cycles
of length some multiple of three like triangles, hexagons, etc.
We can have any disjoint sum of them as well.

% TODO: next sentence could be improved
Mixtures, being fully specified biomolecular networks
with respect to the type $C$,
are a natural choice for the states of our dynamical system.
We jump from state to state by the applications of \emph{rules}.

\begin{definition}
  A \emph{rule} $r$ is a pair of contact maps
  $r_L: L \to C$, $r_R: R \to C$
  which differ only in their edge structures,
  \ie $\agents_L = \agents_R$, $\sites_L = \sites_R$,
  $\sitemap_L = \sitemap_R$, $r_{L,\agents} = r_{R,\agents}$
  and $r_{L,\sites} = r_{R,\sites}$.
\end{definition}

In the context of the contact graph $T$,
we define a rule that binds
agents of type $1$ with agents of type $2$ as follows.

\vspace{-.6cm}
\begin{flushright}
\begin{minipage}{.43\linewidth}
\begin{center}
  \begin{tikzpicture}[thick]
    \node[grphnode] (lhs) at (0,0) {
      \tikz[ingrphdiag]{
        \begin{scope}[shift={(0,0)}]
          \n[n1]{x}{0,0};
          \e{x}{.5,0};
          \site{rx}{x.east};
          \node at (26:.42) {\scriptsize $r$};
        \end{scope}
        \begin{scope}[shift={(1.2,0)}]
          \n[n2]{y}{0,0};
          \e{y}{-.5,0};
          \site{ly}{y.west};
          \node at (206:.42) {\scriptsize $l$};
        \end{scope}
      }};
    \path (lhs.east) +(.3,0) edge[rule] +(1,0)
      +(1.3,0) coordinate (r);
    \node[grphnode,anchor=west] (rhs) at (r) {
      \tikz[ingrphdiag]{
        \e{0,0}{1.1,0};
        \begin{scope}
          \n[n1]{x}{0,0};
          \site{rx}{x.east};
          \node at (26:.42) {\scriptsize $r$};
        \end{scope}
        \begin{scope}[shift={(1.1,0)}]
          \n[n2]{y}{0,0};
          \site{ly}{y.west};
          \node at (206:.42) {\scriptsize $l$};
        \end{scope}
      }};
  \end{tikzpicture}
\end{center}
\end{minipage}
\begin{minipage}{.5\linewidth}
\begin{equation*}
  \arraycolsep=2pt
  \begin{array}{ccccl}
    \agents_L &=& \agents_R &=& \set{u,v} \\
    \sites_L &=& \sites_R &=& \set{r_u, l_v} \\
    \sitemap_L &=& \sitemap_R &=& \set{r_u \mapsto u, l_v \mapsto v} \\
    r_{L,\agents} &=& r_{R,\agents} &=& \set{u \mapsto 1, v \mapsto 2} \\
    r_{L,\sites} &=& r_{R,\sites} &=& \set{r_u \mapsto r_1, l_v \mapsto l_2} \\
    && \edges_L &=& \emptyset \\
    && \edges_R &=& \set{(r_u,l_v),(l_v,r_u)}
  \end{array}
\end{equation*}
\end{minipage}
\end{flushright}

Note that there is no site $l$ in $u$ and no site $r$ in $v$.
Hence $r_L$ and $r_R$ are not mixtures
as they are only partially specified.
Intuitevely, this means that the rule can be applied
regardless of whether those sites are bound or free.

\begin{wrapfigure}[5]{r}{0.35\textwidth}
  \vspace{-1.7em}
  \begin{equation}
    \label{eq:rewrite-step}
    \tikz[baseline=-2.5]{
      \matrix (m) [matrix of math nodes,row sep=35pt,column sep=35pt] {
        r_L & r_R \\
        h & \comatch{h} \\};
      \draw[hom,dotted] (m-1-1) -- (m-1-2);
      \draw[hom,dotted] (m-2-1) -- (m-2-2);
      \draw[hom] (m-1-1) -- node[left] {$\psi$} (m-2-1);
      \draw[hom] (m-1-2) -- node[right] {$\comatch{\psi}$} (m-2-2);
    }
  \end{equation}
\end{wrapfigure}

When a rule $r$ is applied to an embedding $\psi: r_L \to m$
it induces a \emph{rewrite} of the mixture $m$
by modifying the edge structure of the image of $\psi$
from that of $r_L$ to that of $r_R$.
The result of rewriting is a new mixture $\comatch{m}$
and an embedding $\comatch{\psi}: r_R \to \comatch{m}$, where
$\anon{\comatch{m}}$ has the same agents and sites as $\anon{m}$,
\ie $\agents_{\anon{\comatch{m}}} = \agents_{\anon{m}}$,
$\sites_{\anon{\comatch{m}}} = \sites_{\anon{m}}$,
$\sitemap_{\anon{\comatch{m}}} = \sitemap_{\anon{m}}$,
$\comatch{m}_\agents = m_\agents$,
$\comatch{m}_\sites = m_\sites$,
and $\edges_{\anon{\comatch{m}}} = \edges_{\anon{m}} -
\psi_\sites(\edges_{\anon{r_L}}) +
\comatch{\psi}_\sites(\edges_{\anon{r_R}})$.
The embedding $\comatch{\psi}$ is simply defined by
$\comatch{\psi}_\agents = \psi_\agents$ and
$\comatch{\psi}_\sites = \psi_\sites$.
The inverse of $r$,
defined as $\inv{r} := (r_R,r_L)$ is also a valid rule.
By applying $\inv{r}$ to $\comatch{\psi}$
we recover $m$ and $\psi$.
% By rewriting $\comatch{m}$ with $\inv{r}$ via $\comatch{\psi}$,
% we recover $m$ and $\psi$.

\begin{lemma}
  Let $r = (r_L, r_R)$ be a rule,
  $r_L/\rSGe_C$ the coslice category under $r_L$, and
  $r_R/\rSGe_C$ the coslice category under $r_R$.
  The categories $r_L/\rSGe_C$ and $r_R/\rSGe_C$ are isomorphic.
\end{lemma}
\begin{proof}
  We construct a functor $F: r_L/\rSGe_C \to r_R/\rSGe_C$
  by mapping an embedding $\psi: r_L \to m$
  to the result of applying $r$ to it,
  $\comatch{\psi}: r_R \to \comatch{m}$.
  By definition $\comatch{\psi}_\agents = \psi_\agents$ and
  $\comatch{\psi}_\sites = \psi_\sites$.
  Hence, the mapping of embeddings induced by $F$ is injective:
  whenever the application of $r$ to two embeddings
  $\psi: r_L \to m$ and $\phi: r_L \to m'$
  results in $\comatch{\phi}$ and $\comatch{\psi}$
  with $\comatch{\phi} = \comatch{\psi}$,
  then $\phi = \psi$.
  By an analogous argument, we construct a functor
  $G: r_R/\rSGe_C \to r_L/\rSGe_C$ using $\inv{r}$ that
  maps embeddings injectively in the reverse direction.
  Applying $r$ followed by $\inv{r}$ to $\psi$
  results in $\psi$ itself.
  Therefore $GF = \id_{r_L/\rSGe_C}$ and $FG = \id_{r_R/\rSGe_C}$.
\end{proof}

Intuitively, this property characterises a \emph{reversible} rule.
Reversibility will be important to obtain detailed balance as
every rule $r$ must be in balance with its inverse $\inv{r}$.

Given a finite set of rules $\generators$ over $C$,
an initial mixture $m_0$
and a rate map $k$ from $\generators$ to $\RR_{\geqslant 0}$,
we construct a CTMC $\tuple{\states, \state(0), \qm}$ as follows.
\begin{align}
  \label{eq:kappa-ctmc}
  \states &= \MM(C) \nonumber \\
  \state(0)(x) &= \begin{cases}
    1 \quad\text{if } x = m_0 \\
    0 \quad\text{if } x \neq m_0
  \end{cases} \\
  q_{mn} &= \sum_{\substack{r \in \generators\\r = \tuple{r_L,r_R}}}
    k(r) \; \nitoj_{mn}(r_L,r_R) \nonumber
\end{align}
with
\begin{equation*}
  \nitoj_{mn}(r_L,r_R) = \left\{\begin{array}{ll}
    \abs{\matches{r_L}{m}} & \text{if } \comatch{m} = n \\
    0 & \text{otherwise}
  \end{array}\right.
\end{equation*}
It is easy to see that the number of embeddings
between any two contact maps $x,y$ is finite.
Hence, $q_{mm} = -\sum_{n \neq m} q_{mn} > -\infty$
for any fixed $m \in \states$
and $\qm$ is a well-defined \qmatrix.
Here we use the law of mass action for the rate of rules,
where the rate is proportional to the number of embeddings (matches)
of the left-hand side of our rules.
In fact, the law of mass action simply amounts to say that
rewrites induced by rule $r$ are independent processes,
each one occurring at rate $k(r)$.
This is clear in the formulation of
a \emph{labelled transition system} for Kappa.

\begin{definition}
  A \emph{labelled transition system} $\LTS$ is a tuple
  $\tuple{\states,\labels,\to}$ with
  $\states$ a set of states,
  $\labels$ a set of labels,
  and $\to\; \subseteq \states \times \labels \times \states$
  a set of transitions $\tuple{x,\alpha,y}$
  between states $x,y \in \states$
  labelled by $\alpha \in \labels$.
\end{definition}

Given $\generators$,
we define a labelled transition system $\LTS_\generators$
on mixtures over $C$ where
a transition from a mixture $m$ is labelled by
an \emph{event} $(r,\psi)$, as in \diagram{rewrite-step},
with $r$ in $\generators$ and $\psi$ in $\matches{r_L}{m}$.
The CTMC presented in \eqn{kappa-ctmc} can then be equivalently
constructed from $\LTS_\generators$ by simply assigning
rate $k(r)$ to an event of the form $(r,\psi)$.
Thus we write $\LTS^k_\generators$ for the CTMC.
Note that the (strongly) connected components of $\LTS_\generators$
are finite as agents cannot be destroyed nor created by rules.

With all the ingredients on the table
we can proceed now to formulate more precisely
the main question addressed in this thesis.
In the context of the dynamics of biomolecular networks,
the direct problem is stated as,
given a contact graph $C$ and
an energy function on mixtures over $C$,
how do we generate a set of rules $\rules$
with a corresponding rate map $k: \rules \to \RR_{\geqslant 0}$
such that the CTMC $\LTS^k_\rules$ has detailed balance with respect to
the probability distribution $\ip$ as defined in \eqn{energy}?
In practice, just an energy function won't be enough,
we will have to ask for an initial set of rules (without rates)
from which to derive the set with detailed balance.
We write $\generators$ for the generator set that is given as input.
Finally, detailed balance only constrains the ratio between
the rates of a rule and its inverse so to pick concrete rates
we impose a linear model which is explained in \sct{kinetic-model}.
The full method is presented in \chp{direct}.

On the other hand, the inverse problem is stated as,
given a contact graph $C$,
a set of rules $\rules$ over $C$
and a rate map $k$,
does the CTMC $\LTS^k_\rules$ have detailed balance?
If so, how do we obtain
its invariant probability distribution and energy function?
The former question has been proven to be undecidable by \citet{et1}
using an encoding of the Post correspondence problem \citep{post}
in Kappa.
In \chp{inverse} we address the inverse problem for
restricted versions of the Kappa language
that are decidable.


\section{Related work}

\citet{anc} have developed the Allosteric Network Compiler
(ANC), a language to describe biomolecular interaction networks
that have an energy function and a tool to generate
a chemical reaction system (\ie a Petri net) from them.
% The language presents a restricted form of the rules and
% energy functions studied here.
The biomolecular interaction networks they introduce
have ANC structures as nodes.
These structures can contain hierarchical components and
interaction sites, which in turn can be catalytic,
covalently modified or ligand-binding sites.
Hierarchical components can contain any number of
interaction sites and nested hierarchical components.
If a hierarchical component is marked as allosteric,
it can transition between two conformational states.
The transition rates can be modified by the state of the sites and
other components present in the same structure
according to parameters given for each of them.
An additional parameter is required for covalently modified
and ligand-binding sites which determines the change in the ratio
between the two conformational states when a ligand is bound
or the site is modified.

The edges of the ANC network are connections between ligand-binding sites
or between a catalytic site and a covalently modified site.
Rules can be of two types, binding or enzymatic,
and can only depend on the conformational state of the participants.
Binding rules specify the association and dissociation of
two ligand-binding sites.
Enzymatic rules follow a Michaelis-Menten mechanism in which
the enzymatic site reversibly binds a covalently modified site first
and then changes the state of the covalently modified site
as it unbinds it.
Each type of rule is parameterised differently.
The energy function is obtained implicitly
from all the parameters of the model.
The language formalises concepts that are familiar to
molecular biologists and biochemists
in a way that reflects the apparent complexity
of these interaction networks.
However, the language's many concepts and classifications
makes it difficult to see the big picture
and obfuscates the energy function. % obscures
Also it arbitrarily restricts the form and size of the rules
and the form of the energy function.
The work presented in \chp{direct} is a generalisation of ANC.
In \chp{inverse} we will show a language similar to ANC
for which the form of the energy function is uniquely determined
from the kinetic rate parameters.

% Herzfeld and Stanley general allosteric model
% http://cps.bu.edu/hes/articles/hs74.pdf

% http://www.sciencedirect.com/science/article/pii/S0301462202000911
% http://onlinelibrary.wiley.com/doi/10.1080/15216540701272380/pdf

% Calculus of biochemical complexes at equilibrium
% http://bib.oxfordjournals.org/content/8/4/226.long

Another related development is that of
biomolecular interaction databases.
During the last two decades
databases like BindingDB \citep{bindingdb,bindingdb15},
BIND \citep{bind}, MINT \citep{mint,mint12},
MatrixDB \citep{matrixdb09} and BioLiP \citep{biolip} have appeared
with the aim to collect data pertinent to biomolecular interactions
that have been described in the scientific literature.
Of particular interest to us is that they gather information
about thermodynamic and kinetic parameters
like equilibrium and rate constants for these interactions.
The proliferation of molecular interaction databases
have prompted the creation of
the International Molecular Exchange (IMEx) consortium \citep{imex}
and a common data format,
the Human Proteome Organization Proteomics Standards Initiative
Molecular Interactions (HUPO-PSI-MI) format
\citep{hupo-psi-mi-1,hupo-psi-mi-2.5}.
The latest version of this format,
which is used by all molecular interaction databases,
% provides a means for the specification of
% provides the possibility of specifying
provides a means to specify
thermodynamic and kinetic parameters of an interaction.
% http://psidev.sourceforge.net/molecular_interactions//rel25/doc/
% #element_parameter_Link03B317A8
%
Also of interest are databases
specialised in the thermodynamic properties of reactions like
the Thermodynamics of Enzyme-Catalyzed Reactions Database
at the National Institute of Standards and Technology (NIST)
\citep{tecrdb} and others \citep{biocoda-thermo}.
% NOTE: What happened to biocoda?
% NOTE: Biocoda uses ThermoML, so I thought it was related
% but it seems it's much more focused in organic chemistry
% than biochemistry. (see the files in the repository).
% The NIST % Thermodynamics Research Center at the NIST
% has also developed an XML-based format to store
% experimental thermophysical and thermochemical property data
% named ThermoML \citep{thermoml} and used by several journals
% in the chemistry and biochemistry literature.
% They maintain a web-repository of all published ThermoML files
% at \url{http://www.trc.nist.gov/ThermoML.html}.
Moreover, sometimes it is possible to predict these % parameters
thermodynamic constants and how they vary in different solutions
\citep{equilibrator,group-contrib}.
Together all of these tools, databases and algorithms
provide a strong infrastructure that facilitates the construction
of thermodynamic models of biomolecular interaction networks.

Thermodynamic models have already been used successfully
in many areas of biology.
For instance, in chemotaxis\footnote{
  Chemotaxis is the process by which a living organism
  avoid certain molecules like poisons and
  chase other molecules like food.}
thermodynamic models have been put forward to explain
the positioning of the chemoreceptors on the membrane \citep{wingreen},
their cooperative adaptation mechanisms to keep a high sensitivity
for different ligands in different environments \citep{sourjik},
and the action of motor that rotates the flagellum
for the bacterium to move or stay in a given place \citep{teuta}.
By recreating the latter model we show in \sct{alloring}
how to use our method to construct such thermodynamic models.
Other examples include models in metabolism \citep{seba-tca,cannon},
macromolecular assembly \citep{saiz},
transcription regulation \citep{bintu}
and more \citep{kiselev}.

Understanding the relationship between the kinetics and thermodynamics
of biomolecular interactions may help understand
the relationship between animate and inanimate matter.
It has been argued by \citet{pross},
in an attempt to bridge the worlds of animate and inanimate matter,
that the former % animate matter (\ie life)
corresponds to the set of \emph{persisting molecular replicators}.
A molecular replicator is a molecule or set of molecules that,
usually in several steps,
can fulfil the following transformation,
\[ R + F \to R + R + W \]
where $R$ is the replicator,
$F$ is a molecule consumed by $R$ to construct a second $R$
(mnemonically named $F$ for food)
and $W$ (for waste) is what is left of $F$
that was not used for the replication.
This transformation is known as \emph{autocatalysis}.
Under unlimited resources a molecular replicator
grows exponentially fast.
On the other hand, when $F$ is exhausted
the concentration of $R$ will converge to that
dictated by the thermodynamic equilibrium.
Persistent replicators are those that manage to
keep themselves in a far-from-equilibrium regime
and thus continue replicating to maintain its population.
It is my belief that this type of processes can be investigated
by introducing additional rules
to the set of thermodynamically-consistent rules
generated by our method.\footnote{
  Note that when the additional irreversible rules
  do not intersect the reversible ones,
  one still can get an extended notion of detailed balance
  by an analogous argument to that put forward by \citet{gorban}.}
These sorts of modifications opens up
a possible new line of work which we might call
far-from-equilibrium graph thermodynamics.
The convergence properties
of thermodynamically-consistent sets of rules
presented in \sct{energy-gp}
would not hold in a far-from-equilibrium regime
but might nevertheless serve as a reference for comparison.
Moreover, the rates calculated for the rules by the method
would still be valid as these depend only on
the chemical properties of molecules in solution
(reactants, products, solvent, etc).
In \sct{kinetic-model} a framework is proposed
to systematically assign rates to rules based on
some of these chemical considerations.
It allows the exploration of kinetics
in a thermodynamically consistent way.
Hence this framework might play an important role
in the study of far-from-equilibrium systems.

\if0
The importance of the relationship
between kinetics and thermodynamics
in the understanding of fundamental
physico-chemical principles of living systems
is also highlighted by the recent work of \citet{pross}.
The author generalises Darwin's concept of evolution
and natural selection to the realm of autocatalytic
replicating molecules and molecular assemblies.
In particular, he extends the idea of fitness
to that of \emph{kinetic stability}.
A kinetically stable replicator is defined as
one that can maintain a significant equilibrium population
of members under the existing environmental conditions.
He argues that elements within the non-replicative region
of general chemical space are directed toward lower free energy states,
whereas elements of replicator space are driven by autocatalysis
and tend from kinetically less stable to kinetically more stable.
Replicators may indeed constantly stay out of equilibrium
by using a source of energy,
a phenomenon that is ubiquitous in living systems.
It is therefore perhaps questionable
whether using a thermodynamic approach
can shed any light on the understanding of these systems.
\fi




%%% Local Variables:
%%% mode: latex
%%% TeX-master: "thesis"
%%% End:

