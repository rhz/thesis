In this chapter
we would like to explore restricted versions of Kappa
for which it is possible to infer the energy function
from the rewriting rules and their associated rates.
Recall from the introduction
that in Kappa itself this problem is undecidable \citep{et1}.
One such restriction is when agents do not have sites
and thus cannot bind.
This is Petri nets.
We briefly present here the result obtained by \citet{et2}
and show the construction of the energy function
for \emph{simple} and \emph{symmetric} Petri nets
with \emph{mass action} semantics
(sisma Petri net for short).
\begin{definition}
A \emph{sisma Petri net} is a Petri net
on species $\species$ and reactions $\reactions$ for which:
\begin{enumerate}[label={(\roman*)}]
\item (simple) there are no two reactions that have
  the same stoichiometry (net change in species).
\item (symmetric) for each reaction $r \in \reactions$,
  there is a reaction $\inv{r} \in \reactions$
  that has the reverse direction,
  \ie the inputs of one are the outputs of the other.
\item (mass action) the jumping rate $q_{xy,r}$
  of going from a state $x$ to $y$
  by a reaction $r$ is proportional to
  the number of ocurrences of its left-hand side in $x$.
  In particular,
  given a rate constant $k(r)$ for reaction $r$,
  % the jumping rate of $r$ at state $x$ is
  we have
  \[ q_{xy,r} = k(r) \prod_{A \in \species}
     \frac{x(A)!}{(x(A)-\Delta_r(A))!} \]
  where $x(A)$ is the number of $A$s in $x$
  and $\Delta_r(A)$ is the net change of $A$ in reaction $r$.
\end{enumerate}
\end{definition}
Note that simple implies
no two distinct reactions can be applied
to a state $x$ to obtain state $y$.
A sisma Petri net has an energy function
\begin{equation}\label{eq:pn-energy}
  E(x) = \sum_{A \in \species} \cost(A) x(A) + \ln\bigl(x(A)!\bigr)
\end{equation}
for some function $\cost: \species \to \RR$ such that,
for all $r \in \reactions$,
\[ \sum_{A \in \species} \Delta_r(A) \cost(A) = \ln(k(r^\star)) - \ln(k(r)). \]
If there is no such function $\cost$,
the Petri net does not have detailed balance and
an energy function.
%
In the rest of the chapter
we introduce two other restrictions of Kappa
and show how to construct their energy function.


\section{Cooperative assembly systems}
\label{sec:cas}

The first restriction is when
rules can only create or destroy one edge at a time
and their rates can only depend on
how many bound sites the endpoints of the edge have.
% the type of the neighbours of the two endpoints of the edge
% but not the state of their sites
% (whether they are free or bound).
% Moreover, sites in an agent are regarded as \emph{indistinguishable}
% and thus rules can only count how many are bound.
Therefore sites are treated as \emph{indistinguishable}.
In addition, agents of the same type cannot bind.
\citet{cas} have proposed these restrictions
and a simple formalism incorporating them
to study the thermodynamics of polymer formation
when there are two types of monomers.
Here we extend their result to any number of monomer types.
% TODO: perhaps take the idea in the next sentence
% and integrate into the paragraph:
% Cooperativity means that the rate of binding (and unbinding)
% depends on the number of connections that have been established
% by the participating nodes.

In the case of two monomers,
rules are of the form
\begin{center}
  \begin{tikzpicture}
    \node[grphnode,anchor=east] (lhs) at (0,0) {
      \tikz[ingrphdiag]{
        \begin{scope}[shift={(0,0)}]
          \e{0,0}{90:1.1};
          \e{0,0}{150:1.1};
          \e{0,0}{270:1.1};
          % a
          \n[n1]{a}{0,0};
          \e{a}{.5,0};
          \site{x1}{a.north};
          \site{x2}{150:.25};
          \site{xn}{a.south};
          \site{xn+1}{a.east};
          % b1
          \n[n2]{b1}{90:1.1};
          \site{z}{b1.south};
          % b2
          \n[n2]{b2}{150:1.1};
          \site[shift={(150:1.1)}]{z}{-30:.25};
          % ellipsis
          \node at (200:.79) {\Large .};
          \node at (210:.8) {\Large .};
          \node at (220:.79) {\Large .};
          % bn
          \n[n2]{bn}{270:1.1};
          \site{z}{bn.north};
        \end{scope}
        \begin{scope}[shift={(1.2,0)}]
          \e{0,0}{90:1.1};
          \e{0,0}{30:1.1};
          \e{0,0}{270:1.1};
          % b
          \n[n2]{b}{0,0};
          \e{b}{-.5,0};
          \site{y1}{b.north};
          \site{y2}{30:7pt};
          \site{yn}{b.south};
          \site{yn+1}{b.west};
          % a1
          \n[n1]{a1}{90:1.1};
          \site{z}{a1.south};
          % a2
          \n[n1]{a2}{30:1.1};
          \site[shift={(30:1.1)}]{z}{210:.25};
          % ellipsis
          \node at (-20:.79) {\Large .};
          \node at (-30:.8) {\Large .};
          \node at (-40:.79) {\Large .};
          % an
          \n[n1]{an}{270:1.1};
          \site{z}{an.north};
        \end{scope}
      }};
    \path (lhs.east) +(.3,.09) edge[rule] +(1,.09)
      +(1.3,0) coordinate (r);
    \path (lhs.east) +(1,-.09) edge[rule] +(.3,-.09);
    \node[grphnode,anchor=west] (rhs) at (r) {
      \tikz[ingrphdiag]{
        \e{0,0}{1.1,0};
        \begin{scope}[shift={(0,0)}]
          \e{0,0}{90:1.1};
          \e{0,0}{150:1.1};
          \e{0,0}{270:1.1};
          % a
          \n[n1]{a}{0,0};
          \site{x1}{a.north};
          \site{x2}{150:.25};
          \site{xn}{a.south};
          \site{xn+1}{a.east};
          % b1
          \n[n2]{b1}{90:1.1};
          \site{z}{b1.south};
          % b2
          \n[n2]{b2}{150:1.1};
          \site[shift={(150:1.1)}]{z}{-30:.25};
          % ellipsis
          \node at (200:.79) {\Large .};
          \node at (210:.8) {\Large .};
          \node at (220:.79) {\Large .};
          % bn
          \n[n2]{bn}{270:1.1};
          \site{z}{bn.north};
        \end{scope}
        \begin{scope}[shift={(1.1,0)}]
          \e{0,0}{90:1.1};
          \e{0,0}{30:1.1};
          \e{0,0}{270:1.1};
          % b
          \n[n2]{b}{0,0};
          \site{y1}{b.north};
          \site{y2}{30:7pt};
          \site{yn}{b.south};
          \site{yn+1}{b.west};
          % a1
          \n[n1]{a1}{90:1.1};
          \site{z}{a1.south};
          % a2
          \n[n1]{a2}{30:1.1};
          \site[shift={(30:1.1)}]{z}{210:.25};
          % ellipsis
          \node at (-20:.79) {\Large .};
          \node at (-30:.8) {\Large .};
          \node at (-40:.79) {\Large .};
          % an
          \n[n1]{an}{270:1.1};
          \site{z}{an.north};
        \end{scope}
      }};
  \end{tikzpicture}
\end{center}
where the three black dots on the sides of each graph
are an ellipsis to mean that monomers can be bound
to an arbitrary number of monomers of the other type
as long as each site is bound only once
and there is a finite number of sites per monomer
fixed by the monomer type.
% TODO: is this sentence clear?
Hence, the rule schema % formula?
represents a family of rules indexed by % $i,j$
the number of bound sites in the two monomers.
% TODO: add the following?
% Note that no nodes can be created or destroyed
% by this family of rules
% and so the total number of nodes is fixed.

We formalise the ideas in the first paragraph as follows.
Let $\types$ be the set of monomer types % $A,B,\dots$
and $\valence: \types \to \NN$ the map that assigns
% for each type the number of sites they have,
to each type the number of sites a monomer of that type has,
which we refer to as their valence.
A~monomer $u$ has type $\typeof(u) \in \types$
and degree $\degree_x(u) \in \NN$ in state $x$.
We simply write $\valence(u)$ for $\valence(\typeof(u))$.
The rate constant of a rule that binds
a monomer of type $\tp \in \types$ and degree $i$ with
a monomer of type $\tp' \in \types$ and degree $j$
is $\gamma^+_{\tp,i,\tp',j}$.
The rate constant of the reverse rule (unbinding)
is $\gamma^-_{\tp,i,\tp',j}$.
%
The jumping rate $q_{xy}$ from state $x$ to $y$
% is then determined by mass action semantics,
% \ie is linear in the number of ocurrences
% of the left-hand side of the rule that operates the transition.
is then linearly determined by the number of ocurrences
of the left-hand side of the rule and the rate constant
(mass action semantics).

The binding or unbinding of any two nodes $u,v$ in $x$
can only be carried out by one rule,
namely the one that operates on degrees $\degree_x(u),\degree_x(v)$.
The binding rule has rate constant
$\alpha(u,v) := \gamma^+_{\typeof(u),\degree_x(u),\typeof(v),\degree_x(v)}$
while the unbinding rate constant is
$\beta(u,v) := \gamma^-_{\typeof(u),\degree_x(u),\typeof(v),\degree_x(v)}$.
Moreover, we are free to choose
one site among the $\valence(u)-\degree_x(u)$ free sites of $u$
and one site among the $\valence(v)-\degree_x(v)$ free sites of $v$
in order to apply the binding rule.
Hence, $q_{xy}$ is equal to $\alpha(u,v)
(\valence(u) - \degree_x(u)) (\valence(v) - \degree_x(v))$
% in the forward direction (binding)
when the binding rule is applied to $x$ to obtain $y$
and $\beta(u,v)$
% in the backward direction (unbinding).
when the unbinding rule is.

Below it is the theorem that shows under which conditions
the systems presented in this section have an energy function.

\if0
\begin{proposition}
  Let $\types$ be a finite set of monomer types.
  The system has detailed balance
  (and thus an energy function)
  if and only if
  \begin{equation}
    \label{eq:cas-db}
    \gamma^-_{\tp,i,\tp',j} / \gamma^+_{\tp,i,\tp',j} =
    \Gamma_\tp(i) \Gamma_{\tp'}(j)
  \end{equation}
  for all $\tp,\tp' \in \types$,
  $i \in \set{0,1,\ldots,\valence(\tp)}$ and
  $j \in \set{0,1,\ldots,\valence(\tp')}$.
\end{proposition}
\begin{proof}
  % Let $n_\tp$ be the total number of nodes of type $\tp$ and
  Let $x_0$ the state where all nodes have degree $0$,
  \ie the completely disconnected graph.
  % We define the energy of a state $x$ as
  We construct a function $E: \states \to \RR$ as
  \[ E(x) = \sum_{\tp \in \types} \sum_{0 < i \leqslant \valence(\tp)}
            \cost_\tp(i) x(\tp_i) \]
  with $x(\tp_i)$ the number of nodes of type $\tp$
  with degree $i$ in $x$ and
  % TODO: which other variable name is good for the iterator a?
  \[ \cost_\tp(i) = \sum_{0 \leqslant a < i}
     \ln\, \frac{\Gamma_\tp(a)}{\valence_\tp - a} \]
  Note that $E(x_0) = 0$ since $x_0(\tp_i) = 0$ for all $i > 0$.

  ($\Leftarrow$):
  We can now verify that $E$ is an energy function
  that describes an invariant distribution $\pi$
  of the continuous-time Markov chain
  by showing detailed balance with respect to $\pi$.
  Recall from \eqn{detailed-balance}
  the detailed balance condition,
  that for all $x,y \in \states$ 
  \[ \pi_x q_{xy} = \pi_y q_{yx} \]
  When $q_{xy} = 0$ this equality holds % trivially holds
  % as $q_{yx}$ is also zero.
  as \eqn{cas-db} dictates that
  a rate constant is non-zero only if
  the reverse rate constant is also non-zero.
  % NOTE: q_{xy} = 0 might be because there is no transition
  % between x and y or because the rate constant is zero.
  % Also, by reversibility of rules
  % % TODO: better word for enabled?
  % whenever a transition from $x$ to $y$ is enabled by a rule $r$,
  % its inverse $\inv{r}$ enables a transition in the opposite direction,
  % \ie from $y$ to $x$.
  In the case when $y = x + (u,v)$
  the equality becomes,
  % FIXME: here it seems that we are proving the right-direction
  % but before we said we are proving the left-direction
  by substituting $\pi(x)$ and $\pi(y)$ as in \eqn{energy},
  % and $q(\_,\!\_)$ as in \eqn{cas-jump-rate},
  \begin{equation*}
    \expn\left[
    -\sum_{\tp \in \types} \sum_{0 < i \leqslant \valence(\tp)}
    \cost_\tp(i) x(\tp_i)
    \right] q_{xy} =
    % \gamma^+_{\typeof(u),\degree_x(u),\typeof(v),\degree_x(v)}
    % (\valence(u) - \degree_x(u)) (\valence(v) - \degree_x(v)) = \\
    \expn\left[
    -\sum_{\tp \in \types} \sum_{0 < i \leqslant \valence(\tp)}
    \cost_\tp(i) y(\tp_i)
    \right] q_{yx}
    % \gamma^-_{\typeof(u),\degree_x(u),\typeof(v),\degree_x(v)}
  \end{equation*}
  % with $\expn(x) = e^x$.
  where the divisors from \eqn{energy} have cancelled each other out.
  By rearranging
  \[ \prod_{\tp \in \types} \prod_{0 < i \leqslant \valence(\tp)}
     e^{\,\cost_\tp(i) (y(\tp_i) - x(\tp_i))} =
     \frac{q_{yx}}{q_{xy}} \]
  % FIXME: rephrase
  However $y(\tp_i) - x(\tp_i) = 0$ for most $\tp$ and $i$
  since $y = x + (u,v)$.
  Only when $\tp$ is $\typeof(u)$ and $i$ is $\degree_x(u)$
  or $\tp$ is $\typeof(v)$ and $i$ is $\degree_x(v)$
  will $y(\tp_i) - x(\tp_i)$ equal $-1$.
  On the other hand, when $\tp$ is $\typeof(u)$
  and $i$ is $\degree_y(u) = \degree_x(u) + 1$
  or $\tp$ is $\typeof(v)$
  and $i$ is $\degree_y(v) = \degree_x(v) + 1$,
  $y(\tp_i) - x(\tp_i)$ will be $1$.
  For conciseness, let $\tp = \typeof(u)$, $\tp' = \typeof(v)$,
  $i = \degree_x(u)$ and $j = \degree_x(v)$.
  It follows the above equality can be rewritten as
  \[ \expn\left(
       \cost_{\tp}(i+1) +
       \cost_{\tp'}(j+1) -
       \cost_{\tp}(i) -
       \cost_{\tp'}(j) \right) =
     \frac{q_{yx}}{q_{xy}} \]
  % By expanding ... % Philipp didn't like expanding
  By substituting $\cost$ and $q$ we obtain
  \[ \frac{
       \Gamma_\tp(i) \Gamma_{\tp'}(j)}{
       (\valence(u) - i) (\valence(v) - j)} =
     \frac{
       \gamma^-_{\tp,i,\tp',j}}{
       \gamma^+_{\tp,i,\tp',j}
       (\valence(u) - i) (\valence(v) - j)}, \]
  which in turn can be trivially obtained from
  \[ \Gamma_\tp(i) \Gamma_{\tp'}(j) =
     \gamma^-_{\tp,i,\tp',j} / \gamma^+_{\tp,i,\tp',j}.
     \qedhere \]
\end{proof}
\fi

\if0
\section{Flipping and binding} % co-ANC
\label{sec:fb}
\fi







%%% Local Variables:
%%% mode: latex
%%% TeX-master: "thesis"
%%% End:
