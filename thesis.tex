\documentclass[phd,lfcs]{infthesis}
% \documentclass[phd,lfcs,twoside]{infthesis}
% \documentclass{report}

\usepackage[utf8]{inputenc}
\usepackage[T1]{fontenc}
\usepackage[british]{babel}
\usepackage{microtype}
\usepackage[usenames,dvipsnames,svgnames,table]{xcolor}
% \usepackage{natbib}
% \usepackage[natbibapa]{apacite}
\usepackage[english=british,autopunct=false]{csquotes}
% maxcitenames=2,uniquelist=false
\usepackage[natbib=true,style=authoryear,maxbibnames=6]{biblatex}
\usepackage{graphicx}
\usepackage{textcomp}
\usepackage{wrapfig}
\usepackage{xfrac}
\usepackage{xspace}
% \usepackage{spacing}
% \usepackage{epigraph}
% \usepackage{gnuplot-lua-tikz}
\usepackage{mathcommon}
\usepackage{kappa}
% \usepackage{kappalistings}
\usepackage[sc]{mathpazo}
\usepackage{hyperref}
% \usepackage{blindtext}
\usepackage{expl3}

% Bibliography
\addbibresource{thesis.bib}

% Text
\newcommand{\ie}{i.e.\xspace}
\newcommand{\eg}{e.g.\xspace}

% Referencing
\newcommand{\chp}[1]{\S\ref{chp:#1}}
\newcommand{\sct}[1]{\S\ref{sec:#1}}
\newcommand{\subsct}[1]{\S\ref{subsec:#1}}
\newcommand{\eqn}[1]{Eq.~\ref{eq:#1}}
\newcommand{\eqns}[2]{Eq. \ref{eq:#1} and \ref{eq:#2}}
\newcommand{\lem}[1]{Lemma~\ref{lemma:#1}}
\newcommand{\lems}[2]{Lemmas \ref{lemma:#1} and \ref{lemma:#2}}
\newcommand{\thm}[1]{Th.~\ref{thm:#1}}
\newcommand{\fig}[1]{Fig.~\ref{fig:#1}}
\newcommand{\diagram}[1]{diagram~\ref{eq:#1}}
\newcommand{\app}[1]{Appendix~\ref{app:#1}}
\newcommand{\mcite}[1]{\textcolor{gray}{#1}} % missing cite

% Thermodynamic graph rewriting
\newcommand{\rules}{\mathcal{R}}
\newcommand{\generators}{\mathcal{G}}
\newcommand{\shapes}{\mathcal{P}}
\newcommand{\cost}{\epsilon} % \varepsilon is the same with this font
% TODO: should I enclose the result of \refinedrules in parenthesis?
\newcommand{\refinedrules}{\generators_\shapes}
\newcommand{\gp}{\Gamma}

% Labelled transition systems
\newcommand{\LTS}{\mathcal{L}}
\newcommand{\labels}{\Lambda}

% Probability
\newcommand{\pmf}{%
  probability distribution\xspace} % probability mass function

% Markov chains
\newcommand{\states}{\mathcal{S}} % state space
\renewcommand{\state}{\xi}
\newcommand{\qm}{Q} % q matrix
\newcommand{\ip}{\pi} % invariant probability
\newcommand{\nitoj}{\gamma}
% \newcommand{\qmatrix}{infinitesimal generator\xspace}
\newcommand{\qmatrix}{$q$-matrix\xspace}
% \newcommand{\qmatrices}{infinitesimal generators\xspace}
\newcommand{\qmatrices}{$q$-matrices\xspace}

% Petri nets
\newcommand{\species}{\Sigma} %\mathcal{A}}
\newcommand{\reactions}{\mathfrak{R}} %\mathcal{R}}
\newcommand{\matches}[2]{\left[#1;#2\right]}
\newcommand{\MM}{\mathbf{M}}

% Cooperative assembly systems
\newcommand{\valence}{\nu} %\upsilon}
\newcommand{\types}{\mathcal{T}} % set of types
\newcommand{\type}{t} % a type (variable)
\newcommand{\typeof}{\tau} % map from nodes to types
\newcommand{\degree}{d}

% FB-systems
\newcommand{\sitesof}{\sigma}
\newcommand{\emptyvec}{\varnothing}
\newcommand{\neighbours}{\vec{n}}
\newcommand{\sitestates}{\vec{s}}
\newcommand{\sitestatesof}{\iota}
\newcommand{\neighboursof}{\eta}

% Kappa
\newcommand{\edges}{\mathcal{E}}
\newcommand{\SG}{\mathbf{SG}}
\newcommand{\rSGe}{\mathbf{rSGe}}
\newcommand{\anon}[1]{\left|#1\right|}
\newcommand{\inv}[1]{#1^\dagger}
\newcommand{\comatch}[1]{#1^\star}

% Category theory
\newcommand{\homset}{\Upsilon}
\newcommand{\mSet}{\mathbf{mSet}}
\newcommand{\Set}{\mathbf{Set}}
\newcommand{\iso}{\simeq}

% Triangles
\newcommand{\cyc}[1]{{}\cdot #1 \cdot{}}

% Math
\renewcommand{\tuple}[1]{\left(#1\right)}
\DeclareMathOperator*{\expn}{exp}
\renewcommand{\qedsymbol}{\ensuremath{\blacksquare}}
\newcommand{\partialto}{\rightharpoonup}
\newcommand{\id}{\vec{1}} % identity function

% Other stuff
\newcommand{\maybe}[1]{\textcolor{gray}{#1}}
% \renewcommand{\maybe}[1]{}
\newcommand{\todo}[1]{\textcolor{red}{TODO: #1}}
% \renewcommand{\todo}[1]{}

% Styles for nodes
\tikzstyle{n}=[circle,draw=Black,fill=White] %,thick,minimum size=6pt,inner sep=0pt]
\tikzstyle{n1}=[n,fill=Orange]
\tikzstyle{n2}=[n,fill=Blue!60!White]
\tikzstyle{n3}=[n,fill=Green!40!White] % Green!40!Ocre?
% \tikzstyle{n4}=[n,fill=White!95!Black]
% \tikzstyle{n5}=[n,fill=White]
% \tikzstyle{n6}=[n,fill=Black]
% \tikzstyle{n7}=[n,fill=Yellow!50!White]
\tikzstyle{p}=[n,draw=none] % phantom
\tikzstyle{e}=[thick]
\tikzstyle{rule}=[thick,->,>=angle 90]
\tikzstyle{onarrow}=[fill=White,inner sep=2pt]
\tikzstyle{site}=[n,fill=Yellow!50!White,inner sep=1.5pt]

% Binding types
\tikzstyle{bt1}=[n1,dashed,inner sep=3.5pt]
\tikzstyle{bt2}=[n2,dashed,inner sep=3.5pt]
\tikzstyle{bt3}=[n3,dashed,inner sep=3.5pt]

% Background box for graph diagrams
\tikzstyle{grphdiag-bg}=[rounded corners,fill=White!85!Black]

% Graph diagrams
\tikzstyle{grphnode}=[rectangle,grphdiag-bg]
% \tikzstyle{ingrphdiag}=[>=stealth,thick,semithick,
%   n/.append style={semithick,minimum size=4pt}]
\tikzstyle{ingrphdiag}=[thick,anchor=center]

% Use different arrow tips to distinguish edges from morphism
% and a light gray backdrop.
\usetikzlibrary{backgrounds}
\tikzstyle{grphdiag}=[>=stealth,framed,thick,%
  background rectangle/.style=grphdiag-bg]

% Draw a node
\newcommand{\n}[3][n]{\node[inner sep=5pt,#1] (#2) at (#3) {}}
\newcommand{\nn}[4][n]{\node[inner sep=3.5pt,#1] (#2) at (#3) {#4}}
% \renewcommand{\site}[3][n7]{\node[inner sep=1.5pt,#1] (#2) at (#3) {}}

% Draw an edge
\newcommand{\e}[3][e]{\draw[#1] (#2) -- (#3)}

% Draw a directed edge
\newcommand{\de}[3][e]{\draw (#2) edge[->,#1] (#3)}

% Draw a bi-directional rule
\newcommand{\birulearrow}[2]{\hspace{1ex}\tikz{%
  \draw[rule] (0, .15) -- node[above] {#1} ++( .5, 0);
  \draw[rule] (.5,  0) -- node[below] {#2} ++(-.5, 0);
}\hspace{1ex}}

% Arrows in minimal gluings diagram
\newcommand{\arrsn}[3][]{\arr[#1]{#2.south}{#3.north}}
\newcommand{\arr}[3][]{%
  \draw[-bigto,#1] ($(#2)!.1!(#3)$) -- ($(#2)!.9!(#3)$);}

% Header
\title{On a thermodynamic approach to
  biomolecular interaction networks}
\author{Ricardo Honorato-Zimmer}

% Abstract
\if0
\abstract{
  We develop a new thermodynamic approach to stochastic graph rewriting.
  The ingredients are a finite set of reversible
  graph rewriting rules $\generators$ (called generating rules),
  a finite set of connected graphs $\shapes$ (called energy patterns),
  and an energy cost function $\cost$
  which associates real values to each of these energy patterns.
  % 
  The idea is that $\generators$ defines the qualitative dynamics,
  by showing which transformations are possible,
  while $\shapes$ and $\cost$ allow one to attach an energy
  to the reachable graphs and, thereby,
  describe their long-term probability distribution $\ip$.
  % 
  Given $\generators$ and $\shapes$,
  we construct a finite set of rules $\generators_{\shapes}$ which
  (i) has the same qualitative transition system as $\generators$; and
  (ii) when equipped with rates according to $\cost$,
  defines a continuous-time Markov chain of which $\ip$ is
  the \emph{stationary} and \emph{limiting} probability distribution.
  % 
  The construction relies on the use of site graphs and
  a technique of `growth policy' for quantitative rule refinement.
  % which is of independent interest.
  % 
  % Nothing else is assumed of $\generators$ or $\shapes$,
  % and only the rates on the generated rule set $\generators_{\shapes}$
  % depend on $\cost$.

  This division of labour between the qualitative and long-term
  quantitative aspects of the dynamics leads to intuitive and concise
  descriptions for realistic models.
  % 
  It also guarantees thermodynamical consistency
  (\emph{aka} detailed balance),
  otherwise known to be undecidable.
  % which is important for some applications.
  % 
  Finally, it leads to parsimonious parameterisations of models,
  % again an important point in some applications.
  an important point in the application of stochastic graph rewriting
  to the modelling of biochemical interaction networks.
}
\fi

\begin{document}

%% First, the preliminary pages
\begin{preliminary}

\maketitle

\if0
\begin{acknowledgements}
  I'm really thankful to everyone that's helped me along the way.
  It's been a marvelous journey with all its highs and lows
  and so much learning.
  % This experience has made me feel alive and transformed me
  % in so many ways.
  This experience has transformed me in so many ways.
  % TODO: perhaps rewrite this section in an inline, less verbose,
  % less direct style like the following
  I'd like to thank Vincent for taking me as a PhD student,
  forming me as a scientist and as a better person.
  It's difficult to condese in a few sentences the gratitude I feel
  when I think of the so many things
  he patiently taught me during these years.
  I'm lucky to have had a genius supervisor.
  I'd like to thank him particularly for having faith in me
  even when things didn't look good.

  % Vincent, I'd like to thank you first for having me in your group.
  % For having patiently taught me so many things
  % from math to good life lessons.
  % Thank you for blowing my mind so many times
  % with your brilliant ideas and ways to look at things.
  % Thank you for sharing your time with me
  % and having faith in me even when things didn't look good.

  Renate y Ricardo, muchas gracias por haberme dado la vida,
  por haberme criado, por haberme enseñado y nutrido
  de tan buenas ideas, creencias y alimentos.
  Muchas gracias por haberse preocupado por mi en todo momento,
  tanto por mi salud, como bienestar y desarrollo.
  De m\'as est\'a decir, sin uds nada de esto habr\'ia sido posible.
  Muchas gracias por ser los maravillosos padres que son.

  Dani, muchas gracias por ser la maravillosa hermana que eres.
  Como poner en palabras todas las cosas que hemos compartido
  y todas las cosas que me has enseñado en toda esta vida juntos,
  que me han formado y me han hecho en lo que soy.
  Muchas gracias por tu eterna y sincera preocupacion por mi.
  Te llevo siempre en mi corazon.

  Froko, muchas gracias mi compañero de la vida
  por todo el tiempo que hemos compartido,
  por toda la invaluable compañia,
  por todo el amor en los momentos dificiles
  y por todas las cosas que gracias a ti he logrado comprender.
  No hay suficientes palabras para decirte todo lo que te aprecio.

  An\'ibal and Gr\'ainne, thank you so much guys for all the love
  and good vibes in the good and bad moments of this journey.
  It was amazing to share this time together
  and having the comfort of your words and hugs with me.
  Without your reassuring empathy I wouldn't have made it.

  Mark, thank you so much
  % for teaching me the good lessons and what life ultimately is all about.
  for showing me the tools and techniques to open the heart
  to the expected as well as the unexpected
  and naturally find peace in doing so.
  I could see all along how you embody this compassionate way of being
  and that gave me the strength to start softening my own heart.
  % for giving me tools that have given me some much needed space
  % and freedom.
  % Listening to you was challenging at the beginning
  % but I could see all the time how you embody this compassionate way
  % of being.
  % but slowly started changing my life.

  Sophie, Stuart, Ayleen and Susan, thank you so much guys for
  being the great friends and flatmates that you were.
  Thank you for the countless laughs and the support and love.
  I'm so happy to have you guys in my life.

  Philipp, thank you so much for the wonderful time together.

  Seba,

  Alejandro,

  Claudia,

  Sandro, thank you for patiently explaining me so many important things
  and thank you for sharing your bright ideas with me.
  Thank you so much for all the honest love.

  Tobias, thank you so much for being so nice, soft and caring to me. % I needed it.
  Thank you so much also for showing me some of your math wizardry,
  I really appreaciate it.

  William, thank you for the wonderful and insightful conversations. % sharing your knowledge and opinions
  I really like having met you, having you in the group
  and having you as one of my friends.

  Illias, your cool style and humility are true sources of inspiration.

  Gordon,

  Thank you very much to all the other people in the group
  for the stimulating conversation and companionship,
  Milana, Andrea, Guoli, Matteo, Guillaume, Fredrik, Matthias, Nick,
  Katharina, Emilia, David, Oksana, John, Hristiana,
  Argyris, Andreea and Simon.
  I would like to thank Russ, Jean and J\'er\^ome
  for the camaraderie, the good feedback and
  the advice regarding my PhD and academic career.
  Thanks also to Dr Alfred Hofmann and Dr Alexander Shulgin
  for their life-changing inventions that have provided me
  with beautiful insights during this time.
  I'd like to thank as well everyone that helped me get to the PhD,
  in particular special thanks go to Dr Tomas Perez-Acle
  and Dr Juan Carlos Letelier. % who helped me when.../in this way...
  Last and perhaps most importantly to the present version
  of this manuscript, I'd like to thank Paweł Sobociński,
  whose invaluable feedback and dedicated reading of the first
  version of this manuscript catalysed % have been crucial and
  the improvement that gave life to this final version.
\end{acknowledgements}
\fi

%% Next we need to have the declaration.
\standarddeclaration

%% Finally, a dedication (this is optional --
%% uncomment the following line if you want one).
% \dedication{To my mummy.}

%% Create the table of contents
\tableofcontents

%% If you want a list of figures or tables,
%% uncomment the appropriate line(s)
% \listoffigures
% \listoftables

\end{preliminary}

\chapter{Introduction}
\label{chp:intro}
% this needs package epigraph
% \setlength{\epigraphwidth}{.8\textwidth}
% \setlength{\epigraphrule}{0pt}
% \epigraph{
%   I regard as quite useless the reading of
%   large treatises of pure analysis:
%   too large a number of methods pass at once before the eyes.
%   It is in the works of applications that one must study them;
%   one judges their ability there and
%   one apprises the manner of making use of them.}{
%   --- Joseph Louis Lagrange}

% \begin{quotation}
%   \textit{
%     ``I regard as quite useless the reading of
%     large treatises of pure analysis:
%     too large a number of methods pass at once before the eyes.
%     It is in the works of applications that one must study them;
%     one judges their ability there and
%     one apprises the manner of making use of them.''} \\
%   \par\raggedleft--- Joseph Louis Lagrange
% \end{quotation}

% from https://hbfs.wordpress.com/2011/01/18/epigraphs-in-latex/
% epigraph with 3 params: width, text, author
\newcommand{\epigraph}[3]{
\vspace{1em}\hfill{}\begin{minipage}{#1}{\begin{spacing}{0.9}
\small\noindent\textit{#2}\end{spacing}
\vspace{1em}
\hfill{}{#3}}\vspace{2em}
\end{minipage}}
\epigraph{.8\textwidth}{
  I regard as quite useless the reading of
  large treatises of pure analysis:
  too large a number of methods pass at once before the eyes.
  It is in the works of applications that one must study them;
  one judges their ability there and
  one apprises the manner of making use of them.}{
  --- Joseph Louis Lagrange}

\noindent
In the history of natural sciences,
there has been two main approaches to describe dynamical systems:
\emph{Kinetics} and \emph{Thermodynamics}.
The former goes all the way back to
Newton's laws of motion~\citep{newton}.
% In particular, the second law of motion describes
% how these forces translate into the second derivative of
% the position of the particles over time, \ie their acceleration.
Broadly speaking, in the kinetic approach we describe the forces
acting on each particle at every point in time.
The net force determines how
the position of the particle changes over time.
The fact this net force can be decomposed into separate forces
that can be measured independently is very useful in practice,
\eg the gravitational force on an object does not depend on
the other forces acting on that object.
% In a way this is the most detailed description
% that we can have at the level of point particles.

% On the other hand,
In the thermodynamic approach
% the movement and transformation of particles
% is represented in the energy function of the system.
% is seen as a consequence of the minimisation of the energy of the system.
all information concerning the system and the forces acting on it
is contained in the energy function.
This approach first appeared in the work of
\citet{lagrange2} and \citet{hamilton}.
It endowed the description of a dynamical system in classical mechanics
with a remarkable conciseness, simplicity and elegance. % :
% just one function could describe the entirety of the system.
The kinetic description can then be derived from this energy function.
However the opposite is not true:
in general a kinetic description might not have an energy function
from which it can be derived in the classical framework.
This is because thermodynamics puts restrictions on possible processes.
The specifics of this correspondence have been described
in detail by . % FIXME: cite

Later on in the history of natural sciences,
% when chemistry started to become a quantitative science,
the dynamics of gas compression and expansion
and its relation to heat was established.
% also came in the two variants.
First came the Kinetic theory of gases by . % FIXME: cite
% First the Kinetic theory of gases was developed by .
This theory single-handledly gave birth to
the field of statistical mechanics. % TODO: check stat mech wiki page
% Is the kinetic theory of gases the one that has two sides?
% The equations that link energy, entropy, temperature, pressure,
% volume and so on are part of that theory?
Then the laws of thermodynamics came about
to describe the macro behaviour of systems
with an underlying energy function. % TODO: check this please
Similarly to classical mechanics,
in the thermodynamic approach to statistical mechanics
some processes are improbable and some outright impossible.

% Later still,
Half a century had passed since the publication of
the Kinetic theory of gases and 
a formal language to describe chemical reaction systems
% came to be by the hand of Petri
was invented by Petri. % FIXME: cite
% the description of dynamical reaction systems
% was also done in both ways.
This language, later called Petri nets,
would define a reaction as a transformation of
multisets of chemical species.
% What happens historically between the invention of Petri nets
% and an algorithm to simulate them?
% When can we start talking about the two approaches in Petri nets?
In this framework we find the two approaches too:
SSA and Metropolis-Hastings...

However, Petri nets as a language for describing chemical reactions
has limitations when we take into consideration
how chemistry actually works.
It's an abstraction that leaks because it doesn't fully cover
the underlying phenomena.
In the world of (non-radioactive) chemical reactions
atoms do not transform,
they just change their partners with whom they share their electrons.
It's all about atom binding and unbinding,
establishing connections and breaking them.

More recently,
a formal language to describe biochemical interactions
where molecules not just react but can also bind other molecules
non-covalently has been created by . % FIXME: cite

In this work, we are interested in the correspondence between
the kinetic and thermodynamic descriptions of graph rewriting.
Graph rewriting is ...


















%%% Local Variables:
%%% mode: latex
%%% TeX-master: "thesis"
%%% End:



\chapter[The direct problem: From energy to rules]{
  The direct problem \\
  \LARGE From energy to rules}
\label{chp:direct}
In this chapter we show how to construct a set of reversible rules
and their forward and backward rate constants from an energy function.
In the spirit of rule-based modelling languages like Kappa
where rules and observables are defined in terms of patterns,\footnote{
  Recall that a pattern is a contact map used to find subgraphs in states.}
we use a set of \emph{energy patterns} $\shapes$
for our energy function.
We assign an \emph{energy cost} $\cost(g)$ to each of them
and build the energy function as a linear combination
of their number of ocurrences. % of each energy pattern.
\[ E(m) = \sum_{g \in \shapes} \cost(g) \abs{\matches{g}{m}} \]
This is reminiscent of group contribution methods
used to estimate the standard Gibbs free energy of formation
of biomolecules \citep{group-contrib}.

As mentioned at the end of \sct{kappa},
we will derive the set of rules with detailed balance
from a set of generator rules $\generators$ (without rates).
We suppose that $\generators$ is closed under
rule inversion, \ie $\generators = \inv{\generators}$.
Given a contact graph $C$,
a simple option would be to include
every possible minimal rule in this set,
that is, include a creation and a destruction rule
for each edge in the contact graph.
Each of these rules is minimal in the sense that
it only asks for the presence of
the two participating agents and sites.
The example rule in \sct{kappa} (page~\pageref{p:example})
where agents of type $1$ and $2$ bind
% regardless of the binding state of any other site,
regardless of the context
in which these two agents happen to be,
which we denote by $r^+_{12}$,
is one such minimal rule
that can be derived from contact graph $T$.
This option is \emph{maximally permissive}
% as every possible transformation
% allowed by the contact graph
% is allowed by $\generators$.\footnote{
with respect to the contact graph.\footnote{
  Intuitively, this is analogous to the case of classical mechanics
  % where the topology of the space gives us the possible transformations
  where, a priori, movement is not constrained along any coordinate.}
Even if all transformations are possible,
many of them may be unlikely due to having a high energy.
Still one might prefer to forbid certain transformations
in some scenarios.
This is indeed the case in the example
that will be presented in \sct{alloring}.

In our previous example (\sct{kappa}),
we might want to favour the formation of
triangles over chains and other cycles.
For this we give a negative energy cost to the trinagle $t$,
\ie $\cost(t) < 0$.
If $t$ is the only energy pattern,
then the energy of a state $m$ is
$E(m) = \cost(t) \abs{\matches{t}{m}}$.
In this model one might, for instance,
wonder how low the energy cost of $t$ must be
to have at least $90\%$ of all agents in a triangle
at equilibrium at least $90\%$ of the time.

We would like to find rules that have detailed balance
with respect to this energy function.
Consider the rule $r^+_{12}$ and its inverse $r^-_{12}$,
the unbinding of agents $1$ and $2$.
% Given the maximally permissive set of generator rules
% $\generators=\set{r^+_{12},r^-_{12},r^+_{23},r^-_{23},r^+_{31},r^-_{31}}$,
% we first ask ourselves if these reversible rules
We first ask ourselves if this pair of rules
could have detailed balance
for some assignment of kinetic rates.
% to the forward and backward rule.
Suppose we assign kinetic rates $k^+$ and $k^-$
to $r^+_{12}$ and $r^-_{12}$.
Recall from \sct{bg} that $e^{E(n)-E(m)} = q_{nm}/q_{mn}$
for systems with detailed balance.
From \eqn{kappa-ctmc}
\[ q_{mn} = \sum_{\substack{r \in \generators\\r = \tuple{r_L,r_R}}}
   k(r) \; \abs{\setof{\psi \in \matches{r_L}{m}}{m^{(r,\psi)} = n}}
\]
It is clear that at most one of the two rules
can bring us from state $m$ to $n$, say it is $r^+_{12}$.
By rule reversibility (\lem{reversibility})
$r^-_{12}$ brings us from $n$ back to $m$
and the number of matches of $r^-_{12}$ in $n$
is equal to the number of matches of $r^+_{12}$ in $m$.
Hence, $e^{E(n)-E(m)} = k^+/k^-$.
In words, the change in energy produced by the rule application
fixes the ratio between the kinetic rates.
As a consequence,
each rule application should produce the same energy change
for there to be an assignment of kinetic rates with detailed balance.
Whenever a rule produces the same energy change
regardless of where it is applied
we say that the rule has an \emph{unambiguous energy balance}
or is $\shapes$-balanced.
More generally, we define $\shapes$-balance as follows.

\begin{definition}
  Given a contact graph $C$
  and a set $\shapes$ of contact maps over $C$,
  a rule $r$ is $\shapes$-balanced
  if, for all mixtures $m$ and embeddings $\psi: r_L \to m$,
  the number of ocurrences of $p \in \shapes$
  produced and consumed by $r$ when applied to $\psi$
  is a fixed number
  $\Delta_r p = |[p;m^{(r,\psi)}]| - \abs{\matches{p}{m}}$.
  % is a fixed number $\Delta_r p$,
  % \ie $|[p;m^{(r,\psi)}]| - \abs{\matches{p}{m}} = \Delta_r p\;$
  % for all $p \in \shapes$.
  We refer to $\Delta_r p$ as the balance of $r$ with respect to $p$.
  % We refer to the vector of ocurrence changes as $\Delta_r \shapes$.
\end{definition}
% TODO: perhaps add a remark about unambiguous stoichiometry

The following two rule applications show that
$r^+_{12}$ is not $\shapes$-balanced.
\begin{center}
  \resizebox{.9\linewidth}{!}{%
  \begin{tikzpicture}[thick]
    % first row
    \node[grphnode,anchor=east] (lhs1) at (0,0) {
      \tikz[ingrphdiag]{
        \begin{scope}[shift={(0,0)}]
          \n[n1]{x}{0,0};
          \e{x}{.5,0};
          \site{rx}{x.east};
          \node at (26:.42) {\scriptsize $r$};
        \end{scope}
        \begin{scope}[shift={(1.2,0)}]
          \n[n2]{y}{0,0};
          \e{y}{-.5,0};
          \site{ly}{y.west};
          \node at (206:.42) {\scriptsize $l$};
        \end{scope}
      }};
    \path (lhs1.east) +(.3,0) edge[rule,dotted] +(1,0)
      +(1.3,0) coordinate (r1);
    \node[grphnode,anchor=west] (rhs1) at (r1) {
      \tikz[ingrphdiag]{
        \e{0,0}{1.1,0};
        \begin{scope}
          \n[n1]{x}{0,0};
          \site{rx}{x.east};
          \node at (26:.42) {\scriptsize $r$};
        \end{scope}
        \begin{scope}[shift={(1.1,0)}]
          \n[n2]{y}{0,0};
          \site{ly}{y.west};
          \node at (206:.42) {\scriptsize $l$};
        \end{scope}
      }};
    % second column
    \node[grphnode,anchor=east] (lhs2) at (9,0) {
      \tikz[ingrphdiag]{
        \begin{scope}[shift={(0,0)}]
          \n[n1]{x}{0,0};
          \e{x}{.5,0};
          \site{rx}{x.east};
          \node at (26:.42) {\scriptsize $r$};
        \end{scope}
        \begin{scope}[shift={(1.2,0)}]
          \n[n2]{y}{0,0};
          \e{y}{-.5,0};
          \site{ly}{y.west};
          \node at (206:.42) {\scriptsize $l$};
        \end{scope}
      }};
    \path (lhs2.east) +(.3,0) edge[rule,dotted] +(1,0)
      +(1.3,0) coordinate (r2);
    \node[grphnode,anchor=west] (rhs2) at (r2) {
      \tikz[ingrphdiag]{
        \e{0,0}{1.1,0};
        \begin{scope}
          \n[n1]{x}{0,0};
          \site{rx}{x.east};
          \node at (26:.42) {\scriptsize $r$};
        \end{scope}
        \begin{scope}[shift={(1.1,0)}]
          \n[n2]{y}{0,0};
          \site{ly}{y.west};
          \node at (206:.42) {\scriptsize $l$};
        \end{scope}
      }};
    % second row
    \path (lhs1.south) +(0,-.2) edge[rule] +(0,-.6);
    \node[grphnode,anchor=east] (lhs3) at (0,-2) {
      \tikz[ingrphdiag]{
        \begin{scope}[shift={(0,0)}]
          \n[n1]{x}{0,0};
          \e{x}{.5,0};
          \e{x}{-.5,0};
          \site{lx}{x.west};
          \site{rx}{x.east};
          \node at (206:.42) {\scriptsize $l$};
          \node at (26:.42) {\scriptsize $r$};
        \end{scope}
        \e{1.2,0}{2.3,0};
        \begin{scope}[shift={(1.2,0)}]
          \n[n2]{y}{0,0};
          \e{y}{-.5,0};
          \site{ly}{y.west};
          \site{ry}{y.east};
          \node at (206:.42) {\scriptsize $l$};
          \node at (26:.42) {\scriptsize $r$};
        \end{scope}
        \begin{scope}[shift={(2.3,0)}]
          \n[n3]{z}{0,0};
          \e{z}{.5,0};
          \site{lz}{z.west};
          \site{rz}{z.east};
          \node at (206:.42) {\scriptsize $l$};
          \node at (26:.42) {\scriptsize $r$};
        \end{scope}
      }};
    \path (lhs3.east) +(.3,0) edge[rule,dotted] +(1,0)
      +(1.3,0) coordinate (r3);
    \path (rhs1.south) +(0,-.2) edge[rule] +(0,-.6);
    \node[grphnode,anchor=west] (rhs3) at (r3) {
      \tikz[ingrphdiag]{
        \e{0,0}{2.2,0};
        \begin{scope}[shift={(0,0)}]
          \n[n1]{x}{0,0};
          \e{x}{-.5,0};
          \site{lx}{x.west};
          \site{rx}{x.east};
          \node at (206:.42) {\scriptsize $l$};
          \node at (26:.42) {\scriptsize $r$};
        \end{scope}
        \begin{scope}[shift={(1.1,0)}]
          \n[n2]{y}{0,0};
          \site{ly}{y.west};
          \site{ry}{y.east};
          \node at (206:.42) {\scriptsize $l$};
          \node at (26:.42) {\scriptsize $r$};
        \end{scope}
        \begin{scope}[shift={(2.2,0)}]
          \n[n3]{z}{0,0};
          \e{z}{.5,0};
          \site{lz}{z.west};
          \site{rz}{z.east};
          \node at (206:.42) {\scriptsize $l$};
          \node at (26:.42) {\scriptsize $r$};
        \end{scope}
      }};
    % second row, second column
    \path (lhs2.south) +(0,-.2) edge[rule] +(0,-.6);
    \node[grphnode,anchor=east] (lhs4) at (9,-2.4) {
      \tikz[ingrphdiag]{
        \e{0,0}{-56.944:1.1};
        \e{0:1.2}{-56.944:1.1};
        \begin{scope}[shift={(0,0)}]
          \n[n1]{x}{0,0};
          \e{x}{.5,0};
          \site{r1}{0:7pt};
          \site{l1}{-60:7pt};
          \node at (-86:12pt) {\scriptsize $l$};
          \node at (26:12pt) {\scriptsize $r$};
        \end{scope}
        \begin{scope}[shift={(0:1.2)}]
          \n[n2]{y}{0,0};
          \e{y}{-.5,0};
          \site{r2}{180:7pt};
          \site{l2}{-120:7pt};
          \node at (154:12pt) {\scriptsize $l$};
          \node at (-94:12pt) {\scriptsize $r$};
        \end{scope}
        \begin{scope}[shift={(-56.944:1.1)}]
          \n[n3]{z}{0,0};
          % angle is 66.111 deg
          \site{r3}{123.0555:7pt};
          \site{l3}{56.9445:7pt};
          \node at (146:12pt) {\scriptsize $r$};
          \node at (34:12pt) {\scriptsize $l$};
        \end{scope}
      }};
    \path (lhs4.east) +(.3,0) edge[rule,dotted] +(1,0)
      +(1.3,0) coordinate (r4);
    \path (rhs2.south) +(0,-.2) edge[rule] +(0,-.6);
    \node[grphnode,anchor=west] (rhs4) at (r4) {
      \tikz[ingrphdiag]{
        \e{0,0}{0:1.1};
        \e{0,0}{-60:1.1};
        \e{0:1.1}{-60:1.1};
        \begin{scope}[shift={(0,0)}]
          \n[n1]{x}{0,0};
          \site{r1}{0:7pt};
          \site{l1}{-60:7pt};
          \node at (-86:12pt) {\scriptsize $l$};
          \node at (26:12pt) {\scriptsize $r$};
        \end{scope}
        \begin{scope}[shift={(0:1.1)}]
          \n[n2]{y}{0,0};
          \site{r2}{180:7pt};
          \site{l2}{-120:7pt};
          \node at (154:12pt) {\scriptsize $l$};
          \node at (-94:12pt) {\scriptsize $r$};
        \end{scope}
        \begin{scope}[shift={(-60:1.1)}]
          \n[n3]{z}{0,0};
          \site{r3}{120:7pt};
          \site{l3}{60:7pt};
          \node at (146:12pt) {\scriptsize $r$};
          \node at (34:12pt) {\scriptsize $l$};
        \end{scope}
      }};
  \end{tikzpicture}}
\end{center}

We see that, while the application on the left
does not produce any change in energy ($\Delta E = 0$),
the one on the right creates a triangle
and thus $\Delta E = \cost(t)$. %\footnote{
%   And we won't tolerate energetical ambiguity in this house!}
We must then split $r^+_{12}$ into subrules that check
the surroundings of the rule application
to make sure that, for instance,
every application of such a subrule
creates one triangle or none at all.
It is important that the partition of the rule
has certain properties.
In particular, one would like that every match of the rule
can be mapped to exactly one match of one of the subrules.
Prior work by \citet{refinement} has shown how
one can obtain a partition of rules with this property
and will be presented, in a slightly modified version,
in \sct{refinements}. % the next section.
% NOTE: not possible to put refinement section
% before minimal glueings because the proof of the
% unique decomposition theorem uses minimal glueings.

But before diving into rule partitioning,
or rule refinement as we call it,
it would be good to have a more rigourous idea of
when a rule is $\shapes$-balanced or not.
In the examples shown above we see that
our energy pattern, the triangle,
must be fully incorporated into
the left- or the right-hand side of the rule
to be sure it produces or consumes it in every application.
On the other hand, a rule that is incompatible
with our energy pattern will also be $\shapes$-balanced
by making it impossible for the rule to match a triangle.
This is true whenever there is no glueing % union
of the left-hand side of a rule with the energy pattern
where they overlap in a site that is modified by the rule.
In the next section,
we introduce the concept of overlapping glueings
of contact maps by means of multi-sums,
a concept related to local coproducts and relative pushouts.
% in $\rSGe_C$.


\section{Minimal glueings}
\label{sec:mg}

The category $\SG$ has all pullbacks,
constructed from those in $\Set$,
and they indeed restrict to $\rSGe_C$.

\begin{lemma}\label{lemma:pullbacks}
  Given a cospan $\phi_1: g_1 \to h \gets g_2 :\phi_2$ in $\rSGe_C$
  there is a unique span $\psi_1: g_1 \gets p \to g_2 :\psi_2$
  (up to unique isomorphism)
  such that any span $\omega_1: g_1 \gets q \to g_2 :\omega_2$
  that forms a commuting square $\omega_1,\omega_2,\phi_1,\phi_2\;$
  factors \emph{uniquely} through it.
  \begin{center}
    \begin{tikzpicture}
      \node (h1) at (0,0) {$g_1$};
      \node (h2) at (6,0) {$g_2$};
      \node (h) at (3,-1) {$h$};
      \node (p) at (3,1) {$p$};
      \node (q) at (3,2.2) {$q$};
      \draw (q) edge[hom,bend right=20] node[above left] {$\omega_1$} (h1);
      \draw (q) edge[hom,bend left=20] node[above right] {$\omega_2$} (h2);
      \draw[hom] (p) -- node[above] {$\psi_1$} (h1);
      \draw[hom] (p) -- node[above] {$\psi_2$} (h2);
      \draw[hom] (h1) -- node[below] {$\phi_1$} (h);
      \draw[hom] (h2) -- node[below] {$\phi_2$} (h);
      \draw[hom,dotted] (q) -- node[right] {$!$} (p);
    \end{tikzpicture}
  \end{center}
\end{lemma}
\begin{proof}
  We construct contact map $p: G \to C$ by taking the intersection
  of the agents, sites and edges in the image of $\phi_1,\phi_2$
  and restricting $\sitemap$ accordingly.
  With some abuse of notation, we have
  \begin{alignat*}{3}
    \agents_G & {}= \phi_{1,\agents}(\agents_{\anon{g_1}}) & {}\cap{} &
                  \,\phi_{2,\agents}(\agents_{\anon{g_2}}) \\
    \sites_G & {}= \,\phi_{1,\sites}(\sites_{\anon{g_1}}) & {}\cap{} &
                 \,\,\phi_{2,\sites}(\sites_{\anon{g_2}}) \\
    \edges_G & {}= \,\phi_{1,\sites}(\edges_{\anon{g_1}}) & {}\cap{} &
                 \,\,\phi_{2,\sites}(\edges_{\anon{g_2}})
  \end{alignat*}
  and $\sitemap_G = \rest{\sitemap_{\anon{h}}}{\sites_G}$.
  Functions $p_\agents,p_\sites$ are the restriction of
  $h_\agents,h_\sites$ to $\agents_G,\sites_G$, respectively.
  Embeddings $\psi_1$ and $\psi_2$ map agents and sites
  in $G$ to their pre-images along $\phi_1$ and $\phi_2$;
  by construction, all agents and sites in $G$
  are guaranteed to have such a pre-image.
  It is easy to see that
  (i) $\psi_1$ and $\psi_2$ are type-preserving
  and thus embeddings in $\rSGe_C$; and that
  (ii) the square formed by $\psi_1,\psi_2,\phi_1,\phi_2$ commutes.

  Consider any span $\omega_1: g_1 \gets q \to g_2 :\omega_2$ in $\rSGe_C$.
  If the square formed by $\omega_1$, $\omega_2,\phi_1,\phi_2$ commutes,
  then $q$ can have at most one copy of each agent and site
  in the intersection of the images of $\phi_1$ and $\phi_2$
  because $\phi_1\,\omega_1$ and $\phi_2\,\omega_2$ are injective.
  Hence, every agent and site in the image of $\omega_1,\omega_2$
  has a \emph{unique} pre-image along $\psi_1,\psi_2$, respectively,
  with the same type.
  This fixes a pair of functions $\omega_\agents,\omega_\sites$
  that map agents and sites in $q$ to those in $p$ injectively
  and form an embedding $\omega$ in $\rSGe_C$.
  Since the pre-image along $\psi_1,\psi_2$ always exists and is unique,
  any embedding $\omega': p \to q$ must be equal to $\omega$
  whenever $\phi_1\,\omega' = \omega_1$ and
  $\phi_2\,\omega' = \omega_2$.
\end{proof}

$\SG$ also has all pushouts and all sums,
but these do not in general restrict to $\rSGe_C$,
just as pushouts and sums in $\Set$ do not restrict to
the subcategory of injective functions.
% all pushouts; but these do not generally restrict to $\rSGe_C$ since
% (i) the pushout object need not be realisable,
% even if all objects in the starting span were;
% (ii) the arrows in the resulting cospan need not be embeddings,
% even if all arrows in the starting span were;
% and (iii) the mediating arrow need not even be injective
% (on agents or sites).
However, $\rSGe_C$ has \emph{multi-sums}.

\begin{lemma}\label{lemma:mg}
  For all pairs of contact maps over $C$,
  $g_1: G_1 \to C$ and $g_2: G_2 \to C$,
  % there exists a finite set $I$ and a family of cospans ... with i \in I
  there exists a finite family of cospans
  $\theta^i_1: g_1 \to s_i \gets g_2 :\theta^i_2$,
  such that any cospan $\phi_1: g_1 \to h \gets g_2 :\phi_2\;$
  factors through \emph{exactly one} of the family
  and does so \emph{uniquely}.
  \begin{center}
    \begin{tikzpicture}
      \node (h1) at (0,0) {$g_1$};
      \node (si) at (1.8,0) {$s_i$};
      \node (h2) at (3.6,0) {$g_2$};
      \node (h) at (1.8,-1.8) {$h$};
      \draw[hom] (h1) -- node[above] {$\theta^i_1$} (si);
      \draw[hom] (h2) -- node[above] {$\theta^i_2$} (si);
      \draw[hom] (h1) -- node[below left] {$\phi_1$} (h);
      \draw[hom] (h2) -- node[below right] {$\phi_2$} (h);
      \draw[hom,dotted] (si) -- node[right] {$!$} (h);
    \end{tikzpicture}
  \end{center}
\end{lemma}
\begin{proof}
  Take subsets $A_i$ of the cartesian product
  of $\agents_{\anon{g_1}}$ and $\agents_{\anon{g_2}}$
  that have each agent of $g_1$ and $g_2$ at most once
  ($(a,b) \in A_i \wedge (a,b') \in A_i \then b = b'$)
  and where each pair $(a,b) \in A_i$ has the same type,
  % that are type-compatible,
  \ie $g_{1,\agents}(a) = g_{2,\agents}(b)$.
  % for all $(a,b) \in A_i$,
  To each $A_i$ assign all subsets $S_{ij}$ of
  $\sites_{\anon{g_1}} \times \sites_{\anon{g_2}}$
  that are type-compatible
  and whose elements belong to agents paired in $A_i$,
  that is, if $(x,y) \in S_{ij}$
  then $g_{1,\sites}(x) = g_{2,\sites}(y)$
  and $(\sitemap_{\anon{g_1}}(x),\sitemap_{\anon{g_2}}(y)) \in A_i$.
  % Note that the latter predicate fixes ...
  Note how this fixes a mapping $\sitemap_{ij}$
  between elements of $S_{ij}$ to elements of $A_i$
  defined by
  $\sitemap_{ij}((x,y)) =
     (\sitemap_{\anon{g_1}}(x),\sitemap_{\anon{g_2}}(y))$.
  % Discard all sets $S_ij$ that are subsets
  % of a set $S_jk$ with $j \neq k$.
  For each $A_i$ keep only the set $S_{ij}$
  that is a superset of all other sets $S_{ik}$ ($k \neq j$).
  % and discard all others.
  There must be one such maximal set because
  if any two pairs of sites $(x_1,y_1),(x_2,y_2)$
  are type-preserving and belong to the same agents,
  then there will be one set among the $S_{ij}$s that has both
  and thus $\{S_{ij}\}_j$ is a directed partial order
  for the inclusion relation.
  % Hence, we can drop the $j$ subscript
  % in $S_{ij}$ and $\sitemap_{ij}$.
  Let $S_i$ be the maximal element of $\{S_{ij}\}_j$,
  which exists by directedness and finiteness of this family,
  and $\sitemap_i$ the corresponding mapping to $A_i$.
  Intuitively, the maximal set $S_i$ is the set of all sites
  that are defined in both agents at the same time.
  Next we discard those pairs $A_i,S_i$
  whose elements do not agree on their edge structure;
  if $(x,y) \in S_i$ then either both sites must be free
  or connected to sites $(x',y') \in S_i$.

  We construct a family of contact maps $p_i: P_i \to C$
  using $\agents_{P_i} = A_i$ as its agents,
  $\sites_{P_i} = S_i$ as its sites,
  $\sitemap_{P_i} = \sitemap_i$ and
  $\edges_{P_i} = \{((x_1,y_1), (x_2,y_2)) \in S_i \times S_i \st
     x_1 \edges_{\anon{g_1}} x_2 \wedge
     y_1 \edges_{\anon{g_2}} y_2\}$.
  Functions $p_{i,\agents},p_{i,\sites}$
  are defined straightforwardly.
  Spans $\psi^i_1: g_1 \gets p_i \to g_2 :\psi^i_2$
  are then obtained by mapping agents $(a,b)$ in $p_i$
  to $a$ in $g_1$ and $b$ in $g_2$
  and similarly for sites.
  Multi-sums $\theta^i_1: g_1 \to s_i \gets g_2 :\theta^i_2$
  are pushouts of such spans:
  they are obtained by adding to $p_i$
  all the missing agents, sites and edges from $g_1$ and $g_2$.
  Since all sites that are in $g_1$ but not in $p_i$
  cannot be in $g_2$ by maximality of $S_i$,
  there can be no conflict when adding sites or edges.
  The same argument holds for sites in $g_2$ that are not in $p_i$.

  Note that the family $A_i$ is finite
  and thus the family of multi-sums is finite as well.
  Also, it is easy to see that the spans $\psi^i_1,\psi^i_2$
  are pullbacks of $\theta^i_1,\theta^i_2$.
  Hence, (isomorphism classes of) multi-sums
  are in a one-to-one correspondence
  with (isomorphism classes of) pullbacks.
  This implies that there is only one multi-sum
  that factors any given cospan.
\end{proof}

The pairs $\theta^i_1,\theta^i_2$ enumerate
all minimal ways in which one can glue $g_1$ and $g_2$.
% and thus all the minimal contexts in which they can occur.
Hence, we refer to them as minimal glueings.
%
The notion of multi-sum dates back to \citet{diers}.
% We call them \emph{minimal glueings} in $\rSGe_C$
% according to their intuition in this concrete context
% and use them in \sct{energy-gp} to construct balanced rules.
% TODO: elaborate on relation to RPOs
They are very close to relative pushouts \citep{leifer}
and will be used in the same way,
to minimise rewriting contexts.
Indeed, each minimal glueing $i$
in the family of cospans $\theta^i_1,\theta^i_2$
accounts for one minimal rewriting context.

To illustrate how this construction operates,
consider the minimal glueings of the following
two contact maps over $T$ % (the triangle)
with their respective pullbacks.
% as shown in the following diagram.
\begin{center}
  \resizebox{.9\linewidth}{!}{%
  \begin{tikzpicture}[thick]

    % TODO: 23123 is missing

    \begin{scope}[shift={(0,0)}]
      %%% Empty intersection %%%
      \node[grphnode,anchor=south] (pb1) at (0,0) {
        \tikz[ingrphdiag]{
          \node {\large $\varnothing$};
          \node[yshift=.1em] {\large\phantom{$\varnothing$}};
        }};

      %%% 1-2-3 %%%
      \node[grphnode] (g1-1) at (-135:3) {
        \tikz[ingrphdiag]{
          \e{0,0}{2.2,0};
          \begin{scope}
            \n[n1]{n1}{0,0};
            \site{r1}{n1.east};
            \node at (26:.42) {\scriptsize $r$};
          \end{scope}
          \begin{scope}[shift={(1.1,0)}]
            \n[n2]{n2}{0,0};
            \site{l2}{n2.west};
            \site{r2}{n2.east};
            \node at (206:.42) {\scriptsize $l$};
            \node at (26:.42) {\scriptsize $r$};
          \end{scope}
          \begin{scope}[shift={(2.2,0)}]
            \n[n3]{n3}{0,0};
            \site{l3}{n3.west};
            \node at (206:.42) {\scriptsize $l$};
          \end{scope}
        }};

      %%% 2-3-1 %%%
      \node[grphnode] (g2-1) at (-45:3) {
        \tikz[ingrphdiag]{
          \e{0,0}{2.2,0};
          \begin{scope}
            \n[n2]{n2}{0,0};
            \site{r2}{n2.east};
            \node at (26:.42) {\scriptsize $r$};
          \end{scope}
          \begin{scope}[shift={(1.1,0)}]
            \n[n3]{n3}{0,0};
            \site{l3}{n3.west};
            \site{r3}{n3.east};
            \node at (206:.42) {\scriptsize $l$};
            \node at (26:.42) {\scriptsize $r$};
          \end{scope}
          \begin{scope}[shift={(2.2,0)}]
            \n[n1]{n1}{0,0};
            \site{l1}{n1.west};
            \node at (206:.42) {\scriptsize $l$};
          \end{scope}
        }};

      % cos(45) = 0.7071, * 3 = 2.1213, * 2 = 4.2426
      %%% Disjoint union: 1-2-3 2-3-1 %%%
      \node[grphnode,anchor=north] (po1) at (-90:4.2426) {
        \tikz[ingrphdiag]{
          \e{0,0}{2.2,0};
          \begin{scope}
            \n[n1]{n1}{0,0};
            \site{r1}{n1.east};
            \node at (26:.42) {\scriptsize $r$};
          \end{scope}
          \begin{scope}[shift={(1.1,0)}]
            \n[n2]{n2}{0,0};
            \site{l2}{n2.west};
            \site{r2}{n2.east};
            \node at (206:.42) {\scriptsize $l$};
            \node at (26:.42) {\scriptsize $r$};
          \end{scope}
          \begin{scope}[shift={(2.2,0)}]
            \n[n3]{n3}{0,0};
            \site{l3}{n3.west};
            \node at (206:.42) {\scriptsize $l$};
          \end{scope}

          \e{0,-.9526}{2.2,-.9526};
          \begin{scope}[shift={(0,-.9526)}]
            \n[n2]{n4}{0,0};
            \site{r4}{n4.east};
            \node at (26:.42) {\scriptsize $r$};
          \end{scope}
          \begin{scope}[shift={(1.1,-.9526)}]
            \n[n3]{n5}{0,0};
            \site{l5}{n5.west};
            \site{r5}{n5.east};
            \node at (206:.42) {\scriptsize $l$};
            \node at (26:.42) {\scriptsize $r$};
          \end{scope}
          \begin{scope}[shift={(2.2,-.9526)}]
            \n[n1]{n6}{0,0};
            \site{l6}{n6.west};
            \node at (206:.42) {\scriptsize $l$};
          \end{scope}
        }};

      \arrsn[opacity=.7]{pb1}{g1-1};
      \arrsn[opacity=.7]{pb1}{g2-1};
      \arrsn[opacity=.7]{g1-1}{po1};
      \arrsn[opacity=.7]{g2-1}{po1};
    \end{scope}

    \begin{scope}[shift={(8,0)}]
      %%% 1 %%%
      \node[grphnode,anchor=south] (pb2) at (0,0) {
        \tikz[ingrphdiag]{
          \n[n1]{n1}{0,0};
        }};

      %%% 1-2-3 %%%
      \node[grphnode] (g1-2) at (-135:3) {
        \tikz[ingrphdiag]{
          \e{0,0}{2.2,0};
          \begin{scope}
            \n[n1]{n1}{0,0};
            \site{r1}{n1.east};
            \node at (26:.42) {\scriptsize $r$};
          \end{scope}
          \begin{scope}[shift={(1.1,0)}]
            \n[n2]{n2}{0,0};
            \site{l2}{n2.west};
            \site{r2}{n2.east};
            \node at (206:.42) {\scriptsize $l$};
            \node at (26:.42) {\scriptsize $r$};
          \end{scope}
          \begin{scope}[shift={(2.2,0)}]
            \n[n3]{n3}{0,0};
            \site{l3}{n3.west};
            \node at (206:.42) {\scriptsize $l$};
          \end{scope}
        }};

      %%% 2-3-1 %%%
      \node[grphnode] (g2-2) at (-45:3) {
        \tikz[ingrphdiag]{
          \e{0,0}{2.2,0};
          \begin{scope}
            \n[n2]{n2}{0,0};
            \site{r2}{n2.east};
            \node at (26:.42) {\scriptsize $r$};
          \end{scope}
          \begin{scope}[shift={(1.1,0)}]
            \n[n3]{n3}{0,0};
            \site{l3}{n3.west};
            \site{r3}{n3.east};
            \node at (206:.42) {\scriptsize $l$};
            \node at (26:.42) {\scriptsize $r$};
          \end{scope}
          \begin{scope}[shift={(2.2,0)}]
            \n[n1]{n1}{0,0};
            \site{l1}{n1.west};
            \node at (206:.42) {\scriptsize $l$};
          \end{scope}
        }};

      %%% 2-3-1-2-3 %%%
      \node[grphnode,anchor=north] (po2) at (-90:4.2426) {
        \tikz[ingrphdiag]{
          \e{-2.2,0}{2.2,0};
          \begin{scope}[shift={(-2.2,0)}]
            \n[n2]{n2}{0,0};
            \site{r2}{n2.east};
            \node at (26:.42) {\scriptsize $r$};
          \end{scope}
          \begin{scope}[shift={(-1.1,0)}]
            \n[n3]{n3}{0,0};
            \site{l3}{n3.west};
            \site{r3}{n3.east};
            \node at (206:.42) {\scriptsize $l$};
            \node at (26:.42) {\scriptsize $r$};
          \end{scope}
          \begin{scope}
            \n[n1]{n1}{0,0};
            \site{l1}{n1.west};
            \site{r1}{n1.east};
            \node at (206:.42) {\scriptsize $l$};
            \node at (26:.42) {\scriptsize $r$};
          \end{scope}
          \begin{scope}[shift={(1.1,0)}]
            \n[n2]{n2}{0,0};
            \site{l2}{n2.west};
            \site{r2}{n2.east};
            \node at (206:.42) {\scriptsize $l$};
            \node at (26:.42) {\scriptsize $r$};
          \end{scope}
          \begin{scope}[shift={(2.2,0)}]
            \n[n3]{n3}{0,0};
            \site{l3}{n3.west};
            \node at (206:.42) {\scriptsize $l$};
          \end{scope}
        }};

      \arrsn[opacity=.7]{pb2}{g1-2};
      \arrsn[opacity=.7]{pb2}{g2-2};
      \arrsn[opacity=.7]{g1-2}{po2};
      \arrsn[opacity=.7]{g2-2}{po2};
    \end{scope}

    \begin{scope}[shift={(0,-9)}]
      %%% 2-3 %%%
      \node[grphnode,anchor=south] (pb3) at (0,0) {
        \tikz[ingrphdiag]{
          \e{0,0}{1.1,0};
          \begin{scope}
            \n[n2]{n2}{0,0};
            \site{r2}{n2.east};
            \node at (26:.42) {\scriptsize $r$};
          \end{scope}
          \begin{scope}[shift={(1.1,0)}]
            \n[n3]{n3}{0,0};
            \site{l3}{n3.west};
            \node at (206:.42) {\scriptsize $l$};
          \end{scope}
        }};

      %%% 1-2-3 %%%
      \node[grphnode] (g1-3) at (-135:3) {
        \tikz[ingrphdiag]{
          \e{0,0}{2.2,0};
          \begin{scope}
            \n[n1]{n1}{0,0};
            \site{r1}{n1.east};
            \node at (26:.42) {\scriptsize $r$};
          \end{scope}
          \begin{scope}[shift={(1.1,0)}]
            \n[n2]{n2}{0,0};
            \site{l2}{n2.west};
            \site{r2}{n2.east};
            \node at (206:.42) {\scriptsize $l$};
            \node at (26:.42) {\scriptsize $r$};
          \end{scope}
          \begin{scope}[shift={(2.2,0)}]
            \n[n3]{n3}{0,0};
            \site{l3}{n3.west};
            \node at (206:.42) {\scriptsize $l$};
          \end{scope}
        }};

      %%% 2-3-1 %%%
      \node[grphnode] (g2-3) at (-45:3) {
        \tikz[ingrphdiag]{
          \e{0,0}{2.2,0};
          \begin{scope}
            \n[n2]{n2}{0,0};
            \site{r2}{n2.east};
            \node at (26:.42) {\scriptsize $r$};
          \end{scope}
          \begin{scope}[shift={(1.1,0)}]
            \n[n3]{n3}{0,0};
            \site{l3}{n3.west};
            \site{r3}{n3.east};
            \node at (206:.42) {\scriptsize $l$};
            \node at (26:.42) {\scriptsize $r$};
          \end{scope}
          \begin{scope}[shift={(2.2,0)}]
            \n[n1]{n1}{0,0};
            \site{l1}{n1.west};
            \node at (206:.42) {\scriptsize $l$};
          \end{scope}
        }};

      %%% 1-2-3-1 %%%
      \node[grphnode,anchor=north] (po3) at (-90:4.2426) {
        \tikz[ingrphdiag]{
          \e{0,0}{3.3,0};
          \begin{scope}
            \n[n1]{n1}{0,0};
            \site{r1}{n1.east};
            \node at (26:.42) {\scriptsize $r$};
          \end{scope}
          \begin{scope}[shift={(1.1,0)}]
            \n[n2]{n2}{0,0};
            \site{l2}{n2.west};
            \site{r2}{n2.east};
            \node at (206:.42) {\scriptsize $l$};
            \node at (26:.42) {\scriptsize $r$};
          \end{scope}
          \begin{scope}[shift={(2.2,0)}]
            \n[n3]{n3}{0,0};
            \site{l3}{n3.west};
            \site{r3}{n3.east};
            \node at (206:.42) {\scriptsize $l$};
            \node at (26:.42) {\scriptsize $r$};
          \end{scope}
          \begin{scope}[shift={(3.3,0)}]
            \n[n1]{n4}{0,0};
            \site{l4}{n4.west};
            \node at (206:.42) {\scriptsize $l$};
          \end{scope}
        }};

      \arrsn[opacity=.7]{pb3}{g1-3};
      \arrsn[opacity=.7]{pb3}{g2-3};
      \arrsn[opacity=.7]{g1-3}{po3};
      \arrsn[opacity=.7]{g2-3}{po3};
    \end{scope}

    \begin{scope}[shift={(8,-9)}]
      %%% 1 2-3 %%%
      \node[grphnode,anchor=south] (pb4) at (0,0) {
        \tikz[ingrphdiag]{
          \e{1,0}{2.2,0};
          \n[n1]{n1}{0,0};
          \begin{scope}[shift={(1,0)}]
            \n[n2]{n2}{0,0};
            \site{r2}{n2.east};
            \node at (26:.42) {\scriptsize $r$};
          \end{scope}
          \begin{scope}[shift={(2.1,0)}]
            \n[n3]{n3}{0,0};
            \site{l3}{n3.west};
            \node at (206:.42) {\scriptsize $l$};
          \end{scope}
        }};

      %%% 1-2-3 %%%
      \node[grphnode] (g1-4) at (-135:3) {
        \tikz[ingrphdiag]{
          \e{0,0}{2.2,0};
          \begin{scope}
            \n[n1]{n1}{0,0};
            \site{r1}{n1.east};
            \node at (26:.42) {\scriptsize $r$};
          \end{scope}
          \begin{scope}[shift={(1.1,0)}]
            \n[n2]{n2}{0,0};
            \site{l2}{n2.west};
            \site{r2}{n2.east};
            \node at (206:.42) {\scriptsize $l$};
            \node at (26:.42) {\scriptsize $r$};
          \end{scope}
          \begin{scope}[shift={(2.2,0)}]
            \n[n3]{n3}{0,0};
            \site{l3}{n3.west};
            \node at (206:.42) {\scriptsize $l$};
          \end{scope}
        }};

      %%% 2-3-1 %%%
      \node[grphnode] (g2-4) at (-45:3) {
        \tikz[ingrphdiag]{
          \e{0,0}{2.2,0};
          \begin{scope}
            \n[n2]{n2}{0,0};
            \site{r2}{n2.east};
            \node at (26:.42) {\scriptsize $r$};
          \end{scope}
          \begin{scope}[shift={(1.1,0)}]
            \n[n3]{n3}{0,0};
            \site{l3}{n3.west};
            \site{r3}{n3.east};
            \node at (206:.42) {\scriptsize $l$};
            \node at (26:.42) {\scriptsize $r$};
          \end{scope}
          \begin{scope}[shift={(2.2,0)}]
            \n[n1]{n1}{0,0};
            \site{l1}{n1.west};
            \node at (206:.42) {\scriptsize $l$};
          \end{scope}
        }};

      %%% Triangle %%%
      \node[grphnode,anchor=north] (po4) at (-90:4.2426) {
        \tikz[ingrphdiag]{
          \e{0,0}{0:1.1};
          \e{0,0}{-60:1.1};
          \e{0:1.1}{-60:1.1};
          \begin{scope}[shift={(0,0)}]
            \n[n1]{x}{0,0};
            \site{r1}{0:7pt};
            \site{l1}{-60:7pt};
            \node at (-86:12pt) {\scriptsize $l$};
            \node at (26:12pt) {\scriptsize $r$};
          \end{scope}
          \begin{scope}[shift={(0:1.1)}]
            \n[n2]{y}{0,0};
            \site{r2}{180:7pt};
            \site{l2}{-120:7pt};
            \node at (154:12pt) {\scriptsize $l$};
            \node at (-94:12pt) {\scriptsize $r$};
          \end{scope}
          \begin{scope}[shift={(-60:1.1)}]
            \n[n3]{z}{0,0};
            \site{r3}{120:7pt};
            \site{l3}{60:7pt};
            \node at (146:12pt) {\scriptsize $r$};
            \node at (34:12pt) {\scriptsize $l$};
          \end{scope}
        }};

      \arrsn[opacity=.7]{pb4}{g1-4};
      \arrsn[opacity=.7]{pb4}{g2-4};
      \arrsn[opacity=.7]{g1-4}{po4};
      \arrsn[opacity=.7]{g2-4}{po4};
    \end{scope}

  \end{tikzpicture}}%
\end{center}

%%% Local Variables:
%%% mode: latex
%%% TeX-master: "thesis"
%%% End:


I have implemented an online tool that computes minimal glueings
available at \url{https://rhz.github.com/thesis/mg.html}.
Its source code can be found at \url{https://github.com/rhz/thesis/}.

Using minimal glueings we can test whether
a rule $r$ is $\shapes$-balanced,
that is, whether $r$ consumes and produces
the same number of instances of each energy pattern $p$
when applied to any mixture $m$.
In particular, for an $r$-event $\psi$
to \emph{consume} an instance $\phi$ of $p$ in a mixture $m$,
$\phi_\sites$ and $\psi_\sites$ must have images
which intersect on at least one site which is modified by $r$
(\eg by adding an edge if it was free). % or removing its edge).
% Otherwise the energy pattern is left intact by the action of the rule.
This is the case iff
the minimal glueing $\phi',\psi'$ of $r_L$ and $p$
\begin{wrapfigure}[5]{r}{0.41\textwidth}
  \vspace{-1.8em}
  \begin{equation}
    \label{eq:p-balanced}
    \tikz[baseline=-1.1cm]{
  % \begin{center}
  %   \begin{tikzpicture}
      \node (p) at (0,0) {$p$};
      \node (s) at (1.8,0) {$s$};
      \node (l) at (3.6,0) {$r_L$};
      \node (m) at (1.8,-1.8) {$m$};
      \draw[hom] (p) -- node[above] {$\phi'$} (s);
      \draw[hom] (l) -- node[above] {$\psi'$} (s);
      \draw[hom] (p) -- node[below left] {$\phi$} (m);
      \draw[hom] (l) -- node[below right] {$\psi$} (m);
      \draw[hom,dotted] (s) -- (m);}
  %   \end{tikzpicture}
  % \end{center}
    \end{equation}
\end{wrapfigure}
that factors the cospan $\phi,\psi$ has the same property.
Likewise, for an $r$-event to \emph{produce} an instance of $p$,
the associated minimal glueing between $p$ and $r_R$
must have a modified intersection.
We call such minimal glueings \emph{relevant}.
% ; they are the ones which underlie events
% that can affect the instances of $p$.

To illustrate the idea of relevant minimal glueings,
let us consider a different example.
In this example, the contact graph is very simple:
just one agent type with two sites, $l$ and $r$,
that can bind each other.
% The maximally permissive set of generators rules
% contains only one reversible rule.
% One extension of this rule is
Imagine we have the following reversible rule.
\begin{center}
  \begin{tikzpicture}[thick]
    \node[grphnode,anchor=east] (lhs1) at (0,0) {
      \tikz[ingrphdiag]{
        \e{0,0}{2.2,0};
        \begin{scope}
          \n[n]{x}{0,0};
          \site{rx}{x.east};
          \node at (26:.42) {\scriptsize $r$};
        \end{scope}
        \begin{scope}[shift={(1.1,0)}]
          \n[n]{y}{0,0};
          \site{ly}{y.west};
          \site{ry}{y.east};
          \node at (206:.42) {\scriptsize $l$};
          \node at (26:.42) {\scriptsize $r$};
        \end{scope}
        \begin{scope}[shift={(2.2,0)}]
          \n[n]{z}{0,0};
          \site{lz}{z.west};
          \node at (206:.42) {\scriptsize $l$};
        \end{scope}
      }};
    \path (lhs1.east) +(.3,0) edge[rule] +(1,0)
      +(1.3,0) coordinate (r1);
    \node[grphnode,anchor=west] (rhs1) at (r1) {
      \tikz[ingrphdiag]{
        \e{1.1,0}{2.3,0};
        \begin{scope}[shift={(0,0)}]
          \n[n]{x}{0,0};
          \e{x}{.5,0};
          \site{rx}{x.east};
          \node at (26:.42) {\scriptsize $r$};
        \end{scope}
        \begin{scope}[shift={(1.2,0)}]
          \n[n]{y}{0,0};
          \e{y}{-.5,0};
          \site{ly}{y.west};
          \site{ry}{y.east};
          \node at (206:.42) {\scriptsize $l$};
          \node at (26:.42) {\scriptsize $r$};
        \end{scope}
        \begin{scope}[shift={(2.3,0)}]
          \n[n]{z}{0,0};
          \site{lz}{z.west};
          \node at (206:.42) {\scriptsize $l$};
        \end{scope}
      }};
  \end{tikzpicture}
\end{center}
Take the chain of 3 agents as our energy pattern.
The minimal glueings of the left-hand side of the rule
with the energy pattern are shown below.
On the left of each diagram is the energy pattern.
The relevant minimal glueings are marked
with a light green background.
% \tikzstyle{site}=[font=\scriptsize\itshape,inner sep=1pt,above]
\tikzstyle{s}=[font=\scriptsize,yshift=6pt]
\tikzstyle{emb}=[->,dashed,thin]
\tikzstyle{relevant}=[show background rectangle,
background rectangle/.style={fill=Green!40,rounded corners=4pt}]
\tikzstyle{non-relevant}=[show background rectangle,
background rectangle/.style={fill=White,rounded corners=4pt}]

% Draw a chain of agents
% Parameters:
%  - Position of the first agent
%  - Distance between agents
%  - List of node ids
%  - Name of left site
%  - Name of right site
\newcommand{\createchain}[5]{%
  \draw #1
    \foreach \aid [count=\ai] in {#3} {
      \ifnum \ai = 1
      \else -- node[s,pos=.33] {$#5$} node[s,pos=.65] {$#4$} ++(#2,0)
      \fi
      node[inner sep=5pt,n] (\aid) {} };
  \putsites{#3}
}

\ExplSyntaxOn
\int_new:N \sites_len 
\newcommand{\putsites}[1]{%
  \clist_set:Nn \sites_clist {#1}
  \int_set:Nn \sites_len {\clist_count:N \sites_clist}
  \int_step_inline:nnnn {1}{1}{\sites_len}
  { \edef\aid{\clist_item:Nn \sites_clist {##1}}
    \begin{scope}[shift={(\aid)}]
      \int_compare:nTF { ##1 > 1 }
      { \site{l\aid}{\aid.west}; }{}
      \int_compare:nTF { ##1 < \sites_len }
      { \site{r\aid}{\aid.east}; }{}
    \end{scope}}
}
\ExplSyntaxOff

% \newcommand{\putsites}[1]{
%   \foreach \aid in {#1} {
%     \begin{scope}[shift={(\aid)}]
%       \site{l\aid}{\aid.west};
%       \site{r\aid}{\aid.east};
%     \end{scope}};
% }

% Draw the arrows of embeddings
% Parameters:
%  - List of pairs from/to of node ids
\newcommand{\dembs}[2][.1]{% diagram embeddings
  \foreach \ai / \aj in {#2}
    \draw[emb] ($(\ai)!#1!(\aj.north)$) -- ($(\ai)!.9!(\aj.north)$);
}

% \newcommand{\vsep}{.3cm}
% first row
\begin{center}
\begin{minipage}{.48\textwidth}
  \begin{center} % no overlap
    \begin{tikzpicture}[thick, non-relevant]
      \createchain{(0pt,0pt)}{30pt}{a1,a2,a3}{l}{r};
      \createchain{(90pt,0pt)}{30pt}{a4,a5,a6}{l}{r};
      \createchain{(0pt,-40pt)}{30pt}{a7,a8,a9}{l}{r};
      \createchain{(90pt,-40pt)}{30pt}{a10,a11,a12}{l}{r};
      \dembs[.35]{a1/a7,a2/a8,a3/a9,a4/a10,a5/a11,a6/a12};
    \end{tikzpicture}
  \end{center}
\end{minipage}
\begin{minipage}{.48\textwidth}
  \begin{center} % all overlap
    \begin{tikzpicture}[thick, relevant]
      \createchain{(0pt,0pt)}{30pt}{a1,a2,a3}{l}{r};
      \createchain{(90pt,0pt)}{30pt}{a4,a5,a6}{l}{r};
      \createchain{(45pt,-40pt)}{30pt}{a7,a8,a9}{l}{r};
      \dembs[.25]{a1/a7,a4/a7,a2/a8,a5/a8,a3/a9,a6/a9};
    \end{tikzpicture}
  \end{center}
\end{minipage}
\end{center}
% second row
% \vspace{\vsep}
\begin{center}
\begin{minipage}{.48\textwidth}
  \begin{center} % two overlap
    \begin{tikzpicture}[thick, relevant]
      \createchain{(0pt,0pt)}{30pt}{a1,a2,a3}{l}{r};
      \createchain{(90pt,0pt)}{30pt}{a4,a5,a6}{l}{r};
      \createchain{(30pt,-40pt)}{30pt}{a7,a8,a9,a10}{l}{r};
      \dembs[.28]{a1/a7,a4/a8,a2/a8,a5/a9,a3/a9,a6/a10};
    \end{tikzpicture}
  \end{center}
\end{minipage}
\begin{minipage}{.48\textwidth}
  \begin{center} % two overlap
    \begin{tikzpicture}[thick, non-relevant]
      \createchain{(0pt,0pt)}{30pt}{a1,a2,a3}{l}{r};
      \createchain{(90pt,0pt)}{30pt}{a4,a5,a6}{l}{r};
      \createchain{(30pt,-40pt)}{30pt}{a7,a8,a9,a10}{l}{r};
      \dembs[.2]{a4/a7,a1/a8,a5/a8,a2/a9,a6/a9,a3/a10};
    \end{tikzpicture}
  \end{center}
\end{minipage}
\end{center}
% third row
% \vspace{\vsep}
\begin{center}
\begin{minipage}{.48\textwidth}
  \begin{center} % one overlap
    \begin{tikzpicture}[thick, non-relevant]
      \createchain{(0pt,0pt)}{30pt}{a1,a2,a3}{l}{r};
      \createchain{(90pt,0pt)}{30pt}{a4,a5,a6}{l}{r};
      \createchain{(15pt,-40pt)}{30pt}{a7,a8,a9,a10,a11}{l}{r};
      \dembs[.31]{a1/a7,a2/a8,a3/a9,a4/a9,a5/a10,a6/a11};
    \end{tikzpicture}
  \end{center}
\end{minipage}
\begin{minipage}{.48\textwidth}
  \begin{center} % one overlap
    \begin{tikzpicture}[thick, non-relevant]
      \createchain{(0pt,0pt)}{30pt}{a1,a2,a3}{l}{r};
      \createchain{(90pt,0pt)}{30pt}{a4,a5,a6}{l}{r};
      \createchain{(15pt,-40pt)}{30pt}{a7,a8,a9,a10,a11}{l}{r};
      \dembs[.18]{a4/a7,a5/a8,a6/a9,a1/a9,a2/a10,a3/a11};
    \end{tikzpicture}
  \end{center}
\end{minipage}
\end{center}
% fourth row
% \vspace{\vsep}
\begin{center}
\begin{minipage}{.48\textwidth}
  \begin{center} % triangle
    \begin{tikzpicture}[thick, relevant]
      \createchain{(0pt,0pt)}{30pt}{a1,a2,a3}{l}{r};
      \createchain{(90pt,0pt)}{30pt}{a4,a5,a6}{l}{r};

      \begin{scope}[shift={(60pt,-40pt)}]
        \e{0,0}{0:1.1};
        \e{0,0}{-60:1.1};
        \e{0:1.1}{-60:1.1};
        \begin{scope}[shift={(0,0)}]
          \n[n]{a7}{0,0};
          \site{r7}{0:7pt};
          \site{l7}{-60:7pt};
          \node at (-86:12pt) {\scriptsize $l$};
          \node at (26:12pt) {\scriptsize $r$};
        \end{scope}
        \begin{scope}[shift={(0:1.1)}]
          \n[n]{a8}{0,0};
          \site{r8}{180:7pt};
          \site{l8}{-120:7pt};
          \node at (154:12pt) {\scriptsize $l$};
          \node at (-94:12pt) {\scriptsize $r$};
        \end{scope}
        \begin{scope}[shift={(-60:1.1)}]
          \n[n]{a9}{0,0};
          \site{r9}{120:7pt};
          \site{l9}{60:7pt};
          \node at (146:12pt) {\scriptsize $r$};
          \node at (34:12pt) {\scriptsize $l$};
        \end{scope}
      \end{scope}

      \dembs[.3]{a2/a7,a3/a8,a4/a7,a5/a8};
      \draw[emb] ($(a1)+(-65:12pt)$) to [out=-65,in=180] ($(a9)+(180:12pt)$);
      \draw[emb] ($(a6)+(245:12pt)$) to [out=245,in=0]   ($(a9)+(0:12pt)$);
    \end{tikzpicture}
  \end{center}
\end{minipage}
\begin{minipage}{.48\textwidth}
  \begin{center} % triangle
    \begin{tikzpicture}[thick, non-relevant]
      \createchain{(0pt,0pt)}{30pt}{a1,a2,a3}{l}{r};
      \createchain{(90pt,0pt)}{30pt}{a4,a5,a6}{l}{r};

      \begin{scope}[shift={(60pt,-40pt)}]
        \e{0,0}{0:1.1};
        \e{0,0}{-60:1.1};
        \e{0:1.1}{-60:1.1};
        \begin{scope}[shift={(0,0)}]
          \n[n]{a7}{0,0};
          \site{r7}{0:7pt};
          \site{l7}{-60:7pt};
          \node at (-86:12pt) {\scriptsize $l$};
          \node at (26:12pt) {\scriptsize $r$};
        \end{scope}
        \begin{scope}[shift={(0:1.1)}]
          \n[n]{a8}{0,0};
          \site{r8}{180:7pt};
          \site{l8}{-120:7pt};
          \node at (154:12pt) {\scriptsize $l$};
          \node at (-94:12pt) {\scriptsize $r$};
        \end{scope}
        \begin{scope}[shift={(-60:1.1)}]
          \n[n]{a9}{0,0};
          \site{r9}{120:7pt};
          \site{l9}{60:7pt};
          \node at (146:12pt) {\scriptsize $r$};
          \node at (34:12pt) {\scriptsize $l$};
        \end{scope}
      \end{scope}

      \dembs[.2]{a5/a7,a6/a8,a1/a7,a2/a8};
      \draw[emb] ($(a3)+(240:12pt)$) to [out=240,in=180] ($(a9)+(180:12pt)$);
      \draw[emb] ($(a4)+(-60:12pt)$) to [out=-60,in=0]   ($(a9)+(0:12pt)$);
    \end{tikzpicture}
  \end{center}
\end{minipage}
\end{center}
% \end{minipage}
% \begin{minipage}{.48\textwidth}
%   \begin{center} % all overlap
%     \begin{tikzpicture}[thick, relevant]
%       \createchain{(0pt,0pt)}{30pt}{a1,a2,a3}{l}{r};
%       \createchain{(90pt,0pt)}{30pt}{a4,a5,a6}{l}{r};
%       \createchain{(45pt,-40pt)}{30pt}{a7,a8,a9}{l}{r};
%       \dembs[.25]{a1/a7,a4/a7,a2/a8,a5/a8,a3/a9,a6/a9};
%     \end{tikzpicture}
%   \end{center}
%   \vspace{\vsep}
%   \begin{center} % two overlap
%     \begin{tikzpicture}[thick, non-relevant]
%       \createchain{(0pt,0pt)}{30pt}{a1,a2,a3}{l}{r};
%       \createchain{(90pt,0pt)}{30pt}{a4,a5,a6}{l}{r};
%       \createchain{(30pt,-40pt)}{30pt}{a7,a8,a9,a10}{l}{r};
%       \dembs[.2]{a4/a7,a1/a8,a5/a8,a2/a9,a6/a9,a3/a10};
%     \end{tikzpicture}
%   \end{center}
%   \vspace{\vsep}
%   \begin{center} % one overlap
%     \begin{tikzpicture}[thick, non-relevant]
%       \createchain{(0pt,0pt)}{30pt}{a1,a2,a3}{l}{r};
%       \createchain{(90pt,0pt)}{30pt}{a4,a5,a6}{l}{r};
%       \createchain{(15pt,-40pt)}{30pt}{a7,a8,a9,a10,a11}{l}{r};
%       \dembs[.18]{a4/a7,a5/a8,a6/a9,a1/a9,a2/a10,a3/a11};
%     \end{tikzpicture}
%   \end{center}
%   \vspace{\vsep}
%   \begin{center} % triangle
%     \begin{tikzpicture}[thick, non-relevant]
%       \createchain{(0pt,0pt)}{30pt}{a1,a2,a3}{l}{r};
%       \createchain{(90pt,0pt)}{30pt}{a4,a5,a6}{l}{r};
%
%       \begin{scope}[shift={(60pt,-40pt)}]
%         \e{0,0}{0:1.1};
%         \e{0,0}{-60:1.1};
%         \e{0:1.1}{-60:1.1};
%         \begin{scope}[shift={(0,0)}]
%           \n[n]{a7}{0,0};
%           \site{r7}{0:7pt};
%           \site{l7}{-60:7pt};
%           \node at (-86:12pt) {\scriptsize $l$};
%           \node at (26:12pt) {\scriptsize $r$};
%         \end{scope}
%         \begin{scope}[shift={(0:1.1)}]
%           \n[n]{a8}{0,0};
%           \site{r8}{180:7pt};
%           \site{l8}{-120:7pt};
%           \node at (154:12pt) {\scriptsize $l$};
%           \node at (-94:12pt) {\scriptsize $r$};
%         \end{scope}
%         \begin{scope}[shift={(-60:1.1)}]
%           \n[n]{a9}{0,0};
%           \site{r9}{120:7pt};
%           \site{l9}{60:7pt};
%           \node at (146:12pt) {\scriptsize $r$};
%           \node at (34:12pt) {\scriptsize $l$};
%         \end{scope}
%       \end{scope}
%
%       \dembs[.2]{a5/a7,a6/a8,a1/a7,a2/a8};
%       \draw[emb] ($(a3)+(240:12pt)$) to [out=240,in=180] ($(a9)+(180:12pt)$);
%       \draw[emb] ($(a4)+(-60:12pt)$) to [out=-60,in=0]   ($(a9)+(0:12pt)$);
%     \end{tikzpicture}
%   \end{center}
% \end{minipage}
% \end{center}
%
% fifth row
% \vspace{\vsep}
\begin{center} % square
  \begin{tikzpicture}[thick, non-relevant]
    \createchain{(0pt,0pt)}{30pt}{a1,a2,a3}{l}{r};
    \createchain{(90pt,0pt)}{30pt}{a4,a5,a6}{l}{r};

    \begin{scope}[shift={(75pt,-40pt)}]
      \draw (0,0) -- ++(-30:1.1) coordinate (c8)
                  -- ++(-150:1.1) coordinate (c9)
                  -- ++(150:1.1) coordinate (c10)
                  -- cycle;
      \begin{scope}[shift={(0,0)}]
        \n[n]{a7}{0,0};
        \site{r7}{-30:7pt};
        \site{l7}{-150:7pt};
        \node at (-176:12pt) {\scriptsize $l$};
        \node at (-4:12pt) {\scriptsize $r$};
      \end{scope}
      \begin{scope}[shift={(c8)}]
        \n[n]{a8}{0,0};
        \site{r8}{150:7pt};
        \site{l8}{-150:7pt};
        \node at (124:12pt) {\scriptsize $l$};
        \node at (-124:12pt) {\scriptsize $r$};
      \end{scope}
      \begin{scope}[shift={(c9)}]
        \n[n]{a9}{0,0};
        \site{r9}{150:7pt};
        \site{l9}{30:7pt};
        \node at (4:12pt) {\scriptsize $l$};
        \node at (176:12pt) {\scriptsize $r$};
      \end{scope}
      \begin{scope}[shift={(c10)}]
        \n[n]{a10}{0,0};
        \site{r10}{30:7pt};
        \site{l10}{-30:7pt};
        \node at (-56:12pt) {\scriptsize $l$};
        \node at (56:12pt) {\scriptsize $r$};
      \end{scope}
    \end{scope}

    \draw[emb] ($(a1)+(-55:12pt)$) to ($(a10)+(125:12pt)$);
    \draw[emb] ($(a2)+(-40:12pt)$) to ($(a7)+(140:12pt)$);
    \draw[emb] ($(a3)+(-30:12pt)$) to [out=-30,in=100] ($(a8)+(100:12pt)$);
    \draw[emb] ($(a4)+(-60:12pt)$) to [out=-60,in=80 ] ($(a8)+(80:12pt)$);
    \draw[emb] ($(a5)+(-70:12pt)$) to [out=-70,in=-30] ($(a9)+(-30:12pt)$);
    \draw[emb] ($(a6)+(210:12pt)$) to [out=210,in=90 ] ($(a10)+(90:12pt)$);
  \end{tikzpicture}
\end{center}


Whenever $\psi': r_L \to s$ in \diagram{p-balanced} is an iso,
then the energy pattern $p$ is fully included % contained
in the left-hand side of rule $r$.
This implies the rule contains all the relevant context needed
to make sure that an instance of $p$ is consumed
by any $r$-event $\psi: r_L \to m$.
We say that $r$ is $\shapes$-\emph{left-balanced} iff,
for all $p \in \shapes$ and relevant minimal glueings
$\theta^i_1: p \to s_i \gets r_L :\theta^i_2$,
the right leg $\theta^i_2$ is an isomorphism.
Symmetrically, one says that $r$ is $\shapes$-\emph{right-balanced}
iff $\inv{r}$ is $\shapes$-left-balanced.
Then $r$ is $\shapes$-\emph{balanced}
iff it is $\shapes$-left- and $\shapes$-right-balanced.

\begin{lemma}
  Rule $r$ is $\shapes$-balanced if and only if
  $r$ is $\shapes$-left- and $\shapes$-right-balanced.
  Moreover, for any mixture $m$ and embedding $\psi: r_L \to m$,
  \[ \Delta_r p = |[p;m^{(r,\psi)}]| % \abs{\matches{p}{\comatch{m}}}
                - \abs{\matches{p}{m}}
                = \abs{\matches{p}{r_R}}
                - \abs{\matches{p}{r_L}} \]
\end{lemma}
\begin{proof}
  Suppose there are two mixtures $m$, $n$
  and embeddings $\psi: r_L \to m$, $\phi: r_L \to n$
  such that, when $r$ is applied to $\psi$ and $\phi$,
  it has a different balance
  with respect to a pattern $p \in \shapes$,
  \ie $|[p;m^{(r,\psi)}]| - \abs{\matches{p}{m}} \neq
  |[p;n^{(r,\phi)}]| - \abs{\matches{p}{n}}$.
  %
  We have
  \begin{equation*}
    \abs{\matches{p}{m}} = |\{p \to m \getsby{\psi} r_L\}|
    = \abs{\set{\tikz[baseline=-.6cm,x=1.2cm,y=1.2cm]{
      \node (p) at (0,0) {$p$};
      \node (s) at (1,0) {$s$};
      \node (l) at (2,0) {$r_L$};
      \node (m) at (1,-1) {$m$};
      \draw[hom] (p) -- (s);
      \draw[hom] (p) -- (m);
      \draw[hom] (l) -- (s);
      \draw[hom] (l) -- node[below right] {$\psi$} (m);
      \draw[hom,dotted] (s) -- (m);}}}
  \end{equation*}
  where $p \to s \gets r_L$ is the minimal glueing
  that factors the cospan $p \to m \getsby{\psi} r_L$.
  A similar equality can be obtained for $r_R$,
  $m^{(r,\psi)}$ and $\comatch{\psi}$.
  %
  The \emph{irrelevant} minimal glueings on each side of the rule
  are in bijection: the rule does not destroy nor create them.
  Hence, when taking the difference
  $|[p;m^{(r,\psi)}]| - \abs{\matches{p}{m}}$
  they cancel each other out and we are left with
  a difference of \emph{relevant} minimal glueings on each side.
  %
  Since $s \iso r_L$ for each relevant minimal glueing on the left
  then
  \begin{equation*}
    \abs{\set{\tikz[baseline=-.6cm,x=1.2cm,y=1.2cm]{
      \node (p) at (0,0) {$p$};
      \node (s) at (1,0) {$s$};
      \node (l) at (2,0) {$r_L$};
      \node (m) at (1,-1) {$m$};
      \draw[hom] (p) -- (s);
      \draw[hom] (p) -- (m);
      \path (l) -- node[onarrow] {$\iso$} (s);
      \draw[hom] (l) -- node[below right] {$\psi$} (m);
      \draw[hom,dotted] (s) -- (m);}}}
    = \abs{\matches{p}{r_L}}
  \end{equation*}
  Again, a similar equality can be obtained for $r_R$,
  $m^{(r,\psi)}$ and $\comatch{\psi}$.
  Thus we have proved that
  $|[p;m^{(r,\psi)}]| - \abs{\matches{p}{m}} =
  \abs{\matches{p}{r_R}} - \abs{\matches{p}{r_L}}$
  for any $m$ and $\psi$,
  contradicting our original assumption.
\end{proof}


\section{Refinements}
\label{sec:refinements}

A rule is refined into another rule by adding context.
For example, we can add a common neighbour
to the agents in $r^+_{12}$ to obtain a refinement.
% \begin{center}
%   \begin{tikzpicture}
\begin{equation}
  \label{eq:refined1}
  \tikz[baseline=-.16cm]{
    \node[grphnode,anchor=east] (lhs) at (0,0) {
      \tikz[ingrphdiag]{
        \e{0,0}{-56.944:1.1};
        \e{0:1.2}{-56.944:1.1};
        \begin{scope}[shift={(0,0)}]
          \n[n1]{x}{0,0};
          \e{x}{.5,0};
          \site{r1}{0:7pt};
          \site{l1}{-60:7pt};
          \node at (-86:12pt) {\scriptsize $l$};
          \node at (26:12pt) {\scriptsize $r$};
        \end{scope}
        \begin{scope}[shift={(0:1.2)}]
          \n[n2]{y}{0,0};
          \e{y}{-.5,0};
          \site{r2}{180:7pt};
          \site{l2}{-120:7pt};
          \node at (154:12pt) {\scriptsize $l$};
          \node at (-94:12pt) {\scriptsize $r$};
        \end{scope}
        \begin{scope}[shift={(-56.944:1.1)}]
          \n[n3]{z}{0,0};
          % angle is 66.111 deg
          \site{r3}{123.0555:7pt};
          \site{l3}{56.9445:7pt};
          \node at (146:12pt) {\scriptsize $r$};
          \node at (34:12pt) {\scriptsize $l$};
        \end{scope}
      }};
    \path (lhs.east) +(.3,0) edge[rule] +(1,0)
      +(1.3,0) coordinate (r);
    \node[grphnode,anchor=west] (rhs) at (r) {
      \tikz[ingrphdiag]{
        \e{0,0}{0:1.1};
        \e{0,0}{-60:1.1};
        \e{0:1.1}{-60:1.1};
        \begin{scope}[shift={(0,0)}]
          \n[n1]{x}{0,0};
          \site{r1}{0:7pt};
          \site{l1}{-60:7pt};
          \node at (-86:12pt) {\scriptsize $l$};
          \node at (26:12pt) {\scriptsize $r$};
        \end{scope}
        \begin{scope}[shift={(0:1.1)}]
          \n[n2]{y}{0,0};
          \site{r2}{180:7pt};
          \site{l2}{-120:7pt};
          \node at (154:12pt) {\scriptsize $l$};
          \node at (-94:12pt) {\scriptsize $r$};
        \end{scope}
        \begin{scope}[shift={(-60:1.1)}]
          \n[n3]{z}{0,0};
          \site{r3}{120:7pt};
          \site{l3}{60:7pt};
          \node at (146:12pt) {\scriptsize $r$};
          \node at (34:12pt) {\scriptsize $l$};
        \end{scope}
      }};
  }
\end{equation}
%   \end{tikzpicture}
% \end{center}
This refinement happens to be $\shapes$-balanced.
Another refinement of $r^+_{12}$ could be
% \begin{center}
%   \begin{tikzpicture}
\begin{equation}
  \label{eq:refined2}
  \tikz[baseline=-.16cm]{
    \node[grphnode,anchor=east] (lhs) at (0,0) {
      \tikz[ingrphdiag]{
        \begin{scope}[shift={(0,0)}]
          \n[n1]{x}{0,0};
          \e{x}{.5,0};
          \site{rx}{x.east};
          \node at (26:.42) {\scriptsize $r$};
        \end{scope}
        \begin{scope}[shift={(1.2,0)}]
          \n[n2]{y}{0,0};
          \e{y}{-.5,0};
          \e{y}{.5,0};
          \site{ly}{y.west};
          \site{ry}{y.east};
          \node at (206:.42) {\scriptsize $l$};
          \node at (26:.42) {\scriptsize $r$};
        \end{scope}
      }};
    \path (lhs.east) +(.3,0) edge[rule] +(1,0)
      +(1.3,0) coordinate (r);
    \node[grphnode,anchor=west] (rhs) at (r) {
      \tikz[ingrphdiag]{
        \e{0,0}{1.1,0};
        \begin{scope}
          \n[n1]{x}{0,0};
          \site{rx}{x.east};
          \node at (26:.42) {\scriptsize $r$};
        \end{scope}
        \begin{scope}[shift={(1.1,0)}]
          \n[n2]{y}{0,0};
          \e{y}{-.5,0};
          \e{y}{.5,0};
          \site{ly}{y.west};
          \site{ry}{y.east};
          \node at (206:.42) {\scriptsize $l$};
          \node at (26:.42) {\scriptsize $r$};
        \end{scope}
      }};
  }
\end{equation}
%   \end{tikzpicture}
% \end{center}
Here we have added a free site to the blue node.
This second refinement is also $\shapes$-balanced
because the free $r$ site on the blue node guarantees that
(i) the rule will never create a triangle and
(ii) there is no embedding from the left-hand side
into a triangle and hence no triangle can be destroyed
by the action of the rule.
The following refinement, however, is not $\shapes$-balanced.
\begin{center}
  \begin{tikzpicture}
    \node[grphnode,anchor=east] (lhs) at (0,0) {
      \tikz[ingrphdiag]{
        \begin{scope}[shift={(0,0)}]
          \n[n1]{x}{0,0};
          \e{x}{.5,0};
          \site{rx}{x.east};
          \node at (26:.42) {\scriptsize $r$};
        \end{scope}
        \e{1.2,0}{2.3,0};
        \begin{scope}[shift={(1.2,0)}]
          \n[n2]{y}{0,0};
          \e{y}{-.5,0};
          \site{ly}{y.west};
          \site{ry}{y.east};
          \node at (206:.42) {\scriptsize $l$};
          \node at (26:.42) {\scriptsize $r$};
        \end{scope}
        \begin{scope}[shift={(2.3,0)}]
          \n[n3]{z}{0,0};
          \site{lz}{z.west};
          \node at (206:.42) {\scriptsize $l$};
        \end{scope}
      }};
    \path (lhs.east) +(.3,0) edge[rule] +(1,0)
      +(1.3,0) coordinate (r);
    \node[grphnode,anchor=west] (rhs) at (r) {
      \tikz[ingrphdiag]{
        \e{0,0}{2.2,0};
        \begin{scope}[shift={(0,0)}]
          \n[n1]{x}{0,0};
          \site{rx}{x.east};
          \node at (26:.42) {\scriptsize $r$};
        \end{scope}
        \begin{scope}[shift={(1.1,0)}]
          \n[n2]{y}{0,0};
          \site{ly}{y.west};
          \site{ry}{y.east};
          \node at (206:.42) {\scriptsize $l$};
          \node at (26:.42) {\scriptsize $r$};
        \end{scope}
        \begin{scope}[shift={(2.2,0)}]
          \n[n3]{z}{0,0};
          \site{lz}{z.west};
          \node at (206:.42) {\scriptsize $l$};
        \end{scope}
      }};
  \end{tikzpicture}
\end{center}

We add context to a rule $r = \tuple{r_L,r_R}$
by applying the rule to an embedding $\psi: r_L \to g$.
This operation is well-defined
even if the codomain of the embedding is not a mixture.
% The result of the rewrite $g^{(r,\psi)}$
The pair of contact maps $(g,g^{(r,\psi)})$
% with $g^{(r,\psi)}$ the result of the rewrite
is itself a valid rule
since they only differ in their edge structure.
In this way, an extension of a rule
is determined uniquely by an embedding.

Epis\footnote{
  Epi, mono and iso are short for
  epimorphism, monomorphism and isomorphism.}
of $\rSGe_C$ are good candidates for extensions. % relevant
They are characterised as follows:
an embedding $\psi: g \to h$ is an epi iff
every connected component of $\anon{h}$ contains
at least one agent in the image of $\psi_\agents$.
This ensures that no new connected component is added to the rule
while extending it.
However, for technical reasons
that will become apparent in \thm{unique-decomposition},
we use prefixes of epis
instead of epis to extend rules ---
an embedding $\psi: g \to h$ is said to be
a \emph{prefix} of $\phi: g \to h'$
if there is some embedding $\theta: h \to h'$
that makes the composition of $\psi$ and $\theta$ equal to $\phi$
(\ie $\theta \, \psi = \phi$) % psychology + tetas = philosophy
and write $\psi \leq \phi$ for this.
We refer to a prefix
\begin{wrapfigure}[5]{r}{0.27\textwidth}
  \vspace{-2em}
  \begin{center}
    \begin{tikzpicture}
      \matrix (m) [matrix of math nodes,row sep=25pt,column sep=25pt] {
        & g & \\
        h & & h' \\};
      \draw[hom] (m-1-2) -- node[above left] {$\psi$} (m-2-1);
      \draw[hom] (m-1-2) -- node[above right] {$\phi$} (m-2-3);
      \draw[hom] (m-2-1) -- node[below] {$\theta$} (m-2-3);
    \end{tikzpicture}
  \end{center}
\end{wrapfigure}
of an epi $\psi: g \to h$ as an \emph{extension} of $g$.
In the category of extensions of $g$,
a morphism between objects $\psi: g \to h$ and $\phi: g \to h'$
is an embedding $\theta: h \to h'$
such that the triangle on the right commutes.
If $\theta$ is an iso we write $\psi \cong_g \phi$.

One might wonder when the prefix of an epi is not itself an epi.
The following diagram illustrates such a situation,
% where $\theta$ is the witness of $\psi \leq \phi$.
where $\psi$ is a prefix of epi $\phi$
but is not itself an epi since the connected component
of the blue node in the codomain of $\psi$
is not in the image of $\psi_\agents$.
\begin{center}
  \begin{tikzpicture}
    \node[grphnode,outer sep=.3cm] (g) at (0,0) {
      \tikz[ingrphdiag]{
        \n[n1]{x}{0,0};
      }};
    \node[grphnode,outer sep=.3cm] (h) at (-135:3) {
      \tikz[ingrphdiag]{
        \n[n1]{x}{0,0};
        \n[n2]{y}{.9,0};
      }};
    \node[grphnode,outer sep=.3cm] (h') at (-45:3) {
      \tikz[ingrphdiag,outer sep=0]{
        \e{0,0}{1.1,0};
        \begin{scope}[shift={(0,0)}]
          \n[n1]{x}{0,0};
          \site{rx}{x.east};
          \node at (26:.42) {\scriptsize $r$};
        \end{scope}
        \begin{scope}[shift={(1.1,0)}]
          \n[n2]{y}{0,0};
          \site{ly}{y.west};
          \node at (206:.42) {\scriptsize $l$};
        \end{scope}
      }};
    \path (g) edge[rule] node[above left] {$\psi$} (h);
    \path (g) edge[rule] node[above right] {$\phi$} (h');
    \path (h) edge[rule] node[below] {$\theta$} (h');
  \end{tikzpicture}
\end{center}

% workaround to push the next lonely sentence to the next page
% \bigskip

Rule application preserves epis
and in fact also prefixes of epis:
\begin{lemma}
  \label{lem:epi-prefix}
  Let $r = \tuple{r_L,r_R}$ be a rule
  and $\psi: r_L \to g$ be an embedding
  with $r_L,r_R,g$ contact maps in $\rSGe_C$.
  The embedding $\comatch{\psi}: r_R \to \comatch{g}$
  that results from applying $r$ to $\psi$
  is a prefix of an epi iff $\psi$ is.
\end{lemma}
\begin{proof}
  % Here we just prove that rule application preserves epis.
  % For prefixes of epis we have to make sure that
  % the mediating arrow (ie the witness of \phi \geq \psi)
  % is preserved as well.
  % This works because the new connected components in g
  % (added in the codomain of the prefix of epi)
  % are then connected to those in r_L through sites
  % that are not involved in the action of the rule,
  % since an edge addition requires the sites to be free
  % and an edge deletion requires them to be bound
  % but in no case they can be used to bind
  % the new connected components.
  % Embeddings preserve edges and free sites
  % so the sites involved in the action of the rule
  % have to be mentioned in the codomain of the prefix of epi.
  % Because rule applications will leave everything else intact
  % the mediating arrow is preserved.
  This amounts to proving that
  some embedding $\comatch{\phi} \geq \comatch{\psi}$
  is an epi if there is an epi $\phi \geq \psi$;
  the converse is true by symmetry of rules.
  For this it is enough to consider the case
  where the rule adds or deletes exactly one edge
  since rules that modify more than one edge at a time
  can be decomposed as sequences of deletions and insertions of edges;
  given that each deletion and insertion preserves the property,
  the sequence will preserve it as well.

  The case of adding an edge is easy as the image of $\comatch{\phi}$
  has fewer connected components to intersect than $\phi$.
  The case where $r$ deletes an edge
  can introduce new connected components,
  however in this case both ends $u,v$
  of the deleted edge must be in $r_L$,
  so whether the deletion disconnects or not the codomain of $\psi$,
  the components of $\comatch{\phi}(u)$ and $\comatch{\phi}(v)$
  will have a pre-image, namely $u$ and $v$.
\end{proof}

It follows that the category of extensions
of $r_L$ and $r_R$ are isomorphic.
Hence, any extension $\phi$ to a rule $r$ can be mapped to
an extension of its inverse rule $\inv{r}$.

A family of epis $\phi_i: g \to g_i$ \emph{uniquely decomposes} $g$,
or is a \emph{refinement} of $g$, if,
for all mixtures $m$ and embeddings $\psi: g \to m$,
there exists a unique $i$ and $\psi'$ such that $\psi = \psi' \phi_i$.
%; uniqueness of $i$ prevents the $\phi_i$s from overlapping.
%; since $\phi_i$ is an epi, there can be at most one such $\psi$.
This is the basic requirement
for a reasonable notion of rule refinement:
it guarantees that the left-hand side $g$ of a given rule
splits into a non-overlapping and exhaustive collection
of more specific cases $g_i$.

% For the partitioning of rules
% we need a guiding principle.

A method to easily construct such decompositions
was proposed by \citet{refinement}
which works by detailing
which agents and sites should be added to $g$.
This ``extension plan'' is called growth policy.
A \emph{growth policy} $\gp$ for contact map $g$ over $C$
is a family of functions $\gp_\phi$,
indexed by all extensions $\phi: g \to h$,
where $\gp_\phi$ maps $u \in \agents_{\anon{h}}$ to
a subset $\gp_\phi(u)$ of $\sitemap_C^{-1}(h_\agents(u))$,
\ie each agent in $\anon{h}$ is allocated
a subset of the sites belonging to the agent type $h_\agents(u)$
it is mapped to in the contact graph.
%
An agent in $\anon{h}$ may cover some, or all,
of these sites or even completely extraneous sites:
if the former, \ie if for all $u$ in $\agents_{\anon{h}}$,
$h_\sites(\sitemap_{\anon{h}}^{-1}(u)) \subset \gp_\phi(u)$,
we say that $\phi$ is \emph{immature};
if for all $u$s the inclusion is an equality
and $\phi$ is an epi,
% $h_\sites(\sitemap_{\anon{h}}^{-1}(u)) = \gp_\phi(u)$,
we say that $\phi$ is \emph{mature};
otherwise $\phi$ is said to be \emph{overgrown}.
The functions $\gp_\phi$ must satisfy,
for all extensions $\phi$ and $\phi' \geq \phi$,
the \emph{faithfulness} property,
$\gp_\phi = \gp_{\phi'} \, \psi_\agents$
with $\psi$ such that $\psi \, \phi = \phi'$;
so a site requested by $\phi$
must be requested by any further extension.
Additionally, this property forces $\gp$ to eagerly ask
for all sites that will be eventually requested
at any given agent in the codomain of $\phi$.
If $\phi$ is not overgrown
then no $\phi' \leq \phi$ is overgrown either.
% Also, note that the union of two growth policies
% is itself a growth policy.

Given a contact map $g$ over $C$ and a growth policy $\gp$ for $g$,
we define $\gp(g)$ by choosing one representative
per $\cong_g$-isomorphism class of the set of all extensions of $g$
which are mature according to $\gp$.

The following theorem guarantees that
factorisations through $\gp(g)$ are unique when they exist,
but \emph{not} that they necessarily do exist.
In section \sct{energy-gp},
we will construct a specific growth policy % of interest
for which this property of exhaustivity of the decomposition
can be proved by hand.
As such, it fulfils our desired criteria of providing
an exhaustive collection of mutually exclusive sub-cases.

\begin{theorem}
  \label{thm:unique-decomposition}
  % If $\gp$ is a growth policy for $g$,
  % $\gp(g)$ uniquely decomposes $g$.
  Let $g$ and $m$ be contact maps over $C$
  and $\gp$ a growth policy for $g$.
  If an embedding $\psi: g \to m$ can be decomposed
  in two ways as $\gamma_1 \phi_1$ and $\gamma_2 \phi_2$
  with $\phi_i: g \to h_i$ in $\gp(g)$ and $\gamma_i: h_i \to m$,
  then $\phi_1 = \phi_2$ and $\gamma_1 = \gamma_2$.
  \begin{equation}
    \label{eq:gp}
    \tikz[baseline=-4.3]{
      \matrix (m) [matrix of math nodes,row sep=25pt,column sep=25pt]{
        g & & & h_1 \\
        & p & & \\
        & & m & \\
        h_2 & & & m \\};
      % outer square
      \draw[hom] (m-1-1) -- node[above] {$\phi_1$} (m-1-4);
      \draw[hom] (m-1-1) -- node[left] {$\phi_2$} (m-4-1);
      \draw[hom] (m-4-1) -- node[below] {$\gamma_2$} (m-4-4);
      \draw[hom] (m-1-4) -- node[right] {$\gamma_1$} (m-4-4);
      % inner square
      \draw[hom] (m-2-2) -- node[onarrow] {$\pi_1$} (m-4-1);
      \draw[hom] (m-2-2) -- node[onarrow] {$\pi_2$} (m-1-4);
      \draw[hom] (m-4-1) -- node[onarrow] {$\theta_1$} (m-3-3);
      \draw[hom] (m-1-4) -- node[onarrow] {$\theta_2$} (m-3-3);
      % mediating arrows
      \draw[hom] (m-1-1) -- node[onarrow] {$\phi$} (m-2-2);
      \draw[hom,dashed] (m-3-3) -- (m-4-4);
    }
  \end{equation}
\end{theorem}
\begin{proof}
  Suppose that $\gamma_1 \phi_1 = \gamma_2 \phi_2$,
  where $\phi_1$ and $\phi_2$ are mature extensions of $g$
  according to $\gp$ and $\phi_1 \neq \phi_2$.
  As shown in \diagram{gp},
  we have an inner square formed by the pullback $\pi_1,\pi_2$,
  and the minimal glueing $\theta_1,\theta_2$ of $h_1,h_2$
  that factors $\gamma_1,\gamma_2$.
  Every connected component of $m$
  has a pre-image in $h_1$ or $h_2$,
  and thus also in $g$,
  since $\phi_1$ and $\phi_2$ are epis
  as mature extensions.
  Because every connected component of $m$
  has an image in $h_1$ and $h_2$,
  then evey connected component of $m$
  has a pre-image in both $h_1$ and $h_2$.
  Hence $\theta_1$ and $\theta_2$ are epis.
  % Also $\theta_1$ and $\theta_2$ are epis,
  % as every connected component of $m$
  % has a pre-image in $h_1$ or $h_2$
  % and so also in $g$, since the $\phi_i$s are epis,
  % and so also in the other of $h_2$ and $h_1$.

  The nodes in the images of $\theta_1$ and $\theta_2$
  might be the same or differ.
  When they differ, some site $z$ sitting on a node
  in the intersection of the images of $\theta_1,\theta_2$
  is connected to a node outside the image,
  since $\theta_1,\theta_2$ are epis.
  However, $z$ cannot be in the intersection of the images
  unless the site it is connected to is also part of the intersection
  (\lem{mg}).
  Therefore the nodes in the images must be the same.
  In this case there has to be a site $z$
  that is not in the image of one of them
  or $\theta_1,\theta_2$ are both isos.
  So there must be a pair $u,z$,
  consisting of a node $u$ in $m$
  with pre-images $u_1,u_2$ in $h_1,h_2$
  and a site $z$ of $u$,
  such that $z$ has no pre-image
  in exactly one of $\theta_1,\theta_2$.
  Say it is $\theta_2$.
  Since $\phi_1$ is not overgrown,
  $z \in \gp_{\phi_1}(u_1)$ and, by faithfulness,
  $z \in \gp_\phi(\tuple{u_1,u_2})$,
  where $\tuple{u_1,u_2}$ is
  the pullback pre-image of $u_1$ and $u_2$.
  So again, by faithfulness, $z \in \gp_{\phi_2}(u_2)$
  which contradicts our original assumption.
  Hence, $\theta_1$ and $\theta_2$ are isos.
  It follows that $\phi_1 = \phi_2$ as there is only
  one representative per $\cong_g$-isomorphism class in $\gp(g)$.
  Finally, $\gamma_1 = \gamma_2$ because $\phi_1$ is an epi.
\end{proof}
% NB: the argument uses the faithful condition
% in both directions to push around the $z$ site.

Given a rule $r$ and an extension $\phi: r_L \to g$, % of $r_L$,
we write $r_\phi$ for the refined rule associated to $\phi$,
% $r_\phi$ denotes the refined rule associated to $\phi$,
that is, $r_\phi$ is the pair $(g,g^{(r,\phi)})$.
%
Given $\gp$ a growth policy for $r_L$,
we write $\gp(r)$ for the family of rules
obtained by refining $r$ according to $\gp$,
that is, $\gp(r)$ is the family of rules $r_\phi$
for $\phi$ ranging in $\gp(r_L)$.

An example of growth policy is the \emph{ground} policy
which assigns all possible sites to all agents.
In this case, $\gp(g)$ is simply the set, possibly infinite,
of all epis of $g$ into mixtures, considered up to $\cong_g$.
The ground refinement $\gp(r)$ % of $r$
contains all refinements of $r$ along those epis.
The refined rules therefore manipulate mixtures directly.
It is easy to see that the ground refinement of $r^+_{12}$
in our example is infinite,
since $r^+_{12}$ % each of the three rules
can trigger the extension of a chain of any length.
A similar argument is true for $r^-_{12}$.
Note that ground refinements of a rule $r$
are trivially $\shapes$-balanced but, in general,
the set of refined rules is impractically large or infinite as above.
Instead, the growth policy that we introduce
in the next section % \sct{energy-gp}
will always be finite.


\section{Thermodynamic growth policy} % Energy-based refinement}
\label{sec:energy-gp}

An extension $\phi$ of a rule $r$ is $\shapes$-balanced
if it generates a refined rule $r_\phi$ that is $\shapes$-balanced.
To find such extensions % $\shapes$-balanced extensions of a rule $r$,
it seems natural to use minimal glueings:
take as extensions the right leg $\theta^i_2$
of each relevant minimal glueing
$\theta^i_1: p \to s_i \gets r_L :\theta^i_2$
of $p \in \shapes$ and $r_L$ (or $r_R$).
For instance, the only relevant minimal glueing of
the right-hand side of $r^+_{12}$ and the triangle is
\begin{center}
  \begin{tikzpicture}[thick]
    \begin{scope}
      %%% Rhs: 1-2 %%%
      \node[grphnode,anchor=south] (rr) at (150:2.5) {
        \tikz[ingrphdiag]{
          \e{0,0}{1.1,0};
          \begin{scope}
            \n[n1]{n1}{0,0};
            \site{r1}{n1.east};
            \node at (26:.42) {\scriptsize $r$};
          \end{scope}
          \begin{scope}[shift={(1.1,0)}]
            \n[n2]{n2}{0,0};
            \site{l2}{n2.west};
            \node at (206:.42) {\scriptsize $l$};
          \end{scope}
        }};

      %%% Triangle %%%
      \node[grphnode,anchor=south] (p) at (30:2.5) {
        \tikz[ingrphdiag]{
          \e{0,0}{0:1.1};
          \e{0,0}{-60:1.1};
          \e{0:1.1}{-60:1.1};
          \begin{scope}[shift={(0,0)}]
            \n[n1]{x}{0,0};
            \site{r1}{0:7pt};
            \site{l1}{-60:7pt};
            \node at (-86:12pt) {\scriptsize $l$};
            \node at (26:12pt) {\scriptsize $r$};
          \end{scope}
          \begin{scope}[shift={(0:1.1)}]
            \n[n2]{y}{0,0};
            \site{r2}{180:7pt};
            \site{l2}{-120:7pt};
            \node at (154:12pt) {\scriptsize $l$};
            \node at (-94:12pt) {\scriptsize $r$};
          \end{scope}
          \begin{scope}[shift={(-60:1.1)}]
            \n[n3]{z}{0,0};
            \site{r3}{120:7pt};
            \site{l3}{60:7pt};
            \node at (146:12pt) {\scriptsize $r$};
            \node at (34:12pt) {\scriptsize $l$};
          \end{scope}
        }};

      %%% Triangle %%%
      \node[grphnode,anchor=north] (mg) at (0,0) {
        \tikz[ingrphdiag]{
          \e{0,0}{0:1.1};
          \e{0,0}{-60:1.1};
          \e{0:1.1}{-60:1.1};
          \begin{scope}[shift={(0,0)}]
            \n[n1]{x}{0,0};
            \site{r1}{0:7pt};
            \site{l1}{-60:7pt};
            \node at (-86:12pt) {\scriptsize $l$};
            \node at (26:12pt) {\scriptsize $r$};
          \end{scope}
          \begin{scope}[shift={(0:1.1)}]
            \n[n2]{y}{0,0};
            \site{r2}{180:7pt};
            \site{l2}{-120:7pt};
            \node at (154:12pt) {\scriptsize $l$};
            \node at (-94:12pt) {\scriptsize $r$};
          \end{scope}
          \begin{scope}[shift={(-60:1.1)}]
            \n[n3]{z}{0,0};
            \site{r3}{120:7pt};
            \site{l3}{60:7pt};
            \node at (146:12pt) {\scriptsize $r$};
            \node at (34:12pt) {\scriptsize $l$};
          \end{scope}
        }};

      \draw[-bigto,opacity=.7]
      ($(rr.south)!.1!(mg.north)$)
      -- node[pos=.4,below left,opacity=1] {$\comatch{\phi}$}
      ($(rr.south)!.9!(mg.north)$);
      % \arrsn[opacity=.7]{rr}{mg};
      \arrsn[opacity=.7]{p}{mg};
    \end{scope}
  \end{tikzpicture}
\end{center}
If we use $\phi$ ---
the embedding coresponding to $\comatch{\phi}$
on the left-hand side ---
as an extension of $r^+_{12}$
we obtain rule~\ref{eq:refined1}.
Now, having found the only extension of $r^+_{12}$
that produces a triangle,
we are left with the problem of finding
the extensions that cover the cases when $r^+_{12}$
can be applied without producing a triangle.
Intuitively, one must handle the cases
when the $l$ site of the orange node
or the $r$ site of the blue node are free
(as in rule~\ref{eq:refined2}).

% We say that a balanced extension $\phi$ is \emph{prime}
% iff it is minimally so,
% \ie any prefix of $\phi$ that is $\shapes$-balanced
% is isomorphic to $\phi$ as an extension of $r_L$.
% Prime extensions are epis since erasing an `untouched'
% connected component in the codomain preserves balance.

% If $\phi$ is a $\shapes$-balanced extension of $r$,
% the refined rule $r_\phi$
% has a \emph{balance vector} in $\ZZ^\shapes$,
% written $\Delta\phi$, where, for each $p \in \shapes$,
% $\Delta\phi(p)$ is the number of copies of $p$
% produced by \emph{any} $r_\phi$-event,
% which is also the difference between the number of embeddings
% of $p$ in the right-hand and the left-hand side of $r_\phi$.


\section{Linear kinetic model}
\label{sec:kinetic-model}

% is the linear kinetic model related to "Parameters for
% the description of transition states", John Leffler, Science, 1953
% https://sci-hub.ac/10.2307/1680906
% it says "we approximate the transition state as a hybrid between
% the reagent and product states"
% "whenever the plot of the logarithm of the rate constant
%  against the equilibrium constant is a straight line,
%  the approximation is justified"
% "it should then be possible to predict
%  the free energy of the transition state by a linear combination of
%  the predictions made for the reagents and for the products"
%
% if we then relate the free energy of the transition state to
% the rate constants using Arrhenius?
% this has been done in transition state theory
% https://en.wikipedia.org/wiki/Eyring_equation
%
% do we get additional constraints on kinetic rates from cycles
% in the transition graph? when two cycles share an edge?


\section{Example: Flagellum's motor}
\label{sec:alloring}




%%% Local Variables:
%%% mode: latex
%%% TeX-master: "thesis"
%%% End:


\chapter[The inverse problem: From rules to energy]{
  The inverse problem \\
  \LARGE From rules to energy}
\label{chp:inverse}
% In this chapter
we would like to explore restricted versions of Kappa
for which it is possible to infer the energy function
from the rewriting rules and their associated rates.
Recall from the introduction
that in Kappa itself this problem is undecidable \citep{et1}.
One such restriction is when agents do not have sites
and thus cannot bind.
This is Petri nets.
We briefly present here the result obtained by \citet{et2}
and show the construction of the energy function
for \emph{simple} and \emph{symmetric} Petri nets
with \emph{mass action} semantics
(sisma Petri net for short).
\begin{definition}
A \emph{sisma Petri net} is a Petri net
on species $\species$ and reactions $\reactions$ for which:
\begin{enumerate}[label={(\roman*)}]
\item (simple) there are no two reactions that have
  the same stoichiometry (net change in species).
\item (symmetric) for each reaction $r \in \reactions$,
  there is a reaction $\inv{r} \in \reactions$
  that has the reverse direction,
  \ie the inputs of one are the outputs of the other.
\item (mass action) the jumping rate $q_{xy,r}$
  of going from a state $x$ to $y$
  by a reaction $r$ is proportional to
  the number of ocurrences of its left-hand side in $x$.
  In particular,
  given a rate constant $k(r)$ for reaction $r$,
  % the jumping rate of $r$ at state $x$ is
  we have
  \[ q_{xy,r} = k(r) \prod_{A \in \species}
     \frac{x(A)!}{(x(A)-\Delta_r(A))!} \]
  where $x(A)$ is the number of $A$s in $x$
  and $\Delta_r(A)$ is the net change of $A$ in reaction $r$.
\end{enumerate}
\end{definition}
Note that simple implies
no two distinct reactions can be applied
to a state $x$ to obtain state $y$.
A sisma Petri net has an energy function
\begin{equation}\label{eq:pn-energy}
  E(x) = \sum_{A \in \species} \cost(A) x(A) + \ln\bigl(x(A)!\bigr)
\end{equation}
for some function $\cost: \species \to \RR$ such that,
for all $r \in \reactions$,
\[ \sum_{A \in \species} \Delta_r(A) \cost(A) = \ln(k(r^\star)) - \ln(k(r)). \]
If there is no such function $\cost$,
the Petri net does not have detailed balance and
an energy function.
%
In the rest of the chapter
we introduce two other restrictions of Kappa
and show how to construct their energy function.


\section{Cooperative assembly systems}
\label{sec:cas}

The first restriction is when
rules can only create or destroy one edge at a time
and their rates can only depend on
how many bound sites the endpoints of the edge have.
% the type of the neighbours of the two endpoints of the edge
% but not the state of their sites
% (whether they are free or bound).
% Moreover, sites in an agent are regarded as \emph{indistinguishable}
% and thus rules can only count how many are bound.
Therefore sites are treated as \emph{indistinguishable}.
In addition, agents of the same type cannot bind.
\citet{cas} have proposed these restrictions
and a simple formalism incorporating them
to study the thermodynamics of polymer formation
when there are two types of monomers.
Here we extend their result to any number of monomer types.
% TODO: perhaps take the idea in the next sentence
% and integrate into the paragraph:
% Cooperativity means that the rate of binding (and unbinding)
% depends on the number of connections that have been established
% by the participating nodes.

In the case of two monomers,
rules are of the form
\begin{center}
  \begin{tikzpicture}
    \node[grphnode,anchor=east] (lhs) at (0,0) {
      \tikz[ingrphdiag]{
        \begin{scope}[shift={(0,0)}]
          \e{0,0}{90:1.1};
          \e{0,0}{150:1.1};
          \e{0,0}{270:1.1};
          % a
          \n[n1]{a}{0,0};
          \e{a}{.5,0};
          \site{x1}{a.north};
          \site{x2}{150:.25};
          \site{xn}{a.south};
          \site{xn+1}{a.east};
          % b1
          \n[n2]{b1}{90:1.1};
          \site{z}{b1.south};
          % b2
          \n[n2]{b2}{150:1.1};
          \site[shift={(150:1.1)}]{z}{-30:.25};
          % ellipsis
          \node[Black!60!White] at (200:.8) {\Large .};
          \node[Black!60!White] at (210:.8) {\Large .};
          \node[Black!60!White] at (220:.8) {\Large .};
          % bn
          \n[n2]{bn}{270:1.1};
          \site{z}{bn.north};
        \end{scope}
        \begin{scope}[shift={(1.2,0)}]
          \e{0,0}{90:1.1};
          \e{0,0}{30:1.1};
          \e{0,0}{270:1.1};
          % b
          \n[n2]{b}{0,0};
          \e{b}{-.5,0};
          \site{y1}{b.north};
          \site{y2}{30:7pt};
          \site{yn}{b.south};
          \site{yn+1}{b.west};
          % a1
          \n[n1]{a1}{90:1.1};
          \site{z}{a1.south};
          % a2
          \n[n1]{a2}{30:1.1};
          \site[shift={(30:1.1)}]{z}{210:.25};
          % ellipsis
          \node[Black!60!White] at (-20:.8) {\Large .};
          \node[Black!60!White] at (-30:.8) {\Large .};
          \node[Black!60!White] at (-40:.8) {\Large .};
          % an
          \n[n1]{an}{270:1.1};
          \site{z}{an.north};
        \end{scope}
      }};
    \path (lhs.east) +(.3,.09) edge[rule] +(1,.09)
      +(1.3,0) coordinate (r);
    \path (lhs.east) +(1,-.09) edge[rule] +(.3,-.09);
    \node[grphnode,anchor=west] (rhs) at (r) {
      \tikz[ingrphdiag]{
        \e{0,0}{1.1,0};
        \begin{scope}[shift={(0,0)}]
          \e{0,0}{90:1.1};
          \e{0,0}{150:1.1};
          \e{0,0}{270:1.1};
          % a
          \n[n1]{a}{0,0};
          \site{x1}{a.north};
          \site{x2}{150:.25};
          \site{xn}{a.south};
          \site{xn+1}{a.east};
          % b1
          \n[n2]{b1}{90:1.1};
          \site{z}{b1.south};
          % b2
          \n[n2]{b2}{150:1.1};
          \site[shift={(150:1.1)}]{z}{-30:.25};
          % ellipsis
          \node[Black!60!White] at (200:.8) {\Large .};
          \node[Black!60!White] at (210:.8) {\Large .};
          \node[Black!60!White] at (220:.8) {\Large .};
          % bn
          \n[n2]{bn}{270:1.1};
          \site{z}{bn.north};
        \end{scope}
        \begin{scope}[shift={(1.1,0)}]
          \e{0,0}{90:1.1};
          \e{0,0}{30:1.1};
          \e{0,0}{270:1.1};
          % b
          \n[n2]{b}{0,0};
          \site{y1}{b.north};
          \site{y2}{30:7pt};
          \site{yn}{b.south};
          \site{yn+1}{b.west};
          % a1
          \n[n1]{a1}{90:1.1};
          \site{z}{a1.south};
          % a2
          \n[n1]{a2}{30:1.1};
          \site[shift={(30:1.1)}]{z}{210:.25};
          % ellipsis
          \node[Black!60!White] at (-20:.8) {\Large .};
          \node[Black!60!White] at (-30:.8) {\Large .};
          \node[Black!60!White] at (-40:.8) {\Large .};
          % an
          \n[n1]{an}{270:1.1};
          \site{z}{an.north};
        \end{scope}
      }};
  \end{tikzpicture}
\end{center}
where the three grey dots on the sides of each graph
are an ellipsis to mean that monomers can be bound
to an arbitrary number of monomers of the other type
as long as each site is bound only once
and there is a finite number of sites per monomer
fixed by the monomer type.
% TODO: is this sentence clear?
Hence, the rule schema % formula?
represents a family of rules indexed by % $i,j$
the number of bound sites in the two monomers.
% TODO: add the following?
% Note that no nodes can be created or destroyed
% by this family of rules
% and so the total number of nodes is fixed.

We formalise the ideas in the first paragraph as follows.
Let $\types$ be the set of monomer types % $A,B,\dots$
and $\valence: \types \to \NN$ the map that assigns
% for each type the number of sites they have,
to each type the number of sites a monomer of that type has,
which we refer to as their valence.
A~monomer $u$ has type $\typeof(u) \in \types$
and degree $\degree_x(u) \in \NN$ in state $x$.
We simply write $\valence(u)$ for $\valence(\typeof(u))$.
The rate constant of a rule that binds
a monomer of type $\tp \in \types$ and degree $i$ with
a monomer of type $\tp' \in \types$ and degree $j$
is $\gamma^+_{\tp,i,\tp',j}$.
The rate constant of the reverse rule (unbinding)
is $\gamma^-_{\tp,i,\tp',j}$.
%
The jumping rate $q_{xy}$ from state $x$ to $y$
% is then determined by mass action semantics,
% \ie is linear in the number of ocurrences
% of the left-hand side of the rule that operates the transition.
is then linearly determined by the number of ocurrences
of the left-hand side of the rule and the rate constant
(mass action semantics).
We assume that any two agents can be bound only once.

The binding or unbinding of any two nodes $u,v$ in $x$
can only be carried out by one rule,
namely the one that operates on degrees $\degree_x(u),\degree_x(v)$.
The binding rule has rate constant
$\alpha(u,v) := \gamma^+_{\typeof(u),\degree_x(u),\typeof(v),\degree_x(v)}$
while the unbinding rate constant is
$\beta(u,v) := \gamma^-_{\typeof(u),\degree_x(u),\typeof(v),\degree_x(v)}$.
When binding we are free to choose
one site among the $\valence(u)-\degree_x(u)$ free sites of $u$
and one among the $\valence(v)-\degree_x(v)$ free sites of $v$
in order to apply the binding rule.
On the other hand, when unbinding we have only one choice,
namely removing the only edge between $u$ and $v$.
Hence, $q_{xy}$ is equal to $\alpha(u,v) \,
(\valence(u) - \degree_x(u)) \, (\valence(v) - \degree_x(v))$
% in the forward direction (binding)
when the binding rule is applied to $x$ to obtain $y$
and to $\beta(u,v)$
% in the backward direction (unbinding).
when unbinding.

The theorem below shows under which conditions
the type of systems presented in this section
have an energy function.

\begin{proposition}
  \label{prop:cas}
  Let $\types$ be a finite set of monomer types
  and $\gamma^-_{\tp,i,\tp',j},\gamma^+_{\tp,i,\tp',j}$
  families of real values indexed by types $\tp,\tp' \in \types$,
  $0 \leqslant i < \valence(\tp)$ and
  $0 \leqslant j < \valence(\tp')$ as above.
  Given a family $\Gamma_{\tp,i}$ of non-zero real values
  the following two statements are equivalent
  \begin{enumerate}[label={(\roman*)}]
  \item The \qmatrix $\qm$ as defined above by $q_{xy}$
    has detailed balance with respect to the \pmf $\pi$
    determined by the energy function
    \[ E(x) = \sum_{\tp \in \types} \sum_{0 < i \leqslant \valence(\tp)}
              \cost_\tp(i) \, x(\tp_i) \]
    where $x(\tp_i)$ is the number of nodes of type $\tp$
    with degree $i$ in $x$ and
    % TODO: which other variable name is good for the iterator a?
    \[ \cost_\tp(i) = \sum_{0 \leqslant j < i}
       \ln\, \frac{\Gamma_{\tp,j}}{\valence(\tp) - j} \]
  % \item The ratio $\gamma^-_{\tp,i,\tp',j} / \gamma^+_{\tp,i,\tp',j}$
  %   is equal to $\Gamma_{\tp,i} \, \Gamma_{\tp',j}$
  \item For all $\tp,\tp' \in \types$,
    $0 \leqslant i < \valence(\tp)$ and
    $0 \leqslant j < \valence(\tp')$ we have
    \begin{equation}
      \label{eq:cas-ratio}
      \frac{\gamma^-_{\tp,i,\tp',j}}{\gamma^+_{\tp,i,\tp',j}} =
      \Gamma_{\tp,i} \, \Gamma_{\tp',j}
    \end{equation}
  \end{enumerate}
\end{proposition}
\begin{proof}
  ($\Rightarrow$):
  Recall the detailed balance condition
  from \defn{detailed-balance}
  which says that for all states $x,y$
  \[ \pi_x \, q_{xy} = \pi_y \, q_{yx} \]
  By substituting $\pi_x$ and $\pi_y$ as in \eqn{energy} we obtain
  \begin{equation*}
    \expn\left[
      -\sum_{\tp \in \types} \sum_{0 < i \leqslant \valence(\tp)}
      \cost_\tp(i) \, x(\tp_i)
    \right] q_{xy} =
    \expn\left[
      -\sum_{\tp \in \types} \sum_{0 < i \leqslant \valence(\tp)}
      \cost_\tp(i) \, y(\tp_i)
    \right] q_{yx}
  \end{equation*}
  and by rearranging
  \[ \prod_{\tp \in \types} \prod_{0 < i \leqslant \valence(\tp)}
     e^{\,\cost_\tp(i) \, (y(\tp_i) - x(\tp_i))} =
     \frac{q_{yx}}{q_{xy}} \]

  When $y$ is obtained from $x$ by binding nodes $u,v$,
  the difference $y(\tp_i) - x(\tp_i)$ is equal to $0$
  for all pairs $\tp, i$ except
  i) when $\tp = \typeof(u), i = \degree_x(u)$
  or $\tp = \typeof(v)$, $i = \degree_x(v)$,
  then $y(\tp_i) - x(\tp_i) = -1$; and
  ii) when $\tp = \typeof(u), i = \degree_y(u) = \degree_x(u) + 1$
  or $\tp = \typeof(v), i = \degree_y(v) = \degree_x(v) + 1$,
  then $y(\tp_i) - x(\tp_i) = 1$.
  Let $\tp_u = \typeof(u)$, $\tp_v = \typeof(v)$,
  $d_u = \degree_x(u)$ and $d_v = \degree_x(v)$.
  It follows that the last equation can be rewritten as
  \[ \expn\left[
       \cost_{\tp_u}(d_u+1) +
       \cost_{\tp_v}(d_v+1) -
       \cost_{\tp_u}(d_u) -
       \cost_{\tp_v}(d_v) \right] =
     \frac{q_{yx}}{q_{xy}} \]
  % By expanding ... % Philipp didn't like expanding
  By substituting $\cost$ we get % and $q$ we have
  % \[ \frac{
  %      \Gamma_{\tp_u,d_u} \, \Gamma_{\tp_v,d_v}}{
  %      (\valence(u) - d_u) \, (\valence(v) - d_v)} =
  \[ \frac{
     \prod_{0 \leqslant i < d_u+1}%\limits
     \frac{\Gamma_{\tp_u,i}}{\valence(u) - i} \,
     \prod_{0 \leqslant i < d_v+1}%\limits
     \frac{\Gamma_{\tp_v,i}}{\valence(v) - i}}{
     \prod_{0 \leqslant i < d_u}%\limits
     \frac{\Gamma_{\tp_u,i}}{\valence(u) - i} \,
     \prod_{0 \leqslant i < d_v}%\limits
     \frac{\Gamma_{\tp_v,i}}{\valence(v) - i}} =
     \frac{q_{yx}}{q_{xy}} \]
     % \frac{
     %   % \gamma^-_{\tp_u,d_u,\tp_v,d_v}}{
     %   % \gamma^+_{\tp_u,d_u,\tp_v,d_v} \,
     %   \beta(u,v)}{
     %   \alpha(u,v) \, (\valence(u) - d_u) \, (\valence(v) - d_v)} \]
  Products on the left cancel out and yield,
  after substituting $q$ on the right,
  % which by simplification and substitution of $q$ yields
  \[ \frac{
       \Gamma_{\tp_u,d_u} \, \Gamma_{\tp_v,d_v}}{
       (\valence(u) - d_u) \, (\valence(v) - d_v)} =
     \frac{
       \beta(u,v)}{
       \alpha(u,v) \, (\valence(u) - d_u) \, (\valence(v) - d_v)} \]
  which then simplifies to
  \[ \Gamma_{\tp_u,d_u} \, \Gamma_{\tp_v,d_v} =
    \frac{\gamma^-_{\tp_u,d_u,\tp_v,d_v}}{\gamma^+_{\tp_u,d_u,\tp_v,d_v}} \]
  This equality holds in general for nodes of any degree and type.

  ($\Leftarrow$):
  We prove that, whenever (ii) holds,
  $\pi$ verifies the detailed balance condition.
  For all $x,y$ such that $q_{xy} = 0$
  the equality $\pi_x\,q_{xy} = \pi_y\,q_{yx}$ holds
  as rules are reversible and (ii) dictates that
  a rate constant is zero if the reverse rate constant is.
  % NOTE: q_{xy} = 0 might be because there is no transition
  % between x and y or because the rate constant is zero.
  When $q_{xy} > 0$ then $y$ can be obtained from $x$
  by binding or unbinding some nodes $u,v$.
  By substituting $\tp$ for $\typeof(u)$, $\tp'$ for $\typeof(v)$,
  $i$ for $\degree_x(u)$ and $j$ for $\degree_x(v)$
  in \eqn{cas-ratio} we obtain the last equation
  in the first part of the proof.
  We can replay the transformations backwards
  to obtain $\pi_x\,q_{xy} = \pi_y\,q_{yx}$
  when $y$ is obtained by binding.
  The case of unbinding follows a similar argument.
\end{proof}


\section{Flipping and binding} % co-ANC
\label{sec:fb}

Now we look at systems whose nodes have sites
that possess an internal state.
This internal state is used to decide when to bind other nodes
or change the internal state of other sites.
For simplicity, internal states can
take one of only two possible values. %, 0 and 1.
Unlike cooperative assembly systems,
% where sites were undistinguishable from each other,
here sites are \emph{distinguishable}
% and thus have a unique name that distinguishes it from the rest
as in Kappa.
Hence, we will extend contact maps $g$ as defined in \sct{kappa}
with maps that assign an internal state in $\set{0,1}$
to sites in $\sites_{\anon{g}}$
and use that as graphs in this section.
% TODO: add the order of sites wherever it's used.
% In addition, sites are \emph{ordered}
% by a total order $<_a$ with a an agent type
% (node in the contact graph).

A site's internal state can be changed by rules
we call \emph{flips}.
The rate at which we flip a site may depend on
the type of the site and the node it belongs to,
the internal state of sites on the same node
and the type of the neighbours.
Note that it cannot depend on
the internal states of the neighbours' sites
or the nodes they are bound to.
Graphically, flips are of the form
\begin{center}
  \begin{tikzpicture}[thick]
    \node[grphnode,anchor=east] (lhs) at (0,0) {
      \tikz[ingrphdiag]{
        \nn[n4]{a}{0,0}{$u$};
        \n[n4,dotted]{b1}{10:1.1};
        \n[n4,dotted]{b2}{170:1.1};
        \n[n4,dotted]{b3}{-90:1.1};
        \e[dotted]{a}{b1};
        \e[dotted]{a}{b2};
        \e[dotted]{a}{b3};
        \site[n]{s1}{a.south};
        \site[n6]{s2}{170:.33};
        \site[n6]{s3}{10:.33};
        \site[n4,dotted,shift={(10:1.1)}]{s4}{190:.25};
        \site[n4,dotted,shift={(170:1.1)}]{s5}{-10:.25};
        \site[n4,dotted]{s6}{b3.north};
        \node at (-65:.48) {\scriptsize x};
        \node at (110:.7) {\Large .};
        \node at (90:.7) {\Large .};
        \node at (70:.7) {\Large .};
      }};
    \path (lhs.east) +(.3,.09) edge[rule] +(1,.09)
      +(1.3,0) coordinate (r);
    \path (lhs.east) +(1,-.09) edge[rule] +(.3,-.09);
    \node[grphnode,anchor=west] (rhs) at (r) {
      \tikz[ingrphdiag]{
        \nn[n4]{a}{0,0}{$u$};
        \n[n4,dotted]{b1}{10:1.1};
        \n[n4,dotted]{b2}{170:1.1};
        \n[n4,dotted]{b3}{-90:1.1};
        \e[dotted]{a}{b1};
        \e[dotted]{a}{b2};
        \e[dotted]{a}{b3};
        \site[n5]{s1}{a.south};
        \site[n6]{s2}{170:.33};
        \site[n6]{s3}{10:.33};
        \site[n4,dotted,shift={(10:1.1)}]{s4}{190:.25};
        \site[n4,dotted,shift={(170:1.1)}]{s5}{-10:.25};
        \site[n4,dotted]{s6}{b3.north};
        \node at (-65:.48) {\scriptsize x};
        \node at (110:.7) {\Large .};
        \node at (90:.7) {\Large .};
        \node at (70:.7) {\Large .};
      }};
  \end{tikzpicture}
\end{center}
where the dotted lines % in the rule diagram
denote an optional node or edge
and site $x$ changes state from white to black.
As usual, we write $r_L$ for the left-hand side contact map of rule $r$
and $r_R$ for that of the right-hand side,
both contact maps over some fixed contact graph $C$.

In a flip we have a complete view
over the internal state of sites in $u$
and those of the neighbours,
which we characterise as vectors indexed by site types
in $I := \sitemap_C^{-1}(r_{L,\agents}(u))$.
The rate constants of the forward and backward
flip rules are then parametrised by the agent type
$a := r_{L,\agents}(u) = r_{R,\agents}(u)$ of $u$ in $C$,
the site type $i := r_{L,\sites}(x) = r_{R,\sites}(x)$ of $x$,
the internal state vector $\sitestates \in \set{0,1}^I$ of $u$,
% TODO: how do i get the internal state vector of the neighbour?
and the binding state vector $\neighbours \in
((\agents_C \times \sites_C) \union \set{\star})^I$
% ((\agents_C \times \sites_C \times \set{0,1}^I) \union \set{\star})^I
where $\star$ is used to denote a free site.
% When all sites are free we use $\emptyvec$.
We write $\lambda^+_{a,i,\sitestates,\neighbours}$
for the rate constant of the forward rule,
$\lambda^-_{a,i,\sitestates,\neighbours}$
for that of the backward rule,
and $\Lambda_{a,i,\sitestates,\neighbours} =
\lambda^-_{a,i,\sitestates,\neighbours}/
\lambda^+_{a,i,\sitestates,\neighbours}$ for their ratio.
% between the backward and forward rate constants. % of flips.
% TODO: do we (have to) use them?
% Also, later we use $\sitestatesof_x(u)$ and $\neighboursof_x(u)$
% for the internal state vector and neighbour vector % of neighbours' types
% of node $u$ in $x$.

The second type of rules we allow are \emph{binds}.
They are of the form
\begin{center}
  \begin{tikzpicture}[thick]
    \node[grphnode,anchor=east] (lhs) at (0,0) {
      \tikz[ingrphdiag]{
        \begin{scope}[shift={(0,0)}]
          \nn[n4]{a}{0,0}{$u$};
          \node[Black!60!White] at (155:.48) {\Large .};
          \node[Black!60!White] at (180:.48) {\Large .};
          \node[Black!60!White] at (205:.48) {\Large .};
          \node at (26:.48) {\scriptsize x};
          \e{a}{.55,0};
          \site[n6]{s1}{a.east};
          \site[n6]{s2}{110:.33};
          \site[n6]{s3}{-110:.33};
        \end{scope}
        \begin{scope}[shift={(1.4,0)}]
          \nn[n4]{b}{0,0}{$v$};
          \node[Black!60!White] at (25:.48) {\Large .};
          \node[Black!60!White] at (0:.48) {\Large .};
          \node[Black!60!White] at (-25:.48) {\Large .};
          \node at (206:.51) {\scriptsize y};
          \e{b}{-.55,0};
          \site[n6]{s1}{b.west};
          \site[n6]{s2}{70:.33};
          \site[n6]{s3}{-70:.33};
        \end{scope}
      }};
    \path (lhs.east) +(.3,.09) edge[rule] +(1,.09)
      +(1.3,0) coordinate (r);
    \path (lhs.east) +(1,-.09) edge[rule] +(.3,-.09);
    \node[grphnode,anchor=west] (rhs) at (r) {
      \tikz[ingrphdiag]{
        \begin{scope}[shift={(0,0)}]
          \nn[n4]{a}{0,0}{$u$};
          \node[Black!60!White] at (155:.48) {\Large .};
          \node[Black!60!White] at (180:.48) {\Large .};
          \node[Black!60!White] at (205:.48) {\Large .};
          \node at (26:.48) {\scriptsize x};
          \site[n6]{s2}{110:.33};
          \site[n6]{s3}{-110:.33};
        \end{scope}
        \begin{scope}[shift={(1.2,0)}]
          \nn[n4]{b}{0,0}{$v$};
          \node[Black!60!White] at (25:.48) {\Large .};
          \node[Black!60!White] at (0:.48) {\Large .};
          \node[Black!60!White] at (-25:.48) {\Large .};
          \node at (206:.51) {\scriptsize y};
          \site[n6]{s2}{70:.33};
          \site[n6]{s3}{-70:.33};
        \end{scope}
        \e{a}{b};
        \site[n6]{s1}{a.east};
        \site[n6]{s1}{b.west};
      }};
  \end{tikzpicture}
\end{center}
where the three grey dots are an ellipsis
meaning that nodes $u$ and $v$ must declare
an internal state to all sites they have.
The rate constant of the binding and unbinding rules
depends on the internal state of the sites
on the participating nodes $\sitestates$ and $\sitestates'$
as well as the type of the nodes $a,b$ and the bound sites~$i,j$.
We write $\Gamma_{a,i,\sitestates,b,j,\sitestates'} =
\gamma^-_{a,i,\sitestates,b,j,\sitestates'}/
\gamma^+_{a,i,\sitestates,b,j,\sitestates'}$ for
the ratio between the backward and forward rate constants of binds.

We refer to systems composed by flips and binds
as \emph{flip-bind systems} or FB-systems for short.
We will show how the energy function of an FB-system looks like.
But first, we prove two lemmas that show us how detailed balance
fixes the value of some ratios of rate constants,
reducing so the number of free parameters in the system.
To simplify notation in the following lemmas,
let $\sitestates+i$ be the vector $\sitestates$
with site $i$ flipped.

\begin{lemma}
  \label{lemma:flipflip}
  % Let $x$ be a state of an FB-system with detailed balance.
  % For all nodes $u$ in $x$ and sites $i,j$ in $u$ it is true that
  Let $C$ be a contact graph.
  For all agent types $a \in \agents_C$,
  let $I = \sitemap_C^{-1}(a)$,
  and for all
  site types $i \in I$, % \sites_C such that $\sitemap_C(i) = a$,
  internal state vectors
  $\sitestates \in \set{0,1}^I$ and
  binding state vectors
  $\neighbours \in ((\agents_C\times\sites_C)\union\set{\star})^I$,
  an FB-system with detailed balance verifies
  \begin{equation}
    \label{eq:flipflip}
    \Lambda_{a,i,\sitestates,\neighbours} \,
    \Lambda_{a,j,\sitestates+i,\neighbours} =
    \Lambda_{a,j,\sitestates,\neighbours} \,
    \Lambda_{a,i,\sitestates+j,\neighbours}
  \end{equation}
  % where $a$ is the type of $u$,
  % $\sitestates$ its internal state vector
  % and $\neighbours$ its vector of neighbours.
\end{lemma}
\begin{proof}
  Pick a state $x$ and a node $u$ in $x$.
  We can find a square in the transition graph
  with 4 flips starting from $x$:
  flip site $i$ first then $j$ and flip $j$ first then $i$.
  \begin{center}
    \begin{tikzpicture}
      \node (x) at (0,0) {$x$};
      \node (xi) at (4,0) {$x_i$};
      \node (xj) at (0,-3) {$x_j$};
      \node (xij) at (4,-3) {$x_{ij}$};
      % x -- xi
      \path (.5,.09) edge[rule]
        node[above] {$\lambda^+_{a,i,\sitestates,\neighbours}$}
        (3.5,.09);
      \path (3.5,-.09) edge[rule]
        node[below] {$\lambda^-_{a,i,\sitestates,\neighbours}$}
        (.5,-.09);
      % x -- xj
      \path (.09,-.5) edge[rule]
        node[right] {$\lambda^+_{a,j,\sitestates,\neighbours}$}
        (.09,-2.5);
      \path (-.09,-2.5) edge[rule]
        node[left] {$\lambda^-_{a,j,\sitestates,\neighbours}$}
        (-.09,-.5);
      % xi -- xij
      \path (4.09,-.5) edge[rule]
        node[right] {$\lambda^+_{a,j,\sitestates+i,\neighbours}$}
        (4.09,-2.5);
      \path (3.91,-2.5) edge[rule]
        node[left] {$\lambda^-_{a,j,\sitestates+i,\neighbours}$}
        (3.91,-.5);
      % x -- xi
      \path (.5,-2.91) edge[rule]
        node[above] {$\lambda^+_{a,i,\sitestates+j,\neighbours}$}
        (3.5,-2.91);
      \path (3.5,-3.09) edge[rule]
        node[below] {$\lambda^-_{a,i,\sitestates+j,\neighbours}$}
        (.5,-3.09);
    \end{tikzpicture}
  \end{center}
  By detailed balance we have that
  the product of rates along a cycle must be equal to 1.
  Hence, starting from $x$ and going through the cycle
  in one direction and the reverse we get
  \begin{equation*}
    \lambda^+_{a,i,\sitestates,\neighbours} \,
    \lambda^+_{a,j,\sitestates+i,\neighbours} \,
    \lambda^-_{a,i,\sitestates+j,\neighbours} \,
    \lambda^-_{a,j,\sitestates,\neighbours} =
    \lambda^+_{a,j,\sitestates,\neighbours} \,
    \lambda^+_{a,i,\sitestates+j,\neighbours} \,
    \lambda^-_{a,j,\sitestates+i,\neighbours} \,
    \lambda^-_{a,i,\sitestates,\neighbours}
  \end{equation*}
  By rearranging we obtain \eqn{flipflip}.
\end{proof}

When we consider all flips for a node with $n$ sites,
we find an $n$-hypercube in the transition graph.
This hypercube has $2^{n-1} n$ edges
where an edge corresponds to a pair of forward and backward flips.
It follows that there are the same number of $\Lambda$ ratios.
In addition, each face of the hypercube generates an equation
by \lem{flipflip} % detailed balance
and there are $2^{n-3} (n-1) n$ faces in an $n$-hypercube.
This is a severe constraint on the number of parameters
that can be freely set in an FB-system with detailed balance.
% Note also that the number of faces of an $n$-hypercube grows faster
% than the number of edges and thus when $n \geqslant 5$
% there are no free parameters.
% What does this mean practically?
% What is the concrete value that $\Lambda$s take?

A further constrain to the values of
the rate constant ratios of flips and bindings
can be obtained and its proved in the following lemma.
For this lemma we will use a fixed total order $<_a$
on the sites of an agent type $a$ in the contact graph $C$.

\begin{lemma}
  \label{lemma:flipbind}
  % Let $x$ be a state of an FB-system with detailed balance.
  % For all nodes $u$ in $x$ and sites $i$ in $u$ it is true that
  Let $C$ be a contact graph.
  For all agent types $a \in \agents_C$,
  let $I = \sitemap_C^{-1}(a)$,
  and for all
  site types $i \in I$, % \sites_C such that $\sitemap_C(i) = a$,
  binding state vectors
  $\neighbours \in ((\agents_C\times\sites_C)\union\set{\star})^I$
  and internal state vectors
  $\sitestates \in \set{0,1}^I$,
  % $\sitestates_{jb} \in \set{0,1}^{\sitemap_C^{-1}(b)}$
  % with $j \in I$, $b \in \agents_C$,
  $\sitestates_{j} \in \set{0,1}^{\sitemap_C^{-1}(b)}$
  with $j \in I$ and $b$ the agent type in $\neighbours(j)$,
  an FB-system with detailed balance verifies
  \begin{equation}
    \label{eq:flipbind}
    \frac{\Lambda_{a,i,\sitestates,\neighbours}}{
          \Lambda_{a,i,\sitestates,\emptyvec}} =
    \prod_{\substack{i' \in I\\\neighbours(i') \neq \star\\\tuple{b,j} := \neighbours(i')}}
    \frac{\Gamma_{a,i',\sitestates,b,j,\sitestates_{i'}}}{
          \Gamma_{a,i',\sitestates+i,b,j,\sitestates_{i'}}}
  \end{equation}
  % where $a$ is the type of $u$,
  % $\sitestates$ its internal state vector,
  % $\neighbours$ its vector of neighbours
  % and $\emptyvec$ a vector of free sites,
  where $\emptyvec$ a vector of free sites,
  \ie $\emptyvec(i) = \star$ for all $i \in I$.
\end{lemma}
\begin{proof}
  We construct a series of squares
  with 2 flips and 2 bindings each.
  Pick a state $x$ and a node $u$ in $x$.
  Strip $u$ of all its neighbours and
  call that state $x_0$.
  Choose a site $i$ of $u$.
  The first square starts from $x_0$, then
  i) flips site $i$ and
  ii) binds the smallest site $i_1$
  according to the order $<_{x_\agents(u)}$
  to site $j$ of an agent of type $b$
  that has internal vector state $\sitestates_{i_1}$,
  where $(j,b) = \neighbours(i_1)$.
  % or does (ii) first then (i).
  After performing (i) then (ii) or (ii) then (i)
  % the flip and bind in any of the two possible orders
  we reach state $x_1$.\\[.09cm]
  % \begin{center}
  % \vspace{.2cm}
  \hspace*{-.3cm}
    \begin{tikzpicture}
      %
      % first square
      %
      \node[outer sep=.2cm] (x0) at (0,0) {$x_0$};
      \node[outer sep=.2cm] (x0f) at (0,-3) {$\widehat{x_0}$};
      \node[outer sep=.2cm] (x1f) at (4,0) {$\widehat{x_1}$};
      \node[outer sep=.2cm] (x1) at (4,-3) {$x_1$};
      % x0 -- x1f
      \draw[rule,transform canvas={yshift=.09cm}] (x0) --
        node[above] {$\gamma^+_{a,i_1,\sitestates,b,j,\sitestates_{i_1}}$}
        (x1f);
      \draw[rule,transform canvas={yshift=-.09cm}] (x1f) --
        node[below] {$\gamma^-_{a,i_1,\sitestates,b,j,\sitestates_{i_1}}$}
        (x0);
      % x0 -- x0f
      \draw[rule,transform canvas={xshift=.09cm}] (x0) --
        node[right] {$\lambda^+_{a,i,\sitestates,\emptyvec}$}
        (x0f);
      \draw[rule,transform canvas={xshift=-.09cm}] (x0f) --
        node[left] {$\lambda^-_{a,i,\sitestates,\emptyvec}$}
        (x0);
      % x1f -- x1
      \draw[rule,transform canvas={xshift=.09cm}] (x1f) --
        node[right] {$\lambda^+_{a,i,\sitestates,\neighbours_1}$}
        (x1);
      \draw[rule,transform canvas={xshift=-.09cm}] (x1) --
        node[left] {$\lambda^-_{a,i,\sitestates,\neighbours_1}$}
        (x1f);
      % x0f -- x1
      \draw[rule,transform canvas={yshift=.09cm}] (x0f) --
        node[above] {$\gamma^+_{a,i_1,\sitestates+i,b,j,\sitestates_{i_1}}$}
        (x1);
      \draw[rule,transform canvas={yshift=-.09cm}] (x1) --
        node[below] {$\gamma^-_{a,i_1,\sitestates+i,b,j,\sitestates_{i_1}}$}
        (x0f);
      %
      % middle square
      %
      \node[outer sep=.2cm] (xn) at (7.5,0) {$x_n$};
      \node[outer sep=.2cm] (xnf) at (7.5,-3) {$\widehat{x_n}$};
      % x1f -- xn
      \draw[rule,transform canvas={yshift=.09cm}] (x1f) -- (xn);
      \draw[rule,transform canvas={yshift=-.09cm}] (xn) --
        node[below] {$\ldots$} (x1f);
      % xn -- xnf
      \draw[rule,transform canvas={xshift=.09cm}] (xn) --
        node[right] {$\lambda^+_{a,i,\sitestates,\neighbours_n}$}
        (xnf);
      \draw[rule,transform canvas={xshift=-.09cm}] (xnf) --
        node[left] {$\lambda^-_{a,i,\sitestates,\neighbours_n}$}
        (xn);
      % x1 -- xnf
      \draw[rule,transform canvas={yshift=.09cm}] (x1) --
        node[above] {$\ldots$} (xnf);
      \draw[rule,transform canvas={yshift=-.09cm}] (xnf) -- (x1);
      %
      % nth square
      %
      \node (xn+1) at (11.5,-3) {$x_{n+1}$};
      \node (xn+1f) at (11.5,0) {$\widehat{x_n}_{+1}$};
      % xn -- xn+1f
      \draw[rule,transform canvas={yshift=.09cm}] (xn) --
        node[above] {$\gamma^+_{a,i_n,\sitestates,b',j',\sitestates_{i_n}}$}
        (xn+1f);
      \draw[rule,transform canvas={yshift=-.09cm}] (xn+1f) --
        node[below] {$\gamma^-_{a,i_n,\sitestates,b',j',\sitestates_{i_n}}$}
        (xn);
      % xn+1f -- xn+1
      \draw[rule,transform canvas={xshift=.09cm}] (xn+1f) --
        node[right] {$\lambda^+_{a,i,\sitestates,\neighbours_{n+1}}$}
        (xn+1);
      \draw[rule,transform canvas={xshift=-.09cm}] (xn+1) --
        node[left] {$\lambda^-_{a,i,\sitestates,\neighbours_{n+1}}$}
        (xn+1f);
      % x0f -- x1
      \draw[rule,transform canvas={yshift=.09cm}] (xnf) --
        node[above] {$\gamma^+_{a,i_n,\sitestates+i,b',j',\sitestates_{i_n}}$}
        (xn+1);
      \draw[rule,transform canvas={yshift=-.09cm}] (xn+1) --
        node[below] {$\gamma^-_{a,i_n,\sitestates+i,b',j',\sitestates_{i_n}}$}
        (xnf);
    \end{tikzpicture}
  % \end{center}
  % with $\neighbours_1$ the neighbour vector of $u$ in $x_1$.
  with $\neighbours_1(i) = \star$ for all $i$ except $i_1$,
  where $\neighbours_1(i_1) = \neighbours(i_1)$.
  In general, $\neighbours_n(i)$ is equal to $\neighbours(i)$
  if $i \leq_{x_\agents(u)} i_n$ and $\star$ otherwise.

  By detailed balance we obtain the following relation
  for the first square
  \begin{equation}
    \label{eq:fb1}
    \Lambda_{a,i,\sitestates,\emptyvec} \,
    \Gamma_{a,i_1,\sitestates+i,b,j,\sitestates_{i_1}} =
    \Gamma_{a,i_1,\sitestates,b,j,\sitestates_{i_1}} \,
    \Lambda_{a,i,\sitestates,\neighbours_1}
  \end{equation}

  We construct the $n$th square
  starting from $x_n$ by flipping site $i$
  and binding site $i_n$ to site $j'$ in an agent of type $b'$
  as indicated by $\neighbours(i_n)$.
  This neighbour has an internal state vector $\sitestates_{i_n}$.
  Again, by detailed balance we get
  \begin{equation*}
    \label{eq:fbn}
    \Lambda_{a,i,\sitestates,\neighbours_n} \,
    \Gamma_{a,i,\sitestates+i,b',j',\sitestates_{i_n}} =
    \Gamma_{a,i,\sitestates,b',j',\sitestates_{i_n}} \,
    \Lambda_{a,i,\sitestates,\neighbours_{n+1}}
  \end{equation*}
  % with $\neighbours_n$ the neighbour vector of $u$ in $x_n$.

  \eqn{fb1} can be rewritten as
  \begin{equation*}
    \frac{
      \Lambda_{a,i,\sitestates,\neighbours_1}}{
      \Lambda_{a,i,\sitestates,\emptyvec}} =
    \frac{
      \Gamma_{a,i,\sitestates+i,
              b,j,\sitestates_{i_1}}}{
      \Gamma_{a,i,\sitestates,
              b,j,\sitestates_{i_1}}}
  \end{equation*}
  Then we substitute $\Lambda_{a,i,\sitestates,\neighbours_1}$
  according to \eqn{fbn} and obtain
  \begin{equation*}
    \frac{
      \Lambda_{a,i,\sitestates,\neighbours_2}}{
      \Lambda_{a,i,\sitestates,\emptyvec}} =
    \frac{
      \Gamma_{a,i,\sitestates+i,
              b,j,\sitestates_{i_1}}}{
      \Gamma_{a,i,\sitestates,
              b,j,\sitestates_{i_1}}} \,
    \frac{
      \Gamma_{a,i,\sitestates+i,
              c,k,\sitestates_{i_2}}}{
      \Gamma_{a,i,\sitestates,
              c,k,\sitestates_{i_2}}}
  \end{equation*}
  We repeat until we recover \eqn{flipbind}.
\end{proof}





%%% Local Variables:
%%% mode: latex
%%% TeX-master: "thesis"
%%% End:


\if0
%% Appendix
\appendix
\chapter{Model: Assembling triangles}
\label{app:triangles}
\include{model}
\fi

%% Choose your favourite bibliography style here.
% \bibliographystyle{apacite}

%% If you want the bibliography single-spaced (which is allowed), uncomment
%% the next line.
\singlespace

%% Specify the bibliography file. Default is thesis.bib.
% \bibliography{thesis}

\printbibliography

%% ... that's all, folks!
\end{document}

%%% Local Variables:
%%% mode: latex
%%% TeX-master: t
%%% End:
