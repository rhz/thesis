\documentclass[phd,lfcs]{infthesis}
% \documentclass[phd,lfcs,twoside]{infthesis}
% \documentclass{report}

\usepackage[utf8]{inputenc}
\usepackage[T1]{fontenc}
\usepackage[british]{babel}
\usepackage{microtype}
\usepackage[usenames,dvipsnames,svgnames,table]{xcolor}
% \usepackage{natbib}
% \usepackage[natbibapa]{apacite}
\usepackage{csquotes}
\usepackage[natbib=true,style=authoryear]{biblatex}
\usepackage{graphicx}
\usepackage{textcomp}
\usepackage{wrapfig}
\usepackage{xfrac}
\usepackage{xspace}
% \usepackage{spacing}
% \usepackage{epigraph}
% \usepackage{gnuplot-lua-tikz}
\usepackage{mathcommon}
% \usepackage{kappa}
% \usepackage{kappalistings}
\usepackage[sc]{mathpazo}
\usepackage{hyperref}

% Bibliography
\addbibresource{thesis.bib}

% Text
\newcommand{\ie}{i.e.\xspace}
\newcommand{\eg}{e.g.\xspace}

% Referencing
\newcommand{\chp}[1]{\S\ref{chp:#1}}
\newcommand{\sct}[1]{\S\ref{sec:#1}}
\newcommand{\subsct}[1]{\S\ref{subsec:#1}}
\newcommand{\eqn}[1]{Eq.~\ref{eq:#1}}
\newcommand{\eqns}[2]{Eq. \ref{eq:#1} and \ref{eq:#2}}
\newcommand{\lem}[1]{Lemma~\ref{lemma:#1}}
\newcommand{\lems}[2]{Lemmas \ref{lemma:#1} and \ref{lemma:#2}}
\newcommand{\thm}[1]{Th.~\ref{thm:#1}}
\newcommand{\fig}[1]{Fig.~\ref{fig:#1}}
\newcommand{\diagram}[1]{diagram~\ref{eq:#1}}
\newcommand{\app}[1]{Appendix~\ref{app:#1}}
\newcommand{\mcite}[1]{\textcolor{gray}{#1}} % missing cite

% Thermodynamic graph rewriting
\newcommand{\rules}{\mathcal{R}}
\newcommand{\generators}{\mathcal{G}}
\newcommand{\shapes}{\mathcal{P}}
\newcommand{\cost}{\epsilon} % \varepsilon is the same with this font
% TODO: should I enclose the result of \refinedrules in parenthesis?
\newcommand{\refinedrules}{\generators_\shapes}
\newcommand{\gp}{\Gamma}

% Labelled transition systems
\newcommand{\LTS}{\mathcal{L}}

% Markov chains
\newcommand{\states}{\mathcal{S}} % state space
\newcommand{\qm}{Q} % q matrix
\newcommand{\ip}{\pi} % invariant probability

% Petri nets
\newcommand{\species}{\mathcal{A}}
\newcommand{\reactions}{\mathcal{R}}

% Cooperative assembly systems
\newcommand{\valence}{\nu} %\upsilon}
\newcommand{\types}{\mathcal{T}} % set of types
\newcommand{\type}{t} % a type (variable)
\newcommand{\typeof}{\tau} % map from nodes to types
\newcommand{\degree}{d}

% FB-systems
\newcommand{\sitesof}{\sigma}
\newcommand{\emptyvec}{\varnothing}
\newcommand{\neighbours}{\vec{n}}
\newcommand{\sitestates}{\vec{s}}
\newcommand{\sitestatesof}{\iota}
\newcommand{\neighboursof}{\eta}

% Kappa
\newcommand{\edges}{\mathcal{E}}
\newcommand{\SG}{\mathbf{SG}}
\newcommand{\SGe}{\mathbf{SGe}}
\newcommand{\rSGe}{\mathbf{rSGe}}
\newcommand{\anon}[1]{\left|#1\right|}

% Category theory
\newcommand{\homset}{\Upsilon}
\newcommand{\mSet}{\mathbf{mSet}}
\newcommand{\Set}{\mathbf{Set}}
\newcommand{\iso}{\simeq}

% Triangles
\newcommand{\cyc}[1]{{}\cdot #1 \cdot{}}

% Math
\renewcommand{\tuple}[1]{\left(#1\right)}
\DeclareMathOperator*{\expn}{exp}
\renewcommand{\qedsymbol}{\ensuremath{\blacksquare}}
\newcommand{\partialto}{\rightharpoonup}
\newcommand{\id}{\vec{1}} % identity function

% Other stuff
\newcommand{\maybe}[1]{\textcolor{gray}{#1}}
% \renewcommand{\maybe}[1]{}
\newcommand{\todo}[1]{\textcolor{red}{TODO: #1}}
% \renewcommand{\todo}[1]{}

% Styles for nodes
\tikzstyle{n}=[circle,draw=Black] %,thick,minimum size=6pt,inner sep=0pt]
\tikzstyle{n1}=[n,fill=Orange]
\tikzstyle{n2}=[n,fill=Blue!60!White]
\tikzstyle{n3}=[n,fill=Green!40!White] % Green!40!Ocre?
\tikzstyle{n4}=[n,fill=White!95!Black]
\tikzstyle{n5}=[n,fill=White]
\tikzstyle{n6}=[n,fill=Black]
\tikzstyle{n7}=[n,fill=Yellow!50!White]
\tikzstyle{p}=[n,draw=none] % phantom
\tikzstyle{e}=[thick]
\tikzstyle{rule}=[thick,->,>=angle 90]
\tikzstyle{onarrow}=[fill=White,inner sep=2pt]
\tikzstyle{site}=[n7,inner sep=1.5pt]

% Binding types
\tikzstyle{bt1}=[n1,dashed,inner sep=3.5pt]
\tikzstyle{bt2}=[n2,dashed,inner sep=3.5pt]
\tikzstyle{bt3}=[n3,dashed,inner sep=3.5pt]

% Background box for graph diagrams
\tikzstyle{grphdiag-bg}=[rounded corners,fill=White!85!Black]

% Graph diagrams
\tikzstyle{grphnode}=[rectangle,grphdiag-bg]
% \tikzstyle{ingrphdiag}=[>=stealth,thick,semithick,
%   n/.append style={semithick,minimum size=4pt}]
\tikzstyle{ingrphdiag}=[thick,anchor=center]

% Use different arrow tips to distinguish edges from morphism
% and a light gray backdrop.
\usetikzlibrary{backgrounds}
\tikzstyle{grphdiag}=[>=stealth,framed,thick,%
  background rectangle/.style=grphdiag-bg]

% Draw a node
\newcommand{\n}[3][n]{\node[inner sep=5pt,#1] (#2) at (#3) {}}
\newcommand{\nn}[4][n]{\node[inner sep=3.5pt,#1] (#2) at (#3) {#4}}
% \renewcommand{\site}[3][n7]{\node[inner sep=1.5pt,#1] (#2) at (#3) {}}

% Draw an edge
\newcommand{\e}[3][e]{\draw[#1] (#2) -- (#3)}

% Draw a directed edge
\newcommand{\de}[3][e]{\draw (#2) edge[->,#1] (#3)}

% Draw a bi-directional rule
\newcommand{\birulearrow}[2]{\hspace{1ex}\tikz{%
  \draw[rule] (0, .15) -- node[above] {#1} ++( .5, 0);
  \draw[rule] (.5,  0) -- node[below] {#2} ++(-.5, 0);
}\hspace{1ex}}

% Arrows in minimal gluings diagram
\newcommand{\arrsn}[3][]{\arr[#1]{#2.south}{#3.north}}
\newcommand{\arr}[3][]{%
  \draw[-bigto,#1] ($(#2)!.1!(#3)$) -- ($(#2)!.9!(#3)$);}

% Header
\title{Thermodynamic graph rewriting}
\author{Ricardo Honorato-Zimmer}

% Abstract
\abstract{
  We develop a new thermodynamic approach to stochastic graph rewriting.
  The ingredients are a finite set of reversible
  graph rewriting rules $\generators$ (called generating rules),
  a finite set of connected graphs $\shapes$ (called energy patterns),
  and an energy cost function $\cost$
  which associates real values to each of these energy patterns.
  % 
  The idea is that $\generators$ defines the qualitative dynamics,
  by showing which transformations are possible,
  while $\shapes$ and $\cost$ allow one to attach an energy
  to the reachable graphs and, thereby,
  describe their long-term probability distribution $\ip$.
  % 
  Given $\generators$ and $\shapes$,
  we construct a finite set of rules $\generators_{\shapes}$ which
  (i) has the same qualitative transition system as $\generators$; and
  (ii) when equipped with rates according to $\cost$,
  defines a continuous-time Markov chain of which $\ip$ is
  the \emph{stationary} and \emph{limiting} probability distribution.
  % 
  The construction relies on the use of site graphs and
  a technique of `growth policy' for quantitative rule refinement.
  % which is of independent interest.
  % 
  % Nothing else is assumed of $\generators$ or $\shapes$,
  % and only the rates on the generated rule set $\generators_{\shapes}$
  % depend on $\cost$.

  This division of labour between the qualitative and long-term
  quantitative aspects of the dynamics leads to intuitive and concise
  descriptions for realistic models.
  % 
  It also guarantees thermodynamical consistency
  (\emph{aka} detailed balance),
  otherwise known to be undecidable.
  % which is important for some applications.
  % 
  Finally, it leads to parsimonious parameterisations of models,
  % again an important point in some applications.
  an important point in the application of stochastic graph rewriting
  to the modelling of biochemical interaction networks.
}

\begin{document}

%% First, the preliminary pages
\begin{preliminary}

\maketitle

\if0
\begin{acknowledgements}
  I'm really thankful to everyone that's helped me along the way.
  It's been a marvelous journey with all its highs and lows and so much learning.
  This experience has made me feel alive and transformed me in so many ways.

  Vincent, I'd like to thank you first for having me in your group.
  For having patiently taught me so many things from math to good life lessons.
  Thank you for blowing my mind so many times
  with your brilliant ideas and ways to look at things.
  Thank you for sharing your time with me
  and having faith in me even when things didn't look good.

  Renate y Ricardo, muchas gracias por haberme dado la vida, por haberme criado,
  por haberme enseñado y nutrido de tan buenas ideas, creencias y alimentos.
  Muchas gracias por haberse preocupado por mi en todo momento,
  tanto por mi salud, como bienestar y desarrollo.
  De m\'as est\'a decir, sin uds nada de esto habr\'ia sido posible.
  Muchas gracias por ser los maravillosos padres que son.

  Dani, muchas gracias por ser la maravillosa hermana que eres.
  Como poner en palabras todas las cosas que hemos compartido
  y todas las cosas que me has enseñado en toda esta vida juntos,
  que me han formado y me han hecho en lo que soy.
  Muchas gracias por tu eterna y sincera preocupacion por mi.
  Te llevo siempre en mi corazon.

  Froko, muchas gracias mi compañero de la vida
  por todo el tiempo que hemos compartido,
  por toda la invaluable compañia,
  por todo el amor en los momentos dificiles
  y por todas las cosas que gracias a ti he logrado comprender.
  No hay suficientes palabras para decirte todo lo que te aprecio.

  An\'ibal and Gr\'ainne, thank you so much guys for all the love
  and good vibes in the good and bad moments of this journey.
  It was amazing to share this time together
  and having the comfort of your words and hugs with me.
  Without your reassuring empathy I wouldn't have made it.

  Mark, thank you so much
  % for teaching me the good lessons and what life ultimately is all about.
  for showing me the tools and techniques to open the heart
  to the expected as well as the unexpected
  and naturally find peace in doing so.
  I could see all along how you embody this compassionate way of being
  and that gave me the strength to start softening my own heart.
  % for giving me tools that have given me some much needed space and freedom.
  % Listening to you was challenging at the beginning
  % but I could see all the time how you embody this compassionate way of being.
  % but slowly started changing my life.

  Sophie, Stuart, Ayleen and Susan, thank you so much guys for
  being the great friends and flatmates that you were.
  Thank you for the countless laughs and the support and love.
  I'm so happy to have you guys in my life.

  Philipp, thank you so much for the wonderful time together.

  Seba,

  Alejandro,

  Claudia,

  Sandro, thank you for patiently explaining me so many important things
  and thank you for sharing your bright ideas with me.
  Thank you so much for all the honest love.

  Tobias, thank you so much for being so nice, soft and caring to me. % I needed it.
  Thank you so much also for showing me some of your math wizardry,
  I really appreaciate it.

  William, thank you for the wonderful and insightful conversations. % sharing your knowledge and opinions
  I really like having met you, having you in the group
  and having you as one of my friends.

  % TODO: Finish Ilias acknoledgement
  Illias, your cool style and humility are true sources of inspiration.

  Gordon,

  Thank you very much to all the other people in the group
  for the good vibes and good advise,
  Milana, Andrea, Guoli, Matteo, Guillaume, Fredrik, Matthias, Nick,
  Katharina, Emilia, David, Oksana, John, Hristiana,
  Argyris, Andreea, Simon, Jean and J\'er\^ome.
  Thanks also to Dr Alfred Hofmann and Dr Alexander Shulgin
  for their life-changing inventions that have provided me
  with beautiful insights during this time.
  I'd like to thank as well everyone that helped me get to the PhD,
  in particular special thanks go to Dr Tomas Perez-Acle
  and Dr Juan Carlos Letelier. % who helped me when.../in this way...
\end{acknowledgements}
\fi

%% Next we need to have the declaration.
\standarddeclaration

%% Finally, a dedication (this is optional -- uncomment the following line if
%% you want one).
% \dedication{To my mummy.}

%% Create the table of contents
\tableofcontents

%% If you want a list of figures or tables, uncomment the appropriate line(s)
% \listoffigures
% \listoftables

\end{preliminary}

\chapter{Introduction}
\label{ch:intro}
% % this needs package epigraph
% \setlength{\epigraphwidth}{.8\textwidth}
% \setlength{\epigraphrule}{0pt}
% \epigraph{
%   I regard as quite useless the reading of
%   large treatises of pure analysis:
%   too large a number of methods pass at once before the eyes.
%   It is in the works of applications that one must study them;
%   one judges their ability there and
%   one apprises the manner of making use of them.}{
%   --- Joseph Louis Lagrange}

% \begin{quotation}
%   \textit{
%     ``I regard as quite useless the reading of
%     large treatises of pure analysis:
%     too large a number of methods pass at once before the eyes.
%     It is in the works of applications that one must study them;
%     one judges their ability there and
%     one apprises the manner of making use of them.''} \\
%   \par\raggedleft--- Joseph Louis Lagrange
% \end{quotation}

% this needs package spacing
% from https://hbfs.wordpress.com/2011/01/18/epigraphs-in-latex/
% epigraph with 3 params: width, text, author
\newcommand{\epigraph}[3]{
\vspace{1em}\hfill{}\begin{minipage}{#1}{\begin{spacing}{0.9}
\small\noindent\textit{#2}\end{spacing}
\vspace{1em}
\hfill{}{#3}}\vspace{2em}
\end{minipage}}
\epigraph{.8\textwidth}{
  I regard as quite useless the reading of
  large treatises of pure analysis:
  too large a number of methods pass at once before the eyes.
  It is in the works of applications that one must study them;
  one judges their ability there and
  one apprises the manner of making use of them.}{
  --- Joseph Louis Lagrange}


%%% Local Variables:
%%% mode: latex
%%% TeX-master: "thesis"
%%% End:


% \noindent
\section{Historical background}

In the history of natural sciences,
there has been two main approaches to describe dynamical systems,
which I call here
\emph{kinetics} and \emph{thermodynamics}.
Loosely speaking, in the kinetic approach the system is
described by the positions and momenta of each particle.
This approach goes all the way back to
Newton's laws of motion~\citep{newton}.
Intuitively, it is a ``ground'' description in the sense that
it is as explicit and detailed as possible.
If a (classical) mechanical system has $N$ particles
then a state of the system is described by
a vector in $\RR^{2 \cdot 3 \cdot N}$.
% Regarding the role of forces in Newtonian mechanics:
% Interactions producing a change of momentum on a particle
% that can be measured independently do not interact between them
% and thus the resulting derivative of the momentum (force)
% is just the sum of the forces as measured independently.

On the other hand,
the thermodynamic approach shows how
all information about the change of the system in time
is contained in the energy function (for conservative systems).
Given a set of states $\states$ for the system,
an energy function $E: \states \to \RR$ maps a state to its energy.
In this way a state is described by a single scalar
regardless of how many particles it comprises.
Naturally, this approach endowed the description of
a dynamical system in classical mechanics
with a remarkable conciseness, simplicity and elegance.
It first appeared in the work of
\citet{lagrange2} and \citet{hamilton},
and has been subsequently used as the basis for most of modern physics.
Once in possession of the energy function,
the kinetic description (\ie the equations of motion)
can then be derived from it.
However the converse is not true:
in general a kinetic description might not have an energy function
from which it can be derived \citep{santilli},
partly because of non-conservative (\eg dissipative) forces.
Obtaining an energy function from the equations of motion
is called the \emph{inverse problem} in classical mechanics
and it was first attended to by \citet{helmholtz}.
Both the direct and the inverse problem are the interest of this thesis.
Note that this approach has been given the name `thermodynamic'
not because of thermodynamics,
the science that studies the dynamics of heat and temperature,
but because of the protagonical role of the energy
in driving the system's evolution.
Certainly, there are connections to thermodynamics
that will be highlighted as they arise.

Half a century after Hamilton's work
researchers like Maxwell, Boltzmann, and Gibbs
applied the ideas of classical mechanics to \emph{atoms}
in order to describe physical properties of matter like pressure,
the capacity to transfer heat, and others.
This body of work came to be known as \emph{statistical mechanics}
and was used to explain Brownian motion by \citet{einstein-brownian},
which after its experimental verification \citep{perrin}
settled the debate about the existence of atoms.
This work however did not attempt to explain
the chemical interactions and reactions that molecules undergo.
That would have to wait yet half a century
for the axiomatisation of probability theory by \citet{kolmogorov}
and the further developments by \citet{doob} and \citet{feller},
who, among others, established the theoretical framework
for continuous-time Markov chains (CTMCs).
Below you can find the definitions for a CTMC
and its infinitesimal generator that will be used in this thesis.
In particular, we work with time-homogeneous CTMCs.

\begin{definition}%[Infinitesimal generator]
  An \emph{infinitesimal generator} $\qm$
  on an at most countable set of states $\states$
  is an $\states \times \states$ matrix
  with elements $q_{ij} \in \RR$, $i,j \in \states$
  such that $0 \leqslant q_{ij} < \infty$ when $i \neq j$
  and $q_{ii} = - \sum_{j \neq i} q_{ij} < \infty$.
\end{definition}

The infinitesimal generator plays the role of
the time derivative of the transition probabilities at time $0$
and induces the evolution of a probabilistic state
according to the Kolmogorov backward equation,
\begin{equation}
  \label{eq:transition-function}
  \ddt P(t) = Q P(t), \quad P(0) = I
\end{equation}
where $P(t)$ is the $\states \times \states$ matrix
with elements $p_{ij}(t) \in \RR$ the probability that
we were in state $i$ at time $0$ and are in state $j$ at time $t$.
% transition probability from state $i$ to $j$ at time $t$.
When the infinitesimal generator is stable and conservative
there is a unique minimal solution to \eqn{transition-function}
\citep{anderson}.
We shall work with this type of infinitesimal generators
and assume there is a transition function $P(t)$
whenever we have an infinitesimal generator $\qm$ and vice versa.

Given a \pmf $s(0)$ on $\states$ (seen as a row vector)
as an initial probabilistic state,
the probability distribution $s(t)$ after time $t$
is given by $s(t) = s(0) P(t)$.
% We can obtain the time derivative of this distribution
% from \eqn{transition-function}.
% % $s_i(t)$ that the Markov chain is in state $i$ at time $t$.
% In coordinate form, we have
% \begin{equation} % TODO: is this equation correct?
%   \label{eq:prob-deriv}
%   \ddt s_i(t) = \sum_{j \in \states} q_{ji} s_j(t)
% \end{equation}
We say the infinitesimal generator is \emph{irreducible}
if every state is reachable regardless of the initial state,
\ie $p_{ij}(t) > 0$ for all $i,j \in \states$
and some $t \geqslant 0$.

\begin{definition}[CTMC]%[continuous-time Markov chain]
  A \emph{continuous-time Markov chain} is a tuple
  $\tuple{\states, s(0), \qm}$ with
  $\states$ an at most countable set of states,
  $s(0)$ a \pmf on $\states$
  representing the initial probabilistic state and
  $\qm$ the infinitesimal generator of the Markov chain.
  % The Markov chain can be presented as a time-indexed family
  % of random variables $X_t$.
\end{definition}

% TODO: Am I here introducing something after using/mentioning it?
An important property of CTMCs for the present work is that of
\emph{time reversibility}, also known as \emph{detailed balance},
which we introduce below.

\begin{definition}[detailed balance]
  An infinitesimal generator $\qm$ on $\states$
  is said to be \emph{time reversible} iff
  there is a \pmf $\ip$ on $\states$ such that
  \begin{equation}
    \label{eq:detailed-balance}
    \ip_i q_{ij} = \ip_j q_{ji}
  \end{equation}
  for all $i,j \in \states$.
  Then $\qm$ is said to have \emph{detailed balance}
  with respect to $\ip$.
\end{definition}

And the related property of an
\emph{invariant} probability measure for the infinitesimal generator.

\begin{definition}
  A \pmf $\ip$ on $\states$ is
  \emph{invariant} for an infinitesimal generator $\qm$
  iff $\ip \qm = 0$, \ie
  \[ -\ip_i q_{ii} = \ip_i \sum_{j \neq i} q_{ij}
                  = \sum_{j \neq i} \ip_j q_{ji} \]
  In other words,
  whenever the action of $\qm$ on it does not change it.
\end{definition}

The relationship between these two properties is established
by the following lemma.

\begin{lemma}
  Suppose the infinitesimal generator $\qm$
  has detailed balance with respect to $\ip$.
  Then $\ip$ is invariant for $\qm$.
\end{lemma}
\begin{proof}
  From \eqn{detailed-balance} we obtain
  \[ \sum_{i \in \states} \ip_i q_{ij} =
     \sum_{i \in \states} \ip_j q_{ji} = -\ip_j q_{jj}, \]
  as $\sum_{i \in \states} q_{ji} = -q_{jj}$ for any fixed state $j$.
\end{proof}

Moreover, we would like to know when this invariant
probability measure is realised by the Markov chain.

\begin{definition}[ergodicity]
  An infinitesimal generator $\qm$ is \emph{ergodic} when
  there is a probability measure $\ip$ on $\states$ such that
  \[ \lim_{t \to \infty} P_{ij}(t) = \ip_j \]
  for all $i,j \in \states$.
\end{definition}

This is equivalent to say that the Markov chain
will converge to the probability measure $\ip$
regardless of the initial state $s(0)$.

\begin{lemma}
  Suppose the infinitesimal generator $\qm$ is irreducible
  and has an invariant probability measure $\ip$.
  Then $\qm$ is ergodic and converges to $\ip$.
\end{lemma}

The proof for this lemma can be found in part 2 of theorem 1.6
in chapter 5 of Anderson's book (\cite*[][pages 160--161]{anderson}).
CTMCs have a strong kinetic flavour as they describe
stochastic processes in terms of probability flows
happening at a certain rate.
They are the ``ground'' description in the stochastic world
and all approaches to describe these processes
are interpreted in terms of them.

It is natural to wonder then how the thermodynamic approach
looks like in the stochastic world.
It turns out the energy function has a very clear interpretation
in this setting, namely, that of defining the probability $\ip_i$
that the system finds itself in state $i \in \states$ as follows.
\begin{equation}
  \label{eq:energy}
  \ip_i = \frac{e^{-E(i)}}{\sum_{j \in \states} e^{-E(j)}}
\end{equation}
This is known as the \emph{Boltzmann distribution}.
Usually the energy is divided by $kT$ the product of
the Boltzmann constant $k$ and the temperature $T$.
However, we can omit this term by expressing the energy
in units of $1/(kT)$.
Note also that (i) \eqn{energy} defines the energy function
uniquely only up to an additive constant
given the probability distribution $\ip$,
that is, if we change the energy of each state by adding
a fixed constant we obtain the same probability distribution $\ip$;
and (ii) by convention the sign of the energy is inverted
so lower energies represent more favourable states.

The next question is how do we construct a CTMC
from an energy function.
What else do we need?
Clearly, we need to know the state space $\states$.
Also, unlike in classical mechanics,
we would need to know which transitions between states are possible
since there are no assumptions of continuity on $\states$.
The first formulation to shed light on this problem
was proposed by \citet{metropolis}.
The algorithm asks for an energy function and an \emph{a priori},
one-step transition probability matrix
that is assumed to be symmetric,
\ie that for any two states $i,j \in \states$
the elements $a_{ij}$ and $a_{ji}$ of the matrix are equal.
This matrix plays the role of the infinitesimal generator in the
discrete-time setting (\ie where time is indexed by the naturals)
and each element $a_{ij}$ denotes the probability that
we choose to jump to state $j$ when we are at state $i$.
Hence $\sum_{j \in \states} a_{ij} = 1$ for any fixed $i$
and we write $a_{i-}$ for this probability distribution.
The algorithm has been generalised to the case of asymmetric
a priori probability matrices by \citet{hastings}.

The construction gives a discrete-time Markov chain that
converges to the probability distribution $\ip$ in \eqn{energy}.
For the sake of simplicity we present here
only the original formulation.
The algorithm then works as follows.
Given any state $i \in \states$ we pick a neighbour state $j$
at random according to the probability distribution $a_{i-}$.
We evaluate the energy function at $i$ and $j$
to compute $\Delta E = E(j)-E(i)$ and proceed with the transition
with probability $1$ if $\Delta E < 0$ and
probability $e^{-\Delta E}$ if $\Delta E > 0$.
Otherwise we stay at state $i$.
In both cases time (a natural number) is increased by 1.

To see that $\ip$ as defined in \eqn{energy} is the invariant
probability distribution of the discrete-time Markov chain
we show that it has (the discrete-time version of)
detailed balance with respect to $\ip$.
The probability $p_{ij}$ of jumping from $i$ to $j$ is
a combination of the a priori probability and
the probability of accepting that transition,
which depends on $\Delta E$.
\[ p_{ij} = a_{ij}\; \min(1, e^{-\Delta E}) \]
By taking the ratio of $p_{ij}$ and $p_{ji}$ we have
\[ \frac{p_{ij}}{p_{ji}} =
   \frac{a_{ij}\; \min(1, e^{E(i)-E(j)})}{
         a_{ji}\; \min(1, e^{E(j)-E(i)})} =
   \frac{\min(1, e^{E(i)-E(j)})}{
         \min(1, e^{E(j)-E(i)})} \]
since $a_{ij} = a_{ji}$ by symmetry of the
a priori transition probability matrix.
Suppose $E(i)-E(j) > 0 > E(j)-E(i)$,
\[ \frac{p_{ij}}{p_{ji}} = e^{E(i)-E(j)}
     = \frac{e^{-E(j)}}{e^{-E(i)}} = \frac{\ip_{j}}{\ip_{i}} \]
It is easy to see that when
$E(j)-E(i) > 0 > E(i)-E(j)$ we obtain the same equation.
Hence the discrete-time Markov chain has detailed balance
with respect to $\ip$ as defined in \eqn{energy}.
Provided the a priori transition probability matrix
makes it possible to reach any state from any other state,
the Markov chain will converge to $\ip$ as $t \to \infty$.

The Metropolis-Hastings algorithm can be generalised
to the continuous-time case \citep{diaconis}.
However, the algorithm require us to either
(i) compute the energy of all states to obtain the probabilities
$p_{ij}$ (or transition rates $q_{ij}$ in the continuous-time case),
or (ii) do rejection sampling, as outlined above.
Option (i) can be very time-consuming when $\states$ is large
% or the evaluation of the energy function is expensive.
or it's costly to evaluate the energy function.
Option (ii) can be inefficient when the rejection rate is high.
For these reasons we explore an alternative method in this thesis.
We partition the state space in regions of equal energy
and group transitions according to these regions.
This is made possible by assuming extra structure on $\states$
(to be introduced in \sct{kappa}).

Let us go back to the stochastic modelling of
chemical interactions mentioned above.
The theory of CTMCs allows one to frame
the dynamics of chemical reaction systems.
However, since the number of molecules of a species
is a priori unbounded and thus $\states$ is infinite,
one would like to have a way to express these systems
in a finite and simple form.
A language that could do this
came to be in the work of \citet{petri}.
This language, later called \emph{Petri nets},
sees reactions as transformations of
multisets of chemical species.

\begin{definition}
  A \emph{multiset} $M$ over a set $X$ is a map from $X$ to
  the naturals assigning to each element $x \in X$
  the number of copies $M(x) \in \NN$ of that element
  in the multiset.
\end{definition}

There is a natural partial order $\leqslant$ on multisets over $X$.
We say $M \leqslant N$ if for each element $x \in X$,
$M(x) \leqslant N(x)$.
We write $\MM(X)$ for the set of all multisets over $X$.


\begin{definition}
  Given a set of species $\species$,
  a \emph{reaction} $r$ is a pair $\tuple{L,R}$
  with $L$ and $R$ multisets over $\species$.
  We refer to $L$ and $R$ as the left- and right-hand side of $r$.
  % and write $L \to R$ for the reaction.
\end{definition}

\begin{definition}%[PN]%[Petri net]
  A \emph{Petri net} is a pair $\tuple{\species, \reactions}$ with
  a set of species $\species$ and a set of reactions $\reactions$.
\end{definition}

A state of a Petri net is a multiset over $\species$,
usually called a \emph{marking}.
A reaction can occur in a given state $M$ only if
its left-hand side $L \leqslant M$.

\begin{definition}
  A \emph{match} $m$ of the left-hand side $L$ of a reaction
  on a state $M$ is an injective function from $L$ to $M$,
  \ie a map that identifies each copy of $a \in \species$ in $L$
  with a copy of $a$ in $M$.
\end{definition}

We write $\matches{L}{M}$ for the set of matches from $L$ to $M$.
From this definition we have that the number of matches
$\abs{\matches{L}{M}}$ from $L$ to $M$ is
\[ \abs{\matches{L}{M}} = \prod_{a \in \species} \binom{L(a)}{M(a)} \]
A reaction is said to be elementary iff its rate is
proportional to the number of matches of its left-hand side.
This is known as the \emph{law of mass action} in chemistry.
Here we consider only elementary reactions.

Petri nets can be given a stochastic interpretation
in terms of a CTMC.
Given a Petri net $\tuple{\species, \reactions}$,
an initial marking $M_0$ and
a kinetic map $k: \reactions \to \RR^+$ that assigns
kinetic rates $k(r)$ to reactions $r \in \reactions$,
we construct a CTMC $\tuple{\states, s(0), \qm}$ as follows.
\begin{align*}
  \states &{} = \MM(\species) \\
  s(0)(x) &{} = \begin{cases}
    1 \quad\text{if } x = M_0 \\
    0 \quad\text{if } x \neq M_0
  \end{cases} \\
  q_{MN} &{} = \sum_{\tuple{L,R} \in \reactions} n_{MN}(L,R)
\end{align*}
with
\begin{equation*}
  n_{MN}(L,R) = \left\{\begin{array}{ll}
    \abs{\matches{L}{M}} & \text{if } M - L + R = N \\
    0 & \text{otherwise}
  \end{array}\right.
\end{equation*}

The physical validity of this stochastic approach
and the physical conditions under which
it can be used has been argued by \citet{gillespie76}.
Interestingly, \citet{et2} have solved
the \emph{direct} and \emph{inverse} problem for Petri nets,
that is, the problem of constructing a Petri net
from an energy function on $\states$ and vice versa.

Petri nets have limitations when we take into consideration
what happens inside molecules in a chemical reaction.
The chemical transformation taking place amounts to
a change in the way electrons are shared by atoms
resulting in a relocation of chemical bonds.
In other words, (non-radioactive) reactions are all about
the binding and unbinding of atoms,
how they establish connections and break them.
This is poorly captured by a change of species,
as it is modelled by Petri nets.
A consequence of this lack of a formal representation for
molecular bonds is that certain systems of chemical reactions
cannot be described in a finite way using Petri nets,
\eg unbounded polymerisation
(think of a molecular chain that can always attach new links).

Recently,
a formal language to describe biochemical interactions
using rewriting rules,
where molecules not just react but also can bind other molecules
has been proposed by \citet{danoslaneve2002a}.
In the next section we introduce this language, called Kappa,
% and some of its properties,
keeping in mind that we want to address
the \emph{direct} and \emph{inverse} problem mentioned above,
namely, the problem of generating a set of rewriting rules
from an energy function and vice versa.


\section{Kappa}
\label{sec:kappa}

Kappa represents interactions among proteins,
nucleic acids and other biomolecules as
connections in a biomolecular network.
In these networks, nodes symbolise the biomolecules
while connections stand for transient molecular bonds
(\eg non-covalent interactions like hydrogen bonds).
This network is constantly changing as molecules
travel and interact with other molecules in a cell,
which is viewed as the constant destruction and creation
of the connections that make up the network.

Due to spatial constraints,
molecules can physically interact with
just so many other molecules at once.
Exactly how many will depend on multiple factors like
the size of the two interacting molecules and
the region where they come in contact.
These regions, known in molecular biology by the names of domains,
motifs or binding sites, are simply called \emph{sites} in Kappa.
Any such site can bind at most one other site at a time.
These sites belong to the nodes of the graph,
which Kappa calls \emph{agents}.
In the same way a molecule is of a certain species,
agents can be of different types.
These types also live in a network,
a static network which represents the ``network of possibilities''.
It tell us which sites a site \emph{can} bind
instead of what is actually bound to at a given moment.

To make these ideas formal we will use
the category-theoretical approach introduced in \citet{kappadpo}.
We will first introduce the networks for types
and build on them to construct the biomolecular network.%
\footnote{Below we use the words graph and edge
  as synonyms for network and connection.}

\begin{definition}%[site graph]
  A \emph{site graph} $G$ consists of
  a finite set of agents $\agents_G$,
  a finite set of sites $\sites_G$,
  a map $\sitemap_G: \sites_G \to \agents_G$
  that assigns sites to agents
  and a symmetric edge relation $\edges_G$ on $\sites_G$.
\end{definition}

The pair $\sites_G$, $\edges_G$ form an undirected graph.
Note that site graphs do not impose a bound on
the number of connections a site can have,
it just lists the possibilities.
Indeed there is no restriction at all so far.

Sites not in the domain of $\edges_G$ are said to be \emph{free}.
One says $G$ is \emph{realisable} iff
(i) no site has an edge to itself and
(ii) sites have at most one incident edge.
Each realisable site graph represents a state
in which our biomolecular network can be.
However, it contains no typing information.
We have to assign to each agent and site in the graph
an agent and site in the type graph.
More precisely, we need a map from a realisable site graph
to a site graph.
Below we introduce such maps.

\begin{flushleft}
\begin{minipage}{.71\linewidth}
\begin{definition}
  A \emph{homomorphism} $h: G \to G'$ of site graphs is
  a pair of functions, $h_\sites: \sites_G \to \sites_{G'}$
  and $h_\agents: \agents_G \to \agents_{G'}$, such that
  (i) $h_\agents(\sitemap_G(s)) = \sitemap_{G'}(h_\sites(s))$
  and (ii) if $s \mathbin{\edges_G} s'$ then
  $h_\sites(s) \mathbin{\edges_{G'}} h_\sites(s')$.
\end{definition}
\end{minipage}
\begin{minipage}{.28\linewidth}
\begin{flushright}
  \begin{tikzpicture}
    \matrix (m) [matrix of math nodes,row sep=30pt,column sep=30pt] {
      \sites_G & \sites_{G'} \\
      \agents_G & \agents_{G'} \\};
    \draw[hom] (m-1-1) -- node[above] {$h_\sites$} (m-1-2);
    \draw[hom] (m-2-1) -- node[below] {$h_\agents$} (m-2-2);
    \draw[hom] (m-1-1) -- node[left] {$\sitemap_G$} (m-2-1);
    \draw[hom] (m-1-2) -- node[right] {$\sitemap_{G'}$} (m-2-2);
  \end{tikzpicture}
\end{flushright}
\end{minipage}
\end{flushleft}

Put simply, homomorphisms preserve site ownership and connections.
The diagram to the right is the corresponding
commutative diagram in the category of sets and
is equivalent to condition (i) in the definition.
We call the typing map $h: G \to C$ a contact map over $C$
and refer to $C$ as the contact graph.
% TODO: say something more about typing

% TODO: explain
Site graphs and homomorphisms form a category $\SG$.

A homomorphism $h: G \to G'$ is an \emph{embedding} iff
(i) $h_\agents$ and $h_\sites$ are injective;
and (ii) if $s$ is free in $G$, so is $h_\sites(s)$ in $G'$.
Injectivity of $h_\agents$ and $h_\sites$ implies that
whenever $h: G \to G'$ is an embedding and $G'$ is realisable
then $G$ is also realisable.

\begin{wrapfigure}[4]{r}{0.28\textwidth}
  \vspace{-2.4em}
  \begin{center}
    \begin{tikzpicture}
      \matrix (m) [matrix of math nodes,row sep=20pt,column sep=20pt] {
        G & & G' \\
        & C & \\};
      \draw[hom] (m-1-1) -- node[above] {$\psi$} (m-1-3);
      \draw[hom] (m-1-1) -- node[below left] {$h$} (m-2-2);
      \draw[hom] (m-1-3) -- node[below right] {$h'$} (m-2-2);
    \end{tikzpicture}
  \end{center}
\end{wrapfigure}

An embedding $h: G \to G'$ between realisable site graphs
can be lifted to an embedding between contact maps $g: G \to C$
and $g': G' \to C$ iff the diagram on the left commutes
% in the category of site graphs and homomorphisms $\SG$.
in $\SG$.

% TODO: explain
Contact maps over $C$ and embeddings form a category $\rSGe_C$.













%%% Local Variables:
%%% mode: latex
%%% TeX-master: "thesis"
%%% End:



% \chapter{An energy function by design}
% \chapter{Imposing an energy function}
\if0
\chapter{Energy-driven rewriting} % by design}
\label{ch:energy}
\input{energy}

\chapter{Conclusions}
\label{ch:conclusions}
\input{conclusions}

%% Appendix
\appendix
\chapter{Model: Assembling triangles}
\label{app:triangles}
\include{model}
\fi

%% Choose your favourite bibliography style here.
% \bibliographystyle{apacite}

%% If you want the bibliography single-spaced (which is allowed), uncomment
%% the next line.
\singlespace

%% Specify the bibliography file. Default is thesis.bib.
% \bibliography{thesis}

\printbibliography

%% ... that's all, folks!
\end{document}

%%% Local Variables:
%%% mode: latex
%%% TeX-master: t
%%% End:
