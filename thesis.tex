\documentclass[phd,lfcs]{infthesis}
% \documentclass[phd,lfcs,twoside]{infthesis}
% \documentclass{report}

\usepackage[utf8]{inputenc}
\usepackage[T1]{fontenc}
\usepackage[british]{babel}
\usepackage{microtype}
\usepackage[usenames,dvipsnames,svgnames,table]{xcolor}
% \usepackage{natbib}
% \usepackage[natbibapa]{apacite}
\usepackage[english=british,autopunct=false]{csquotes}
% maxcitenames=2,uniquelist=false
\usepackage[natbib=true,style=authoryear-comp,maxbibnames=6]{biblatex}
\usepackage{graphicx}
\usepackage{textcomp}
\usepackage{wrapfig}
\usepackage{xfrac}
\usepackage{xspace}
% \usepackage{spacing}
% \usepackage{epigraph}
% \usepackage{gnuplot-lua-tikz}
\usepackage{mathcommon}
\usepackage{kappa}
\usepackage[sc]{mathpazo}
\usepackage{hyperref}
% \usepackage{blindtext}
\usepackage{expl3}
\usepackage{kappalst}
\usepackage{enumitem}

\frenchspacing

% Bibliography
\addbibresource{thesis.bib}

% Text
\newcommand{\ie}{i.e.\xspace}
\newcommand{\eg}{e.g.\xspace}

% Referencing
\newcommand{\chp}[1]{\S\ref{chp:#1}}
\newcommand{\sct}[1]{\S\ref{sec:#1}}
\newcommand{\subsct}[1]{\S\ref{subsec:#1}}
\newcommand{\eqn}[1]{Eq.~\ref{eq:#1}}
\newcommand{\eqns}[2]{Eq. \ref{eq:#1} and \ref{eq:#2}}
\newcommand{\lem}[1]{Lemma~\ref{lemma:#1}}
\newcommand{\lems}[2]{Lemmas \ref{lemma:#1} and \ref{lemma:#2}}
\newcommand{\thm}[1]{Th.~\ref{thm:#1}}
\newcommand{\fig}[1]{Fig.~\ref{fig:#1}}
\newcommand{\diagram}[1]{diagram~\ref{eq:#1}}
\newcommand{\app}[1]{Appendix~\ref{app:#1}}
\newcommand{\mcite}[1]{\textcolor{gray}{#1}} % missing cite
\newcommand{\defn}[1]{Def.~\ref{def:#1}}
\newcommand{\prop}[1]{Prop.~\ref{prop:#1}}

% Thermodynamics
\newcommand{\kB}{k_B}

% Thermodynamic graph rewriting
\newcommand{\rules}{\mathcal{R}}
\newcommand{\generators}{\mathcal{G}}
\newcommand{\shapes}{\mathcal{P}}
\newcommand{\cost}{\epsilon} % \varepsilon is the same with this font
% TODO: should I enclose the result of \refinedrules in parenthesis?
\newcommand{\refinedrules}{\generators_\shapes}
\newcommand{\gp}{\Gamma}

% Labelled transition systems
\newcommand{\LTS}{\mathcal{L}}
\newcommand{\labels}{\Lambda}

% Probability
\newcommand{\pmf}{%
  probability distribution\xspace} % probability mass function

% Markov chains
\newcommand{\states}{\mathcal{S}} % state space
\renewcommand{\state}{\xi}
\newcommand{\qm}{Q} % q matrix
\newcommand{\ip}{\pi} % invariant probability
\newcommand{\nitoj}{\gamma}
% \newcommand{\qmatrix}{infinitesimal generator\xspace}
\newcommand{\qmatrix}{$q$-matrix\xspace}
% \newcommand{\qmatrices}{infinitesimal generators\xspace}
\newcommand{\qmatrices}{$q$-matrices\xspace}

% Petri nets
\newcommand{\species}{\Sigma} %\mathcal{A}}
\newcommand{\reactions}{\mathfrak{R}} %\mathcal{R}}
\newcommand{\matches}[2]{\left[#1;#2\right]}
\newcommand{\MM}{\mathbf{M}}

% Cooperative assembly systems
\newcommand{\valence}{\nu} %\upsilon}
\newcommand{\types}{\mathcal{T}} % set of types
\newcommand{\tp}{t} % a type (variable)
\newcommand{\typeof}{\tau} % map from nodes to types
\newcommand{\degree}{d}

% FB-systems
\newcommand{\sitesof}{\sigma}
\newcommand{\emptyvec}{\varnothing}
\newcommand{\neighbours}{\vec{n}}
\newcommand{\sitestates}{\vec{s}}
\newcommand{\sitestatesof}{\iota}
\newcommand{\neighboursof}{\eta}
\newcommand{\sitestate}{\delta}

% Kappa
\newcommand{\edges}{\mathcal{E}}
\newcommand{\SG}{\mathbf{SG}}
\newcommand{\rSGe}{\mathbf{rSGe}}
\newcommand{\anon}[1]{\left|#1\right|}
\newcommand{\inv}[1]{#1^\dagger}
\newcommand{\comatch}[1]{#1^\star}

% Category theory
% \newcommand{\homset}{\Upsilon}
\newcommand{\mSet}{\mathbf{mSet}}
\newcommand{\Set}{\mathbf{Set}}
\newcommand{\iso}{\simeq}

% Triangles
% \newcommand{\cyc}[1]{{}\cdot #1 \cdot{}}

% Math
\renewcommand{\tuple}[1]{\left(#1\right)}
\DeclareMathOperator*{\expn}{exp}
\renewcommand*{\exp}[1]{e^{\,#1}} % \mathrm{e}^{#1}}
\renewcommand{\qedsymbol}{\ensuremath{\blacksquare}}
\newcommand{\partialto}{\rightharpoonup}
\newcommand{\id}{\vec{1}} % identity function

% Other stuff
\newcommand{\maybe}[1]{\textcolor{gray}{#1}}
% \renewcommand{\maybe}[1]{}
\newcommand{\todo}[1]{\textcolor{red}{TODO: #1}}
% \renewcommand{\todo}[1]{}

% Styles for nodes
\tikzstyle{n}=[circle,draw=Black,fill=White] %,thick,minimum size=6pt,inner sep=0pt]
\tikzstyle{n1}=[n,fill=Orange]
\tikzstyle{n2}=[n,fill=Blue!60!White]
\tikzstyle{n3}=[n,fill=Green!40!White] % Green!40!Ocre?
\tikzstyle{n4}=[n,fill=White!95!Black]
\tikzstyle{n5}=[n,fill=Black]
\tikzstyle{n6}=[n,fill=Yellow!50!White]
\tikzstyle{p}=[n,draw=none] % phantom
\tikzstyle{e}=[thick]
\tikzstyle{rule}=[thick,->,>=angle 90]
\tikzstyle{onarrow}=[fill=White,inner sep=2pt]
\tikzstyle{site}=[n,fill=Yellow!50!White,inner sep=1.5pt]

% Binding types
\tikzstyle{bt1}=[n1,dashed,inner sep=3.5pt]
\tikzstyle{bt2}=[n2,dashed,inner sep=3.5pt]
\tikzstyle{bt3}=[n3,dashed,inner sep=3.5pt]

% Background box for graph diagrams
\tikzstyle{grphdiag-bg}=[rounded corners,fill=White!85!Black]

% Graph diagrams
\tikzstyle{grphnode}=[rectangle,grphdiag-bg]
% \tikzstyle{ingrphdiag}=[>=stealth,thick,semithick,
%   n/.append style={semithick,minimum size=4pt}]
\tikzstyle{ingrphdiag}=[thick,anchor=center]

% Use different arrow tips to distinguish edges from morphism
% and a light gray backdrop.
\usetikzlibrary{backgrounds}
\tikzstyle{grphdiag}=[>=stealth,framed,thick,%
  background rectangle/.style=grphdiag-bg]

% Draw a node
\newcommand{\n}[3][n]{\node[inner sep=5pt,#1] (#2) at (#3) {}}
\newcommand{\nn}[4][n]{\node[inner sep=3.5pt,#1] (#2) at (#3) {#4}}
% \renewcommand{\site}[3][n7]{\node[inner sep=1.5pt,#1] (#2) at (#3) {}}

% Draw an edge
\newcommand{\e}[3][e]{\draw[#1] (#2) -- (#3)}

% Draw a directed edge
\newcommand{\de}[3][e]{\draw (#2) edge[->,#1] (#3)}

% Draw a bi-directional rule
\newcommand{\birulearrow}[2]{\hspace{1ex}\tikz{%
  \draw[rule] (0, .15) -- node[above] {#1} ++( .5, 0);
  \draw[rule] (.5,  0) -- node[below] {#2} ++(-.5, 0);
}\hspace{1ex}}

% Arrows in minimal gluings diagram
\newcommand{\arrsn}[3][]{\arr[#1]{#2.south}{#3.north}}
\newcommand{\arr}[3][]{%
  \draw[-bigto,#1] ($(#2)!.1!(#3)$) -- ($(#2)!.9!(#3)$);}

% Header
\title{On a thermodynamic approach to
  biomolecular interaction networks}
\author{Ricardo Honorato-Zimmer}

% Abstract
\abstract{
  We explore the direct and inverse problem of thermodynamics
  in the context of rule-based modelling.
  The direct problem can be concisely stated as
  obtaining a set of rewriting rules and their rates
  from the description of the energy landscape
  such that their asymptotic behaviour when $t \to \infty$ coincide.
  To tackle this problem,
  we describe an energy function as
  a finite set of connected patterns $\shapes$
  and an energy cost function $\cost$
  which associates real values to each of these energy patterns.
  We use a finite set of reversible
  graph rewriting rules $\generators$
  to define the qualitative dynamics
  by showing which transformations are possible.
  Given $\generators$ and $\shapes$,
  we construct a finite set of rules $\generators_{\shapes}$ which
  i) has the same qualitative transition system as $\generators$ and
  ii) when equipped with rates according to $\cost$,
  defines a continuous-time Markov chain
  that has detailed balance with respect to
  the invariant \pmf determined by the energy function.
  The construction relies on a technique for rule refinement
  described in earlier work
  and allows us to represent
  thermodynamically consistent models
  of biochemical interaction networks
  in a concise manner.

  The inverse problem, on the other hand,
  is to i) check whether a rule-based model
  has an energy function that describes its asymptotic behaviour
  and if so ii) obtain the energy function
  from the graph rewriting rules and their rates.
  Although this problem is known to be undecidable
  in the general case,
  we find two suitable subsets of Kappa,
  our rule-based modelling framework of choice,
  were this question can be answer positively
  and the form of their energy functions described analytically.
  \if0
  We develop a new thermodynamic approach to stochastic graph rewriting.
  The ingredients are a finite set of reversible
  graph rewriting rules $\generators$ (called generating rules),
  a finite set of connected graphs $\shapes$ (called energy patterns),
  and an energy cost function $\cost$
  which associates real values to each of these energy patterns.
  % 
  The idea is that $\generators$ defines the qualitative dynamics,
  by showing which transformations are possible,
  while $\shapes$ and $\cost$ allow one to attach an energy
  to the reachable graphs and, thereby,
  describe their long-term probability distribution $\ip$.
  % 
  Given $\generators$ and $\shapes$,
  we construct a finite set of rules $\generators_{\shapes}$ which
  (i) has the same qualitative transition system as $\generators$; and
  (ii) when equipped with rates according to $\cost$,
  defines a continuous-time Markov chain of which $\ip$ is
  the \emph{stationary} and \emph{limiting} probability distribution.
  % 
  The construction relies on the use of site graphs and
  a technique of `growth policy' for quantitative rule refinement.
  % which is of independent interest.
  % 
  % Nothing else is assumed of $\generators$ or $\shapes$,
  % and only the rates on the generated rule set $\generators_{\shapes}$
  % depend on $\cost$.

  This division of labour between the qualitative and long-term
  quantitative aspects of the dynamics leads to intuitive and concise
  descriptions for realistic models.
  % 
  It also guarantees thermodynamical consistency
  (\emph{aka} detailed balance),
  otherwise known to be undecidable.
  % which is important for some applications.
  % 
  Finally, it leads to parsimonious parameterisations of models,
  % again an important point in some applications.
  an important point in the application of stochastic graph rewriting
  to the modelling of biochemical interaction networks.
  \fi
}

\begin{document}

%% First, the preliminary pages
\begin{preliminary}

\maketitle

\hyphenation{Guillaume}
\hyphenation{Milana}

\begin{acknowledgements}
  \vspace{.1cm}
  I would like to thank everyone that has helped me along the way.
  It has been a marvelous journey with all its highs and lows
  and has involved much learning and growth
  in my scientific as well as personal life.
  % This experience has made me feel alive
  % and transformed me in so many ways.
  This experience has transformed me in many ways.

  I am eternally grateful to my supervisor Vincent Danos.
  % I thank him for taking me as a PhD student,
  He took me as a PhD student,
  forming me as a scientist and as a better person.
  I feel an immense sense of gratitude
  % It is difficult to condese in a few sentences the gratitude I feel
  when I think of the many things
  he patiently taught me during all these years.
  It is not an exaggeration to say that he taught me how to think.
  I am lucky to have had a genius supervisor.
  % I would like to thank him particularly for having faith in me
  % when things did not look good.
  % even when things did not look good.

  % I would also like to thank my parents,
  Many thanks go to my parents,
  Renate Zimmer and Ricardo Honorato,
  for having given me the life and energy,
  the teachings and nourishment,
  the good ideas, beliefs and love that
  prepared me to succeed in this adventure.
  % Let me state here the obvious and fundamental truth:
  Without them none of this would have been possible.
  Importantly, I would like to thank as well
  my sister Daniela and brother Francisco,
  who, together with my parents,
  shaped my thinking since an early age,
  % encouraged me in this endeavour,
  encouraged me in my scientific career,
  and from whom I have learned to question
  % otherwise unnoticed assumptions.
  my own beliefs and understanding.
  I have still much to learn from them.

  My deepest feelings of gratitude go % as well
  to my friends Philipp Thomas, Sandro Stucki,
  Sebastian Jaramillo, Ilias Garnier and Tobias Heindel
  for sharing their time and knowledge
  with me since I met them. % over the last few years.
  They have also contributed enormously
  to develop the precision and clarity of mind
  that was necessary
  in a multitude of topics of relevance to this thesis.

  I would like to thank my friends
  An\'ibal Ravani, Gr\'ainne Foster,
  Sophie Currie, Stuart Burrows, Aileen Dickson, Susan Robinson,
  Alejandro Granados, Sebastian, Benura Azeroglu,
  Philipp, Claudia Cianci, Sandro,
  Nicolas Behr, Laura Kreiling,
  Patric Fulop, Andreea Beica,
  Ilias, Huisong Li and Guillaume Terradot
  for the various forms of unconditional support,
  companionship and encouragement
  that kept me positive and made possible
  the realisation of this manuscript.
  % Thank you so much guys for all the love
  % and good vibes in the good and bad moments of this journey.
  % Without their reassuring empathy and kind hospitality
  % I would not have made it.
  Their reassuring empathy
  in the good and bad moments of this journey
  and kind hospitality
  have been the emotional fuel of the writing process.

  \pagebreak

  I feel a great sense of gratitude towards
  Mark Miller for the tremendous gift
  that his teachings and loving presence
  have been. % in my life.
  He has changed the direction of my life
  and the impact that this change has had
  on the making of the final version of this manuscript
  cannot be overstated.
  %
  I am also grateful to Alfred Hofmann and Alexander Shulgin
  for their wonderful inventions that provided me
  with the beautiful insights, % during this time
  at times clearly perceived
  and other times just subconciously at work,
  that started this transition in a subtle
  and sometimes not so subtle way.

  This list of acknowledgements would not be complete
  without a special mention to all the people in the group,
  % Many thanks are due to all the people in the group,
  namely and in no particular order,
  Milana Filatenkova, Andrea Weise,
  William Waites, Guoli Yang, Matteo Cavaliere,
  Fredrik Dahlqvist, Matthias Sachs,
  Katharina Heil, Emilia Wysocka,
  David Sterratt, Oksana Sorokina, John Wilson-Kanamori,
  Hristiana Pashkuleva, Argyris Zardilis, Simon Lyons, Gaëlle Candel,
  Nick, Patric, Andreea, Guillaume, Sandro, Ilias, Tobias and Vincent,
  for the stimulating conversations, interesting ideas
  and sincere desire to understand
  that created the excellent environment
  for which our meetings are famous.
  %
  I would also like to thank
  Russ Harmer, Jean Krivine and J\'er\^ome Feret
  for the camaraderie, good feedback and helpful advice.
  % regarding my PhD and academic career.

  I am grateful to % would like to thank as well
  everyone that helped me get to the PhD.
  In particular, special thanks go to Tomas Perez-Acle
  and Juan Carlos Letelier. % who helped me when.../in this way...
  %
  Last and perhaps most importantly
  to the present version of this manuscript,
  I would like to thank Paweł Sobociński and Ilias
  whose invaluable feedback and dedicated reading
  of the first version of this manuscript catalysed
  the improvements that gave life to this final version.

  \if0
  Vincent, I'd like to thank you first for having me in your group.
  For having patiently taught me so many things
  from math to good life lessons.
  Thank you for blowing my mind so many times
  with your brilliant ideas and ways to look at things.
  Thank you for sharing your time with me
  and having faith in me even when things didn't look good.

  Renate y Ricardo, muchas gracias por haberme dado la vida,
  por haberme criado, por haberme enseñado y nutrido
  de tan buenas ideas, creencias y alimentos.
  Muchas gracias por haberse preocupado por mi en todo momento,
  tanto por mi salud, como bienestar y desarrollo.
  De m\'as est\'a decir, sin uds nada de esto habr\'ia sido posible.
  Muchas gracias por ser los maravillosos padres que son.

  Dani, muchas gracias por ser la maravillosa hermana que eres.
  Como poner en palabras todas las cosas que hemos compartido
  y todas las cosas que me has enseñado en toda esta vida juntos,
  que me han formado y me han hecho en lo que soy.
  Muchas gracias por tu eterna y sincera preocupacion por mi.
  Te llevo siempre en mi corazon.

  Froko, muchas gracias mi compañero de la vida
  por todo el tiempo que hemos compartido,
  por toda la invaluable compañia,
  por todo el amor en los momentos dificiles
  y por todas las cosas que gracias a ti he logrado comprender.
  No hay suficientes palabras para decirte todo lo que te aprecio.

  An\'ibal and Gr\'ainne, thank you so much guys for all the love
  and good vibes in the good and bad moments of this journey.
  It was amazing to share this time together
  and having the comfort of your words and hugs with me.
  Without your reassuring empathy I wouldn't have made it.

  Mark, thank you so much
  % for teaching me the good lessons and what life ultimately is all about.
  for showing me the tools and techniques to open the heart
  to the expected as well as the unexpected
  and naturally find peace in doing so.
  I could see all along how you embody this compassionate way of being
  and that gave me the strength to start softening my own heart.
  % for giving me tools that have given me some much needed space
  % and freedom.
  % Listening to you was challenging at the beginning
  % but I could see all the time how you embody this compassionate way
  % of being.
  % but slowly started changing my life.

  Sophie, Stuart, Ayleen and Susan, thank you so much guys for
  being the great friends and flatmates that you were.
  Thank you for the countless laughs and the support and love.
  I'm so happy to have you guys in my life.

  Philipp, thank you so much for the wonderful time together.
  and an important minus sign in \prop{cas}.

  Sandro, thank you for patiently explaining me so many important things
  and thank you for sharing your bright ideas with me.
  Thank you so much for all the honest love.

  Tobias, thank you so much for being so nice, soft and caring to me. % I needed it.
  Thank you so much also for showing me some of your math wizardry,
  I really appreaciate it.

  William, thank you for the wonderful and insightful conversations. % sharing your knowledge and opinions
  I really like having met you, having you in the group
  and having you as one of my friends.

  Illias, your cool style and humility are true sources of inspiration.

  Thank you very much to all the other people in the group
  for the stimulating conversation and companionship,
  Milana, Andrea, Guoli, Matteo, Guillaume, Fredrik, Matthias, Nick,
  Katharina, Emilia, David, Oksana, John, Hristiana,
  Argyris, Andreea and Simon.
  I would like to thank Russ, Jean and J\'er\^ome
  for the camaraderie, the good feedback and
  the advice regarding my PhD and academic career.
  Thanks also to Dr Alfred Hofmann and Dr Alexander Shulgin
  for their life-changing inventions that have provided me
  with beautiful insights during this time.
  I'd like to thank as well everyone that helped me get to the PhD,
  in particular special thanks go to Dr Tomas Perez-Acle
  and Dr Juan Carlos Letelier. % who helped me when.../in this way...
  Last and perhaps most importantly to the present version
  of this manuscript, I'd like to thank Paweł Sobociński,
  whose invaluable feedback and dedicated reading of the first
  version of this manuscript catalysed % have been crucial and
  the improvement that gave life to this final version.
  \fi
\end{acknowledgements}

%% Next we need to have the declaration.
\standarddeclaration

%% Finally, a dedication (this is optional --
%% uncomment the following line if you want one).
% \dedication{To my mummy.}

%% Create the table of contents
\tableofcontents

%% If you want a list of figures or tables,
%% uncomment the appropriate line(s)
% \listoffigures
% \listoftables

\end{preliminary}

\chapter{Introduction}
\label{chp:intro}
% % this needs package epigraph
% \setlength{\epigraphwidth}{.8\textwidth}
% \setlength{\epigraphrule}{0pt}
% \epigraph{
%   I regard as quite useless the reading of
%   large treatises of pure analysis:
%   too large a number of methods pass at once before the eyes.
%   It is in the works of applications that one must study them;
%   one judges their ability there and
%   one apprises the manner of making use of them.}{
%   --- Joseph Louis Lagrange}

% \begin{quotation}
%   \textit{
%     ``I regard as quite useless the reading of
%     large treatises of pure analysis:
%     too large a number of methods pass at once before the eyes.
%     It is in the works of applications that one must study them;
%     one judges their ability there and
%     one apprises the manner of making use of them.''} \\
%   \par\raggedleft--- Joseph Louis Lagrange
% \end{quotation}

% this needs package spacing
% from https://hbfs.wordpress.com/2011/01/18/epigraphs-in-latex/
% epigraph with 3 params: width, text, author
\newcommand{\epigraph}[3]{
\vspace{1em}\hfill{}\begin{minipage}{#1}{\begin{spacing}{0.9}
\small\noindent\textit{#2}\end{spacing}
\vspace{1em}
\hfill{}{#3}}\vspace{2em}
\end{minipage}}
\epigraph{.8\textwidth}{
  I regard as quite useless the reading of
  large treatises of pure analysis:
  too large a number of methods pass at once before the eyes.
  It is in the works of applications that one must study them;
  one judges their ability there and
  one apprises the manner of making use of them.}{
  --- Joseph Louis Lagrange}


%%% Local Variables:
%%% mode: latex
%%% TeX-master: "thesis"
%%% End:


% \noindent
\section{Historical background}

In the history of natural sciences,
there has been two main approaches to describe dynamical systems,
which I call here
\emph{kinetics} and \emph{thermodynamics}.
Loosely speaking, in the kinetic approach the system is
described by the positions and momenta of each particle.
This approach goes all the way back to
Newton's laws of motion~\citep{newton}.
Intuitively, it is a ``ground'' description in the sense that
it is as explicit and detailed as possible.
If a (classical) mechanical system has $N$ particles
then a state of the system is described by
a vector in $\RR^{2 \cdot 3 \cdot N}$.
% Regarding the role of forces in Newtonian mechanics:
% Interactions producing a change of momentum on a particle
% that can be measured independently do not interact between them
% and thus the resulting derivative of the momentum (force)
% is just the sum of the forces as measured independently.

On the other hand,
the thermodynamic approach shows how
all information about the change of the system in time
is contained in the energy function (for conservative systems).
Given a set of states $\states$ for the system,
an energy function $E: \states \to \RR$ maps a state to its energy.
In this way a state is described by a single scalar
regardless of how many particles it comprises.
Naturally, this approach endowed the description of
a dynamical system in classical mechanics
with a remarkable conciseness, simplicity and elegance.
It first appeared in the work of
\citet{lagrange2} and \citet{hamilton},
and has been subsequently used as the basis for most of modern physics.
Once in possession of the energy function,
the kinetic description (\ie the equations of motion)
can then be derived from it.
However the converse is not true:
in general a kinetic description might not have an energy function
from which it can be derived \citep{santilli},
partly because of non-conservative (\eg dissipative) forces.
Obtaining an energy function from the equations of motion
is called the \emph{inverse problem} in classical mechanics
and it was first attended to by \citet{helmholtz}.
Both the direct and the inverse problem are the interest of this thesis.
Note that this approach has been given the name `thermodynamic'
not because of thermodynamics,
the science that studies the dynamics of heat and temperature,
but because of the protagonical role of the energy
in driving the system's evolution.
Certainly, there are connections to thermodynamics
that will be highlighted as they arise.

Half a century after Hamilton's work
researchers like Maxwell, Boltzmann, and Gibbs
applied the ideas of classical mechanics to \emph{atoms}
in order to describe physical properties of matter like pressure,
the capacity to transfer heat, and others.
This body of work came to be known as \emph{statistical mechanics}
and was used to explain Brownian motion by \citet{einstein-brownian},
which after its experimental verification \citep{perrin}
settled the debate about the existence of atoms.
This work however did not attempt to explain
the chemical interactions and reactions that molecules undergo.
That would have to wait yet half a century
for the axiomatisation of probability theory by \citet{kolmogorov}
and the further developments by \citet{doob} and \citet{feller},
who, among others, established the theoretical framework
for continuous-time Markov chains (CTMCs).
Below you can find the definitions for a CTMC
and its infinitesimal generator that will be used in this thesis.
In particular, we work with time-homogeneous CTMCs.

\begin{definition}%[Infinitesimal generator]
  An \emph{infinitesimal generator} $\qm$
  on an at most countable set of states $\states$
  is an $\states \times \states$ matrix
  with elements $q_{ij} \in \RR$, $i,j \in \states$
  such that $0 \leqslant q_{ij} < \infty$ when $i \neq j$
  and $q_{ii} = - \sum_{j \neq i} q_{ij} < \infty$.
\end{definition}

The infinitesimal generator plays the role of
the time derivative of the transition probabilities at time $0$
and induces the evolution of a probabilistic state
according to the Kolmogorov backward equation,
\begin{equation}
  \label{eq:transition-function}
  \ddt P(t) = Q P(t), \quad P(0) = I
\end{equation}
where $P(t)$ is the $\states \times \states$ matrix
with elements $p_{ij}(t) \in \RR$ the probability that
we were in state $i$ at time $0$ and are in state $j$ at time $t$.
% transition probability from state $i$ to $j$ at time $t$.
When the infinitesimal generator is stable and conservative
there is a unique minimal solution to \eqn{transition-function}
\citep{anderson}.
We shall work with this type of infinitesimal generators
and assume there is a transition function $P(t)$
whenever we have an infinitesimal generator $\qm$ and vice versa.

Given a \pmf $s(0)$ on $\states$ (seen as a row vector)
as an initial probabilistic state,
the probability distribution $s(t)$ after time $t$
is given by $s(t) = s(0) P(t)$.
% We can obtain the time derivative of this distribution
% from \eqn{transition-function}.
% % $s_i(t)$ that the Markov chain is in state $i$ at time $t$.
% In coordinate form, we have
% \begin{equation} % TODO: is this equation correct?
%   \label{eq:prob-deriv}
%   \ddt s_i(t) = \sum_{j \in \states} q_{ji} s_j(t)
% \end{equation}
We say the infinitesimal generator is \emph{irreducible}
if every state is reachable regardless of the initial state,
\ie $p_{ij}(t) > 0$ for all $i,j \in \states$
and some $t \geqslant 0$.

\begin{definition}[CTMC]%[continuous-time Markov chain]
  A \emph{continuous-time Markov chain} is a tuple
  $\tuple{\states, s(0), \qm}$ with
  $\states$ an at most countable set of states,
  $s(0)$ a \pmf on $\states$
  representing the initial probabilistic state and
  $\qm$ the infinitesimal generator of the Markov chain.
  % The Markov chain can be presented as a time-indexed family
  % of random variables $X_t$.
\end{definition}

% TODO: Am I here introducing something after using/mentioning it?
An important property of CTMCs for the present work is that of
\emph{time reversibility}, also known as \emph{detailed balance},
which we introduce below.

\begin{definition}[detailed balance]
  An infinitesimal generator $\qm$ on $\states$
  is said to be \emph{time reversible} iff
  there is a \pmf $\ip$ on $\states$ such that
  \begin{equation}
    \label{eq:detailed-balance}
    \ip_i q_{ij} = \ip_j q_{ji}
  \end{equation}
  for all $i,j \in \states$.
  Then $\qm$ is said to have \emph{detailed balance}
  with respect to $\ip$.
\end{definition}

And the related property of an
\emph{invariant} probability measure for the infinitesimal generator.

\begin{definition}
  A \pmf $\ip$ on $\states$ is
  \emph{invariant} for an infinitesimal generator $\qm$
  iff $\ip \qm = 0$, \ie
  \[ -\ip_i q_{ii} = \ip_i \sum_{j \neq i} q_{ij}
                  = \sum_{j \neq i} \ip_j q_{ji} \]
  In other words,
  whenever the action of $\qm$ on it does not change it.
\end{definition}

The relationship between these two properties is established
by the following lemma.

\begin{lemma}
  Suppose the infinitesimal generator $\qm$
  has detailed balance with respect to $\ip$.
  Then $\ip$ is invariant for $\qm$.
\end{lemma}
\begin{proof}
  From \eqn{detailed-balance} we obtain
  \[ \sum_{i \in \states} \ip_i q_{ij} =
     \sum_{i \in \states} \ip_j q_{ji} = -\ip_j q_{jj}, \]
  as $\sum_{i \in \states} q_{ji} = -q_{jj}$ for any fixed state $j$.
\end{proof}

Moreover, we would like to know when this invariant
probability measure is realised by the Markov chain.

\begin{definition}[ergodicity]
  An infinitesimal generator $\qm$ is \emph{ergodic} when
  there is a probability measure $\ip$ on $\states$ such that
  \[ \lim_{t \to \infty} P_{ij}(t) = \ip_j \]
  for all $i,j \in \states$.
\end{definition}

This is equivalent to say that the Markov chain
will converge to the probability measure $\ip$
regardless of the initial state $s(0)$.

\begin{lemma}
  Suppose the infinitesimal generator $\qm$ is irreducible
  and has an invariant probability measure $\ip$.
  Then $\qm$ is ergodic and converges to $\ip$.
\end{lemma}

The proof for this lemma can be found in part 2 of theorem 1.6
in chapter 5 of Anderson's book (\cite*[][pages 160--161]{anderson}).
CTMCs have a strong kinetic flavour as they describe
stochastic processes in terms of probability flows
happening at a certain rate.
They are the ``ground'' description in the stochastic world
and all approaches to describe these processes
are interpreted in terms of them.

It is natural to wonder then how the thermodynamic approach
looks like in the stochastic world.
It turns out the energy function has a very clear interpretation
in this setting, namely, that of defining the probability $\ip_i$
that the system finds itself in state $i \in \states$ as follows.
\begin{equation}
  \label{eq:energy}
  \ip_i = \frac{e^{-E(i)}}{\sum_{j \in \states} e^{-E(j)}}
\end{equation}
This is known as the \emph{Boltzmann distribution}.
Usually the energy is divided by $kT$ the product of
the Boltzmann constant $k$ and the temperature $T$.
However, we can omit this term by expressing the energy
in units of $1/(kT)$.
Note also that (i) \eqn{energy} defines the energy function
uniquely only up to an additive constant
given the probability distribution $\ip$,
that is, if we change the energy of each state by adding
a fixed constant we obtain the same probability distribution $\ip$;
and (ii) by convention the sign of the energy is inverted
so lower energies represent more favourable states.

The next question is how do we construct a CTMC
from an energy function.
What else do we need?
Clearly, we need to know the state space $\states$.
Also, unlike in classical mechanics,
we would need to know which transitions between states are possible
since there are no assumptions of continuity on $\states$.
The first formulation to shed light on this problem
was proposed by \citet{metropolis}.
The algorithm asks for an energy function and an \emph{a priori},
one-step transition probability matrix
that is assumed to be symmetric,
\ie that for any two states $i,j \in \states$
the elements $a_{ij}$ and $a_{ji}$ of the matrix are equal.
This matrix plays the role of the infinitesimal generator in the
discrete-time setting (\ie where time is indexed by the naturals)
and each element $a_{ij}$ denotes the probability that
we choose to jump to state $j$ when we are at state $i$.
Hence $\sum_{j \in \states} a_{ij} = 1$ for any fixed $i$
and we write $a_{i-}$ for this probability distribution.
The algorithm has been generalised to the case of asymmetric
a priori probability matrices by \citet{hastings}.

The construction gives a discrete-time Markov chain that
converges to the probability distribution $\ip$ in \eqn{energy}.
For the sake of simplicity we present here
only the original formulation.
The algorithm then works as follows.
Given any state $i \in \states$ we pick a neighbour state $j$
at random according to the probability distribution $a_{i-}$.
We evaluate the energy function at $i$ and $j$
to compute $\Delta E = E(j)-E(i)$ and proceed with the transition
with probability $1$ if $\Delta E < 0$ and
probability $e^{-\Delta E}$ if $\Delta E > 0$.
Otherwise we stay at state $i$.
In both cases time (a natural number) is increased by 1.

To see that $\ip$ as defined in \eqn{energy} is the invariant
probability distribution of the discrete-time Markov chain
we show that it has (the discrete-time version of)
detailed balance with respect to $\ip$.
The probability $p_{ij}$ of jumping from $i$ to $j$ is
a combination of the a priori probability and
the probability of accepting that transition,
which depends on $\Delta E$.
\[ p_{ij} = a_{ij}\; \min(1, e^{-\Delta E}) \]
By taking the ratio of $p_{ij}$ and $p_{ji}$ we have
\[ \frac{p_{ij}}{p_{ji}} =
   \frac{a_{ij}\; \min(1, e^{E(i)-E(j)})}{
         a_{ji}\; \min(1, e^{E(j)-E(i)})} =
   \frac{\min(1, e^{E(i)-E(j)})}{
         \min(1, e^{E(j)-E(i)})} \]
since $a_{ij} = a_{ji}$ by symmetry of the
a priori transition probability matrix.
Suppose $E(i)-E(j) > 0 > E(j)-E(i)$,
\[ \frac{p_{ij}}{p_{ji}} = e^{E(i)-E(j)}
     = \frac{e^{-E(j)}}{e^{-E(i)}} = \frac{\ip_{j}}{\ip_{i}} \]
It is easy to see that when
$E(j)-E(i) > 0 > E(i)-E(j)$ we obtain the same equation.
Hence the discrete-time Markov chain has detailed balance
with respect to $\ip$ as defined in \eqn{energy}.
Provided the a priori transition probability matrix
makes it possible to reach any state from any other state,
the Markov chain will converge to $\ip$ as $t \to \infty$.

The Metropolis-Hastings algorithm can be generalised
to the continuous-time case \citep{diaconis}.
However, the algorithm require us to either
(i) compute the energy of all states to obtain the probabilities
$p_{ij}$ (or transition rates $q_{ij}$ in the continuous-time case),
or (ii) do rejection sampling, as outlined above.
Option (i) can be very time-consuming when $\states$ is large
% or the evaluation of the energy function is expensive.
or it's costly to evaluate the energy function.
Option (ii) can be inefficient when the rejection rate is high.
For these reasons we explore an alternative method in this thesis.
We partition the state space in regions of equal energy
and group transitions according to these regions.
This is made possible by assuming extra structure on $\states$
(to be introduced in \sct{kappa}).

Let us go back to the stochastic modelling of
chemical interactions mentioned above.
The theory of CTMCs allows one to frame
the dynamics of chemical reaction systems.
However, since the number of molecules of a species
is a priori unbounded and thus $\states$ is infinite,
one would like to have a way to express these systems
in a finite and simple form.
A language that could do this
came to be in the work of \citet{petri}.
This language, later called \emph{Petri nets},
sees reactions as transformations of
multisets of chemical species.

\begin{definition}
  A \emph{multiset} $M$ over a set $X$ is a map from $X$ to
  the naturals assigning to each element $x \in X$
  the number of copies $M(x) \in \NN$ of that element
  in the multiset.
\end{definition}

There is a natural partial order $\leqslant$ on multisets over $X$.
We say $M \leqslant N$ if for each element $x \in X$,
$M(x) \leqslant N(x)$.
We write $\MM(X)$ for the set of all multisets over $X$.


\begin{definition}
  Given a set of species $\species$,
  a \emph{reaction} $r$ is a pair $\tuple{L,R}$
  with $L$ and $R$ multisets over $\species$.
  We refer to $L$ and $R$ as the left- and right-hand side of $r$.
  % and write $L \to R$ for the reaction.
\end{definition}

\begin{definition}%[PN]%[Petri net]
  A \emph{Petri net} is a pair $\tuple{\species, \reactions}$ with
  a set of species $\species$ and a set of reactions $\reactions$.
\end{definition}

A state of a Petri net is a multiset over $\species$,
usually called a \emph{marking}.
A reaction can occur in a given state $M$ only if
its left-hand side $L \leqslant M$.

\begin{definition}
  A \emph{match} $m$ of the left-hand side $L$ of a reaction
  on a state $M$ is an injective function from $L$ to $M$,
  \ie a map that identifies each copy of $a \in \species$ in $L$
  with a copy of $a$ in $M$.
\end{definition}

We write $\matches{L}{M}$ for the set of matches from $L$ to $M$.
From this definition we have that the number of matches
$\abs{\matches{L}{M}}$ from $L$ to $M$ is
\[ \abs{\matches{L}{M}} = \prod_{a \in \species} \binom{L(a)}{M(a)} \]
A reaction is said to be elementary iff its rate is
proportional to the number of matches of its left-hand side.
This is known as the \emph{law of mass action} in chemistry.
Here we consider only elementary reactions.

Petri nets can be given a stochastic interpretation
in terms of a CTMC.
Given a Petri net $\tuple{\species, \reactions}$,
an initial marking $M_0$ and
a kinetic map $k: \reactions \to \RR^+$ that assigns
kinetic rates $k(r)$ to reactions $r \in \reactions$,
we construct a CTMC $\tuple{\states, s(0), \qm}$ as follows.
\begin{align*}
  \states &{} = \MM(\species) \\
  s(0)(x) &{} = \begin{cases}
    1 \quad\text{if } x = M_0 \\
    0 \quad\text{if } x \neq M_0
  \end{cases} \\
  q_{MN} &{} = \sum_{\tuple{L,R} \in \reactions} n_{MN}(L,R)
\end{align*}
with
\begin{equation*}
  n_{MN}(L,R) = \left\{\begin{array}{ll}
    \abs{\matches{L}{M}} & \text{if } M - L + R = N \\
    0 & \text{otherwise}
  \end{array}\right.
\end{equation*}

The physical validity of this stochastic approach
and the physical conditions under which
it can be used has been argued by \citet{gillespie76}.
Interestingly, \citet{et2} have solved
the \emph{direct} and \emph{inverse} problem for Petri nets,
that is, the problem of constructing a Petri net
from an energy function on $\states$ and vice versa.

Petri nets have limitations when we take into consideration
what happens inside molecules in a chemical reaction.
The chemical transformation taking place amounts to
a change in the way electrons are shared by atoms
resulting in a relocation of chemical bonds.
In other words, (non-radioactive) reactions are all about
the binding and unbinding of atoms,
how they establish connections and break them.
This is poorly captured by a change of species,
as it is modelled by Petri nets.
A consequence of this lack of a formal representation for
molecular bonds is that certain systems of chemical reactions
cannot be described in a finite way using Petri nets,
\eg unbounded polymerisation
(think of a molecular chain that can always attach new links).

Recently,
a formal language to describe biochemical interactions
using rewriting rules,
where molecules not just react but also can bind other molecules
has been proposed by \citet{danoslaneve2002a}.
In the next section we introduce this language, called Kappa,
% and some of its properties,
keeping in mind that we want to address
the \emph{direct} and \emph{inverse} problem mentioned above,
namely, the problem of generating a set of rewriting rules
from an energy function and vice versa.


\section{Kappa}
\label{sec:kappa}

Kappa represents interactions among proteins,
nucleic acids and other biomolecules as
connections in a biomolecular network.
In these networks, nodes symbolise the biomolecules
while connections stand for transient molecular bonds
(\eg non-covalent interactions like hydrogen bonds).
This network is constantly changing as molecules
travel and interact with other molecules in a cell,
which is viewed as the constant destruction and creation
of the connections that make up the network.

Due to spatial constraints,
molecules can physically interact with
just so many other molecules at once.
Exactly how many will depend on multiple factors like
the size of the two interacting molecules and
the region where they come in contact.
These regions, known in molecular biology by the names of domains,
motifs or binding sites, are simply called \emph{sites} in Kappa.
Any such site can bind at most one other site at a time.
These sites belong to the nodes of the graph,
which Kappa calls \emph{agents}.
In the same way a molecule is of a certain species,
agents can be of different types.
These types also live in a network,
a static network which represents the ``network of possibilities''.
It tell us which sites a site \emph{can} bind
instead of what is actually bound to at a given moment.

To make these ideas formal we will use
the category-theoretical approach introduced in \citet{kappadpo}.
We will first introduce the networks for types
and build on them to construct the biomolecular network.%
\footnote{Below we use the words graph and edge
  as synonyms for network and connection.}

\begin{definition}%[site graph]
  A \emph{site graph} $G$ consists of
  a finite set of agents $\agents_G$,
  a finite set of sites $\sites_G$,
  a map $\sitemap_G: \sites_G \to \agents_G$
  that assigns sites to agents
  and a symmetric edge relation $\edges_G$ on $\sites_G$.
\end{definition}

The pair $\sites_G$, $\edges_G$ form an undirected graph.
Note that site graphs do not impose a bound on
the number of connections a site can have,
it just lists the possibilities.
Indeed there is no restriction at all so far.

Sites not in the domain of $\edges_G$ are said to be \emph{free}.
One says $G$ is \emph{realisable} iff
(i) no site has an edge to itself and
(ii) sites have at most one incident edge.
Each realisable site graph represents a state
in which our biomolecular network can be.
However, it contains no typing information.
We have to assign to each agent and site in the graph
an agent and site in the type graph.
More precisely, we need a map from a realisable site graph
to a site graph.
Below we introduce such maps.

\begin{flushleft}
\begin{minipage}{.71\linewidth}
\begin{definition}
  A \emph{homomorphism} $h: G \to G'$ of site graphs is
  a pair of functions, $h_\sites: \sites_G \to \sites_{G'}$
  and $h_\agents: \agents_G \to \agents_{G'}$, such that
  (i) $h_\agents(\sitemap_G(s)) = \sitemap_{G'}(h_\sites(s))$
  and (ii) if $s \mathbin{\edges_G} s'$ then
  $h_\sites(s) \mathbin{\edges_{G'}} h_\sites(s')$.
\end{definition}
\end{minipage}
\begin{minipage}{.28\linewidth}
\begin{flushright}
  \begin{tikzpicture}
    \matrix (m) [matrix of math nodes,row sep=30pt,column sep=30pt] {
      \sites_G & \sites_{G'} \\
      \agents_G & \agents_{G'} \\};
    \draw[hom] (m-1-1) -- node[above] {$h_\sites$} (m-1-2);
    \draw[hom] (m-2-1) -- node[below] {$h_\agents$} (m-2-2);
    \draw[hom] (m-1-1) -- node[left] {$\sitemap_G$} (m-2-1);
    \draw[hom] (m-1-2) -- node[right] {$\sitemap_{G'}$} (m-2-2);
  \end{tikzpicture}
\end{flushright}
\end{minipage}
\end{flushleft}

Put simply, homomorphisms preserve site ownership and connections.
The diagram to the right is the corresponding
commutative diagram in the category of sets and
is equivalent to condition (i) in the definition.
We call the typing map $h: G \to C$ a contact map over $C$
and refer to $C$ as the contact graph.
% TODO: say something more about typing

% TODO: explain
Site graphs and homomorphisms form a category $\SG$.

A homomorphism $h: G \to G'$ is an \emph{embedding} iff
(i) $h_\agents$ and $h_\sites$ are injective;
and (ii) if $s$ is free in $G$, so is $h_\sites(s)$ in $G'$.
Injectivity of $h_\agents$ and $h_\sites$ implies that
whenever $h: G \to G'$ is an embedding and $G'$ is realisable
then $G$ is also realisable.

\begin{wrapfigure}[4]{r}{0.28\textwidth}
  \vspace{-2.4em}
  \begin{center}
    \begin{tikzpicture}
      \matrix (m) [matrix of math nodes,row sep=20pt,column sep=20pt] {
        G & & G' \\
        & C & \\};
      \draw[hom] (m-1-1) -- node[above] {$\psi$} (m-1-3);
      \draw[hom] (m-1-1) -- node[below left] {$h$} (m-2-2);
      \draw[hom] (m-1-3) -- node[below right] {$h'$} (m-2-2);
    \end{tikzpicture}
  \end{center}
\end{wrapfigure}

An embedding $h: G \to G'$ between realisable site graphs
can be lifted to an embedding between contact maps $g: G \to C$
and $g': G' \to C$ iff the diagram on the left commutes
% in the category of site graphs and homomorphisms $\SG$.
in $\SG$.

% TODO: explain
Contact maps over $C$ and embeddings form a category $\rSGe_C$.













%%% Local Variables:
%%% mode: latex
%%% TeX-master: "thesis"
%%% End:



\chapter[The direct problem: From energy to rules]{
  The direct problem \\
  \LARGE From energy to rules}
\label{chp:direct}
In this chapter we show how to construct a set of reversible rules
and their forward and backward rate constants from an energy function.
In the spirit of rule-based modelling languages like Kappa
where rules and observables are defined in terms of patterns,\footnote{
  Recall that a pattern is a contact map used to find subgraphs in states.}
we use a set of connected \emph{energy patterns} $\shapes$
for our energy function.
We assign an \emph{energy cost} $\cost(g)$ to each of them
and build the energy function as a linear combination
of their number of ocurrences. % of each energy pattern.
\begin{equation}
  \label{eq:graph-energy}
  E(m) = \sum_{g \in \shapes} \cost(g) \abs{\matches{g}{m}}
\end{equation}
This is reminiscent of group contribution methods
used to estimate the standard Gibbs free energy of formation
of biomolecules \citep{group-contrib}.

As mentioned at the end of \sct{kappa},
we will derive the set of rules with detailed balance
from a set of generator rules $\generators$ (without rates).
We suppose that $\generators$ is closed under
rule inversion, \ie $\generators = \inv{\generators}$.
Given a contact graph $C$,
a simple option would be to include
every possible minimal rule in this set,
that is, include a creation and a destruction rule
for each edge in the contact graph.
Each of these rules is minimal in the sense that
it only asks for the presence of
the two participating agents and sites.
The example rule in \sct{kappa}
(page~\pageref{p:example}) %, redisplayed below for convenience)
where agents of type $1$ and $2$ bind
% in whatever context,
% whatever the context,
% in whatever context they are,
% in whatever context they happen to be,
% regardless of the surrounding context,
regardless of the context
% in which they happen to be
% in which these two agents happen to be,
% which we denote here by $r^+_{12}$,
is one such minimal rule
that can be derived from the contact graph $T$.
We call this rule $r^+_{12}$.
\begin{equation}
  \label{eq:r+12}
  r^+_{12} :=\;\; %\resizebox{.37\linewidth}{!}{%
    \tikz[thick,baseline=-.1cm]{
      \node[grphnode] (lhs) at (0,0) {
        \tikz[ingrphdiag]{
          \begin{scope}[shift={(0,0)}]
            \n[n1]{x}{0,0};
            \e{x}{.5,0};
            \site{rx}{x.east};
            \node at (26:.42) {\scriptsize $r$};
          \end{scope}
          \begin{scope}[shift={(1.2,0)}]
            \n[n2]{y}{0,0};
            \e{y}{-.5,0};
            \site{ly}{y.west};
            \node at (206:.42) {\scriptsize $l$};
          \end{scope}
        }};
      \path (lhs.east) +(.3,0) edge[rule] +(1,0)
      +(1.3,0) coordinate (r);
      \node[grphnode,anchor=west] (rhs) at (r) {
        \tikz[ingrphdiag]{
          \e{0,0}{1.1,0};
          \begin{scope}
            \n[n1]{x}{0,0};
            \site{rx}{x.east};
            \node at (26:.42) {\scriptsize $r$};
          \end{scope}
          \begin{scope}[shift={(1.1,0)}]
            \n[n2]{y}{0,0};
            \site{ly}{y.west};
            \node at (206:.42) {\scriptsize $l$};
          \end{scope}
        }};
    }%}
\end{equation}
This option is \emph{maximally permissive}
% as every possible transformation
% allowed by the contact graph
% is allowed by $\generators$.\footnote{
with respect to the contact graph.\footnote{
  Intuitively, this is analogous to the case of classical mechanics
  % where the topology of the space gives us the possible transformations
  where, a priori, movement is not constrained along any coordinate.}
Even if all transformations are possible,
many of them may be unlikely due to having a high energy.
Still one might prefer to forbid certain transformations
in some scenarios.
This is indeed the case in the example
that will be presented in \sct{alloring}.

In our previous example (\sct{kappa}),
we might want to favour the formation of
triangles over chains and other cycles.
For this we give a negative energy cost to the triangle $t$,
\ie $\cost(t) < 0$.
If $t$ is the only energy pattern,
then the energy of a state $m$ is
$E(m) = \cost(t) \abs{\matches{t}{m}}$.
In this model one might, for instance,
wonder how low the energy cost of $t$ must be
to have at least $90\%$ of all agents in a triangle
at equilibrium at least $90\%$ of the time.

We would like to find rules that have detailed balance
with respect to this energy function.
Consider the rule $r^+_{12}$ and its inverse $r^-_{12}$,
the unbinding of agents $1$ and $2$.
% Given the maximally permissive set of generator rules
% $\generators=\set{r^+_{12},r^-_{12},r^+_{23},r^-_{23},r^+_{31},r^-_{31}}$,
% we first ask ourselves if these reversible rules
We first ask ourselves if this pair of rules
could have detailed balance
for some assignment of kinetic rates.
% to the forward and backward rule.
Suppose we assign kinetic rates $k^+$ and $k^-$
to $r^+_{12}$ and $r^-_{12}$.
Recall from \sct{bg} that $\exp{E(n)-E(m)} = q_{nm}/q_{mn}$
for systems with detailed balance.
From \eqn{kappa-ctmc}
\[ q_{mn} = \sum_{\substack{r \in \generators\\r = \tuple{r_L,r_R}}}
   k(r) \; \abs{\setof{\psi \in \matches{r_L}{m}}{m^{(r,\psi)} = n}}
\]
where $m^{(r,\psi)}$ is the outcome of rewriting $m$
% using rule $r$ and embedding $\psi$.
with event $(r,\psi)$.
% It is clear that
At most one of the two rules
can bring us from state $m$ to $n$,
say it is $r^+_{12}$.
By rule reversibility (\lem{reversibility})
$r^-_{12}$ brings us from $n$ back to $m$
and the number of matches of $r^-_{12}$ in $n$
is equal to the number of matches of $r^+_{12}$ in $m$.
Hence, $\exp{E(n)-E(m)} = k^+/k^-$.
In words, the change in energy produced by the rule application
fixes the ratio between the kinetic rates.
As a consequence,
each rule application should produce the same energy change
for there to be an assignment of kinetic rates with detailed balance.
Whenever a rule produces the same energy change
regardless of where it is applied
we say that the rule has an \emph{unambiguous energy balance}
or is $\shapes$-balanced.
More generally, we define $\shapes$-balance as follows.

\begin{definition}
  Given a contact graph $C$
  and a set $\shapes$ of contact maps over $C$,
  a rule $r$ is $\shapes$-balanced
  if, for all mixtures $m$ and embeddings $\psi: r_L \to m$,
  the number of ocurrences of $p \in \shapes$
  produced and consumed by $r$ when applied to $\psi$
  is a fixed number
  $\Delta_r p = |[p;m^{(r,\psi)}]| - \abs{\matches{p}{m}}$.
  % is a fixed number $\Delta_r p$,
  % \ie $|[p;m^{(r,\psi)}]| - \abs{\matches{p}{m}} = \Delta_r p\;$
  % for all $p \in \shapes$.
  We refer to $\Delta_r p$ as the balance of $r$ with respect to $p$.
  % We refer to the vector of ocurrence changes as $\Delta_r \shapes$.
\end{definition}
% TODO: perhaps add a remark about unambiguous stoichiometry

The following two rule applications show that
$r^+_{12}$ is not $\shapes$-balanced.
\begin{center}
  \resizebox{.9\linewidth}{!}{%
  \begin{tikzpicture}[thick]
    % first row
    \node[grphnode,anchor=east] (lhs1) at (0,0) {
      \tikz[ingrphdiag]{
        \begin{scope}[shift={(0,0)}]
          \n[n1]{x}{0,0};
          \e{x}{.5,0};
          \site{rx}{x.east};
          \node at (26:.42) {\scriptsize $r$};
        \end{scope}
        \begin{scope}[shift={(1.2,0)}]
          \n[n2]{y}{0,0};
          \e{y}{-.5,0};
          \site{ly}{y.west};
          \node at (206:.42) {\scriptsize $l$};
        \end{scope}
      }};
    \path (lhs1.east) +(.3,0) edge[rule,dotted] +(1,0)
      +(1.3,0) coordinate (r1);
    \node[grphnode,anchor=west] (rhs1) at (r1) {
      \tikz[ingrphdiag]{
        \e{0,0}{1.1,0};
        \begin{scope}
          \n[n1]{x}{0,0};
          \site{rx}{x.east};
          \node at (26:.42) {\scriptsize $r$};
        \end{scope}
        \begin{scope}[shift={(1.1,0)}]
          \n[n2]{y}{0,0};
          \site{ly}{y.west};
          \node at (206:.42) {\scriptsize $l$};
        \end{scope}
      }};
    % second column
    \node[grphnode,anchor=east] (lhs2) at (9,0) {
      \tikz[ingrphdiag]{
        \begin{scope}[shift={(0,0)}]
          \n[n1]{x}{0,0};
          \e{x}{.5,0};
          \site{rx}{x.east};
          \node at (26:.42) {\scriptsize $r$};
        \end{scope}
        \begin{scope}[shift={(1.2,0)}]
          \n[n2]{y}{0,0};
          \e{y}{-.5,0};
          \site{ly}{y.west};
          \node at (206:.42) {\scriptsize $l$};
        \end{scope}
      }};
    \path (lhs2.east) +(.3,0) edge[rule,dotted] +(1,0)
      +(1.3,0) coordinate (r2);
    \node[grphnode,anchor=west] (rhs2) at (r2) {
      \tikz[ingrphdiag]{
        \e{0,0}{1.1,0};
        \begin{scope}
          \n[n1]{x}{0,0};
          \site{rx}{x.east};
          \node at (26:.42) {\scriptsize $r$};
        \end{scope}
        \begin{scope}[shift={(1.1,0)}]
          \n[n2]{y}{0,0};
          \site{ly}{y.west};
          \node at (206:.42) {\scriptsize $l$};
        \end{scope}
      }};
    % second row
    \path (lhs1.south) +(0,-.2) edge[rule] +(0,-.6);
    \node[grphnode,anchor=east] (lhs3) at (0,-2) {
      \tikz[ingrphdiag]{
        \begin{scope}[shift={(0,0)}]
          \n[n1]{x}{0,0};
          \e{x}{.5,0};
          \e{x}{-.5,0};
          \site{lx}{x.west};
          \site{rx}{x.east};
          \node at (206:.42) {\scriptsize $l$};
          \node at (26:.42) {\scriptsize $r$};
        \end{scope}
        \e{1.2,0}{2.3,0};
        \begin{scope}[shift={(1.2,0)}]
          \n[n2]{y}{0,0};
          \e{y}{-.5,0};
          \site{ly}{y.west};
          \site{ry}{y.east};
          \node at (206:.42) {\scriptsize $l$};
          \node at (26:.42) {\scriptsize $r$};
        \end{scope}
        \begin{scope}[shift={(2.3,0)}]
          \n[n3]{z}{0,0};
          \e{z}{.5,0};
          \site{lz}{z.west};
          \site{rz}{z.east};
          \node at (206:.42) {\scriptsize $l$};
          \node at (26:.42) {\scriptsize $r$};
        \end{scope}
      }};
    \path (lhs3.east) +(.3,0) edge[rule,dotted] +(1,0)
      +(1.3,0) coordinate (r3);
    \path (rhs1.south) +(0,-.2) edge[rule] +(0,-.6);
    \node[grphnode,anchor=west] (rhs3) at (r3) {
      \tikz[ingrphdiag]{
        \e{0,0}{2.2,0};
        \begin{scope}[shift={(0,0)}]
          \n[n1]{x}{0,0};
          \e{x}{-.5,0};
          \site{lx}{x.west};
          \site{rx}{x.east};
          \node at (206:.42) {\scriptsize $l$};
          \node at (26:.42) {\scriptsize $r$};
        \end{scope}
        \begin{scope}[shift={(1.1,0)}]
          \n[n2]{y}{0,0};
          \site{ly}{y.west};
          \site{ry}{y.east};
          \node at (206:.42) {\scriptsize $l$};
          \node at (26:.42) {\scriptsize $r$};
        \end{scope}
        \begin{scope}[shift={(2.2,0)}]
          \n[n3]{z}{0,0};
          \e{z}{.5,0};
          \site{lz}{z.west};
          \site{rz}{z.east};
          \node at (206:.42) {\scriptsize $l$};
          \node at (26:.42) {\scriptsize $r$};
        \end{scope}
      }};
    % second row, second column
    \path (lhs2.south) +(0,-.2) edge[rule] +(0,-.6);
    \node[grphnode,anchor=east] (lhs4) at (9,-2.4) {
      \tikz[ingrphdiag]{
        \path[use as bounding box] (-.3,.38) rectangle (1.5,-1.22);
        \e{0,0}{-56.944:1.1};
        \e{0:1.2}{-56.944:1.1};
        \begin{scope}[shift={(0,0)}]
          \n[n1]{x}{0,0};
          \e{x}{.5,0};
          \site{r1}{0:7pt};
          \site{l1}{-60:7pt};
          \node at (-86:12pt) {\scriptsize $l$};
          \node at (26:12pt) {\scriptsize $r$};
        \end{scope}
        \begin{scope}[shift={(0:1.2)}]
          \n[n2]{y}{0,0};
          \e{y}{-.5,0};
          \site{r2}{180:7pt};
          \site{l2}{-120:7pt};
          \node at (154:12pt) {\scriptsize $l$};
          \node at (-94:12pt) {\scriptsize $r$};
        \end{scope}
        \begin{scope}[shift={(-56.944:1.1)}]
          \n[n3]{z}{0,0};
          % angle is 66.111 deg
          \site{r3}{123.0555:7pt};
          \site{l3}{56.9445:7pt};
          \node at (146:12pt) {\scriptsize $r$};
          \node at (34:12pt) {\scriptsize $l$};
        \end{scope}
      }};
    \path (lhs4.east) +(.3,0) edge[rule,dotted] +(1,0)
      +(1.3,0) coordinate (r4);
    \path (rhs2.south) +(0,-.2) edge[rule] +(0,-.6);
    \node[grphnode,anchor=west] (rhs4) at (r4) {
      \tikz[ingrphdiag]{
        \path[use as bounding box] (-.3,.38) rectangle (1.4,-1.22);
        \e{0,0}{0:1.1};
        \e{0,0}{-60:1.1};
        \e{0:1.1}{-60:1.1};
        \begin{scope}[shift={(0,0)}]
          \n[n1]{x}{0,0};
          \site{r1}{0:7pt};
          \site{l1}{-60:7pt};
          \node at (-86:12pt) {\scriptsize $l$};
          \node at (26:12pt) {\scriptsize $r$};
        \end{scope}
        \begin{scope}[shift={(0:1.1)}]
          \n[n2]{y}{0,0};
          \site{r2}{180:7pt};
          \site{l2}{-120:7pt};
          \node at (154:12pt) {\scriptsize $l$};
          \node at (-94:12pt) {\scriptsize $r$};
        \end{scope}
        \begin{scope}[shift={(-60:1.1)}]
          \n[n3]{z}{0,0};
          \site{r3}{120:7pt};
          \site{l3}{60:7pt};
          \node at (146:12pt) {\scriptsize $r$};
          \node at (34:12pt) {\scriptsize $l$};
        \end{scope}
      }};
  \end{tikzpicture}}
\end{center}

We see that, while the application on the left
does not produce any change in energy ($\Delta E = 0$),
the one on the right creates a triangle
and thus $\Delta E = \cost(t)$. %\footnote{
%   And we won't tolerate energetical ambiguity in this house!}
We must then split $r^+_{12}$ into subrules that check
the surroundings of the rule application
to make sure that, for instance,
every application of such a subrule
creates one triangle or none at all.
It is important that the partition of the rule
has certain properties.
In particular, one would like that every match of the rule
can be mapped to exactly one match of one of the subrules.
Prior work by \citet{refinement} has shown how
one can obtain a partition of rules with this property
and will be presented, in a slightly modified version,
in \sct{refinements}. % the next section.
% NOTE: not possible to put refinement section
% before minimal glueings because the proof of the
% unique decomposition theorem uses minimal glueings.

But before diving into rule partitioning,
or rule refinement as we call it,
it would be good to have a more rigourous idea of
when a rule is $\shapes$-balanced or not.
In the examples shown above we see that
our energy pattern, the triangle,
must be fully incorporated into
the left- or the right-hand side of the rule
to be sure it produces or consumes it in every application.
On the other hand, a rule that is incompatible
with our energy pattern will also be $\shapes$-balanced
by making it impossible for the rule to match a triangle.
This is true whenever there is no glueing % union
of the left-hand side of a rule with the energy pattern
where they overlap in a site that is modified by the rule.
In the next section,
we introduce the concept of overlapping glueings
of contact maps by means of multi-sums,
a concept related to local coproducts and relative pushouts.
% in $\rSGe_C$.


\section{Minimal glueings}
\label{sec:mg}

The category $\SG$ has all pullbacks,
constructed from those in $\Set$,
and they indeed restrict to $\rSGe_C$.

\begin{lemma}\label{lemma:pullbacks}
  Given a cospan $\phi_1: g_1 \to h \gets g_2 :\phi_2$ in $\rSGe_C$
  there is a unique span $\psi_1: g_1 \gets p \to g_2 :\psi_2$
  (up to unique isomorphism)
  such that any span $\omega_1: g_1 \gets q \to g_2 :\omega_2$
  that forms a commuting square $\omega_1,\omega_2,\phi_1,\phi_2\;$
  factors \emph{uniquely} through it.
  \begin{center}
    \begin{tikzpicture}
      \node (h1) at (0,0) {$g_1$};
      \node (h2) at (6,0) {$g_2$};
      \node (h) at (3,-1) {$h$};
      \node (p) at (3,1) {$p$};
      \node (q) at (3,2.2) {$q$};
      \draw (q) edge[hom,bend right=20] node[above left] {$\omega_1$} (h1);
      \draw (q) edge[hom,bend left=20] node[above right] {$\omega_2$} (h2);
      \draw[hom] (p) -- node[above] {$\psi_1$} (h1);
      \draw[hom] (p) -- node[above] {$\psi_2$} (h2);
      \draw[hom] (h1) -- node[below] {$\phi_1$} (h);
      \draw[hom] (h2) -- node[below] {$\phi_2$} (h);
      \draw[hom,dotted] (q) -- node[right] {$!$} (p);
    \end{tikzpicture}
  \end{center}
\end{lemma}
\begin{proof}
  We construct contact map $p: G \to C$ by taking the intersection
  of the agents, sites and edges in the image of $\phi_1,\phi_2$
  and restricting $\sitemap$ accordingly.
  With some abuse of notation, we have
  \begin{alignat*}{3}
    \agents_G & {}= \phi_{1,\agents}(\agents_{\anon{g_1}}) & {}\cap{} &
                  \,\phi_{2,\agents}(\agents_{\anon{g_2}}) \\
    \sites_G & {}= \,\phi_{1,\sites}(\sites_{\anon{g_1}}) & {}\cap{} &
                 \,\,\phi_{2,\sites}(\sites_{\anon{g_2}}) \\
    \edges_G & {}= \,\phi_{1,\sites}(\edges_{\anon{g_1}}) & {}\cap{} &
                 \,\,\phi_{2,\sites}(\edges_{\anon{g_2}})
  \end{alignat*}
  and $\sitemap_G = \rest{\sitemap_{\anon{h}}}{\sites_G}$.
  Functions $p_\agents,p_\sites$ are the restriction of
  $h_\agents,h_\sites$ to $\agents_G,\sites_G$, respectively.
  Embeddings $\psi_1$ and $\psi_2$ map agents and sites
  in $G$ to their pre-images along $\phi_1$ and $\phi_2$;
  by construction, all agents and sites in $G$
  are guaranteed to have such a pre-image.
  It is easy to see that
  (i) $\psi_1$ and $\psi_2$ are type-preserving
  and thus embeddings in $\rSGe_C$; and that
  (ii) the square formed by $\psi_1,\psi_2,\phi_1,\phi_2$ commutes.

  Consider any span $\omega_1: g_1 \gets q \to g_2 :\omega_2$ in $\rSGe_C$.
  If the square formed by $\omega_1$, $\omega_2,\phi_1,\phi_2$ commutes,
  then $q$ can have at most one copy of each agent and site
  in the intersection of the images of $\phi_1$ and $\phi_2$
  because $\phi_1\,\omega_1$ and $\phi_2\,\omega_2$ are injective.
  Hence, every agent and site in the image of $\omega_1,\omega_2$
  has a \emph{unique} pre-image along $\psi_1,\psi_2$, respectively,
  with the same type.
  This fixes a pair of functions $\omega_\agents,\omega_\sites$
  that map agents and sites in $q$ to those in $p$ injectively
  and form an embedding $\omega$ in $\rSGe_C$.
  Since the pre-image along $\psi_1,\psi_2$ always exists and is unique,
  any embedding $\omega': p \to q$ must be equal to $\omega$
  whenever $\phi_1\,\omega' = \omega_1$ and
  $\phi_2\,\omega' = \omega_2$.
\end{proof}

$\SG$ also has all pushouts and all sums,
but these do not in general restrict to $\rSGe_C$,
just as pushouts and sums in $\Set$ do not restrict to
the subcategory of injective functions.
% all pushouts; but these do not generally restrict to $\rSGe_C$ since
% (i) the pushout object need not be realisable,
% even if all objects in the starting span were;
% (ii) the arrows in the resulting cospan need not be embeddings,
% even if all arrows in the starting span were;
% and (iii) the mediating arrow need not even be injective
% (on agents or sites).
However, $\rSGe_C$ has \emph{multi-sums}.

\begin{lemma}\label{lemma:mg}
  For all pairs of contact maps over $C$,
  $g_1: G_1 \to C$ and $g_2: G_2 \to C$,
  % there exists a finite set $I$ and a family of cospans ... with i \in I
  there exists a finite family of cospans
  $\theta^i_1: g_1 \to s_i \gets g_2 :\theta^i_2$,
  such that any cospan $\phi_1: g_1 \to h \gets g_2 :\phi_2\;$
  factors through \emph{exactly one} of the family
  and does so \emph{uniquely}.
  \begin{center}
    \begin{tikzpicture}
      \node (h1) at (0,0) {$g_1$};
      \node (si) at (1.8,0) {$s_i$};
      \node (h2) at (3.6,0) {$g_2$};
      \node (h) at (1.8,-1.8) {$h$};
      \draw[hom] (h1) -- node[above] {$\theta^i_1$} (si);
      \draw[hom] (h2) -- node[above] {$\theta^i_2$} (si);
      \draw[hom] (h1) -- node[below left] {$\phi_1$} (h);
      \draw[hom] (h2) -- node[below right] {$\phi_2$} (h);
      \draw[hom,dotted] (si) -- node[right] {$!$} (h);
    \end{tikzpicture}
  \end{center}
\end{lemma}
\begin{proof}
  Take subsets $A_i$ of the cartesian product
  of $\agents_{\anon{g_1}}$ and $\agents_{\anon{g_2}}$
  that have each agent of $g_1$ and $g_2$ at most once
  ($(a,b) \in A_i \wedge (a,b') \in A_i \then b = b'$)
  and where each pair $(a,b) \in A_i$ has the same type,
  % that are type-compatible,
  \ie $g_{1,\agents}(a) = g_{2,\agents}(b)$.
  % for all $(a,b) \in A_i$,
  To each $A_i$ assign all subsets $S_{ij}$ of
  $\sites_{\anon{g_1}} \times \sites_{\anon{g_2}}$
  that are type-compatible
  and whose elements belong to agents paired in $A_i$,
  that is, if $(x,y) \in S_{ij}$
  then $g_{1,\sites}(x) = g_{2,\sites}(y)$
  and $(\sitemap_{\anon{g_1}}(x),\sitemap_{\anon{g_2}}(y)) \in A_i$.
  % Note that the latter predicate fixes ...
  Note how this fixes a mapping $\sitemap_{ij}$
  between elements of $S_{ij}$ to elements of $A_i$
  defined by
  $\sitemap_{ij}((x,y)) =
     (\sitemap_{\anon{g_1}}(x),\sitemap_{\anon{g_2}}(y))$.
  % Discard all sets $S_ij$ that are subsets
  % of a set $S_jk$ with $j \neq k$.
  For each $A_i$ keep only the set $S_{ij}$
  that is a superset of all other sets $S_{ik}$ ($k \neq j$).
  % and discard all others.
  There must be one such maximal set because
  if any two pairs of sites $(x_1,y_1),(x_2,y_2)$
  are type-preserving and belong to the same agents,
  then there will be one set among the $S_{ij}$s that has both
  and thus $\{S_{ij}\}_j$ is a directed partial order
  for the inclusion relation.
  % Hence, we can drop the $j$ subscript
  % in $S_{ij}$ and $\sitemap_{ij}$.
  Let $S_i$ be the maximal element of $\{S_{ij}\}_j$,
  which exists by directedness and finiteness of this family,
  and $\sitemap_i$ the corresponding mapping to $A_i$.
  Intuitively, the maximal set $S_i$ is the set of all sites
  that are defined in both agents at the same time.
  Next we discard those pairs $A_i,S_i$
  whose elements do not agree on their edge structure;
  if $(x,y) \in S_i$ then either both sites must be free
  or connected to sites $(x',y') \in S_i$.

  We construct a family of contact maps $p_i: P_i \to C$
  using $\agents_{P_i} = A_i$ as its agents,
  $\sites_{P_i} = S_i$ as its sites,
  $\sitemap_{P_i} = \sitemap_i$ and
  $\edges_{P_i} = \{((x_1,y_1), (x_2,y_2)) \in S_i \times S_i \st
     x_1 \edges_{\anon{g_1}} x_2 \wedge
     y_1 \edges_{\anon{g_2}} y_2\}$.
  Functions $p_{i,\agents},p_{i,\sites}$
  are defined straightforwardly.
  Spans $\psi^i_1: g_1 \gets p_i \to g_2 :\psi^i_2$
  are then obtained by mapping agents $(a,b)$ in $p_i$
  to $a$ in $g_1$ and $b$ in $g_2$
  and similarly for sites.
  Multi-sums $\theta^i_1: g_1 \to s_i \gets g_2 :\theta^i_2$
  are pushouts of such spans:
  they are obtained by adding to $p_i$
  all the missing agents, sites and edges from $g_1$ and $g_2$.
  Since all sites that are in $g_1$ but not in $p_i$
  cannot be in $g_2$ by maximality of $S_i$,
  there can be no conflict when adding sites or edges.
  The same argument holds for sites in $g_2$ that are not in $p_i$.

  Note that the family $A_i$ is finite
  and thus the family of multi-sums is finite as well.
  Also, it is easy to see that the spans $\psi^i_1,\psi^i_2$
  are pullbacks of $\theta^i_1,\theta^i_2$.
  Hence, (isomorphism classes of) multi-sums
  are in a one-to-one correspondence
  with (isomorphism classes of) pullbacks.
  This implies that there is only one multi-sum
  that factors any given cospan.
\end{proof}

The pairs $\theta^i_1,\theta^i_2$ enumerate
all minimal ways in which one can glue $g_1$ and $g_2$.
% and thus all the minimal contexts in which they can occur.
Hence, we refer to them as minimal glueings.
%
The notion of multi-sum dates back to \citet{diers}.
% We call them \emph{minimal glueings} in $\rSGe_C$
% according to their intuition in this concrete context
% and use them in \sct{energy-gp} to construct balanced rules.
% TODO: elaborate on relation to RPOs
They are very close to relative pushouts \citep{leifer}
and will be used in the same way,
to minimise rewriting contexts.
Indeed, each minimal glueing $i$
in the family of cospans $\theta^i_1,\theta^i_2$
accounts for one minimal rewriting context.

To illustrate how this construction operates,
consider the minimal glueings of the following
two contact maps over $T$ % (the triangle)
with their respective pullbacks.
% as shown in the following diagram.
\begin{center}
  \resizebox{.9\linewidth}{!}{%
  \begin{tikzpicture}[thick]

    \begin{scope}[shift={(0,0)}]
      %%% Empty intersection %%%
      \node[grphnode,anchor=south] (pb1) at (0,0) {
        \tikz[ingrphdiag]{
          \node {\large $\varnothing$};
          \node[yshift=.1em] {\large\phantom{$\varnothing$}};
        }};

      %%% 1-2-3 %%%
      \node[grphnode] (g1-1) at (-135:3) {
        \tikz[ingrphdiag]{
          \e{0,0}{2.2,0};
          \begin{scope}
            \n[n1]{n1}{0,0};
            \site{r1}{n1.east};
            \node at (26:.42) {\scriptsize $r$};
          \end{scope}
          \begin{scope}[shift={(1.1,0)}]
            \n[n2]{n2}{0,0};
            \site{l2}{n2.west};
            \site{r2}{n2.east};
            \node at (206:.42) {\scriptsize $l$};
            \node at (26:.42) {\scriptsize $r$};
          \end{scope}
          \begin{scope}[shift={(2.2,0)}]
            \n[n3]{n3}{0,0};
            \site{l3}{n3.west};
            \node at (206:.42) {\scriptsize $l$};
          \end{scope}
        }};

      %%% 2-3-1 %%%
      \node[grphnode] (g2-1) at (-45:3) {
        \tikz[ingrphdiag]{
          \e{0,0}{2.2,0};
          \begin{scope}
            \n[n2]{n2}{0,0};
            \site{r2}{n2.east};
            \node at (26:.42) {\scriptsize $r$};
          \end{scope}
          \begin{scope}[shift={(1.1,0)}]
            \n[n3]{n3}{0,0};
            \site{l3}{n3.west};
            \site{r3}{n3.east};
            \node at (206:.42) {\scriptsize $l$};
            \node at (26:.42) {\scriptsize $r$};
          \end{scope}
          \begin{scope}[shift={(2.2,0)}]
            \n[n1]{n1}{0,0};
            \site{l1}{n1.west};
            \node at (206:.42) {\scriptsize $l$};
          \end{scope}
        }};

      % cos(45) = 0.7071, * 3 = 2.1213, * 2 = 4.2426
      %%% Disjoint union: 1-2-3 2-3-1 %%%
      \node[grphnode,anchor=north] (po1) at (-90:4.2426) {
        \tikz[ingrphdiag]{
          \e{0,0}{2.2,0};
          \begin{scope}
            \n[n1]{n1}{0,0};
            \site{r1}{n1.east};
            \node at (26:.42) {\scriptsize $r$};
          \end{scope}
          \begin{scope}[shift={(1.1,0)}]
            \n[n2]{n2}{0,0};
            \site{l2}{n2.west};
            \site{r2}{n2.east};
            \node at (206:.42) {\scriptsize $l$};
            \node at (26:.42) {\scriptsize $r$};
          \end{scope}
          \begin{scope}[shift={(2.2,0)}]
            \n[n3]{n3}{0,0};
            \site{l3}{n3.west};
            \node at (206:.42) {\scriptsize $l$};
          \end{scope}

          \e{0,-.9526}{2.2,-.9526};
          \begin{scope}[shift={(0,-.9526)}]
            \n[n2]{n4}{0,0};
            \site{r4}{n4.east};
            \node at (26:.42) {\scriptsize $r$};
          \end{scope}
          \begin{scope}[shift={(1.1,-.9526)}]
            \n[n3]{n5}{0,0};
            \site{l5}{n5.west};
            \site{r5}{n5.east};
            \node at (206:.42) {\scriptsize $l$};
            \node at (26:.42) {\scriptsize $r$};
          \end{scope}
          \begin{scope}[shift={(2.2,-.9526)}]
            \n[n1]{n6}{0,0};
            \site{l6}{n6.west};
            \node at (206:.42) {\scriptsize $l$};
          \end{scope}
        }};

      \arrsn[opacity=.7]{pb1}{g1-1};
      \arrsn[opacity=.7]{pb1}{g2-1};
      \arrsn[opacity=.7]{g1-1}{po1};
      \arrsn[opacity=.7]{g2-1}{po1};
    \end{scope}

    \begin{scope}[shift={(8,0)}]
      %%% 1 %%%
      \node[grphnode,anchor=south] (pb2) at (0,0) {
        \tikz[ingrphdiag]{
          \n[n1]{n1}{0,0};
        }};

      %%% 1-2-3 %%%
      \node[grphnode] (g1-2) at (-135:3) {
        \tikz[ingrphdiag]{
          \e{0,0}{2.2,0};
          \begin{scope}
            \n[n1]{n1}{0,0};
            \site{r1}{n1.east};
            \node at (26:.42) {\scriptsize $r$};
          \end{scope}
          \begin{scope}[shift={(1.1,0)}]
            \n[n2]{n2}{0,0};
            \site{l2}{n2.west};
            \site{r2}{n2.east};
            \node at (206:.42) {\scriptsize $l$};
            \node at (26:.42) {\scriptsize $r$};
          \end{scope}
          \begin{scope}[shift={(2.2,0)}]
            \n[n3]{n3}{0,0};
            \site{l3}{n3.west};
            \node at (206:.42) {\scriptsize $l$};
          \end{scope}
        }};

      %%% 2-3-1 %%%
      \node[grphnode] (g2-2) at (-45:3) {
        \tikz[ingrphdiag]{
          \e{0,0}{2.2,0};
          \begin{scope}
            \n[n2]{n2}{0,0};
            \site{r2}{n2.east};
            \node at (26:.42) {\scriptsize $r$};
          \end{scope}
          \begin{scope}[shift={(1.1,0)}]
            \n[n3]{n3}{0,0};
            \site{l3}{n3.west};
            \site{r3}{n3.east};
            \node at (206:.42) {\scriptsize $l$};
            \node at (26:.42) {\scriptsize $r$};
          \end{scope}
          \begin{scope}[shift={(2.2,0)}]
            \n[n1]{n1}{0,0};
            \site{l1}{n1.west};
            \node at (206:.42) {\scriptsize $l$};
          \end{scope}
        }};

      %%% 2-3-1-2-3 %%%
      \node[grphnode,anchor=north] (po2) at (-90:4.2426) {
        \tikz[ingrphdiag]{
          \e{-2.2,0}{2.2,0};
          \begin{scope}[shift={(-2.2,0)}]
            \n[n2]{n2}{0,0};
            \site{r2}{n2.east};
            \node at (26:.42) {\scriptsize $r$};
          \end{scope}
          \begin{scope}[shift={(-1.1,0)}]
            \n[n3]{n3}{0,0};
            \site{l3}{n3.west};
            \site{r3}{n3.east};
            \node at (206:.42) {\scriptsize $l$};
            \node at (26:.42) {\scriptsize $r$};
          \end{scope}
          \begin{scope}
            \n[n1]{n1}{0,0};
            \site{l1}{n1.west};
            \site{r1}{n1.east};
            \node at (206:.42) {\scriptsize $l$};
            \node at (26:.42) {\scriptsize $r$};
          \end{scope}
          \begin{scope}[shift={(1.1,0)}]
            \n[n2]{n2}{0,0};
            \site{l2}{n2.west};
            \site{r2}{n2.east};
            \node at (206:.42) {\scriptsize $l$};
            \node at (26:.42) {\scriptsize $r$};
          \end{scope}
          \begin{scope}[shift={(2.2,0)}]
            \n[n3]{n3}{0,0};
            \site{l3}{n3.west};
            \node at (206:.42) {\scriptsize $l$};
          \end{scope}
        }};

      \arrsn[opacity=.7]{pb2}{g1-2};
      \arrsn[opacity=.7]{pb2}{g2-2};
      \arrsn[opacity=.7]{g1-2}{po2};
      \arrsn[opacity=.7]{g2-2}{po2};
    \end{scope}

    \begin{scope}[shift={(0,-9)}]
      %%% 2-3 %%%
      \node[grphnode,anchor=south] (pb3) at (0,0) {
        \tikz[ingrphdiag]{
          \e{0,0}{1.1,0};
          \begin{scope}
            \n[n2]{n2}{0,0};
            \site{r2}{n2.east};
            \node at (26:.42) {\scriptsize $r$};
          \end{scope}
          \begin{scope}[shift={(1.1,0)}]
            \n[n3]{n3}{0,0};
            \site{l3}{n3.west};
            \node at (206:.42) {\scriptsize $l$};
          \end{scope}
        }};

      %%% 1-2-3 %%%
      \node[grphnode] (g1-3) at (-135:3) {
        \tikz[ingrphdiag]{
          \e{0,0}{2.2,0};
          \begin{scope}
            \n[n1]{n1}{0,0};
            \site{r1}{n1.east};
            \node at (26:.42) {\scriptsize $r$};
          \end{scope}
          \begin{scope}[shift={(1.1,0)}]
            \n[n2]{n2}{0,0};
            \site{l2}{n2.west};
            \site{r2}{n2.east};
            \node at (206:.42) {\scriptsize $l$};
            \node at (26:.42) {\scriptsize $r$};
          \end{scope}
          \begin{scope}[shift={(2.2,0)}]
            \n[n3]{n3}{0,0};
            \site{l3}{n3.west};
            \node at (206:.42) {\scriptsize $l$};
          \end{scope}
        }};

      %%% 2-3-1 %%%
      \node[grphnode] (g2-3) at (-45:3) {
        \tikz[ingrphdiag]{
          \e{0,0}{2.2,0};
          \begin{scope}
            \n[n2]{n2}{0,0};
            \site{r2}{n2.east};
            \node at (26:.42) {\scriptsize $r$};
          \end{scope}
          \begin{scope}[shift={(1.1,0)}]
            \n[n3]{n3}{0,0};
            \site{l3}{n3.west};
            \site{r3}{n3.east};
            \node at (206:.42) {\scriptsize $l$};
            \node at (26:.42) {\scriptsize $r$};
          \end{scope}
          \begin{scope}[shift={(2.2,0)}]
            \n[n1]{n1}{0,0};
            \site{l1}{n1.west};
            \node at (206:.42) {\scriptsize $l$};
          \end{scope}
        }};

      %%% 1-2-3-1 %%%
      \node[grphnode,anchor=north] (po3) at (-90:4.2426) {
        \tikz[ingrphdiag]{
          \e{0,0}{3.3,0};
          \begin{scope}
            \n[n1]{n1}{0,0};
            \site{r1}{n1.east};
            \node at (26:.42) {\scriptsize $r$};
          \end{scope}
          \begin{scope}[shift={(1.1,0)}]
            \n[n2]{n2}{0,0};
            \site{l2}{n2.west};
            \site{r2}{n2.east};
            \node at (206:.42) {\scriptsize $l$};
            \node at (26:.42) {\scriptsize $r$};
          \end{scope}
          \begin{scope}[shift={(2.2,0)}]
            \n[n3]{n3}{0,0};
            \site{l3}{n3.west};
            \site{r3}{n3.east};
            \node at (206:.42) {\scriptsize $l$};
            \node at (26:.42) {\scriptsize $r$};
          \end{scope}
          \begin{scope}[shift={(3.3,0)}]
            \n[n1]{n4}{0,0};
            \site{l4}{n4.west};
            \node at (206:.42) {\scriptsize $l$};
          \end{scope}
        }};

      \arrsn[opacity=.7]{pb3}{g1-3};
      \arrsn[opacity=.7]{pb3}{g2-3};
      \arrsn[opacity=.7]{g1-3}{po3};
      \arrsn[opacity=.7]{g2-3}{po3};
    \end{scope}

    \begin{scope}[shift={(8,-9)}]
      %%% 1 2-3 %%%
      \node[grphnode,anchor=south] (pb4) at (0,0) {
        \tikz[ingrphdiag]{
          \e{1,0}{2.2,0};
          \n[n1]{n1}{0,0};
          \begin{scope}[shift={(1,0)}]
            \n[n2]{n2}{0,0};
            \site{r2}{n2.east};
            \node at (26:.42) {\scriptsize $r$};
          \end{scope}
          \begin{scope}[shift={(2.1,0)}]
            \n[n3]{n3}{0,0};
            \site{l3}{n3.west};
            \node at (206:.42) {\scriptsize $l$};
          \end{scope}
        }};

      %%% 1-2-3 %%%
      \node[grphnode] (g1-4) at (-135:3) {
        \tikz[ingrphdiag]{
          \e{0,0}{2.2,0};
          \begin{scope}
            \n[n1]{n1}{0,0};
            \site{r1}{n1.east};
            \node at (26:.42) {\scriptsize $r$};
          \end{scope}
          \begin{scope}[shift={(1.1,0)}]
            \n[n2]{n2}{0,0};
            \site{l2}{n2.west};
            \site{r2}{n2.east};
            \node at (206:.42) {\scriptsize $l$};
            \node at (26:.42) {\scriptsize $r$};
          \end{scope}
          \begin{scope}[shift={(2.2,0)}]
            \n[n3]{n3}{0,0};
            \site{l3}{n3.west};
            \node at (206:.42) {\scriptsize $l$};
          \end{scope}
        }};

      %%% 2-3-1 %%%
      \node[grphnode] (g2-4) at (-45:3) {
        \tikz[ingrphdiag]{
          \e{0,0}{2.2,0};
          \begin{scope}
            \n[n2]{n2}{0,0};
            \site{r2}{n2.east};
            \node at (26:.42) {\scriptsize $r$};
          \end{scope}
          \begin{scope}[shift={(1.1,0)}]
            \n[n3]{n3}{0,0};
            \site{l3}{n3.west};
            \site{r3}{n3.east};
            \node at (206:.42) {\scriptsize $l$};
            \node at (26:.42) {\scriptsize $r$};
          \end{scope}
          \begin{scope}[shift={(2.2,0)}]
            \n[n1]{n1}{0,0};
            \site{l1}{n1.west};
            \node at (206:.42) {\scriptsize $l$};
          \end{scope}
        }};

      %%% Triangle %%%
      \node[grphnode,anchor=north] (po4) at (-90:4.2426) {
        \tikz[ingrphdiag]{
          \path[use as bounding box] (-.3,.38) rectangle (1.4,-1.22);
          \e{0,0}{0:1.1};
          \e{0,0}{-60:1.1};
          \e{0:1.1}{-60:1.1};
          \begin{scope}[shift={(0,0)}]
            \n[n1]{x}{0,0};
            \site{r1}{0:7pt};
            \site{l1}{-60:7pt};
            \node at (-86:12pt) {\scriptsize $l$};
            \node at (26:12pt) {\scriptsize $r$};
          \end{scope}
          \begin{scope}[shift={(0:1.1)}]
            \n[n2]{y}{0,0};
            \site{r2}{180:7pt};
            \site{l2}{-120:7pt};
            \node at (154:12pt) {\scriptsize $l$};
            \node at (-94:12pt) {\scriptsize $r$};
          \end{scope}
          \begin{scope}[shift={(-60:1.1)}]
            \n[n3]{z}{0,0};
            \site{r3}{120:7pt};
            \site{l3}{60:7pt};
            \node at (146:12pt) {\scriptsize $r$};
            \node at (34:12pt) {\scriptsize $l$};
          \end{scope}
        }};

      \arrsn[opacity=.7]{pb4}{g1-4};
      \arrsn[opacity=.7]{pb4}{g2-4};
      \arrsn[opacity=.7]{g1-4}{po4};
      \arrsn[opacity=.7]{g2-4}{po4};
    \end{scope}

  \end{tikzpicture}}%
\end{center}

%%% Local Variables:
%%% mode: latex
%%% TeX-master: "thesis"
%%% End:


I have implemented an online tool that computes minimal glueings
available at \url{https://rhz.github.com/thesis/mg.html}.
Its source code can be found at \url{https://github.com/rhz/thesis/}.

Using minimal glueings we can test whether
a rule $r$ is $\shapes$-balanced,
that is, whether $r$ consumes and produces
the same number of instances of each energy pattern $p$
when applied to any mixture $m$.
In particular, for an $r$-event $\psi$
to \emph{consume} an instance $\phi$ of $p$ in a mixture $m$,
$\phi_\sites$ and $\psi_\sites$ must have images
which intersect on at least one site which is modified by $r$
(\eg by adding an edge if it was free). % or removing its edge).
% Otherwise the energy pattern is left intact by the action of the rule.
This is the case iff
the minimal glueing $\phi',\psi'$ of $r_L$ and $p$
\begin{wrapfigure}[5]{r}{0.41\textwidth}
  \vspace{-1.8em}
  \begin{equation}
    \label{eq:p-balanced}
    \tikz[baseline=-1.1cm]{
  % \begin{center}
  %   \begin{tikzpicture}
      \node (p) at (0,0) {$p$};
      \node (s) at (1.8,0) {$s$};
      \node (l) at (3.6,0) {$r_L$};
      \node (m) at (1.8,-1.8) {$m$};
      \draw[hom] (p) -- node[above] {$\phi'$} (s);
      \draw[hom] (l) -- node[above] {$\psi'$} (s);
      \draw[hom] (p) -- node[below left] {$\phi$} (m);
      \draw[hom] (l) -- node[below right] {$\psi$} (m);
      \draw[hom,dotted] (s) -- (m);}
  %   \end{tikzpicture}
  % \end{center}
    \end{equation}
\end{wrapfigure}
that factors the cospan $\phi,\psi$ has the same property.
Likewise, for an $r$-event to \emph{produce} an instance of $p$,
the associated minimal glueing between $p$ and $r_R$
must have a modified intersection.
We call such minimal glueings \emph{relevant}.
% ; they are the ones which underlie events
% that can affect the instances of $p$.

To illustrate the idea of relevant minimal glueings,
let us consider a different example.
In this example, the contact graph is very simple:
just one agent type with two sites, $l$ and $r$,
that can bind each other.
% The maximally permissive set of generators rules
% contains only one reversible rule.
% One extension of this rule is
Imagine we have the following reversible rule.
\begin{center}
  \begin{tikzpicture}[thick]
    \node[grphnode,anchor=east] (lhs1) at (0,0) {
      \tikz[ingrphdiag]{
        \e{0,0}{2.2,0};
        \begin{scope}
          \n[n]{x}{0,0};
          \site{rx}{x.east};
          \node at (26:.42) {\scriptsize $r$};
        \end{scope}
        \begin{scope}[shift={(1.1,0)}]
          \n[n]{y}{0,0};
          \site{ly}{y.west};
          \site{ry}{y.east};
          \node at (206:.42) {\scriptsize $l$};
          \node at (26:.42) {\scriptsize $r$};
        \end{scope}
        \begin{scope}[shift={(2.2,0)}]
          \n[n]{z}{0,0};
          \site{lz}{z.west};
          \node at (206:.42) {\scriptsize $l$};
        \end{scope}
      }};
    \path (lhs1.east) +(.3,0) edge[rule] +(1,0)
      +(1.3,0) coordinate (r1);
    \node[grphnode,anchor=west] (rhs1) at (r1) {
      \tikz[ingrphdiag]{
        \e{1.1,0}{2.3,0};
        \begin{scope}[shift={(0,0)}]
          \n[n]{x}{0,0};
          \e{x}{.5,0};
          \site{rx}{x.east};
          \node at (26:.42) {\scriptsize $r$};
        \end{scope}
        \begin{scope}[shift={(1.2,0)}]
          \n[n]{y}{0,0};
          \e{y}{-.5,0};
          \site{ly}{y.west};
          \site{ry}{y.east};
          \node at (206:.42) {\scriptsize $l$};
          \node at (26:.42) {\scriptsize $r$};
        \end{scope}
        \begin{scope}[shift={(2.3,0)}]
          \n[n]{z}{0,0};
          \site{lz}{z.west};
          \node at (206:.42) {\scriptsize $l$};
        \end{scope}
      }};
  \end{tikzpicture}
\end{center}
Take the chain of 3 agents as our energy pattern.
The minimal glueings of the left-hand side of the rule
with the energy pattern are shown below.
On the left of each diagram is the energy pattern.
The relevant minimal glueings are marked
with a light green background.
% \tikzstyle{site}=[font=\scriptsize\itshape,inner sep=1pt,above]
\tikzstyle{s}=[font=\scriptsize,yshift=6pt]
\tikzstyle{emb}=[->,dashed,thin]
\tikzstyle{relevant}=[show background rectangle,
background rectangle/.style={fill=Green!40,rounded corners=4pt}]
\tikzstyle{non-relevant}=[show background rectangle,
background rectangle/.style={fill=White,rounded corners=4pt}]

% Draw a chain of agents
% Parameters:
%  - Position of the first agent
%  - Distance between agents
%  - List of node ids
%  - Name of left site
%  - Name of right site
\newcommand{\createchain}[5]{%
  \draw #1
    \foreach \aid [count=\ai] in {#3} {
      \ifnum \ai = 1
      \else -- node[s,pos=.33] {$#5$} node[s,pos=.65] {$#4$} ++(#2,0)
      \fi
      node[inner sep=5pt,n] (\aid) {} };
  \putsites{#3}
}

\ExplSyntaxOn
\int_new:N \sites_len 
\newcommand{\putsites}[1]{%
  \clist_set:Nn \sites_clist {#1}
  \int_set:Nn \sites_len {\clist_count:N \sites_clist}
  \int_step_inline:nnnn {1}{1}{\sites_len}
  { \edef\aid{\clist_item:Nn \sites_clist {##1}}
    \begin{scope}[shift={(\aid)}]
      \int_compare:nTF { ##1 > 1 }
      { \site{l\aid}{\aid.west}; }{}
      \int_compare:nTF { ##1 < \sites_len }
      { \site{r\aid}{\aid.east}; }{}
    \end{scope}}
}
\ExplSyntaxOff

% \newcommand{\putsites}[1]{
%   \foreach \aid in {#1} {
%     \begin{scope}[shift={(\aid)}]
%       \site{l\aid}{\aid.west};
%       \site{r\aid}{\aid.east};
%     \end{scope}};
% }

% Draw the arrows of embeddings
% Parameters:
%  - List of pairs from/to of node ids
\newcommand{\dembs}[2][.1]{% diagram embeddings
  \foreach \ai / \aj in {#2}
    \draw[emb] ($(\ai)!#1!(\aj.north)$) -- ($(\ai)!.9!(\aj.north)$);
}

% \newcommand{\vsep}{.3cm}
% first row
\begin{center}
\begin{minipage}{.48\textwidth}
  \begin{center} % no overlap
    \begin{tikzpicture}[thick, non-relevant]
      \createchain{(0pt,0pt)}{30pt}{a1,a2,a3}{l}{r};
      \createchain{(90pt,0pt)}{30pt}{a4,a5,a6}{l}{r};
      \createchain{(0pt,-40pt)}{30pt}{a7,a8,a9}{l}{r};
      \createchain{(90pt,-40pt)}{30pt}{a10,a11,a12}{l}{r};
      \dembs[.35]{a1/a7,a2/a8,a3/a9,a4/a10,a5/a11,a6/a12};
    \end{tikzpicture}
  \end{center}
\end{minipage}
\begin{minipage}{.48\textwidth}
  \begin{center} % all overlap
    \begin{tikzpicture}[thick, relevant]
      \createchain{(0pt,0pt)}{30pt}{a1,a2,a3}{l}{r};
      \createchain{(90pt,0pt)}{30pt}{a4,a5,a6}{l}{r};
      \createchain{(45pt,-40pt)}{30pt}{a7,a8,a9}{l}{r};
      \dembs[.25]{a1/a7,a4/a7,a2/a8,a5/a8,a3/a9,a6/a9};
    \end{tikzpicture}
  \end{center}
\end{minipage}
\end{center}
% second row
% \vspace{\vsep}
\begin{center}
\begin{minipage}{.48\textwidth}
  \begin{center} % two overlap
    \begin{tikzpicture}[thick, relevant]
      \createchain{(0pt,0pt)}{30pt}{a1,a2,a3}{l}{r};
      \createchain{(90pt,0pt)}{30pt}{a4,a5,a6}{l}{r};
      \createchain{(30pt,-40pt)}{30pt}{a7,a8,a9,a10}{l}{r};
      \dembs[.28]{a1/a7,a4/a8,a2/a8,a5/a9,a3/a9,a6/a10};
    \end{tikzpicture}
  \end{center}
\end{minipage}
\begin{minipage}{.48\textwidth}
  \begin{center} % two overlap
    \begin{tikzpicture}[thick, non-relevant]
      \createchain{(0pt,0pt)}{30pt}{a1,a2,a3}{l}{r};
      \createchain{(90pt,0pt)}{30pt}{a4,a5,a6}{l}{r};
      \createchain{(30pt,-40pt)}{30pt}{a7,a8,a9,a10}{l}{r};
      \dembs[.2]{a4/a7,a1/a8,a5/a8,a2/a9,a6/a9,a3/a10};
    \end{tikzpicture}
  \end{center}
\end{minipage}
\end{center}
% third row
% \vspace{\vsep}
\begin{center}
\begin{minipage}{.48\textwidth}
  \begin{center} % one overlap
    \begin{tikzpicture}[thick, non-relevant]
      \createchain{(0pt,0pt)}{30pt}{a1,a2,a3}{l}{r};
      \createchain{(90pt,0pt)}{30pt}{a4,a5,a6}{l}{r};
      \createchain{(15pt,-40pt)}{30pt}{a7,a8,a9,a10,a11}{l}{r};
      \dembs[.31]{a1/a7,a2/a8,a3/a9,a4/a9,a5/a10,a6/a11};
    \end{tikzpicture}
  \end{center}
\end{minipage}
\begin{minipage}{.48\textwidth}
  \begin{center} % one overlap
    \begin{tikzpicture}[thick, non-relevant]
      \createchain{(0pt,0pt)}{30pt}{a1,a2,a3}{l}{r};
      \createchain{(90pt,0pt)}{30pt}{a4,a5,a6}{l}{r};
      \createchain{(15pt,-40pt)}{30pt}{a7,a8,a9,a10,a11}{l}{r};
      \dembs[.18]{a4/a7,a5/a8,a6/a9,a1/a9,a2/a10,a3/a11};
    \end{tikzpicture}
  \end{center}
\end{minipage}
\end{center}
% fourth row
% \vspace{\vsep}
\begin{center}
\begin{minipage}{.48\textwidth}
  \begin{center} % triangle
    \begin{tikzpicture}[thick, relevant]
      \createchain{(0pt,0pt)}{30pt}{a1,a2,a3}{l}{r};
      \createchain{(90pt,0pt)}{30pt}{a4,a5,a6}{l}{r};

      \begin{scope}[shift={(60pt,-40pt)}]
        \e{0,0}{0:1.1};
        \e{0,0}{-60:1.1};
        \e{0:1.1}{-60:1.1};
        \begin{scope}[shift={(0,0)}]
          \n[n]{a7}{0,0};
          \site{r7}{0:7pt};
          \site{l7}{-60:7pt};
          \node at (-86:12pt) {\scriptsize $l$};
          \node at (26:12pt) {\scriptsize $r$};
        \end{scope}
        \begin{scope}[shift={(0:1.1)}]
          \n[n]{a8}{0,0};
          \site{r8}{180:7pt};
          \site{l8}{-120:7pt};
          \node at (154:12pt) {\scriptsize $l$};
          \node at (-94:12pt) {\scriptsize $r$};
        \end{scope}
        \begin{scope}[shift={(-60:1.1)}]
          \n[n]{a9}{0,0};
          \site{r9}{120:7pt};
          \site{l9}{60:7pt};
          \node at (146:12pt) {\scriptsize $r$};
          \node at (34:12pt) {\scriptsize $l$};
        \end{scope}
      \end{scope}

      \dembs[.3]{a2/a7,a3/a8,a4/a7,a5/a8};
      \draw[emb] ($(a1)+(-65:12pt)$) to [out=-65,in=180] ($(a9)+(180:12pt)$);
      \draw[emb] ($(a6)+(245:12pt)$) to [out=245,in=0]   ($(a9)+(0:12pt)$);
    \end{tikzpicture}
  \end{center}
\end{minipage}
\begin{minipage}{.48\textwidth}
  \begin{center} % triangle
    \begin{tikzpicture}[thick, non-relevant]
      \createchain{(0pt,0pt)}{30pt}{a1,a2,a3}{l}{r};
      \createchain{(90pt,0pt)}{30pt}{a4,a5,a6}{l}{r};

      \begin{scope}[shift={(60pt,-40pt)}]
        \e{0,0}{0:1.1};
        \e{0,0}{-60:1.1};
        \e{0:1.1}{-60:1.1};
        \begin{scope}[shift={(0,0)}]
          \n[n]{a7}{0,0};
          \site{r7}{0:7pt};
          \site{l7}{-60:7pt};
          \node at (-86:12pt) {\scriptsize $l$};
          \node at (26:12pt) {\scriptsize $r$};
        \end{scope}
        \begin{scope}[shift={(0:1.1)}]
          \n[n]{a8}{0,0};
          \site{r8}{180:7pt};
          \site{l8}{-120:7pt};
          \node at (154:12pt) {\scriptsize $l$};
          \node at (-94:12pt) {\scriptsize $r$};
        \end{scope}
        \begin{scope}[shift={(-60:1.1)}]
          \n[n]{a9}{0,0};
          \site{r9}{120:7pt};
          \site{l9}{60:7pt};
          \node at (146:12pt) {\scriptsize $r$};
          \node at (34:12pt) {\scriptsize $l$};
        \end{scope}
      \end{scope}

      \dembs[.2]{a5/a7,a6/a8,a1/a7,a2/a8};
      \draw[emb] ($(a3)+(240:12pt)$) to [out=240,in=180] ($(a9)+(180:12pt)$);
      \draw[emb] ($(a4)+(-60:12pt)$) to [out=-60,in=0]   ($(a9)+(0:12pt)$);
    \end{tikzpicture}
  \end{center}
\end{minipage}
\end{center}
% \end{minipage}
% \begin{minipage}{.48\textwidth}
%   \begin{center} % all overlap
%     \begin{tikzpicture}[thick, relevant]
%       \createchain{(0pt,0pt)}{30pt}{a1,a2,a3}{l}{r};
%       \createchain{(90pt,0pt)}{30pt}{a4,a5,a6}{l}{r};
%       \createchain{(45pt,-40pt)}{30pt}{a7,a8,a9}{l}{r};
%       \dembs[.25]{a1/a7,a4/a7,a2/a8,a5/a8,a3/a9,a6/a9};
%     \end{tikzpicture}
%   \end{center}
%   \vspace{\vsep}
%   \begin{center} % two overlap
%     \begin{tikzpicture}[thick, non-relevant]
%       \createchain{(0pt,0pt)}{30pt}{a1,a2,a3}{l}{r};
%       \createchain{(90pt,0pt)}{30pt}{a4,a5,a6}{l}{r};
%       \createchain{(30pt,-40pt)}{30pt}{a7,a8,a9,a10}{l}{r};
%       \dembs[.2]{a4/a7,a1/a8,a5/a8,a2/a9,a6/a9,a3/a10};
%     \end{tikzpicture}
%   \end{center}
%   \vspace{\vsep}
%   \begin{center} % one overlap
%     \begin{tikzpicture}[thick, non-relevant]
%       \createchain{(0pt,0pt)}{30pt}{a1,a2,a3}{l}{r};
%       \createchain{(90pt,0pt)}{30pt}{a4,a5,a6}{l}{r};
%       \createchain{(15pt,-40pt)}{30pt}{a7,a8,a9,a10,a11}{l}{r};
%       \dembs[.18]{a4/a7,a5/a8,a6/a9,a1/a9,a2/a10,a3/a11};
%     \end{tikzpicture}
%   \end{center}
%   \vspace{\vsep}
%   \begin{center} % triangle
%     \begin{tikzpicture}[thick, non-relevant]
%       \createchain{(0pt,0pt)}{30pt}{a1,a2,a3}{l}{r};
%       \createchain{(90pt,0pt)}{30pt}{a4,a5,a6}{l}{r};
%
%       \begin{scope}[shift={(60pt,-40pt)}]
%         \e{0,0}{0:1.1};
%         \e{0,0}{-60:1.1};
%         \e{0:1.1}{-60:1.1};
%         \begin{scope}[shift={(0,0)}]
%           \n[n]{a7}{0,0};
%           \site{r7}{0:7pt};
%           \site{l7}{-60:7pt};
%           \node at (-86:12pt) {\scriptsize $l$};
%           \node at (26:12pt) {\scriptsize $r$};
%         \end{scope}
%         \begin{scope}[shift={(0:1.1)}]
%           \n[n]{a8}{0,0};
%           \site{r8}{180:7pt};
%           \site{l8}{-120:7pt};
%           \node at (154:12pt) {\scriptsize $l$};
%           \node at (-94:12pt) {\scriptsize $r$};
%         \end{scope}
%         \begin{scope}[shift={(-60:1.1)}]
%           \n[n]{a9}{0,0};
%           \site{r9}{120:7pt};
%           \site{l9}{60:7pt};
%           \node at (146:12pt) {\scriptsize $r$};
%           \node at (34:12pt) {\scriptsize $l$};
%         \end{scope}
%       \end{scope}
%
%       \dembs[.2]{a5/a7,a6/a8,a1/a7,a2/a8};
%       \draw[emb] ($(a3)+(240:12pt)$) to [out=240,in=180] ($(a9)+(180:12pt)$);
%       \draw[emb] ($(a4)+(-60:12pt)$) to [out=-60,in=0]   ($(a9)+(0:12pt)$);
%     \end{tikzpicture}
%   \end{center}
% \end{minipage}
% \end{center}
%
% fifth row
% \vspace{\vsep}
\begin{center} % square
  \begin{tikzpicture}[thick, non-relevant]
    \createchain{(0pt,0pt)}{30pt}{a1,a2,a3}{l}{r};
    \createchain{(90pt,0pt)}{30pt}{a4,a5,a6}{l}{r};

    \begin{scope}[shift={(75pt,-40pt)}]
      \draw (0,0) -- ++(-30:1.1) coordinate (c8)
                  -- ++(-150:1.1) coordinate (c9)
                  -- ++(150:1.1) coordinate (c10)
                  -- cycle;
      \begin{scope}[shift={(0,0)}]
        \n[n]{a7}{0,0};
        \site{r7}{-30:7pt};
        \site{l7}{-150:7pt};
        \node at (-176:12pt) {\scriptsize $l$};
        \node at (-4:12pt) {\scriptsize $r$};
      \end{scope}
      \begin{scope}[shift={(c8)}]
        \n[n]{a8}{0,0};
        \site{r8}{150:7pt};
        \site{l8}{-150:7pt};
        \node at (124:12pt) {\scriptsize $l$};
        \node at (-124:12pt) {\scriptsize $r$};
      \end{scope}
      \begin{scope}[shift={(c9)}]
        \n[n]{a9}{0,0};
        \site{r9}{150:7pt};
        \site{l9}{30:7pt};
        \node at (4:12pt) {\scriptsize $l$};
        \node at (176:12pt) {\scriptsize $r$};
      \end{scope}
      \begin{scope}[shift={(c10)}]
        \n[n]{a10}{0,0};
        \site{r10}{30:7pt};
        \site{l10}{-30:7pt};
        \node at (-56:12pt) {\scriptsize $l$};
        \node at (56:12pt) {\scriptsize $r$};
      \end{scope}
    \end{scope}

    \draw[emb] ($(a1)+(-55:12pt)$) to ($(a10)+(125:12pt)$);
    \draw[emb] ($(a2)+(-40:12pt)$) to ($(a7)+(140:12pt)$);
    \draw[emb] ($(a3)+(-30:12pt)$) to [out=-30,in=100] ($(a8)+(100:12pt)$);
    \draw[emb] ($(a4)+(-60:12pt)$) to [out=-60,in=80 ] ($(a8)+(80:12pt)$);
    \draw[emb] ($(a5)+(-70:12pt)$) to [out=-70,in=-30] ($(a9)+(-30:12pt)$);
    \draw[emb] ($(a6)+(210:12pt)$) to [out=210,in=90 ] ($(a10)+(90:12pt)$);
  \end{tikzpicture}
\end{center}


Whenever $\psi': r_L \to s$ in \diagram{p-balanced} is an iso,
then the energy pattern $p$ is fully included % contained
in the left-hand side of rule $r$.
This implies the rule contains all the relevant context needed
to make sure that an instance of $p$ is consumed
by any $r$-event $\psi: r_L \to m$.
We say that $r$ is $\shapes$-\emph{left-balanced} iff,
for all $p \in \shapes$ and relevant minimal glueings
$\theta^i_1: p \to s_i \gets r_L :\theta^i_2$,
the right leg $\theta^i_2$ is an isomorphism.
Symmetrically, one says that $r$ is $\shapes$-\emph{right-balanced}
iff $\inv{r}$ is $\shapes$-left-balanced.
Then $r$ is $\shapes$-\emph{balanced}
iff it is $\shapes$-left- and $\shapes$-right-balanced.

% TODO: where should this lemma be cited?
\begin{lemma}
  Rule $r$ is $\shapes$-balanced if and only if
  $r$ is $\shapes$-left- and $\shapes$-right-balanced.
  Moreover, if $r$ is $\shapes$-balanced then,
  for any mixture $m$, embedding $\psi: r_L \to m$,
  and energy pattern $p \in \shapes$,
  \[ \Delta_r p = |[p;m^{(r,\psi)}]| % \abs{\matches{p}{\comatch{m}}}
                - \abs{\matches{p}{m}}
                = \abs{\matches{p}{r_R}}
                - \abs{\matches{p}{r_L}} \]
\end{lemma}
\begin{proof}
  Suppose there are two mixtures $m$, $n$
  and embeddings $\psi: r_L \to m$, $\phi: r_L \to n$
  such that, when $r$ is applied to $\psi$ and $\phi$,
  it has a different balance
  with respect to a pattern $p \in \shapes$,
  \ie $|[p;m^{(r,\psi)}]| - \abs{\matches{p}{m}} \neq
  |[p;n^{(r,\phi)}]| - \abs{\matches{p}{n}}$.
  %
  We have
  \begin{equation*}
    \abs{\matches{p}{m}} = |\{p \to m \getsby{\psi} r_L\}|
    = \abs{\set{\tikz[baseline=-.6cm,x=1.2cm,y=1.2cm]{
      \node (p) at (0,0) {$p$};
      \node (s) at (1,0) {$s$};
      \node (l) at (2,0) {$r_L$};
      \node (m) at (1,-1) {$m$};
      \draw[hom] (p) -- (s);
      \draw[hom] (p) -- (m);
      \draw[hom] (l) -- (s);
      \draw[hom] (l) -- node[below right] {$\psi$} (m);
      \draw[hom,dotted] (s) -- (m);}}}
  \end{equation*}
  where $p \to s \gets r_L$ is the minimal glueing
  that factors the cospan $p \to m \getsby{\psi} r_L$.
  A similar equality can be obtained for $r_R$,
  $m^{(r,\psi)}$ and $\comatch{\psi}$.
  %
  The \emph{irrelevant} minimal glueings on each side of the rule
  are in bijection: the rule does not destroy nor create them.
  Hence, when taking the difference
  $|[p;m^{(r,\psi)}]| - \abs{\matches{p}{m}}$
  they cancel each other out and we are left with
  a difference of \emph{relevant} minimal glueings on each side.
  %
  Since $s \iso r_L$ for each relevant minimal glueing on the left
  then
  \begin{equation*}
    \abs{\set{\tikz[baseline=-.6cm,x=1.2cm,y=1.2cm]{
      \node (p) at (0,0) {$p$};
      \node (s) at (1,0) {$s$};
      \node (l) at (2,0) {$r_L$};
      \node (m) at (1,-1) {$m$};
      \draw[hom] (p) -- (s);
      \draw[hom] (p) -- (m);
      \path (l) -- node[onarrow] {$\iso$} (s);
      \draw[hom] (l) -- node[below right] {$\psi$} (m);
      \draw[hom,dotted] (s) -- (m);}}}
    = \abs{\matches{p}{r_L}}
  \end{equation*}
  Again, a similar equality can be obtained for $r_R$,
  $m^{(r,\psi)}$ and $\comatch{\psi}$.
  Thus we have proved that
  $|[p;m^{(r,\psi)}]| - \abs{\matches{p}{m}} =
  \abs{\matches{p}{r_R}} - \abs{\matches{p}{r_L}}$
  for any $m$ and $\psi$,
  contradicting our original assumption.
\end{proof}


\section{Refinements}
\label{sec:refinements}

A rule is refined into another rule by adding context.
For example, we can add a common neighbour
to the agents in $r^+_{12}$ to obtain a refinement.
% \begin{center}
%   \begin{tikzpicture}
\begin{equation}
  \label{eq:refined1}
  \tikz[baseline=-.16cm]{
    \node[grphnode,anchor=east] (lhs) at (0,0) {
      \tikz[ingrphdiag]{
        \path[use as bounding box] (-.3,.38) rectangle (1.5,-1.22);
        \e{0,0}{-56.944:1.1};
        \e{0:1.2}{-56.944:1.1};
        \begin{scope}[shift={(0,0)}]
          \n[n1]{x}{0,0};
          \e{x}{.5,0};
          \site{r1}{0:7pt};
          \site{l1}{-60:7pt};
          \node at (-86:12pt) {\scriptsize $l$};
          \node at (26:12pt) {\scriptsize $r$};
        \end{scope}
        \begin{scope}[shift={(0:1.2)}]
          \n[n2]{y}{0,0};
          \e{y}{-.5,0};
          \site{r2}{180:7pt};
          \site{l2}{-120:7pt};
          \node at (154:12pt) {\scriptsize $l$};
          \node at (-94:12pt) {\scriptsize $r$};
        \end{scope}
        \begin{scope}[shift={(-56.944:1.1)}]
          \n[n3]{z}{0,0};
          % angle is 66.111 deg
          \site{r3}{123.0555:7pt};
          \site{l3}{56.9445:7pt};
          \node at (146:12pt) {\scriptsize $r$};
          \node at (34:12pt) {\scriptsize $l$};
        \end{scope}
      }};
    \path (lhs.east) +(.3,0) edge[rule] +(1,0)
      +(1.3,0) coordinate (r);
    \node[grphnode,anchor=west] (rhs) at (r) {
      \tikz[ingrphdiag]{
        \path[use as bounding box] (-.3,.38) rectangle (1.4,-1.22);
        \e{0,0}{0:1.1};
        \e{0,0}{-60:1.1};
        \e{0:1.1}{-60:1.1};
        \begin{scope}[shift={(0,0)}]
          \n[n1]{x}{0,0};
          \site{r1}{0:7pt};
          \site{l1}{-60:7pt};
          \node at (-86:12pt) {\scriptsize $l$};
          \node at (26:12pt) {\scriptsize $r$};
        \end{scope}
        \begin{scope}[shift={(0:1.1)}]
          \n[n2]{y}{0,0};
          \site{r2}{180:7pt};
          \site{l2}{-120:7pt};
          \node at (154:12pt) {\scriptsize $l$};
          \node at (-94:12pt) {\scriptsize $r$};
        \end{scope}
        \begin{scope}[shift={(-60:1.1)}]
          \n[n3]{z}{0,0};
          \site{r3}{120:7pt};
          \site{l3}{60:7pt};
          \node at (146:12pt) {\scriptsize $r$};
          \node at (34:12pt) {\scriptsize $l$};
        \end{scope}
      }};
  }
\end{equation}
%   \end{tikzpicture}
% \end{center}
This refinement happens to be $\shapes$-balanced.
Another refinement of $r^+_{12}$ could be
% \begin{center}
%   \begin{tikzpicture}
\begin{equation}
  \label{eq:refined2}
  \tikz[baseline=-.16cm]{
    \node[grphnode,anchor=east] (lhs) at (0,0) {
      \tikz[ingrphdiag]{
        \begin{scope}[shift={(0,0)}]
          \n[n1]{x}{0,0};
          \e{x}{.5,0};
          \site{rx}{x.east};
          \node at (26:.42) {\scriptsize $r$};
        \end{scope}
        \begin{scope}[shift={(1.2,0)}]
          \n[n2]{y}{0,0};
          \e{y}{-.5,0};
          \e{y}{.5,0};
          \site{ly}{y.west};
          \site{ry}{y.east};
          \node at (206:.42) {\scriptsize $l$};
          \node at (26:.42) {\scriptsize $r$};
        \end{scope}
      }};
    \path (lhs.east) +(.3,0) edge[rule] +(1,0)
      +(1.3,0) coordinate (r);
    \node[grphnode,anchor=west] (rhs) at (r) {
      \tikz[ingrphdiag]{
        \e{0,0}{1.1,0};
        \begin{scope}
          \n[n1]{x}{0,0};
          \site{rx}{x.east};
          \node at (26:.42) {\scriptsize $r$};
        \end{scope}
        \begin{scope}[shift={(1.1,0)}]
          \n[n2]{y}{0,0};
          \e{y}{.5,0};
          \site{ly}{y.west};
          \site{ry}{y.east};
          \node at (206:.42) {\scriptsize $l$};
          \node at (26:.42) {\scriptsize $r$};
        \end{scope}
      }};
  }
\end{equation}
%   \end{tikzpicture}
% \end{center}
Here we have added a free site to the blue node.
This second refinement is also $\shapes$-balanced
because the free $r$ site on the blue node guarantees that
(i) the rule will never create a triangle and
(ii) there is no embedding from the left-hand side
into a triangle and hence no triangle can be destroyed
by the action of the rule.
The following refinement, however, is not $\shapes$-balanced.
\begin{center}
  \begin{tikzpicture}
    \node[grphnode,anchor=east] (lhs) at (0,0) {
      \tikz[ingrphdiag]{
        \begin{scope}[shift={(0,0)}]
          \n[n1]{x}{0,0};
          \e{x}{.5,0};
          \site{rx}{x.east};
          \node at (26:.42) {\scriptsize $r$};
        \end{scope}
        \e{1.2,0}{2.3,0};
        \begin{scope}[shift={(1.2,0)}]
          \n[n2]{y}{0,0};
          \e{y}{-.5,0};
          \site{ly}{y.west};
          \site{ry}{y.east};
          \node at (206:.42) {\scriptsize $l$};
          \node at (26:.42) {\scriptsize $r$};
        \end{scope}
        \begin{scope}[shift={(2.3,0)}]
          \n[n3]{z}{0,0};
          \site{lz}{z.west};
          \node at (206:.42) {\scriptsize $l$};
        \end{scope}
      }};
    \path (lhs.east) +(.3,0) edge[rule] +(1,0)
      +(1.3,0) coordinate (r);
    \node[grphnode,anchor=west] (rhs) at (r) {
      \tikz[ingrphdiag]{
        \e{0,0}{2.2,0};
        \begin{scope}[shift={(0,0)}]
          \n[n1]{x}{0,0};
          \site{rx}{x.east};
          \node at (26:.42) {\scriptsize $r$};
        \end{scope}
        \begin{scope}[shift={(1.1,0)}]
          \n[n2]{y}{0,0};
          \site{ly}{y.west};
          \site{ry}{y.east};
          \node at (206:.42) {\scriptsize $l$};
          \node at (26:.42) {\scriptsize $r$};
        \end{scope}
        \begin{scope}[shift={(2.2,0)}]
          \n[n3]{z}{0,0};
          \site{lz}{z.west};
          \node at (206:.42) {\scriptsize $l$};
        \end{scope}
      }};
  \end{tikzpicture}
\end{center}

We add context to a rule $r = \tuple{r_L,r_R}$
by applying the rule to an embedding $\psi: r_L \to g$.
This operation is well-defined
even if the codomain of the embedding is not a mixture.
% The result of the rewrite $g^{(r,\psi)}$
The pair of contact maps $(g,g^{(r,\psi)})$
% with $g^{(r,\psi)}$ the result of the rewrite
is itself a valid rule
since they only differ in their edge structure.
In this way, an extension of a rule
is determined uniquely by an embedding.

% TODO: pagebreaks are ugly
\pagebreak

Epis\footnote{
  Epi, mono and iso are short for
  epimorphism, monomorphism and isomorphism.}
of $\rSGe_C$ are good candidates for extensions. % relevant
They are characterised as follows:
an embedding $\psi: g \to h$ is an epi iff
every connected component of $\anon{h}$ contains
at least one agent in the image of $\psi_\agents$.
This ensures that no new connected component is added to the rule
while extending it.
However, for technical reasons
that will become apparent in \thm{unique-decomposition},
we use prefixes of epis
instead of epis to extend rules ---
an embedding $\psi: g \to h$ is said to be
a \emph{prefix} of $\phi: g \to h'$
if there is some embedding $\theta: h \to h'$
that makes the composition of $\psi$ and $\theta$ equal to $\phi$
(\ie $\theta \, \psi = \phi$) % psychology + tetas = philosophy
and write $\psi \leq \phi$ for this.
We refer to a prefix
\begin{wrapfigure}[5]{r}{0.27\textwidth}
  \vspace{-2em}
  \begin{center}
    \begin{tikzpicture}
      \matrix (m) [matrix of math nodes,row sep=25pt,column sep=25pt] {
        & g & \\
        h & & h' \\};
      \draw[hom] (m-1-2) -- node[above left] {$\psi$} (m-2-1);
      \draw[hom] (m-1-2) -- node[above right] {$\phi$} (m-2-3);
      \draw[hom] (m-2-1) -- node[below] {$\theta$} (m-2-3);
    \end{tikzpicture}
  \end{center}
\end{wrapfigure}
of an epi $\psi: g \to h$ as an \emph{extension} of $g$.
In the category of extensions of $g$,
a morphism between objects $\psi: g \to h$ and $\phi: g \to h'$
is an embedding $\theta: h \to h'$
such that the triangle on the right commutes.
If $\theta$ is an iso we write $\psi \cong_g \phi$.

One might wonder when the prefix of an epi is not itself an epi.
The following diagram illustrates such a situation,
% where $\theta$ is the witness of $\psi \leq \phi$.
where $\psi$ is a prefix of epi $\phi$
but is not itself an epi since the connected component
of the blue node in the codomain of $\psi$
is not in the image of $\psi_\agents$.
\begin{center}
  \begin{tikzpicture}
    \node[grphnode,outer sep=.3cm] (g) at (0,0) {
      \tikz[ingrphdiag]{
        \n[n1]{x}{0,0};
      }};
    \node[grphnode,outer sep=.3cm] (h) at (-135:3) {
      \tikz[ingrphdiag]{
        \n[n1]{x}{0,0};
        \n[n2]{y}{.9,0};
      }};
    \node[grphnode,outer sep=.3cm] (h') at (-45:3) {
      \tikz[ingrphdiag,outer sep=0]{
        \e{0,0}{1.1,0};
        \begin{scope}[shift={(0,0)}]
          \n[n1]{x}{0,0};
          \site{rx}{x.east};
          \node at (26:.42) {\scriptsize $r$};
        \end{scope}
        \begin{scope}[shift={(1.1,0)}]
          \n[n2]{y}{0,0};
          \site{ly}{y.west};
          \node at (206:.42) {\scriptsize $l$};
        \end{scope}
      }};
    \path (g) edge[rule] node[above left] {$\psi$} (h);
    \path (g) edge[rule] node[above right] {$\phi$} (h');
    \path (h) edge[rule] node[below] {$\theta$} (h');
  \end{tikzpicture}
\end{center}

% workaround to push the next lonely sentence to the next page
% \bigskip

Rule application preserves epis
and in fact also prefixes of epis:
\begin{lemma}
  \label{lemma:epi-prefix}
  Let $r = \tuple{r_L,r_R}$ be a rule
  and $\psi: r_L \to g$ be an embedding
  with $r_L,r_R,g$ contact maps in $\rSGe_C$.
  The embedding $\comatch{\psi}: r_R \to \comatch{g}$
  that results from applying $r$ to $\psi$
  is a prefix of an epi iff $\psi$ is.
\end{lemma}
\begin{proof}
  % Here we just prove that rule application preserves epis.
  % For prefixes of epis we have to make sure that
  % the mediating arrow (ie the witness of \phi \geq \psi)
  % is preserved as well.
  % This works because the new connected components in g
  % (added in the codomain of the prefix of epi)
  % are then connected to those in r_L through sites
  % that are not involved in the action of the rule,
  % since an edge addition requires the sites to be free
  % and an edge deletion requires them to be bound
  % but in no case they can be used to bind
  % the new connected components.
  % Embeddings preserve edges and free sites
  % so the sites involved in the action of the rule
  % have to be mentioned in the codomain of the prefix of epi.
  % Because rule applications will leave everything else intact
  % the mediating arrow is preserved.
  This amounts to proving that
  some embedding $\comatch{\phi} \geq \comatch{\psi}$
  is an epi if there is an epi $\phi \geq \psi$;
  the converse is true by symmetry of rules.
  For this it is enough to consider the case
  where the rule adds or deletes exactly one edge
  since rules that modify more than one edge at a time
  can be decomposed as sequences of deletions and insertions of edges;
  given that each deletion and insertion preserves the property,
  the sequence will preserve it as well.

  The case of adding an edge is easy as the image of $\comatch{\phi}$
  has fewer connected components to intersect than $\phi$.
  The case where $r$ deletes an edge
  can introduce new connected components,
  however in this case both ends $u,v$
  of the deleted edge must be in $r_L$,
  so whether the deletion disconnects or not the codomain of $\psi$,
  the components of $\comatch{\phi}(u)$ and $\comatch{\phi}(v)$
  will have a pre-image, namely $u$ and $v$.
\end{proof}

It follows that the category of extensions
of $r_L$ and $r_R$ are isomorphic.
Hence, any extension $\phi$ to a rule $r$ can be mapped to
an extension of its inverse rule $\inv{r}$.

A family of epis $\phi_i: g \to g_i$ \emph{uniquely decomposes} $g$,
or is a \emph{refinement} of $g$, if,
for all mixtures $m$ and embeddings $\psi: g \to m$,
there exists a unique $i$ and $\psi'$ such that $\psi = \psi' \phi_i$.
%; uniqueness of $i$ prevents the $\phi_i$s from overlapping.
%; since $\phi_i$ is an epi, there can be at most one such $\psi$.
This is the basic requirement
for a reasonable notion of rule refinement:
it guarantees that the left-hand side $g$ of a given rule
splits into a non-overlapping and exhaustive collection
of more specific cases $g_i$.

% For the partitioning of rules
% we need a guiding principle.

A method to easily construct such decompositions
was proposed by \citet{refinement}
which works by detailing
which agents and sites should be added to $g$.
This «extension plan» is called growth policy.
A \emph{growth policy} $\gp$ for contact map $g$ over $C$
is a family of functions $\gp_\phi$,
indexed by all extensions $\phi: g \to h$,
where $\gp_\phi$ maps $u \in \agents_{\anon{h}}$ to
a subset $\gp_\phi(u)$ of $\sitemap_C^{-1}(h_\agents(u))$,
\ie each agent in $\anon{h}$ is allocated
a subset of the sites belonging to the agent type $h_\agents(u)$
it is mapped to in the contact graph.
%
An agent in $\anon{h}$ may cover some, or all,
of these sites or even completely extraneous sites:
\begin{enumerate}[label={(\roman*)}]
\item % if the former, \ie
if for all $u$ in $\agents_{\anon{h}}$,
$h_\sites(\sitemap_{\anon{h}}^{-1}(u)) \subseteq \gp_\phi(u)$,
we say that $\phi$ is \emph{immature};
\item if for all $u$ the inclusion is an equality
and $\phi$ is an epi,
% $h_\sites(\sitemap_{\anon{h}}^{-1}(u)) = \gp_\phi(u)$,
% we say that
$\phi$ is \emph{mature};
\item otherwise $\phi$ is said to be \emph{overgrown}.
\end{enumerate}
The functions $\gp_\phi$ must satisfy,
for all extensions $\phi$ and $\phi' \geq \phi$,
the \emph{faithfulness} property,
$\gp_\phi = \gp_{\phi'} \, \psi_\agents$
with $\psi$ such that $\psi \, \phi = \phi'$;
so a site requested by $\phi$
must be requested by any further extension.
Additionally, this property forces $\gp$ to eagerly ask
for all sites that will be eventually requested
at any given agent in the codomain of $\phi$.
If $\phi$ is not overgrown
then no $\phi' \leq \phi$ is overgrown either.
% Also, note that the union of two growth policies
% is itself a growth policy.

Given a contact map $g$ over $C$ and a growth policy $\gp$ for $g$,
we define $\gp(g)$ by choosing one representative
per $\cong_g$-isomorphism class of the set of all extensions of $g$
which are mature according to $\gp$.

The following theorem guarantees that
factorisations through $\gp(g)$ are unique when they exist,
but \emph{not} that they necessarily do exist.
In section \sct{energy-gp},
we will construct a specific growth policy % of interest
% for which this property of exhaustivity of the decomposition
for which the exhaustivity of the decomposition
can be proved by hand.
As such, it fulfils our desired criteria of providing
an exhaustive collection of mutually exclusive subcases.

\begin{theorem}
  \label{thm:unique-decomposition}
  % If $\gp$ is a growth policy for $g$,
  % $\gp(g)$ uniquely decomposes $g$.
  Let $g$ and $m$ be contact maps over $C$
  and $\gp$ a growth policy for $g$.
  If an embedding $\psi: g \to m$ can be decomposed
  in two ways as $\gamma_1 \phi_1$ and $\gamma_2 \phi_2$
  with $\phi_i: g \to h_i$ in $\gp(g)$ and $\gamma_i: h_i \to m$,
  then $\phi_1 = \phi_2$ and $\gamma_1 = \gamma_2$.
  \begin{equation}
    \label{eq:gp}
    \tikz[baseline=-4.3]{
      \matrix (m) [matrix of math nodes,row sep=25pt,column sep=25pt]{
        g & & & h_1 \\
        & p & & \\
        & & m & \\
        h_2 & & & m \\};
      % outer square
      \draw[hom] (m-1-1) -- node[above] {$\phi_1$} (m-1-4);
      \draw[hom] (m-1-1) -- node[left] {$\phi_2$} (m-4-1);
      \draw[hom] (m-4-1) -- node[below] {$\gamma_2$} (m-4-4);
      \draw[hom] (m-1-4) -- node[right] {$\gamma_1$} (m-4-4);
      % inner square
      \draw[hom] (m-2-2) -- node[onarrow] {$\pi_1$} (m-4-1);
      \draw[hom] (m-2-2) -- node[onarrow] {$\pi_2$} (m-1-4);
      \draw[hom] (m-4-1) -- node[onarrow] {$\theta_1$} (m-3-3);
      \draw[hom] (m-1-4) -- node[onarrow] {$\theta_2$} (m-3-3);
      % mediating arrows
      \draw[hom] (m-1-1) -- node[onarrow] {$\phi$} (m-2-2);
      \draw[hom,dashed] (m-3-3) -- (m-4-4);
    }
  \end{equation}
\end{theorem}
\begin{proof}
  Suppose that $\gamma_1 \phi_1 = \gamma_2 \phi_2$,
  where $\phi_1$ and $\phi_2$ are mature extensions of $g$
  according to $\gp$ and $\phi_1 \neq \phi_2$.
  As shown in \diagram{gp},
  we have an inner square formed by the pullback $\pi_1,\pi_2$,
  and the minimal glueing $\theta_1,\theta_2$ of $h_1,h_2$
  that factors $\gamma_1,\gamma_2$.
  Every connected component of $m$
  has a pre-image in $h_1$ or $h_2$,
  and thus also in $g$,
  since $\phi_1$ and $\phi_2$ are epis
  as mature extensions.
  Because every connected component of $m$
  has an image in $h_1$ and $h_2$,
  then every connected component of $m$
  has a pre-image in both $h_1$ and $h_2$.
  Hence $\theta_1$ and $\theta_2$ are epis.
  % Also $\theta_1$ and $\theta_2$ are epis,
  % as every connected component of $m$
  % has a pre-image in $h_1$ or $h_2$
  % and so also in $g$, since the $\phi_i$s are epis,
  % and so also in the other of $h_2$ and $h_1$.

  The nodes in the images of $\theta_1$ and $\theta_2$
  might be the same or differ.
  When they differ, some site $z$ sitting on a node
  in the intersection of the images of $\theta_1,\theta_2$
  is connected to a node outside the image,
  since $\theta_1,\theta_2$ are epis.
  However, $z$ cannot be in the intersection of the images
  unless the site it is connected to is also part of the intersection
  (\lem{mg}).
  Therefore the nodes in the images must be the same.
  In this case there has to be a site $z$
  that is not in the image of one of them
  or $\theta_1,\theta_2$ are both isos.
  So there must be a pair $u,z$,
  consisting of a node $u$ in $m$
  with pre-images $u_1,u_2$ in $h_1,h_2$
  and a site $z$ of $u$,
  such that $z$ has no pre-image
  in exactly one of $\theta_1,\theta_2$.
  Say it is $\theta_2$.
  Since $\phi_1$ is not overgrown,
  $z \in \gp_{\phi_1}(u_1)$ and, by faithfulness,
  $z \in \gp_\phi(\tuple{u_1,u_2})$,
  where $\tuple{u_1,u_2}$ is
  the pullback pre-image of $u_1$ and $u_2$.
  So again, by faithfulness, $z \in \gp_{\phi_2}(u_2)$
  which contradicts our original assumption.
  Hence, $\theta_1$ and $\theta_2$ are isos.
  It follows that $\phi_1 = \phi_2$ as there is only
  one representative per $\cong_g$-isomorphism class in $\gp(g)$.
  Finally, $\gamma_1 = \gamma_2$ because $\phi_1$ is an epi.
\end{proof}
% NB: the argument uses the faithful condition
% in both directions to push around the $z$ site.

\refstepcounter{markpoint}
\label{p:balance-vector}
Given a rule $r$ and an extension $\phi: r_L \to g$, % of $r_L$,
we write $r_\phi$ for the refined rule associated to $\phi$,
% $r_\phi$ denotes the refined rule associated to $\phi$,
that is, $r_\phi$ is the pair $(g,g^{(r,\phi)})$.
%
Given $\gp$ a growth policy for $r_L$,
we write $\gp(r)$ for the family of rules
obtained by refining $r$ according to $\gp$,
that is, $\gp(r)$ is the family of rules $r_\phi$
for $\phi$ ranging in $\gp(r_L)$.
If $\phi$ is a $\shapes$-balanced extension of $r$,
the refined rule $r_\phi$ has a \emph{balance vector}
in $\ZZ^\shapes$, written $\Delta\phi$,
where, for each $p \in \shapes$,
$\Delta\phi(p)$ is the difference in the number of copies of $p$
produced and consumed by \emph{any} $r_\phi$-event.

An example of growth policy is the \emph{ground} policy
which assigns all possible sites to all agents.
In this case, $\gp(g)$ is simply the set, possibly infinite,
of all epis of $g$ into mixtures, considered up to $\cong_g$.
The ground refinement $\gp(r)$ % of $r$
contains all refinements of $r$ along those epis.
The refined rules therefore manipulate mixtures directly.
It is easy to see that the ground refinement of $r^+_{12}$
in our example is infinite,
since $r^+_{12}$ % each of the three rules
can trigger the extension of a chain of any length.
A similar argument is true for $r^-_{12}$.
Note that ground refinements of a rule $r$
are trivially $\shapes$-balanced but, in general,
the set of refined rules is impractically large or infinite as above.
Instead, the growth policy that we introduce
in the next section % \sct{energy-gp}
will always be finite.


\section{Thermodynamic growth policy} % Energy-based refinement}
\label{sec:energy-gp}

An extension $\phi$ of a rule $r$ is $\shapes$-balanced
if it generates a refined rule $r_\phi$ that is $\shapes$-balanced.
To find such extensions % $\shapes$-balanced extensions of a rule $r$,
it seems natural to use minimal glueings:
take as extensions the right leg $\theta^i_2$
of each relevant minimal glueing
$\theta^i_1: p \to s_i \gets r_L :\theta^i_2$
of $p \in \shapes$ and $r_L$ (or $r_R$).
For instance, the only relevant minimal glueing of
the right-hand side of $r^+_{12}$ and the triangle is
% \begin{center}
%   \resizebox{.36\linewidth}{!}{%
%   \begin{tikzpicture}[thick]
\begin{equation}
  \label{eq:triangle-mg}
  \resizebox{.37\linewidth}{!}{%
  \tikz[baseline=-.16cm]{
    \begin{scope}
      %%% Rhs: 1-2 %%%
      \node[grphnode,anchor=south] (rr) at (150:2.5) {
        \tikz[ingrphdiag]{
          \e{0,0}{1.1,0};
          \begin{scope}
            \n[n1]{n1}{0,0};
            \site{r1}{n1.east};
            \node at (26:.42) {\scriptsize $r$};
          \end{scope}
          \begin{scope}[shift={(1.1,0)}]
            \n[n2]{n2}{0,0};
            \site{l2}{n2.west};
            \node at (206:.42) {\scriptsize $l$};
          \end{scope}
        }};

      %%% Triangle %%%
      \node[grphnode,anchor=south] (p) at (30:2.5) {
        \tikz[ingrphdiag]{
          \path[use as bounding box] (-.3,.36) rectangle (1.4,-1.24);
          \e{0,0}{0:1.1};
          \e{0,0}{-60:1.1};
          \e{0:1.1}{-60:1.1};
          \begin{scope}[shift={(0,0)}]
            \n[n1]{x}{0,0};
            \site{r1}{0:7pt};
            \site{l1}{-60:7pt};
            \node at (-86:12pt) {\scriptsize $l$};
            \node at (26:12pt) {\scriptsize $r$};
          \end{scope}
          \begin{scope}[shift={(0:1.1)}]
            \n[n2]{y}{0,0};
            \site{r2}{180:7pt};
            \site{l2}{-120:7pt};
            \node at (154:12pt) {\scriptsize $l$};
            \node at (-94:12pt) {\scriptsize $r$};
          \end{scope}
          \begin{scope}[shift={(-60:1.1)}]
            \n[n3]{z}{0,0};
            \site{r3}{120:7pt};
            \site{l3}{60:7pt};
            \node at (146:12pt) {\scriptsize $r$};
            \node at (34:12pt) {\scriptsize $l$};
          \end{scope}
        }};

      %%% Triangle %%%
      \node[grphnode,anchor=north] (mg) at (0,0) {
        \tikz[ingrphdiag]{
          \path[use as bounding box] (-.3,.36) rectangle (1.4,-1.24);
          \e{0,0}{0:1.1};
          \e{0,0}{-60:1.1};
          \e{0:1.1}{-60:1.1};
          \begin{scope}[shift={(0,0)}]
            \n[n1]{x}{0,0};
            \site{r1}{0:7pt};
            \site{l1}{-60:7pt};
            \node at (-86:12pt) {\scriptsize $l$};
            \node at (26:12pt) {\scriptsize $r$};
          \end{scope}
          \begin{scope}[shift={(0:1.1)}]
            \n[n2]{y}{0,0};
            \site{r2}{180:7pt};
            \site{l2}{-120:7pt};
            \node at (154:12pt) {\scriptsize $l$};
            \node at (-94:12pt) {\scriptsize $r$};
          \end{scope}
          \begin{scope}[shift={(-60:1.1)}]
            \n[n3]{z}{0,0};
            \site{r3}{120:7pt};
            \site{l3}{60:7pt};
            \node at (146:12pt) {\scriptsize $r$};
            \node at (34:12pt) {\scriptsize $l$};
          \end{scope}
        }};

      \draw[-bigto,opacity=.7]
      ($(rr.south)!.1!(mg.north)$)
      -- node[pos=.4,below left,opacity=1] {$\comatch{\phi}$}
      ($(rr.south)!.9!(mg.north)$);
      % \arrsn[opacity=.7]{rr}{mg};
      \arrsn[opacity=.7]{p}{mg};
    \end{scope}
  }}
\end{equation}
%   \end{tikzpicture}}
% \end{center}
If we use $\phi$ ---
the embedding corresponding to $\comatch{\phi}$
on the left-hand side ---
as an extension of $r^+_{12}$
we obtain rule~\ref{eq:refined1}.
Now, having found the only extension of $r^+_{12}$
that produces a triangle,
we are left with the problem of finding
the extensions that cover the cases when $r^+_{12}$
can be applied without producing a triangle.
Otherwise the decomposition would not be exhaustive;
this is in general the case
when using minimal glueings as extensions.

% Intuitively, one must handle the cases
% when the $l$ site of the orange node
% or the $r$ site of the blue node are free
% (as in rule~\ref{eq:refined2}).
Whenever one of the participating agents in $r^+_{12}$
has a free site in addition to the two free sites
that are bound by the rule,
the formation of a triangle is excluded.
In rule~\ref{eq:refined2}
we added a free $r$ site to the blue node.
The following extesion of $r^+_{12}$ adds
a free $l$ site to the orange node.
% \begin{center}
%   \begin{tikzpicture}
\begin{equation}
  \label{eq:refined3}
  \tikz[baseline=-.16cm]{
    \node[grphnode,anchor=east] (lhs) at (0,0) {
      \tikz[ingrphdiag]{
        \begin{scope}[shift={(0,0)}]
          \n[n1]{x}{0,0};
          \e{x}{-.5,0};
          \e{x}{.5,0};
          \site{lx}{x.west};
          \site{rx}{x.east};
          \node at (206:.42) {\scriptsize $l$};
          \node at (26:.42) {\scriptsize $r$};
        \end{scope}
        \begin{scope}[shift={(1.2,0)}]
          \n[n2]{y}{0,0};
          \e{y}{-.5,0};
          \site{ly}{y.west};
          \node at (206:.42) {\scriptsize $l$};
        \end{scope}
      }};
    \path (lhs.east) +(.3,0) edge[rule] +(1,0)
      +(1.3,0) coordinate (r);
    \node[grphnode,anchor=west] (rhs) at (r) {
      \tikz[ingrphdiag]{
        \e{0,0}{1.1,0};
        \begin{scope}
          \n[n1]{x}{0,0};
          \e{x}{-.5,0};
          \site{lx}{x.west};
          \site{rx}{x.east};
          \node at (206:.42) {\scriptsize $l$};
          \node at (26:.42) {\scriptsize $r$};
        \end{scope}
        \begin{scope}[shift={(1.1,0)}]
          \n[n2]{y}{0,0};
          \site{ly}{y.west};
          \node at (206:.42) {\scriptsize $l$};
        \end{scope}
      }};
  }
\end{equation}
%   \end{tikzpicture}
% \end{center}
Both extensions are minimally $\shapes$-balanced
because any prefix of them that is $\shapes$-balanced
is isomorphic to them as an extension of $r_L$.
We call minimally $\shapes$-balanced extensions \emph{primes}.
Prime extensions are epis since erasing an untouched
connected component in the codomain preserves balance.
However, primes may overlap
as shown by the following rule applications
and therefore do not define in general a valid refinement.
% That is, they do not factorise extensions uniquely.
\begin{center}
  \resizebox{.9\linewidth}{!}{%
  \begin{tikzpicture}[thick]
    % first row
    \node[grphnode,anchor=east] (lhs1) at (0,0) {
      \tikz[ingrphdiag]{
        \begin{scope}[shift={(0,0)}]
          \n[n1]{x}{0,0};
          \e{x}{.5,0};
          \site{rx}{x.east};
          \node at (26:.42) {\scriptsize $r$};
        \end{scope}
        \begin{scope}[shift={(1.2,0)}]
          \n[n2]{y}{0,0};
          \e{y}{-.5,0};
          \e{y}{.5,0};
          \site{ly}{y.west};
          \site{ry}{y.east};
          \node at (206:.42) {\scriptsize $l$};
          \node at (26:.42) {\scriptsize $r$};
        \end{scope}
      }};
    \path (lhs1.east) +(.3,0) edge[rule,dotted] +(1,0)
      +(1.3,0) coordinate (r1);
    \node[grphnode,anchor=west] (rhs1) at (r1) {
      \tikz[ingrphdiag]{
        \e{0,0}{1.1,0};
        \begin{scope}
          \n[n1]{x}{0,0};
          \site{rx}{x.east};
          \node at (26:.42) {\scriptsize $r$};
        \end{scope}
        \begin{scope}[shift={(1.1,0)}]
          \n[n2]{y}{0,0};
          \e{y}{.5,0};
          \site{ly}{y.west};
          \site{ry}{y.east};
          \node at (206:.42) {\scriptsize $l$};
          \node at (26:.42) {\scriptsize $r$};
        \end{scope}
      }};
    % second column
    \node[grphnode,anchor=east] (lhs2) at (8.5,0) {
      \tikz[ingrphdiag]{
        \begin{scope}[shift={(0,0)}]
          \n[n1]{x}{0,0};
          \e{x}{-.5,0};
          \e{x}{.5,0};
          \site{lx}{x.west};
          \site{rx}{x.east};
          \node at (206:.42) {\scriptsize $l$};
          \node at (26:.42) {\scriptsize $r$};
        \end{scope}
        \begin{scope}[shift={(1.2,0)}]
          \n[n2]{y}{0,0};
          \e{y}{-.5,0};
          \site{ly}{y.west};
          \node at (206:.42) {\scriptsize $l$};
        \end{scope}
      }};
    \path (lhs2.east) +(.3,0) edge[rule,dotted] +(1,0)
      +(1.3,0) coordinate (r2);
    \node[grphnode,anchor=west] (rhs2) at (r2) {
      \tikz[ingrphdiag]{
        \e{0,0}{1.1,0};
        \begin{scope}
          \n[n1]{x}{0,0};
          \e{x}{-.5,0};
          \site{lx}{x.west};
          \site{rx}{x.east};
          \node at (206:.42) {\scriptsize $l$};
          \node at (26:.42) {\scriptsize $r$};
        \end{scope}
        \begin{scope}[shift={(1.1,0)}]
          \n[n2]{y}{0,0};
          \site{ly}{y.west};
          \node at (206:.42) {\scriptsize $l$};
        \end{scope}
      }};
    % second row
    \path (lhs1.south) +(0,-.2) edge[rule] +(0,-.6);
    \node[grphnode,anchor=east] (lhs3) at (0,-2) {
      \tikz[ingrphdiag]{
        \begin{scope}[shift={(0,0)}]
          \n[n1]{x}{0,0};
          \e{x}{-.5,0};
          \e{x}{.5,0};
          \site{lx}{x.west};
          \site{rx}{x.east};
          \node at (206:.42) {\scriptsize $l$};
          \node at (26:.42) {\scriptsize $r$};
        \end{scope}
        \begin{scope}[shift={(1.2,0)}]
          \n[n2]{y}{0,0};
          \e{y}{-.5,0};
          \e{y}{.5,0};
          \site{ly}{y.west};
          \site{ry}{y.east};
          \node at (206:.42) {\scriptsize $l$};
          \node at (26:.42) {\scriptsize $r$};
        \end{scope}
      }};
    \path (lhs3.east) +(.3,0) edge[rule,dotted] +(1,0)
      +(1.3,0) coordinate (r3);
    \path (rhs1.south) +(0,-.2) edge[rule] +(0,-.6);
    \node[grphnode,anchor=west] (rhs3) at (r3) {
      \tikz[ingrphdiag]{
        \e{0,0}{1.1,0};
        \begin{scope}
          \n[n1]{x}{0,0};
          \e{x}{-.5,0};
          \site{lx}{x.west};
          \site{rx}{x.east};
          \node at (206:.42) {\scriptsize $l$};
          \node at (26:.42) {\scriptsize $r$};
        \end{scope}
        \begin{scope}[shift={(1.1,0)}]
          \n[n2]{y}{0,0};
          \e{y}{.5,0};
          \site{ly}{y.west};
          \site{ry}{y.east};
          \node at (206:.42) {\scriptsize $l$};
          \node at (26:.42) {\scriptsize $r$};
        \end{scope}
      }};
    % second row, second column
    \path (lhs2.south) +(0,-.2) edge[rule] +(0,-.6);
    \node[grphnode,anchor=east] (lhs4) at (8.5,-2) {
      \tikz[ingrphdiag]{
        \begin{scope}[shift={(0,0)}]
          \n[n1]{x}{0,0};
          \e{x}{-.5,0};
          \e{x}{.5,0};
          \site{lx}{x.west};
          \site{rx}{x.east};
          \node at (206:.42) {\scriptsize $l$};
          \node at (26:.42) {\scriptsize $r$};
        \end{scope}
        \begin{scope}[shift={(1.2,0)}]
          \n[n2]{y}{0,0};
          \e{y}{-.5,0};
          \e{y}{.5,0};
          \site{ly}{y.west};
          \site{ry}{y.east};
          \node at (206:.42) {\scriptsize $l$};
          \node at (26:.42) {\scriptsize $r$};
        \end{scope}
      }};
    \path (lhs4.east) +(.3,0) edge[rule,dotted] +(1,0)
      +(1.3,0) coordinate (r4);
    \path (rhs2.south) +(0,-.2) edge[rule] +(0,-.6);
    \node[grphnode,anchor=west] (rhs4) at (r4) {
      \tikz[ingrphdiag]{
        \e{0,0}{1.1,0};
        \begin{scope}
          \n[n1]{x}{0,0};
          \e{x}{-.5,0};
          \site{lx}{x.west};
          \site{rx}{x.east};
          \node at (206:.42) {\scriptsize $l$};
          \node at (26:.42) {\scriptsize $r$};
        \end{scope}
        \begin{scope}[shift={(1.1,0)}]
          \n[n2]{y}{0,0};
          \e{y}{.5,0};
          \site{ly}{y.west};
          \site{ry}{y.east};
          \node at (206:.42) {\scriptsize $l$};
          \node at (26:.42) {\scriptsize $r$};
        \end{scope}
      }};
  \end{tikzpicture}}
\end{center}

\if0
Additionally, cases like the following have to be handled.
\begin{center}
  \begin{tikzpicture}
    \node[grphnode,anchor=east] (lhs) at (0,0) {
      \tikz[ingrphdiag]{
        \e{0.0,0}{1.1,0};
        \e{2.3,0}{3.4,0};
        \begin{scope}[shift={(0,0)}]
          \n[n3]{w}{0,0};
          \e{w}{-.5,0};
          \site{lw}{w.west};
          \site{rw}{w.east};
          \node at (206:.42) {\scriptsize $l$};
          \node at (26:.42) {\scriptsize $r$};
        \end{scope}
        \begin{scope}[shift={(1.1,0)}]
          \n[n1]{x}{0,0};
          \e{x}{.5,0};
          \site{lx}{x.west};
          \site{rx}{x.east};
          \node at (206:.42) {\scriptsize $l$};
          \node at (26:.42) {\scriptsize $r$};
        \end{scope}
        \begin{scope}[shift={(2.3,0)}]
          \n[n2]{y}{0,0};
          \e{y}{-.5,0};
          \site{ly}{y.west};
          \site{ry}{y.east};
          \node at (206:.42) {\scriptsize $l$};
          \node at (26:.42) {\scriptsize $r$};
        \end{scope}
        \begin{scope}[shift={(3.4,0)}]
          \n[n3]{z}{0,0};
          \e{z}{.5,0};
          \site{lz}{z.west};
          \site{rz}{z.east};
          \node at (206:.42) {\scriptsize $l$};
          \node at (26:.42) {\scriptsize $r$};
        \end{scope}
      }};
    \path (lhs.east) +(.3,0) edge[rule] +(1,0)
      +(1.3,0) coordinate (r);
    \node[grphnode,anchor=west] (rhs) at (r) {
      \tikz[ingrphdiag]{
        \e{0,0}{3.3,0};
        \begin{scope}[shift={(0,0)}]
          \n[n3]{w}{0,0};
          \e{w}{-.5,0};
          \site{lw}{w.west};
          \site{rw}{w.east};
          \node at (206:.42) {\scriptsize $l$};
          \node at (26:.42) {\scriptsize $r$};
        \end{scope}
        \begin{scope}[shift={(1.1,0)}]
          \n[n1]{x}{0,0};
          \site{lx}{x.west};
          \site{rx}{x.east};
          \node at (206:.42) {\scriptsize $l$};
          \node at (26:.42) {\scriptsize $r$};
        \end{scope}
        \begin{scope}[shift={(2.2,0)}]
          \n[n2]{y}{0,0};
          \site{ly}{y.west};
          \site{ry}{y.east};
          \node at (206:.42) {\scriptsize $l$};
          \node at (26:.42) {\scriptsize $r$};
        \end{scope}
        \begin{scope}[shift={(3.3,0)}]
          \n[n3]{z}{0,0};
          \e{z}{.5,0};
          \site{lz}{z.west};
          \site{rz}{z.east};
          \node at (206:.42) {\scriptsize $l$};
          \node at (26:.42) {\scriptsize $r$};
        \end{scope}
      }};
  \end{tikzpicture}
\end{center}
\fi

It is thus apparent that
an energy-based rule refinement has to proceed
cautiously to be exhaustive and mutually exclusive. % non-overlapping.
This is where our growth policy technique
comes in handy to define such refinements.
It divides the problem in a much simpler group of problems:
each extension $\phi$ must declare the set of sites
that it requires to be mature and $\shapes$-balanced.
Minimal glueings play a guiding role here.
They tell us whether an extension has successfully
avoided or absorbed completely an energy pattern.

In our example, we extend our rule $r^+_{12}$
step by step to see this idea in action.
First take no extension at all or,
more precisely, take the identity arrow as an extension.
On the left-hand side there is only one minimal glueing,
the disjoint union, which, as it is always the case,
is irrelevant.
On the right-hand side instead we have two minimal glueings:
the disjoint union and the triangle itself,
as in \diagram{triangle-mg}.
The latter is indeed relevant and informs us
of which sites are missing in the extension,
namely the $l$ site on the orange node
and the $r$ site on the blue node.
So we ask for both and set $\gp_{\id_{r_R}}(u) = \set{l,r}$
for all $u \in \agents_{\anon{r_R}}$.
% Due to faithfulness,
% every mature extension of $r^+_{12}$ must include both sites.
Now let us add one of them as a free site
and ask again which sites each agent requires.
This extension, call it $\phi_1$, has codomain
the left-hand side of rule~\ref{eq:refined3}.
The codomain of the corresponding extension $\comatch{\phi_1}$
on the right-hand side
does not glue relevantly with the triangle anymore.
However, $\id$ is a prefix of $\phi_1$
and hence, due to faithfulness,
$\gp_{\phi_1}$ should ask for the same sites
that $\gp_{\id}$ does,
\ie $\gp_{\phi_1}(u) = \gp_{\id}(u)$
for all agents $u$ in the image of $\id$. % in the domain of $\gp_{\id}$.
So here again caution must be exercised.
The solution is to remember which sites have been asked for
in the past and to keep asking for them in future extensions.

Given contact graph $C$ and $r$ in $\generators$
we define our growth policy $\gp$ for $r_L$ as follows.
% We define our growth policy $\gp$ for $r_L$ as follows.
Suppose $\phi: r_L \to g$ is an extension of $r_L$.
We set $\gp_\phi$ to request
a site $z$ in $\sitemap_C^{-1}(g_\agents(u))$
at agent $u$ in $\agents_{\anon{g}}$ iff either
% (i) there is an agent $u_0$ with a site $z_0$ in $r_L$
% such that $u=\phi(u_0)$ and $s = r_L(z_0)$; or
\begin{enumerate}[label={(\roman*)}]
\item % (\emph{monotonicity})
$u = \phi_\agents(u_0)$ and $z = \phi_\sites(z_0)$
for some $u_0$ in $\agents_{\anon{r_L}}$ and
$z_0$ in $\sites_{\anon{r_L}}$; or
\item % (\emph{past-minimal-glueings-completeness})
$\phi$ factorises as $\phi_2 \, \phi_1$,
where $\phi_1: r_L \to g_1$,
and there is a relevant minimal glueing
$\gamma: p \to s \gets g_1 :\theta$,
with $p$ in $\shapes$,
and some $u_1$ in $\agents_{\anon{g_1}}$
and a site $z_1$ in $\sitemap_{\anon{s}}^{-1}(\theta_\agents(u_1))$
such that $u = \phi_{2,\agents}(u_1)$ and $z = s_\sites(z_1)$; or
\begin{equation}
  \label{eq:energy-gp}
  \tikz[baseline=-2.5,thick]{
    \node (p) at (0,0) {$p$};
    \node (s) at (2,-1.2) {$s$};
    \node (l) at (4,1.8) {$r_L$};
    \node (g1) at (4,0) {$g_1$};
    \node[anchor=east] at (g1) {$u_1 \!\in{}\,$};
    \node (g) at (6,-1.2) {$g$};
    \node[anchor=west] at (g) {${}\ni u$};
    \draw[hom] (l) -- node[pos=.45,left] {$\phi_1$} (g1);
    \draw[hom] (p) -- node[below left] {$\gamma$} (s);
    \draw[hom] (g1) -- node[below right] {$\theta$} (s);
    \draw[hom] (g1) -- node[below left] {$\phi_2$} (g);
    \draw (l) edge[hom,bend left=30] node[above right] {$\phi$} (g);
  }
\end{equation}
\item % (\emph{connected-completeness})
$z = g_\sites(z_2)$ for some $z_2$ in $\sites_{\anon{g}}$
such that $z_2 \edges_{\anon{g}} z_3$
and $g_\sites(z_3)$ in $\gp_\phi(u)$.
\end{enumerate}
In words, clause (i) ensures
that all sites in $r_L$ are asked for % \footnote{
%   Otherwise every extension would be overgrown.}
while clause (ii) adds sites $z$ in $\sites_C$
corresponding to sites $z_1$ in $\sites_{\anon{s}}$
which appear by glueing with $p$
at some point between $r_L$ and $g$.
Clause (iii), on the other hand,
asks for sites that are bound to sites
that are requested by the growth policy
so that extensions that avoid minimal glueings are not overgrown.
%
We refer to the extension $\phi_2: g_1 \to g$
as a \emph{rewind} of $\phi$
and say that the request of $z$ at $u$ originates from $u_1$.
By rewinding extensions we can remember
which sites have been asked for in the past.
% The first clause simply ensures
% that all sites in $r_L$ are asked for.\footnote{
%   Otherwise every extension would be overgrown.}
% The second clause adds in sites which appear by
% glueing with $p$ at some point between $r_L$ and $g$.
% and implements the absorb-or-avoid constraint explained beforehand.

Symmetrically, we define a growth policy $\comatch{\gp}$ for $r_R$
by applying the same definition to the inverse rule $\inv{r}$.
% Since extensions of $r_L$ and $r_R$ are isomorphic,
% we can, with a slight abuse of notation,
% define $\gp^\shapes := \gp \union \comatch{\gp}$.
Finally, we define our growth policy $\gp^\shapes$
as the union of both growth policies,
that is, $\gp^\shapes_\phi(u) = \gp_\phi(u)
\,\cup\, \comatch{\gp_{\comatch{\phi}}}(u)$.

% Coming back to our example,
According to this growth policy,
the extension $\comatch{\phi_1}$
of the right-hand side of $r^+_{12}$
in our example
is immature (despite being $\shapes$-balanced)
since the following rewind
asks for a site that is missing in its image.
\begin{center}
  \resizebox{.65\linewidth}{!}{%
  \begin{tikzpicture}

    \node[grphnode,outer sep=5pt] (p) at (0,0) {
      % Triangle
      \tikz[ingrphdiag,outer sep=0pt]{
        \path[use as bounding box] (-.3,.36) rectangle (1.4,-1.24);
        \e{0,0}{0:1.1};
        \e{0,0}{-60:1.1};
        \e{0:1.1}{-60:1.1};
        \begin{scope}[shift={(0,0)}]
          \n[n1]{x}{0,0};
          \site{r1}{0:7pt};
          \site{l1}{-60:7pt};
          \node at (-86:12pt) {\scriptsize $l$};
          \node at (26:12pt) {\scriptsize $r$};
        \end{scope}
        \begin{scope}[shift={(0:1.1)}]
          \n[n2]{y}{0,0};
          \site{r2}{180:7pt};
          \site{l2}{-120:7pt};
          \node at (154:12pt) {\scriptsize $l$};
          \node at (-94:12pt) {\scriptsize $r$};
        \end{scope}
        \begin{scope}[shift={(-60:1.1)}]
          \n[n3]{z}{0,0};
          \site{r3}{120:7pt};
          \site{l3}{60:7pt};
          \node at (146:12pt) {\scriptsize $r$};
          \node at (34:12pt) {\scriptsize $l$};
        \end{scope}
      }};

    \node[grphnode,outer sep=5pt,anchor=north west] (s)
    at ($(p.south east)+(-30:1)$) {
      % Triangle
      \tikz[ingrphdiag,outer sep=0pt]{
        \path[use as bounding box] (-.3,.36) rectangle (1.4,-1.24);
        \e{0,0}{0:1.1};
        \e{0,0}{-60:1.1};
        \e{0:1.1}{-60:1.1};
        \begin{scope}[shift={(0,0)}]
          \n[n1]{x}{0,0};
          \site{r1}{0:7pt};
          \site{l1}{-60:7pt};
          \node at (-86:12pt) {\scriptsize $l$};
          \node at (26:12pt) {\scriptsize $r$};
        \end{scope}
        \begin{scope}[shift={(0:1.1)}]
          \n[n2]{y}{0,0};
          \site{r2}{180:7pt};
          \site{l2}{-120:7pt};
          \node at (154:12pt) {\scriptsize $l$};
          \node at (-94:12pt) {\scriptsize $r$};
        \end{scope}
        \begin{scope}[shift={(-60:1.1)}]
          \n[n3]{z}{0,0};
          \site{r3}{120:7pt};
          \site{l3}{60:7pt};
          \node at (146:12pt) {\scriptsize $r$};
          \node at (34:12pt) {\scriptsize $l$};
        \end{scope}
      }};

    \node[grphnode,outer sep=5pt,anchor=south west] (g1)
    at ($(s.north east)+(30:1)$) {
      % ?12?
      \tikz[ingrphdiag,outer sep=0pt]{
        \e{0,0}{1.1,0};
        \begin{scope}
          \n[n1]{x}{0,0};
          \site{rx}{x.east};
          \node at (26:.42) {\scriptsize $r$};
        \end{scope}
        \begin{scope}[shift={(1.1,0)}]
          \n[n2]{y}{0,0};
          \site{ly}{y.west};
          \node at (206:.42) {\scriptsize $l$};
        \end{scope}
      }};

    \node[grphnode,outer sep=5pt,anchor=south] (l)
    at ($(g1.north)+(90:1)$) {
      % ?12?
      \tikz[ingrphdiag,outer sep=0pt]{
        \e{0,0}{1.1,0};
        \begin{scope}
          \n[n1]{x}{0,0};
          \site{rx}{x.east};
          \node at (26:.42) {\scriptsize $r$};
        \end{scope}
        \begin{scope}[shift={(1.1,0)}]
          \n[n2]{y}{0,0};
          \site{ly}{y.west};
          \node at (206:.42) {\scriptsize $l$};
        \end{scope}
      }};

    \node[grphnode,outer sep=5pt,anchor=north west] (g)
    at ($(g1.south east)+(-30:1)$) {
      % 12?
      \tikz[ingrphdiag,outer sep=0pt]{
        \e{-.5,0}{1.1,0};
        \begin{scope}
          \n[n1]{x}{0,0};
          \site{lx}{x.west};
          \site{rx}{x.east};
          \node at (206:.42) {\scriptsize $l$};
          \node at (26:.42) {\scriptsize $r$};
        \end{scope}
        \begin{scope}[shift={(1.1,0)}]
          \n[n2]{y}{0,0};
          \site{ly}{y.west};
          \node at (206:.42) {\scriptsize $l$};
        \end{scope}
      }};

    \draw[hom] (l) -- node[pos=.45,left] {$\id$} (g1);
    \draw[hom] (p.south east) -- node[below left] {$\gamma$} (s.north west);
    \draw[hom] (g1.south west) -- node[below right] {$\theta$} (s.north east);
    \draw[hom] (g1.south east) -- node[below left] {$\comatch{\phi_1}$} (g.north west);
    \draw (l) edge[hom,bend left=27] node[above right] {$\comatch{\phi_1}$} (g);

  \end{tikzpicture}}
\end{center}

%%% Local Variables:
%%% mode: latex
%%% TeX-master: "thesis"
%%% End:

So we must add an $r$ site to the blue node.
There are two possibilities when the site is added:
it can be free or it can be bound.
In particular, the contact graph $C$ tells us
that an $r$ site on a blue node can only be bound
to an $l$ site on a green node.
We obtain then two new extensions,
with codomains:
\begin{center}
  \begin{tikzpicture}
    \node[grphnode,anchor=east] (g1) at (0,0) {
      \tikz[ingrphdiag]{
        \e{0,0}{1.2,0};
        \begin{scope}[shift={(0,0)}]
          \n[n1]{x}{0,0};
          \e{x}{-.5,0};
          \site{lx}{x.west};
          \site{rx}{x.east};
          \node at (206:.42) {\scriptsize $l$};
          \node at (26:.42) {\scriptsize $r$};
        \end{scope}
        \begin{scope}[shift={(1.2,0)}]
          \n[n2]{y}{0,0};
          \e{y}{.5,0};
          \site{ly}{y.west};
          \site{ry}{y.east};
          \node at (206:.42) {\scriptsize $l$};
          \node at (26:.42) {\scriptsize $r$};
        \end{scope}
      }};
    \node at (.7,0) {and};
    \node[grphnode,anchor=west] (g2) at (1.4,0) {
      \tikz[ingrphdiag]{
        \e{0,0}{2.2,0};
        \begin{scope}[shift={(0,0)}]
          \n[n1]{x}{0,0};
          \e{x}{-.5,0};
          \site{lx}{x.west};
          \site{rx}{x.east};
          \node at (206:.42) {\scriptsize $l$};
          \node at (26:.42) {\scriptsize $r$};
        \end{scope}
        \begin{scope}[shift={(1.1,0)}]
          \n[n2]{y}{0,0};
          \site{ly}{y.west};
          \site{ry}{y.east};
          \node at (206:.42) {\scriptsize $l$};
          \node at (26:.42) {\scriptsize $r$};
        \end{scope}
        \begin{scope}[shift={(2.2,0)}]
          \n[n3]{z}{0,0};
          \site{lz}{z.west};
          \node at (206:.42) {\scriptsize $l$};
        \end{scope}
      }};
  \end{tikzpicture}
\end{center}
The first extension cannot possibly ask for any more sites.
However, the second extension, call it $\comatch{\phi_2}$,
may ask for the $r$ site on the green node.
If this is the case there must be a rewind of $\comatch{\phi_2}$
which contains a pre-image of the green node
and glues relevantly with the triangle.
\begin{center}
  \resizebox{.8\linewidth}{!}{%
  \begin{tikzpicture}

    \node[grphnode,outer sep=5pt] (p) at (0,0) {
      % Triangle
      \tikz[ingrphdiag,outer sep=0pt]{
        \path[use as bounding box] (-.3,.36) rectangle (1.4,-1.24);
        \e{0,0}{0:1.1};
        \e{0,0}{-60:1.1};
        \e{0:1.1}{-60:1.1};
        \begin{scope}[shift={(0,0)}]
          \n[n1]{x}{0,0};
          \site{r1}{0:7pt};
          \site{l1}{-60:7pt};
          \node at (-86:12pt) {\scriptsize $l$};
          \node at (26:12pt) {\scriptsize $r$};
        \end{scope}
        \begin{scope}[shift={(0:1.1)}]
          \n[n2]{y}{0,0};
          \site{r2}{180:7pt};
          \site{l2}{-120:7pt};
          \node at (154:12pt) {\scriptsize $l$};
          \node at (-94:12pt) {\scriptsize $r$};
        \end{scope}
        \begin{scope}[shift={(-60:1.1)}]
          \n[n3]{z}{0,0};
          \site{r3}{120:7pt};
          \site{l3}{60:7pt};
          \node at (146:12pt) {\scriptsize $r$};
          \node at (34:12pt) {\scriptsize $l$};
        \end{scope}
      }};

    \node[grphnode,outer sep=5pt,anchor=north west] (s)
    at ($(p.south east)+(-30:1)$) {
      % Triangle
      \tikz[ingrphdiag,outer sep=0pt]{
        \path[use as bounding box] (-.3,.36) rectangle (1.4,-1.24);
        \e{0,0}{0:1.1};
        \e{0,0}{-60:1.1};
        \e{0:1.1}{-60:1.1};
        \begin{scope}[shift={(0,0)}]
          \n[n1]{x}{0,0};
          \site{r1}{0:7pt};
          \site{l1}{-60:7pt};
          \node at (-86:12pt) {\scriptsize $l$};
          \node at (26:12pt) {\scriptsize $r$};
        \end{scope}
        \begin{scope}[shift={(0:1.1)}]
          \n[n2]{y}{0,0};
          \site{r2}{180:7pt};
          \site{l2}{-120:7pt};
          \node at (154:12pt) {\scriptsize $l$};
          \node at (-94:12pt) {\scriptsize $r$};
        \end{scope}
        \begin{scope}[shift={(-60:1.1)}]
          \n[n3]{z}{0,0};
          \site{r3}{120:7pt};
          \site{l3}{60:7pt};
          \node at (146:12pt) {\scriptsize $r$};
          \node at (34:12pt) {\scriptsize $l$};
        \end{scope}
      }};

    \node[grphnode,outer sep=5pt,anchor=south west] (g1)
    at ($(s.north east)+(30:1)$) {
      % ?12?
      \tikz[ingrphdiag,outer sep=0pt]{
        \e{0,0}{2.2,0};
        \begin{scope}[shift={(0,0)}]
          \n[n1]{x}{0,0};
          \site{rx}{x.east};
          \node at (26:.42) {\scriptsize $r$};
        \end{scope}
        \begin{scope}[shift={(1.1,0)}]
          \n[n2]{y}{0,0};
          \site{ly}{y.west};
          \site{ry}{y.east};
          \node at (206:.42) {\scriptsize $l$};
          \node at (26:.42) {\scriptsize $r$};
        \end{scope}
        \begin{scope}[shift={(2.2,0)}]
          \n[n3]{z}{0,0};
          \site{lz}{z.west};
          \node at (206:.42) {\scriptsize $l$};
        \end{scope}
        % \e{0,0}{1.1,0};
        % \begin{scope}
        %   \n[n1]{x}{0,0};
        %   \site{rx}{x.east};
        %   \node at (26:.42) {\scriptsize $r$};
        % \end{scope}
        % \begin{scope}[shift={(1.1,0)}]
        %   \n[n2]{y}{0,0};
        %   \site{ly}{y.west};
        %   \node at (206:.42) {\scriptsize $l$};
        % \end{scope}
      }};

    \node[grphnode,outer sep=5pt,anchor=south] (l)
    at ($(g1.north)+(90:1)$) {
      % ?12?
      \tikz[ingrphdiag,outer sep=0pt]{
        \e{0,0}{1.1,0};
        \begin{scope}
          \n[n1]{x}{0,0};
          \site{rx}{x.east};
          \node at (26:.42) {\scriptsize $r$};
        \end{scope}
        \begin{scope}[shift={(1.1,0)}]
          \n[n2]{y}{0,0};
          \site{ly}{y.west};
          \node at (206:.42) {\scriptsize $l$};
        \end{scope}
      }};

    \node[grphnode,outer sep=5pt,anchor=north west] (g)
    at ($(g1.south east)+(-30:1)$) {
      % 12?
      \tikz[ingrphdiag,outer sep=0pt]{
        \e{0,0}{2.2,0};
        \begin{scope}[shift={(0,0)}]
          \n[n1]{x}{0,0};
          \e{x}{-.5,0};
          \site{lx}{x.west};
          \site{rx}{x.east};
          \node at (206:.42) {\scriptsize $l$};
          \node at (26:.42) {\scriptsize $r$};
        \end{scope}
        \begin{scope}[shift={(1.1,0)}]
          \n[n2]{y}{0,0};
          \site{ly}{y.west};
          \site{ry}{y.east};
          \node at (206:.42) {\scriptsize $l$};
          \node at (26:.42) {\scriptsize $r$};
        \end{scope}
        \begin{scope}[shift={(2.2,0)}]
          \n[n3]{z}{0,0};
          \site{lz}{z.west};
          \node at (206:.42) {\scriptsize $l$};
        \end{scope}
        % \e{-.5,0}{1.1,0};
        % \begin{scope}
        %   \n[n1]{x}{0,0};
        %   \site{lx}{x.west};
        %   \site{rx}{x.east};
        %   \node at (206:.42) {\scriptsize $l$};
        %   \node at (26:.42) {\scriptsize $r$};
        % \end{scope}
        % \begin{scope}[shift={(1.1,0)}]
        %   \n[n2]{y}{0,0};
        %   \site{ly}{y.west};
        %   \node at (206:.42) {\scriptsize $l$};
        % \end{scope}
      }};

    \draw[hom] (l) -- (g1);
    \draw[hom] (p.south east) -- (s.north west);
    \draw[hom] (g1.south west) -- (s.north east);
    \draw[hom] (g1.south east) -- (g.north west);
    \draw (l) edge[hom,bend left=27] node[above right] {$\comatch{\phi_2}$} (g);

  \end{tikzpicture}}
\end{center}

%%% Local Variables:
%%% mode: latex
%%% TeX-master: "thesis"
%%% End:

Therefore $\comatch{\phi_2}$ is immature as well.
We must reveal the $r$ site on the green node
and so we obtain
\begin{center}
  \begin{tikzpicture}
    \node[grphnode,anchor=east] (g1) at (0,0) {
      \tikz[ingrphdiag]{
        \e{0,0}{2.2,0};
        \begin{scope}[shift={(0,0)}]
          \n[n1]{x}{0,0};
          \e{x}{-.5,0};
          \site{lx}{x.west};
          \site{rx}{x.east};
          \node at (206:.42) {\scriptsize $l$};
          \node at (26:.42) {\scriptsize $r$};
        \end{scope}
        \begin{scope}[shift={(1.1,0)}]
          \n[n2]{y}{0,0};
          \site{ly}{y.west};
          \site{ry}{y.east};
          \node at (206:.42) {\scriptsize $l$};
          \node at (26:.42) {\scriptsize $r$};
        \end{scope}
        \begin{scope}[shift={(2.2,0)}]
          \n[n3]{z}{0,0};
          \e{z}{.5,0};
          \site{lz}{z.west};
          \site{rz}{z.east};
          \node at (206:.42) {\scriptsize $l$};
          \node at (26:.42) {\scriptsize $r$};
        \end{scope}
      }};
    \node at (.2,0) {,};
    \node[grphnode,anchor=west] (g2) at (0.4,0) {
      \tikz[ingrphdiag]{
        \e{0,0}{3.3,0};
        \begin{scope}[shift={(0,0)}]
          \n[n1]{x}{0,0};
          \e{x}{-.5,0};
          \site{lx}{x.west};
          \site{rx}{x.east};
          \node at (206:.42) {\scriptsize $l$};
          \node at (26:.42) {\scriptsize $r$};
        \end{scope}
        \begin{scope}[shift={(1.1,0)}]
          \n[n2]{y}{0,0};
          \site{ly}{y.west};
          \site{ry}{y.east};
          \node at (206:.42) {\scriptsize $l$};
          \node at (26:.42) {\scriptsize $r$};
        \end{scope}
        \begin{scope}[shift={(2.2,0)}]
          \n[n3]{z}{0,0};
          \site{lz}{z.west};
          \site{rz}{z.east};
          \node at (206:.42) {\scriptsize $l$};
          \node at (26:.42) {\scriptsize $r$};
        \end{scope}
        \begin{scope}[shift={(3.3,0)}]
          \n[n1]{w}{0,0};
          \site{lw}{w.west};
          \node at (206:.42) {\scriptsize $l$};
        \end{scope}
      }};
    \begin{scope}[shift={(g2.east)}]
      \node at (.7,0) {and};
      \node[grphnode,anchor=west] (g3) at (1.4,0) {
        \tikz[ingrphdiag]{
          \path[use as bounding box] (-.3,.36) rectangle (1.4,-1.24);
          \e{0,0}{0:1.1};
          \e{0,0}{-60:1.1};
          \e{0:1.1}{-60:1.1};
          \begin{scope}[shift={(0,0)}]
            \n[n1]{x}{0,0};
            \site{r1}{0:7pt};
            \site{l1}{-60:7pt};
            \node at (-86:12pt) {\scriptsize $l$};
            \node at (26:12pt) {\scriptsize $r$};
          \end{scope}
          \begin{scope}[shift={(0:1.1)}]
            \n[n2]{y}{0,0};
            \site{r2}{180:7pt};
            \site{l2}{-120:7pt};
            \node at (154:12pt) {\scriptsize $l$};
            \node at (-94:12pt) {\scriptsize $r$};
          \end{scope}
          \begin{scope}[shift={(-60:1.1)}]
            \n[n3]{z}{0,0};
            \site{r3}{120:7pt};
            \site{l3}{60:7pt};
            \node at (146:12pt) {\scriptsize $r$};
            \node at (34:12pt) {\scriptsize $l$};
          \end{scope}
        }};
    \end{scope}
  \end{tikzpicture}
\end{center}
Finally, all extensions are mature.
Note that the second extension has an $l$ site
on the rightmost orange node
which would not be asked by the growth policy
if it were not for clause (iii).
In the absence of clause (iii) we would have moved
from an immature extension to an overgrown extension
in just one step,
leaving us in a strange situation
% and defeating the purpose of the growth policy.
% and making it possible for the growth policy
% to have no mature extensions at all.
by allowing the growth policy to define an empty refinement.
% $\gp^\shapes(r) = \varnothing$ for rule $r$.
Next we prove that the growth policy
that we have introduced in this section
is in general well-defined and well-behaved.
% and has a number of important properties.

\begin{theorem}
  \label{thm:energy-gp}
  The above $\gp^\shapes$ is indeed a growth policy for $r_L$
  and the induced refined family of rules $\gp^\shapes(r)$ is
  exhaustive,
  non-empty,
  $\shapes$-balanced,
  and finite.
\end{theorem}
\begin{proof}
  We take the same notations as in \diagram{energy-gp}.

  \emph{Growth policy}:
  Clearly, $\gp^\shapes_{\phi_1}(u_1) \subseteq \gp^\shapes_\phi(u)$
  as every request for a site in $g_1$
  will propagate to $g$ by definition.
  To prove the other direction,
  we need to verify that the requests generated by rewinds
  do not depend on the choice of factorisation.
  So, without loss of generality,
  consider glueings on extensions of $r_L$
  and let an alternative factorisation of $\phi$ through $g_2$
  be given which gives rise to a site request in $u$
  originating from some $u_2$ in $g_2$,
  as in the following diagram.
  \begin{center}
    \begin{tikzpicture}
      \node (l) at (3,2.2) {$r_L$};
      \node (g0) at (3,1) {$g_0 \ni \tuple{u_1,u_2}$};
      \node (g1) at (0,0) {$g_1 \ni u_1$};
      \node (g2) at (6,0) {$g_2 \ni u_2$};
      \node (g) at (3,-1) {$g \ni u$};
      \draw (l) edge[hom,bend right=20] node[above left] {$\phi_1$} (g1);
      \draw (l) edge[hom,bend left=20] (g2);
      \draw[hom] (g0) -- (g1);
      \draw[hom] (g0) -- (g2);
      \draw[hom] (g1) -- node[below left] {$\phi_2$} (g);
      \draw[hom] (g2) -- (g);
      \draw[hom,dotted] (l) -- (g0);
    \end{tikzpicture}
  \end{center}
  Consider $g_0$ the pull-back of the two rewinds
  (\ie the lower cospan).
  By construction it contains a pre-image
  for $u_1$ and $u_2$, say $(u_1,u_2)$.
  % It is also a prefix of an extension by construction.
  The relevant minimal glueing of $p$ and $g_2$
  that makes the site request restricts
  to another minimal glueing of $p$ and $g_0$.
  This new minimal glueing is still relevant
  as it contains the same overlap with the original $r_L$.
  As such, the same site request is made
  by the pre-image agent $(u_1,u_2)$ in $g_0$
  which then propagates to $u_1$ in $g_1$ as required.
  % NB: the argument does not rely on $p$ being glueable to $g_1$ -
  % which need not be true; but glueable to a rewind thereof.

  % \emph{Surjectivity}:
  \emph{Exhaustive}:
  Take any embedding $\psi$ of $r_L$ into a mixture $m$.
  We can restrict the co-domain of $\psi$ to be
  the connected closure $n$ of the image of $\psi$ in $m$,
  resulting in an epi $\psi_n: r_L \to n$.
  Let us further restrict $n$ by removing
  (i) all sites not requested by the growth policy and
  (ii) all agents that have no sites requested by the growth policy.
  % Call the result $g$.
  The result, call it $g$,
  has the same number of connected components as $r_L$
  since $\gp^\states$ only requests sites
  which appear by glueing
  and are thus (perhaps indirectly) connected to the sites
  % for which there is a path from the sites
  that are modified by the rule.
  We thus obtain an epi $\phi: r_L \to g$
  which is mature with respect to $\gp^\shapes$ since,
  by construction, its image contains all sites
  requested by $\gp^\shapes$ and no other foreign site.
  It is easy to see that $\phi$ factorises $\psi$.

  \emph{Non-empty}:
  Clause (i) guarantees that we request at least the sites in $r$
  which implies that $\id$ is not overgrown.
  % which implies that $\id_{r_L}$ is not overgrown.
  Due to clause (iii) there is always an extension
  whose image contains exactly all sites requested by $\gp^\states$
  and lies between an immature and an overgrown extension
  according to $\leq$.
  % in the specialisation order $\leq$.
  This extension is an epi because,
  as pointed out for exhaustivity,
  $\gp^\states$ only requests sites
  connected to those modified by the rule.

  \emph{$\shapes$-balanced}:
  If $\phi \in \gp^\shapes(r)$ is not $\shapes$-balanced
  then there must be some relevant minimal glueing
  inducing a further site request.
  Hence, $\phi$ cannot be mature.

  \emph{Finite}:
  A request for a site $a$ at some node in an extension
  $\phi: r_L\to g$, or $\comatch{\phi}: r_R \to g$,
  originates from a relevant minimal glueing
  of some $p$ in $\shapes$ with a prefix $\phi_1$ of $\phi$.
  Because this glueing is relevant,
  it must be that $a$ is at a distance from the image of $r_L$
  in the codomain of $\phi_1$ which is at most $\delta(p)$,
  the diameter of $p$ (else $p$ would not intersect the image of $r_L$).
  The same bound holds in the codomain of $\phi$,
  as distances can only contract by further extension.
  Therefore any site requested in $g$
  has a distance to the image $\phi(r_L)$
  which is bounded by $\max_{p \in \shapes}\delta(p)$.
  If $\phi$ is not overgrown,
  this sets a bound on the diameter of $g$.
  Hence there are finitely many mature extensions.
  % A site request always comes from a relevant minimal glueing
  % of some $p \in \shapes$ and an extension $\phi$ of $r$;
  % while there can be an infinite number of $\phi$s,
  % only a finite number of them can give rise to
  % \emph{relevant} glueings and each of them can give rise to
  % only a finite number of site requests.
  %
  % This control only improves upon further extension
  % (meaning the distance of the added site $a$ to $r_L$
  %  can only decrease along an epi ---
  %  when one adds cycles creating shorter paths,
  %  but certainly it will not increase as old paths
  %  are preserved by extensions)
  %
  % $\delta$ the diameter of a max glueing is
  % $\leq 2 \max_{p \in \shapes}\delta(g)$,
  % hence there is a finite number of direct
  % max-min relevant glueings on any $r$.
\end{proof}

Therefore, given $\generators$ and $\shapes$,
we obtain a finite $\shapes$-balanced rule set $\refinedrules$,
which refines $\generators$ exhaustively,
by taking the disjoint sum of the refined rules
$\refinedrules = \dsum_{r \in \generators} \,\gp^\shapes(r)$.
% (disjoint sum).
To every refinement $r_\phi$ corresponds
an inverse refinement $\inv{r}_{\comatch{\phi}}$.
Hence, $\refinedrules = \inv{\refinedrules}$
is closed under inversion like $\generators$.


\section{Rates and detailed balance}
\label{sec:rates}

To equip $\refinedrules$ with rates
we define a rate map $k: \refinedrules \to \RR_{>0}$.
We use the real-valued vector of \emph{energy costs} $\cost$
introduced at the beginning of this chapter
(page~\pageref{chp:direct})
together with the balance vector $\Delta\phi$
of a refined rule $r_\phi$ in $\refinedrules$
with respect to $\shapes$
(page~\pageref{p:balance-vector})
to constrain the ratio
between the forward and the backward rate:
% in accordance with detailed balance:
% To obtained detailed balance,
% the rate map must conform to the following relation
% for each $r_\phi$ in $\refinedrules$:
\begin{equation}
  \label{eq:rates}
  \ln\, {k(\inv{r}_{\comatch{\phi}})} - \ln\, {k(r_\phi)} =
  \cost \cdot \Delta\phi
\end{equation}
% where $\Delta\phi$ is the balance vector
% of the refined rule $r_\phi$
% with respect to $\shapes$
% (page~\pageref{p:balance-vector}).

% \eqn{rates} tells us that
The pair of rules $r,\inv{r}$ is biassed in the forward direction
if $k_r(\phi) > k_{\inv{r}}(\comatch{\phi})$
% and thus when $\cost \cdot \Delta\phi < 0$.
and \eqn{rates} tells us that this happens
when $\cost \cdot \Delta\phi < 0$,
\ie whenever the energy decreases as we go in that direction.
% so the convention here is that $\cost\cdot\Delta\phi$
% is the after-energy minus the before-energy.
This is the usual convention for energy functions.

We show that the set of refined rules $\refinedrules$
with any such rate map
% for which this equality holds
has detailed balance.
To simplify notation,
we write $\shapes(m)$
for the $\shapes$-indexed vector
which maps $p$ to $\matches{p}{m}$.
Using vector notation,
the energy $E(m)$ of a state $m$
(as defined in \eqn{graph-energy})
can then simply be written as $\cost \cdot \shapes(m)$.
Moreover, we write $\LTS_\generators(m)$ for the finite
strongly connected component of $m$ in $\LTS_\generators$
and recall the definition of a \pmf
$\pi_m$ on $\LTS_\generators(m)$ from \eqn{energy},
which after substituting $E(m)$ reads
\begin{equation}
  \label{eq:boltzmann}
  \pi_m(x) = \dfrac{
    \exp{-\cost \cdot \shapes(x)}}{
    \sum_{y \in \LTS_\generators(m)} \exp{-\cost \cdot \shapes(y)}}
\end{equation}
%
We can now prove the main theorem of this chapter.

\begin{theorem}
  \label{thm:detailed-balance}
  Let $\generators$, $\shapes$, $\refinedrules$,
  $k$, and $\pi_m$ be defined as above;
  then (i) $\LTS_{\refinedrules}$ and $\LTS_\generators$ are isomorphic
  % TODO: is it important to say symmetric?
  as symmetric labelled transition systems;
  and (ii) for any mixture $m$,
  the time-homogeneous continuous-time Markov chain
  $\LTS^k_{\refinedrules}$ has detailed balance for, and converges to,
  $\pi_m$ on $\LTS_{\refinedrules}(m)$.
\end{theorem}
\begin{proof}
  Both $\LTS_\generators$ and $\LTS_{\refinedrules}$
  offer transitions from a mixture $m$:
  the former are labelled by pairs $(r,\psi)$
  with $r$ in $\generators$
  and $\psi$ in $\matches{r_L}{m}$
  while the latter by pairs $(r_\phi,\gamma)$
  with $r_\phi$ the refinement of $r$
  along a mature extension $\phi: r_L \to g$
  % \ie which belongs to $\gp^\shapes(r_L)$,
  and $\gamma$ in $\matches{g}{m}$.
  Steps in the latter can be mapped to steps in the former
  by transforming labels as follows:
  $(r_\phi, \gamma) \mapsto (r, \gamma \, \phi)$.
  By \thm{energy-gp}, each event $(r,\psi)$ is factored by
  exactly one event $(r_\phi,\gamma)$ and thus
  % As $\refinedrules$ refines $\generators$ exhaustively
  % (\thm{energy-gp}),
  this correspondence is a bijection,
  which establishes the first claim.

  % (Pedantically, there is a full and faithful functor
  %  between the two corresponding free categories
  %  which is the identity on objects ---%
  %  incidentally, this bijection is readily seen
  %  to respect the symmetries on labels.)

  Since we have multiple rules in $\LTS_{\refinedrules}$,
  each of which can be applied in several ways,
  there can be more than one transition from $m$ to the same $n$ ---
  each uniquely described by a $(r_\phi,\gamma)$ label.
  Each such $(r_\phi,\gamma)$ has an inverse
  $(\inv{r}_{\comatch{\phi}},\comatch{\gamma})$
  % where:
  % $\inv{r}$ is the rule inverse to $r$;
  % $\comatch{\phi}$ corresponds to $\phi$ in the isomorphism
  % between the categories of extensions of $r_{\phi,L}$ and $r_{\phi,R}$,
  % with $\phi_\agents = \comatch{\phi}_\agents$;
  % % \phi_\sites = \comatch{\phi}_\sites for that matter as well...
  % and $\comatch{\gamma}$ is the embedding corresponding to $\gamma$,
  % also with $\gamma_\agents = \comatch{\gamma}_\agents$.
  % One can easily verify that $\comatch{\phi}$ is an epi,
  % and that $\comatch{\phi}$ is also mature.
  % Hence $(\inv{r}_{\comatch{\phi}},\comatch{\gamma})$
  % determines a valid transition in $\LTS_{\refinedrules}$
  % which is inverse to $(r_\phi,\gamma)$,
  and we have a bijection between them and thus between
  transitions from $m$ to $n$ and those from $n$ to $m$
  due to \lems{reversibility}{epi-prefix}.

  Consider a pair $t,\inv{t}$ of such corresponding events
  due to $r_\phi$ and $\inv{r}_{\comatch{\phi}}$.
  Because $t$ is a transition from $m$ to $n$
  and $\phi$ is $\shapes$-balanced, % (\thm{energy-gp}),
  we have $\shapes(n) = \shapes(m) + \Delta\phi$
  and hence
  $\cost \cdot \Delta\phi = \cost \cdot (\shapes(n)-\shapes(m))$.
  So, by \eqn{rates}, the rates of $t,\inv{t}$ are such that:
  \begin{equation*}
    k(\inv{t})\, \exp{-\cost \cdot \shapes(n)} =
    k(t)\, \exp{-\cost \cdot \shapes(m)}
  \end{equation*}
  and by summing this equation over all pairs,
  we obtain detailed balance
  for the probability local to the component
  $\LTS_{\refinedrules}(m) = \LTS_{\refinedrules}(n)$,
  defined above as $\pi_m = \pi_n$, since:
  \begin{equation*}
    q_{nm}\, \exp{-\cost \cdot \shapes(n)} =
    q_{mn}\, \exp{-\cost \cdot \shapes(m)}
  \end{equation*}
  The convergence statement then follows from \lem{ergodic}
  applied to the finite irreducible continuous-time Markov chain
  $\LTS^k_{\refinedrules}(m)$
  that is obtained by cropping all states
  not in $\LTS_{\refinedrules}(m)$.
\end{proof}

Note that the subset of the state space
which is reachable from $m$ in $\LTS_{\generators}$,
namely $\LTS_{\generators}(m)$, is finite.
Hence, the \emph{partition function}
$Z(m) := \sum_{y \in \LTS_{\generators}(m)} \exp{-E(y)}$
which figures in the denominator of $\pi_m$ is also finite.
In the presence of rules which increase the number of agents,
the components $\LTS_{\generators}(m)$ can be infinite
and $Z(m)$ may diverge.
For mass action stochastic Petri nets (\sct{bg}),
convergence is guaranteed if detailed balance holds,
but it is not true in general for Kappa \citep{et2,et1}.

Another point worth making is that the result holds symbolically % ---
regardless of the energy costs $\cost$.
Therefore $\cost$ can be seen as a set of parameters.
This is an ideal support for machine learning techniques
if one were contemplating fitting a model to data.


\section{Linear kinetic model}
\label{sec:kinetic-model}

The theorem in the previous section holds
for any rate map that agrees with \eqn{rates}.
In this section,
we show how to obtain a concrete rate $k(r_\phi)$
for each refined rule $r_\phi$ in $\refinedrules$.
% In particular,
% We choose rates from a tractable subset of all possible choices
To simplify the task, % of picking rates,
% we delineate a tractable subset of all possible choices
we pick rates from a tractable subset of all possible choices
by performing a log-affine expansion on
% that is parameterised by
% the balance vector $\Delta\phi$ of the refined rule.
the so-called «thermodynamic drive»
$\Delta E = \cost \cdot \Delta\phi$.
% by delineating a tractable subset of all possible choices
% whose size grows quadratically with $\abs{\shapes}$.
% This is a useful log-linear heuristics
% which has been used to model biological systems \citep{cannon}
% and is common in machine learning.
%
The expansion uses,
for each generator rule $r$ in $\generators$,
a constant $c_r \in \RR$
and a real-valued matrix $A_r$
of dimension $\abs{\shapes} \times \abs{\shapes}$.
Then we assign rates according to the following equality
\begin{equation}
  \label{eq:lkm}
  \ln\, k(r_\phi) = c_r - A_r\,\cost \cdot \Delta\phi
\end{equation}
subject to the following constraints
\begin{gather*}
  c_r = c_{\inv{r}} \\
  A_r + A_{\inv{r}} = I
\end{gather*}
with $I$ the $\abs{\shapes} \times \abs{\shapes}$ identity matrix.
%
We verify that $k$ satisfies \eqn{rates}
by substracting
$\ln\, k(\inv{r}_{\comatch{\phi}})$ and $\ln\, k(r_\phi)$,
giving us
\[ \ln\, k(\inv{r}_{\comatch{\phi}}) - \ln\, k(r_\phi)
   = (c_{\inv{r}} - A_{\inv{r}}\,\cost \cdot \Delta\comatch{\phi})
   - (c_r - A_r\,\cost \cdot \Delta\phi) \]
We have $\Delta\comatch{\phi} = -\Delta\phi$
by reversibility of rules
and so
\begin{align*}
  \ln\, k(\inv{r}_{\comatch{\phi}}) - \ln\, k(r_\phi)
  &{}= c_{\inv{r}} - c_r
     + A_r\,\cost \cdot \Delta\phi +
       A_{\inv{r}}\,\cost \cdot \Delta\phi \\
  &{}= (A_r + A_{\inv{r}})\,\cost \cdot \Delta\phi \\
  &{}= I\,\cost \cdot \Delta\phi \;=\; \cost \cdot \Delta\phi
\end{align*}
% \begin{equation*}
%   c_{\inv{r}} - A_{\inv{r}}(\cost) \cdot \Delta\comatch{\phi} =
%   c_r - A_r(\cost) \cdot \Delta\phi + \cost \cdot \Delta\phi
% \end{equation*}
% or, equivalently, using $\Delta\comatch{\phi} = -\Delta\phi$:
% \begin{equation*}
%   c_{\inv{r}} - c_r = -A_{\inv{r}}(\cost) \cdot \Delta\phi -
%   A_r(\cost) \cdot \Delta\phi + \cost \cdot \Delta\phi =
%   (I - A_{\inv{r}} - A_r)(\cost) \cdot \Delta\phi.
% \end{equation*}

% NOTE: do we get additional constraints on kinetic rates from
% cycles in the transition graph? when two cycles share an edge?

The kinetic model of \eqn{lkm} requires
% of the order of $\abs{\shapes}^2 \times \abs{\generators}$ parameters
$\abs{\shapes}^2 \times \abs{\generators} + \abs{\generators}$
parameters: % because we have
one $A_{r,pq}$ for each generator rule $r \in \generators$
and pair $p, q \in \shapes^2$,
plus one $c_r$ for each $r \in \generators$.
In practice one needs even fewer parameters
as only those energy patterns that are relevant
to a given generator rule $r$,
\ie those that have a non-zero balance
for at least one rule in $\gp^\shapes(r)$,
need to be considered when building $A_r$.
Typically, for larger models,
this will be a far smaller number than $\abs{\shapes}$.
This relative parsimony is compounded by the fact that
the number of \emph{independent} parameters will be often lower
because the $\Delta\phi$ family often has low rank,
meaning that, for a set of extensions $\phi$,
the balance vectors $\Delta\phi$ can be determined as
a linear combination of a smaller basis set.
% This kinetic model taps into another source of parsimony,
% namely the dependencies between the $\Delta\phi$s of a refinement.
% The $\Delta\phi$ family has typically low rank,
% and as \eqn{lkm} assigns rates to an extension $\phi$
% solely based on its $\Delta\phi$,
% fewer {independent} parameters will be needed.
% Concretely, all other rates can be expressed once we have a basis,
% \eg if $\Delta\phi = \sum_i \alpha_i \Delta\phi_i$ then:
% \begin{equation*}
%   \ln k_{r_\phi} - c_r
%   = -\sum_i \alpha_i A_r(\cost) \cdot \Delta\phi_i
%   =  \sum_i \alpha_i (\ln k_{r_{\phi_i}} - c_r).
% \end{equation*}
% This imposes extra uniformities on the rate maps.
% In the case of ANC we can fully solve the rank problem and even
% find a canonical set of rates from which to obtain the remainder.
By way of comparison,
if we were to assign kinetic rates to each refined rule,
we would need $\sum_{r \in \generators} |\gp^\shapes(r)|$ parameters.
% It is to be compared with the total number of choices which is
% far greater as it is of the order of the number of refinements,
% that is to say $\sum_{r \in \generators} |\gp^\shapes(r)|$.

We find two special cases for
the kinetic model presented here
that seem appealing as a first choice for parameterisation.
First, by setting $c_r = c_{\inv{r}} = 0$,
$A_r = I$ and $A_{\inv{r}} = 0$,
we get $k(r_\phi) = \exp{-\cost \cdot \Delta\phi}$
and $k(\inv{r}_{\comatch{\phi}}) = 1$.
Whenever $\inv{r}$ is the thermodynamically favoured direction
(and we can always choose it so),
% As $\cost \cdot \Delta\phi$ is the difference of energy,
% between the target and source in any application $r_\phi$,
this choice amounts to being exponentially reluctant
to climb up the energy gradient.
% This is a continuous-time version of the % celebrated
In this way,
this choice can be thought of as
continuous-time version of the
Metropolis algorithm introduced in \sct{bg}.

The second special case,
on the other hand,
is completely symmetric
and can be obtained by fixing $A_r = A_{\inv{r}} = I/2$
and $C_r = \exp{c_r}$:
% as follows.
\begin{equation}
  \label{eq:sym-lkm}
  \begin{array}{ccc}
    k(r_\phi) &=& C_r\,\exp{-\cost \cdot \Delta\phi/2} \\
    k(\inv{r}_{\comatch{\phi}}) &=& C_r\,\exp{\cost \cdot \Delta\phi/2}
  \end{array}
\end{equation}
% with $C_r = \exp{c_r}$.

% Finally, it is interesting to draw a comparison between
% the ascription given in \eqn{lkm} and the Arrhenius rate law.
% This law posits a dependency of the rate constant $k$ of a reaction
% of the form $\ln\, k = c - E_a/kT$,
% where $c$ is a constant (defining the basic time scale of the reaction),
% $E_a$ is the so-called \emph{activation energy} of the reaction
% and $T$ is the temperature.
% In our case, we are not concerned with
% the effect of $T$ on the (logarithm of the) rate
% but with the effect of consuming and producing
% various energy patterns in $\shapes$ at the locus of
% the instance of the generator rule $r$.
% In this view of things, \eqn{lkm} posits that
% % the `activation energy' of $\phi$ depends linearly on
% the activation energy of $\phi$ depends linearly
% on the cost of the various patterns and the balance of $\phi$.

% The similarity of \eqn{sym-lkm} to the Arrhenius equation
% can shed light on an interesting interpretation of
% our kinetic model.
Note the similarity of \eqn{sym-lkm} to the Arrhenius equation.
\begin{equation*}
  % \label{eq:arrhenius}
  k = A \, \exp{-E_a}
\end{equation*}
where $E_a$ is the \emph{activation energy} of the reaction
(expressed in units of $1/\kB T$ as in \eqn{energy})
and $A$ is a pre-exponential factor that defines the rate
at which the molecules involved in the reaction
collide in the correct orientation for the reaction to occur.
\eqn{sym-lkm} is a special case of the Arrhenius equation
when we equate $A = C_r$ and $E_a = \cost \cdot \Delta\phi/2$.
The first equality is therefore interpreted as an assumption
% The first equality means we are assuming
% This is equivalent to assuming
% This amounts to assuming
that the rate of molecules colliding
in the right orientation depends only on
the molecular motifs present in the left-hand side of
the generator rule $r$
(\ie not on the context revealed by the refined rule).
% Albeit a good starting point for an approximation
Albeit an approximation,
it might prove useful whenever the generator rules
specify enough context to determine, for instance,
% the accessibility surface of the reaction centre.
the accessibility of the reaction centre.
Another possible approach would be to compute $A$
based on properties of the refinement,
\eg how big the surrounding molecular complex is.

The second equality, $E_a = \cost \cdot \Delta\phi/2$,
tells us that the energetic barrier between
the reactants and the products is determined
only based on the energy patterns
that are destroyed and created by the refined rule % $r_\phi$
and their energy cost.
% TODO: say/remark that the dependence is linear
Since the activation energy is allowed to depend
% (to an extent)
on the context revealed by the refined rule,
this assumption imposes a softer constrain
than the previous one.

% TODO: mention connection to transition state/activated complex theory?
% http://staff.um.edu.mt/jgri1/teaching/che2372/notes/10/theory.html
% in particular, it's interesting that the pre-exponential factor A
% is related to the entropy of activation while the activation energy
% is related to the enthalpy of activation.

% is the linear kinetic model related to "Parameters for
% the description of transition states", John Leffler, Science, 1953
% https://sci-hub.ac/10.2307/1680906
% it says "we approximate the transition state as a hybrid between
% the reagent and product states"
% "whenever the plot of the logarithm of the rate constant
%  against the equilibrium constant is a straight line,
%  the approximation is justified"
% "it should then be possible to predict
%  the free energy of the transition state by a linear combination of
%  the predictions made for the reagents and for the products"
% if we then relate the free energy of the transition state to
% the rate constants using Arrhenius?
% this has been done in transition state theory
% https://en.wikipedia.org/wiki/Eyring_equation


\section{Example: Triangles all the way down}
\label{sec:triangles}
% why all the way down?

In this section we will complete and conclude the example
on the thermodynamical control of the formation of triangles
that we have used throughout this chapter.
Additionally we present the model in the format
used by the simulation tool
\href{https://github.com/Kappa-Dev/KaSim}{KaSim}
version 4.0
\citep{KaSimManual2014}
and run a few simulations to get an idea
of how the model behaves.
We use the symmetric linear kinetic model of \eqn{sym-lkm}
with $c_r = 0$ to derive the rates
and add three more energy patterns
(in addition to the triangle)
to demonstrate how they interact in the expansion of the rates.
% The three energy patterns are one for each type of edge,
In particular, we add one energy pattern for each type of edge,
that is, for contact maps
\begin{center}
  \begin{tikzpicture}
    \node[grphnode,anchor=east] (g1) at (0,0) {
      \tikz[ingrphdiag]{
        \e{0,0}{1.1,0};
        \begin{scope}[shift={(0,0)}]
          \n[n1]{x}{0,0};
          \site{rx}{x.east};
          \node at (26:.42) {\scriptsize $r$};
        \end{scope}
        \begin{scope}[shift={(1.1,0)}]
          \n[n2]{y}{0,0};
          \site{ly}{y.west};
          \node at (206:.42) {\scriptsize $l$};
        \end{scope}
      }};
    \node at (.2,0) {,};
    \node[grphnode,anchor=west] (g2) at (0.4,0) {
      \tikz[ingrphdiag]{
        \e{0,0}{1.1,0};
        \begin{scope}[shift={(0,0)}]
          \n[n2]{x}{0,0};
          \site{rx}{x.east};
          \node at (26:.42) {\scriptsize $r$};
        \end{scope}
        \begin{scope}[shift={(1.1,0)}]
          \n[n3]{y}{0,0};
          \site{ly}{y.west};
          \node at (206:.42) {\scriptsize $l$};
        \end{scope}
      }};
    \begin{scope}[shift={(g2.east)}]
      \node at (.7,0) {and};
      \node[grphnode,anchor=west] (g3) at (1.4,0) {
        \tikz[ingrphdiag]{
          \e{0,0}{1.1,0};
          \begin{scope}[shift={(0,0)}]
            \n[n3]{x}{0,0};
            \site{rx}{x.east};
            \node at (26:.42) {\scriptsize $r$};
          \end{scope}
          \begin{scope}[shift={(1.1,0)}]
            \n[n1]{y}{0,0};
            \site{ly}{y.west};
            \node at (206:.42) {\scriptsize $l$};
          \end{scope}
        }};
    \end{scope}
    % TODO: el punto y la coma estan muy arriba.
    \begin{scope}[shift={(g3.east)}]
      \node at (.2,0) {.};
    \end{scope}
  \end{tikzpicture}
\end{center}
% We assign the same energy cost to the three of them.

The analysis on \sct{energy-gp} unveiled five refinements
for rule $r^+_{12}$ (\eqn{r+12}) which we enumerate below.
% A(l,r), B(l,r) -> A(l,r!1), B(l!1,r) @ [exp] (-1/2 * 'ab')
\begin{center}
  \begin{tikzpicture}
    \node[grphnode,anchor=east] (lhs) at (0,0) {
      \tikz[ingrphdiag]{
        \begin{scope}[shift={(0,0)}]
          \n[n1]{x}{0,0};
          \e{x}{-.5,0};
          \e{x}{.5,0};
          \site{lx}{x.west};
          \site{rx}{x.east};
          \node at (206:.42) {\scriptsize $l$};
          \node at (26:.42) {\scriptsize $r$};
        \end{scope}
        \begin{scope}[shift={(1.2,0)}]
          \n[n2]{y}{0,0};
          \e{y}{-.5,0};
          \e{y}{.5,0};
          \site{ly}{y.west};
          \site{ry}{y.east};
          \node at (206:.42) {\scriptsize $l$};
          \node at (26:.42) {\scriptsize $r$};
        \end{scope}
      }};
    \path (lhs.east) +(.3,0) edge[rule] +(1,0)
      +(1.3,0) coordinate (r);
    \node[grphnode,anchor=west] (rhs) at (r) {
      \tikz[ingrphdiag]{
        \e{0,0}{1.1,0};
        \begin{scope}
          \n[n1]{x}{0,0};
          \e{x}{-.5,0};
          \site{lx}{x.west};
          \site{rx}{x.east};
          \node at (206:.42) {\scriptsize $l$};
          \node at (26:.42) {\scriptsize $r$};
        \end{scope}
        \begin{scope}[shift={(1.1,0)}]
          \n[n2]{y}{0,0};
          \e{y}{.5,0};
          \site{ly}{y.west};
          \site{ry}{y.east};
          \node at (206:.42) {\scriptsize $l$};
          \node at (26:.42) {\scriptsize $r$};
        \end{scope}
      }};
  \end{tikzpicture}
\end{center}
% A(l!r.C,r), B(l,r) -> A(l!r.C,r!1), B(l!1,r) @ [exp] (-1/2 * 'ab')
\begin{center}
  \begin{tikzpicture}
    \node[grphnode,anchor=east] (lhs) at (0,0) {
      \tikz[ingrphdiag]{
        \e{0,0}{1.2,0};
        \begin{scope}[shift={(0,0)}]
          \n[n3]{x}{0,0};
          \site{rx}{x.east};
          \node at (26:.42) {\scriptsize $r$};
        \end{scope}
        \begin{scope}[shift={(1.1,0)}]
          \n[n1]{y}{0,0};
          \e{y}{.5,0};
          \site{ly}{y.west};
          \site{ry}{y.east};
          \node at (206:.42) {\scriptsize $l$};
          \node at (26:.42) {\scriptsize $r$};
        \end{scope}
        \begin{scope}[shift={(2.3,0)}]
          \n[n2]{z}{0,0};
          \e{z}{-.5,0};
          \e{z}{.5,0};
          \site{lz}{z.west};
          \site{rz}{z.east};
          \node at (206:.42) {\scriptsize $l$};
          \node at (26:.42) {\scriptsize $r$};
        \end{scope}
      }};
    \path (lhs.east) +(.3,0) edge[rule] +(1,0)
      +(1.3,0) coordinate (r);
    \node[grphnode,anchor=west] (rhs) at (r) {
      \tikz[ingrphdiag]{
        \e{0,0}{2.2,0};
        \begin{scope}[shift={(0,0)}]
          \n[n3]{x}{0,0};
          \site{rx}{x.east};
          \node at (26:.42) {\scriptsize $r$};
        \end{scope}
        \begin{scope}[shift={(1.1,0)}]
          \n[n1]{y}{0,0};
          \site{ly}{y.west};
          \site{ry}{y.east};
          \node at (206:.42) {\scriptsize $l$};
          \node at (26:.42) {\scriptsize $r$};
        \end{scope}
        \begin{scope}[shift={(2.2,0)}]
          \n[n2]{z}{0,0};
          \e{z}{.5,0};
          \site{lz}{z.west};
          \site{rz}{z.east};
          \node at (206:.42) {\scriptsize $l$};
          \node at (26:.42) {\scriptsize $r$};
        \end{scope}
      }};
  \end{tikzpicture}
\end{center}
% A(l,r), B(l,r!l.C) -> A(l,r!1), B(l!1,r!l.C) @ [exp] (-1/2 * 'ab')
\begin{center}
  \begin{tikzpicture}
    \node[grphnode,anchor=east] (lhs) at (0,0) {
      \tikz[ingrphdiag]{
        \begin{scope}[shift={(0,0)}]
          \n[n1]{x}{0,0};
          \e{x}{-.5,0};
          \e{x}{.5,0};
          \site{lx}{x.west};
          \site{rx}{x.east};
          \node at (206:.42) {\scriptsize $l$};
          \node at (26:.42) {\scriptsize $r$};
        \end{scope}
        \e{1.2,0}{2.3,0};
        \begin{scope}[shift={(1.2,0)}]
          \n[n2]{y}{0,0};
          \e{y}{-.5,0};
          \site{ly}{y.west};
          \site{ry}{y.east};
          \node at (206:.42) {\scriptsize $l$};
          \node at (26:.42) {\scriptsize $r$};
        \end{scope}
        \begin{scope}[shift={(2.3,0)}]
          \n[n3]{z}{0,0};
          \site{lz}{z.west};
          \node at (206:.42) {\scriptsize $l$};
        \end{scope}
      }};
    \path (lhs.east) +(.3,0) edge[rule] +(1,0)
      +(1.3,0) coordinate (r);
    \node[grphnode,anchor=west] (rhs) at (r) {
      \tikz[ingrphdiag]{
        \e{0,0}{2.2,0};
        \begin{scope}[shift={(0,0)}]
          \n[n1]{x}{0,0};
          \e{x}{-.5,0};
          \site{lx}{x.west};
          \site{rx}{x.east};
          \node at (206:.42) {\scriptsize $l$};
          \node at (26:.42) {\scriptsize $r$};
        \end{scope}
        \begin{scope}[shift={(1.1,0)}]
          \n[n2]{y}{0,0};
          \site{ly}{y.west};
          \site{ry}{y.east};
          \node at (206:.42) {\scriptsize $l$};
          \node at (26:.42) {\scriptsize $r$};
        \end{scope}
        \begin{scope}[shift={(2.2,0)}]
          \n[n3]{z}{0,0};
          \site{lz}{z.west};
          \node at (206:.42) {\scriptsize $l$};
        \end{scope}
      }};
  \end{tikzpicture}
\end{center}
% A(l!1,r  ), B(l  ,r!3), C(l!3,r!1) -> \
% A(l!1,r!2), B(l!2,r!3), C(l!3,r!1) @ [exp] (-1/2 * ('ab' + 't'))
\begin{center}
  \begin{tikzpicture}
    \node[grphnode,anchor=east] (lhs) at (0,0) {
      \tikz[ingrphdiag]{
        \path[use as bounding box] (-.3,.38) rectangle (1.5,-1.22);
        \e{0,0}{-56.944:1.1};
        \e{0:1.2}{-56.944:1.1};
        \begin{scope}[shift={(0,0)}]
          \n[n1]{x}{0,0};
          \e{x}{.5,0};
          \site{r1}{0:7pt};
          \site{l1}{-60:7pt};
          \node at (-86:12pt) {\scriptsize $l$};
          \node at (26:12pt) {\scriptsize $r$};
        \end{scope}
        \begin{scope}[shift={(0:1.2)}]
          \n[n2]{y}{0,0};
          \e{y}{-.5,0};
          \site{r2}{180:7pt};
          \site{l2}{-120:7pt};
          \node at (154:12pt) {\scriptsize $l$};
          \node at (-94:12pt) {\scriptsize $r$};
        \end{scope}
        \begin{scope}[shift={(-56.944:1.1)}]
          \n[n3]{z}{0,0};
          % angle is 66.111 deg
          \site{r3}{123.0555:7pt};
          \site{l3}{56.9445:7pt};
          \node at (146:12pt) {\scriptsize $r$};
          \node at (34:12pt) {\scriptsize $l$};
        \end{scope}
      }};
    \path (lhs.east) +(.3,0) edge[rule] +(1,0)
      +(1.3,0) coordinate (r);
    \node[grphnode,anchor=west] (rhs) at (r) {
      \tikz[ingrphdiag]{
        \path[use as bounding box] (-.3,.38) rectangle (1.4,-1.22);
        \e{0,0}{0:1.1};
        \e{0,0}{-60:1.1};
        \e{0:1.1}{-60:1.1};
        \begin{scope}[shift={(0,0)}]
          \n[n1]{x}{0,0};
          \site{r1}{0:7pt};
          \site{l1}{-60:7pt};
          \node at (-86:12pt) {\scriptsize $l$};
          \node at (26:12pt) {\scriptsize $r$};
        \end{scope}
        \begin{scope}[shift={(0:1.1)}]
          \n[n2]{y}{0,0};
          \site{r2}{180:7pt};
          \site{l2}{-120:7pt};
          \node at (154:12pt) {\scriptsize $l$};
          \node at (-94:12pt) {\scriptsize $r$};
        \end{scope}
        \begin{scope}[shift={(-60:1.1)}]
          \n[n3]{z}{0,0};
          \site{r3}{120:7pt};
          \site{l3}{60:7pt};
          \node at (146:12pt) {\scriptsize $r$};
          \node at (34:12pt) {\scriptsize $l$};
        \end{scope}
      }};
  \end{tikzpicture}
\end{center}
% C(r!1), A(l!1,r  ), B(l  ,r!3), C(l!3) -> \
% C(r!1), A(l!1,r!2), B(l!2,r!3), C(l!3) @ [exp] (-1/2 * 'ab')
\begin{center}
  \begin{tikzpicture}
    \node[grphnode,anchor=east] (lhs) at (0,0) {
      \tikz[ingrphdiag]{
        \e{0,0}{1.2,0};
        \begin{scope}[shift={(0,0)}]
          \n[n3]{x}{0,0};
          \site{rx}{x.east};
          \node at (26:.42) {\scriptsize $r$};
        \end{scope}
        \begin{scope}[shift={(1.1,0)}]
          \n[n1]{y}{0,0};
          \e{y}{.5,0};
          \site{ly}{y.west};
          \site{ry}{y.east};
          \node at (206:.42) {\scriptsize $l$};
          \node at (26:.42) {\scriptsize $r$};
        \end{scope}
        \e{2.3,0}{3.4,0};
        \begin{scope}[shift={(2.3,0)}]
          \n[n2]{z}{0,0};
          \e{z}{-.5,0};
          \site{lz}{z.west};
          \site{rz}{z.east};
          \node at (206:.42) {\scriptsize $l$};
          \node at (26:.42) {\scriptsize $r$};
        \end{scope}
        \begin{scope}[shift={(3.4,0)}]
          \n[n3]{w}{0,0};
          \site{lw}{w.west};
          \node at (206:.42) {\scriptsize $l$};
        \end{scope}
      }};
    \path (lhs.east) +(.3,0) edge[rule] +(1,0)
      +(1.3,0) coordinate (r);
    \node[grphnode,anchor=west] (rhs) at (r) {
      \tikz[ingrphdiag]{
        \e{0,0}{3.3,0};
        \begin{scope}[shift={(0,0)}]
          \n[n3]{x}{0,0};
          \site{rx}{x.east};
          \node at (26:.42) {\scriptsize $r$};
        \end{scope}
        \begin{scope}[shift={(1.1,0)}]
          \n[n1]{y}{0,0};
          \site{ly}{y.west};
          \site{ry}{y.east};
          \node at (206:.42) {\scriptsize $l$};
          \node at (26:.42) {\scriptsize $r$};
        \end{scope}
        \begin{scope}[shift={(2.2,0)}]
          \n[n2]{z}{0,0};
          \site{lz}{z.west};
          \site{rz}{z.east};
          \node at (206:.42) {\scriptsize $l$};
          \node at (26:.42) {\scriptsize $r$};
        \end{scope}
        \begin{scope}[shift={(3.3,0)}]
          \n[n3]{w}{0,0};
          \site{lw}{w.west};
          \node at (206:.42) {\scriptsize $l$};
        \end{scope}
      }};
  \end{tikzpicture}
\end{center}
The four subcases that do not create a triangle have
$\Delta E = \cost \cdot \Delta\phi = \cost(d_{12})$
where $d_{12}$ is the dimer of an agent of type $1$
and an agent of type $2$.
Hence, their rate under the symmetric linear kinetic model
with $c_r = 0$ would be $\exp{-\cost(d_{12})/2}$.
On the other hand, the fourth refined rule creates a triangle
and thus its
$\Delta E = \cost \cdot \Delta\phi = \cost(d_{12}) + \cost(t)$
where $\cost(t)$ is the energy cost of the triangle.
Its rate then is $\exp{-(\cost(d_{12})+\cost(t))/2}$.
The inverse generator rule $r^-_{12}$ produces as refinements
the inverse of the five subrules enumerated above.
The other generator rules follow a similar pattern
of refinement.

Now we present the KaSim model.
The rules of the model have been manually compressed
to take advantage of KaSim's extended syntax (\eg binding types).
Note also that by using KaSim's variables
we can define the rates parametrically,
allowing us to easily try out different values
for the energy costs.

\begin{lstlisting}[language=kappa]
# Agent declarations
%agent: A(l,r)
%agent: B(l,r)
%agent: C(l,r)

# Energy costs
%var: 't' -10
%var: 'ab' 1
%var: 'bc' 1
%var: 'ca' 1

# Observable
%obs: 'T' |A(l!1, r!2), B(l!2, r!3), C(l!3, r!1)|

# Rules
# A(r), B(l) -> A(r!1), B(l!1) refines into:
A(l,r), B(l,r) -> A(l,r!1), B(l!1,r) @ [exp] (-1/2 * 'ab')
A(l!r.C,r), B(l,r) -> A(l!r.C,r!1), B(l!1,r) @ [exp] (-1/2 * 'ab')
A(l,r), B(l,r!l.C) -> A(l,r!1), B(l!1,r!l.C) @ [exp] (-1/2 * 'ab')
A(l!1,r  ), B(l  ,r!3), C(l!3,r!1) -> \
A(l!1,r!2), B(l!2,r!3), C(l!3,r!1) @ [exp] (-1/2 * ('ab' + 't'))
C(r!1), A(l!1,r  ), B(l  ,r!3), C(l!3) -> \
C(r!1), A(l!1,r!2), B(l!2,r!3), C(l!3) @ [exp] (-1/2 * 'ab')

# A(r!1), B(l!1) -> A(r), B(l) refines into:
A(l,r!1), B(l!1,r) -> A(l,r), B(l,r) @ [exp] -(-1/2 * 'ab')
A(l!r.C,r!1), B(l!1,r) -> A(l!r.C,r), B(l,r) @ [exp] -(-1/2 * 'ab')
A(l,r!1), B(l!1,r!l.C) -> A(l,r), B(l,r!l.C) @ [exp] -(-1/2 * 'ab')
A(l!1,r!2), B(l!2,r!3), C(l!3,r!1) -> \
A(l!1,r  ), B(l  ,r!3), C(l!3,r!1) @ [exp] -(-1/2 * ('ab' + 't'))
C(r!1), A(l!1,r!2), B(l!2,r!3), C(l!3) -> \
C(r!1), A(l!1,r  ), B(l  ,r!3), C(l!3) @ [exp] -(-1/2 * 'ab')

# B(r), C(l) -> B(r!1), C(l!1) refines into:
B(l,r), C(l,r) -> B(l,r!1), C(l!1,r) @ [exp] (-1/2 * 'bc')
B(l!r.A,r), C(l,r) -> B(l!r.A,r!1), C(l!1,r) @ [exp] (-1/2 * 'bc')
B(l,r), C(l,r!l.A) -> B(l,r!1), C(l!1,r!l.A) @ [exp] (-1/2 * 'bc')
B(l!1,r  ), C(l  ,r!3), A(l!3,r!1) -> \
B(l!1,r!2), C(l!2,r!3), A(l!3,r!1) @ [exp] (-1/2 * ('bc' + 't'))
A(r!1), B(l!1,r  ), C(l  ,r!3), A(l!3) -> \
A(r!1), B(l!1,r!2), C(l!2,r!3), A(l!3) @ [exp] (-1/2 * 'bc')

# B(r!1), C(l!1) -> B(r), C(l) refines into:
B(l,r!1), C(l!1,r) -> B(l,r), C(l,r) @ [exp] -(-1/2 * 'bc')
B(l!r.A,r!1), C(l!1,r) -> B(l!r.A,r), C(l,r) @ [exp] -(-1/2 * 'bc')
B(l,r!1), C(l!1,r!l.A) -> B(l,r), C(l,r!l.A) @ [exp] -(-1/2 * 'bc')
B(l!1,r!2), C(l!2,r!3), A(l!3,r!1) -> \
B(l!1,r  ), C(l  ,r!3), A(l!3,r!1) @ [exp] -(-1/2 * ('bc' + 't'))
A(r!1), B(l!1,r!2), C(l!2,r!3), A(l!3) -> \
A(r!1), B(l!1,r  ), C(l  ,r!3), A(l!3) @ [exp] -(-1/2 * 'bc')

# C(r), A(l) -> C(r!1), A(l!1) refines into:
C(l,r), A(l,r) -> C(l,r!1), A(l!1,r) @ [exp] (-1/2 * 'ca')
C(l!r.B,r), A(l,r) -> C(l!r.B,r!1), A(l!1,r) @ [exp] (-1/2 * 'ca')
C(l,r), A(l,r!l.B) -> C(l,r!1), A(l!1,r!l.B) @ [exp] (-1/2 * 'ca')
C(l!1,r  ), A(l  ,r!3), B(l!3,r!1) -> \
C(l!1,r!2), A(l!2,r!3), B(l!3,r!1) @ [exp] (-1/2 * ('ca' + 't'))
B(r!1), C(l!1,r  ), A(l  ,r!3), B(l!3) -> \
B(r!1), C(l!1,r!2), A(l!2,r!3), B(l!3) @ [exp] (-1/2 * 'ca')

# C(r!1), A(l!1) -> C(r), A(l) refines into:
C(l,r!1), A(l!1,r) -> C(l,r), A(l,r) @ [exp] -(-1/2 * 'ca')
C(l!r.B,r!1), A(l!1,r) -> C(l!r.B,r), A(l,r) @ [exp] -(-1/2 * 'ca')
C(l,r!1), A(l!1,r!l.B) -> C(l,r), A(l,r!l.B) @ [exp] -(-1/2 * 'ca')
C(l!1,r!2), A(l!2,r!3), B(l!3,r!1) -> \
C(l!1,r  ), A(l  ,r!3), B(l!3,r!1) @ [exp] -(-1/2 * ('ca' + 't'))
B(r!1), C(l!1,r!2), A(l!2,r!3), B(l!3) -> \
B(r!1), C(l!1,r  ), A(l  ,r!3), B(l!3) @ [exp] -(-1/2 * 'ca')

# Initial mixture
%init: 1000 (A(), B(), C())
\end{lstlisting}

The above KaSim model uses
$\cost(d_{12}) = \cost(d_{23}) = \cost(d_{31}) = 1$
and $\cost(t) = -10$.
Below we will change this values to see how
the production of triangles is affected by them.
We have set the initial mixture to contain
$1000$ copies of each type of agent.
To run a simulation for $50$ time units
and take measurements
(\ie count the number of triangles in the mixture)
every $0.1$ time units,
we issue the following command
\begin{lstlisting}[numbers=none]
$ KaSim t.ka -o t-10.tsv -d t-10 -l 50 -p 0.1
\end{lstlisting} %$
The input file is \lstinline|t.ka| and
the measurements are saved in the \lstinline|t-10.tsv|
in the \lstinline|t-10| folder.
The resulting plots are displayed in \fig{triangles}.

\begin{figure}
  \begin{center}
    \includegraphics[width=.9\linewidth]{triangles/e0/t-sa}
  \end{center}
  \begin{center}
    \includegraphics[width=.9\linewidth]{triangles/e1/t-sa}
  \end{center}
  \caption{
    Trajectories for the number of triangles when $\cost(t)$ varies.
    In the plot above the energy cost of the dimers is $0$
    whereas in the plot below they are set to $1$.}
  \label{fig:triangles}
\end{figure}

% TODO: pagebreaks are ugly
\pagebreak

First, we notice that the moderate energy penalty
we impose on dimers in the second plot does not change much
the number of triangles at equilibrium.
It does, however, have an impact
on the speed at which the triangles form.
This effect is perhaps counter-intuitive.
% TODO: explain the effect

Second, notice that % it is interesting to note that
when $\cost(t) = -10$ all agents are used to build triangles.
In contrast, when $\cost(t) = -5$ less than 20\%
of the agents of each type are used.
In both cases the set of states
with a globally minimum energy is the same,
% that minimise the energy function is the same,
namely those states that maximise the amount of triangles.
So then why is it that in the latter case there are so few triangles?
The reason is entropic:
although the probability of being in a state with few triangles is small,
there are many such states and together they outweigh
the probability of being in the few states were the energy is minimal.
By further decreasing the energy of those few states
we compensate for this mass effect,
until at $\cost(t) = -10$, order wins,
and the effect is not noticeable anymore.


\section{Example: Flagellum's engine}
\label{sec:alloring}

In this section we present another model.
This model is inspired in a classical
object of study in molecular biology:
the bacterial flagellar engine.
% In this section we present a model
% of a bacterial flagellar engine.
%
The flagellar engine can rotate clockwise or anti-clockwise
at high angular velocities.
% This decides whether the bacterium tumbles or swims forward.
When it rotates clockwise the filaments of the flagellum
move chaotically in all directions,
making the bacterium tumble % in a fixed position
and thus randomly change the axis % direction
of its body and engine.
When it rotates anti-clockwise
the filaments of the flagellum align
and move synchronously,
propulsing the bacterium in the direction
the engine is pointing to.
In the latter regime the bacterium thus swims forward.
% which reaches an astonishing 100,000 rpm.
% In the first regime, flagella get tangled together
% resulting in the bacterium tumbling;
% in the second one, the bacterium swims forward.
% This random walk is driven by molecular devices,
% embedded in the bacteria's outer membrane,
% that sense various aspects of the chemical environment
% \citep{sourjik}.
When the bacterium detects that the levels of food are decreasing
or the amount of poisonous substances is increasing,
it tumbles to change the direction in which its swimming.
In this way it implements a basic chemotactic system.

A simple model of the switch between the two modes
has been proposed by \citet{teuta}.
In this model
the engine is seen as a ring of $n$ identical components,
called protomers or $P$ for short,
with two possible conformations, $0$ and $1$.
Here we take $n=34$ for simulations and diagrams
but the analysis does not depend on the specific value of $n$.
% (In reality, each of the $n=34$ component protomers is itself
%  a tiny complex made of different subcomponents,
%  but the model ignores this).
A ring homogeneously in state~$0$ ($1$) rotates (anti-)clockwise
and induces tumbling (straight motion).
Importantly, neighbouring $P$s on the ring prefer to have
matching conformations. % (as in the Ising model).
States of the ring with many mismatches thus incur high penalties.
A small diffusible protein named CheY,
which we call $Y$ for short,
binds $P$ when it is activated.
When $Y$ is binding $P$, $P$ favours state~$1$.
Conversely, in the absence of a $Y$ molecule binding $P$,
$P$ favours state~$0$.
CheY, in turn, is activated by the system of chemoreceptors
in the presence of food and abscence of poisions.\footnote{
  Here we assume that every $Y$ is an activated CheY.}
The configuration of the chemoreceptor cluster and its activity
have also been modelled thermodynamically \citep{sourjik}.

\newagent{\nP}{P}{l//west/,r//east/,y//north/}
\newagent{\nY}{Y}{p//south/}
\tikzstyle{on}=[fill=green!60]
\tikzstyle{off}=[fill=black!30]

\begin{figure}
  \begin{center}
    \begin{tikzpicture}[agent/.append style={transform shape}]
      \def\radius{4}
      % TODO: How can I define etoolbox's internal lists directly?
      \def\on{}
      \foreach \i in {4,5,6,7,20}{\listxadd{\on}{\i}}
      \def\ys{}
      \foreach \i in {4,6,15,20}{\listxadd{\ys}{\i}}
      \draw[very thick] (0,0) circle (\radius);
      \foreach \angle [count=\i] in {0,18,...,342} {%
        \begin{scope}[shift={(\angle+90:\radius)},rotate=\angle]
          \def\state{off}
          \xifinlist{\i}{\on}{\gdef\state{on}}{}
          \xifinlist{\i}{\ys}{%
            \e{0,0}{0,1.2};
            \nY{y\i}{0,1.2}{p};
          }{\e{0,0}{0,.6};}
          \nP[\state]{p\i}{0,0}{l,r,y};
        \end{scope}
      };
      % \node at (0.2,0) {\includegraphics[width=5.5cm]{flagella2.jpg}};
    \end{tikzpicture}
  \end{center}
  \caption{Ring of protomers with some $Y$s bound.
    Since only a few $Y$s are bound to the ring,
    the majority of protomers is likely to be in state~$0$
    (visually represented as grey nodes)
    and a minority in state~$1$ (green).}
  \label{fig:ring}
\end{figure}

As each of the $P$s can be in four states,
a ring of size $34$ has on the order of
$10^{18}$ non-isomorphic configurations.
% NOTE: it's not 10^20 because 4^34/34 ~ 8.68*10^18
% All 4-bit strings of length 34 is 4^34 ~ 2.95*10^20
% but each one of them is isomorphic to 33 other strings
% when they are in a cycle.
This precludes a Petri net approach to the dynamics
% This precludes any reaction-based (\eg Petri nets) approach
where each state of the whole ring
is considered as one chemical species.
We thus use the rule-based approach pioneered in Kappa % presented here
that allows us to specify events based only on
a partial and local context around each protomer
% a partial and local representation of the state
% a minimal necessary context around each protomer
and derive the set of rules
by applying the method of \sct{energy-gp}.

We define the contact graph of the model as
\begin{center}
  \begin{tikzpicture}[grphdiag]
    \draw (0,-.3) circle (.5);
    \e{0,0}{0,1.2};
    \draw[on ,draw opacity=0] (135:10pt) arc (135:315:10pt);
    \draw[off,draw opacity=0] (135:10pt) arc (135:-45:10pt);
    \nP[fill opacity=0]{p}{0,0}{l,r,y};
    \node at (0,0) {P};
    \site{p-d}{p.south};
    \node at (158:.5) {\scriptsize $a$};
    \node at (26:.5) {\scriptsize $b$};
    \node at (114:.5) {\scriptsize $c$};
    \node at (-65:.5) {\scriptsize $d$};
    \nY{y}{0,1.2}{p};
  \end{tikzpicture}
\end{center}
where $P$ has 4 sites $a,b,c,d$.
The first two form the backbone of the ring
while $c$ can bind $Y$s. % an agent of type $Y$.
Site $d$ encodes the conformation state of $P$:
we say $P$ is in state~$0$ when site $d$ is bound to an $A$ agent
and is in state~$1$ when bound to a $B$ agent
($A$ and $B$ agents are not displayed in the contact graph above).
We will never mention this site (nor $A$ and $B$ agents) explicitly
but instead will colour the agent of type $P$ accordingly.%
% NOTE: the real encoding is more complex.
% P-A-B is a P in state 0 and P-B-A is a P in state 1.
\footnote{
  The na\"ive encoding where
  i) $A$ and $B$ have a free site that can bind site $d$ of $P$,
  ii) whenever $P$ changes from state $0$ to state $1$
  we detach an $A$ from $P$ and attach a $B$ to it,
  and thus iii) we have a pool of free $A$s and $B$s in the mixture,
  will have a problem with kinetics due to mass action:
  when we attach a $B$ to $P$ we make it less likely
  for the next $P$ to bind a $B$
  since there are less $B$s free in the mixture.
  To solve this issue every $P$ is either bound
  to an $A$ that is in turn bound to a $B$
  or a $B$ that is bound to an $A$.
  Whenever we want to change state
  we only need to exchange the order of the $A$ and $B$.}
Also, we will draw sites $a,b,c$
always on the left, right and top of $P$, respectively,
and thus forgo annotating the name of the site.

The informal statements about the favoured states of $P$
in the different configurations discussed above are captured
in the definition of the energy patterns and associated energy costs.
Note that the various patterns overlap.

\vspace{8pt}
\noindent
\hspace{-5pt}
\begin{tikzpicture}

  \def\dist{1.4}

  %% Energy patterns %%
  \node[grphnode] (e0) at (0,1.4) {
    \tikz[ingrphdiag]{
      \nP[off]{p0}{0,0}{};
    }};
  \node[anchor=east] at (e0.west) {$\cost_0 :=$};
  \node[grphnode] (e1) at (0,0) {
    \tikz[ingrphdiag]{
      \nP[on]{p1}{0,0}{};
    }};
  \node[anchor=east] at (e1.west) {$\cost_1 :=$};

  \path (e0.east) ++(\dist,0) coordinate (c00);
  \path (e1.east) ++(\dist,0) coordinate (c11);
  \node[grphnode,anchor=west] (e00) at (c00) {
    \tikz[ingrphdiag]{
      \nP[off]{p00f}{0,0}{r};
      \nP[off]{p00s}{1.2,0}{l};
      \e{p00f-r}{p00s-l};
      % \node at (26:.49) {\scriptsize $b$};
      % \node[shift={(1.2,0)}] at (-158:.475) {\scriptsize $a$};
    }};
  \node[anchor=east] at (e00.west) {$\cost_{00} :=$};
  \node[grphnode,anchor=west] (e11) at (c11) {
    \tikz[ingrphdiag]{
      \nP[on]{p11f}{0,0}{r};
      \nP[on]{p11s}{1.2,0}{l};
      \e{p11f-r}{p11s-l};
    }};
  \node[anchor=east] at (e11.west) {$\cost_{11} :=$};

  \path (e00.east) ++(\dist,0) coordinate (c10);
  \path (e11.east) ++(\dist,0) coordinate (c01);
  \node[grphnode,anchor=west] (e10) at (c10) {
    \tikz[ingrphdiag]{
      \nP[on]{p10f}{(0,0}{r};
      \nP[off]{p10s}{1.2,0}{l};
      \e{p10f-r}{p10s-l};
    }};
  \node[anchor=east] at (e10.west) {$\cost_{10} :=$};
  \node[grphnode,anchor=west] (e01) at (c01) {
    \tikz[ingrphdiag]{
      \nP[off]{p01f}{0,0}{r};
      \nP[on]{p01s}{1.2,0}{l};
      \e{p01f-r}{p01s-l};
    }};
  \node[anchor=east] at (e01.west) {$\cost_{01} :=$};

  \path (e01.east) ++(\dist,.7) coordinate (c0y);
  \node[grphnode,anchor=west] (e0y) at (c0y) {
    \tikz[ingrphdiag]{
      \nP[off]{p0y}{0,0}{y};
      \nY{y0}{0,1.2}{p};
      \e{p0y-y}{y0-p};
    }};
  \node[anchor=east] at (e0y.west) {$\cost_0^Y :=$};

  \path (e0y.east) ++(\dist,0) coordinate (c1y);
  \node[grphnode,anchor=west] (e1y) at (c1y) {
    \tikz[ingrphdiag]{
      \nP[on]{p1y}{0,0}{y};
      \nY{y1}{0,1.2}{p};
      \e{p1y-y}{y1-p};
    }};
  \node[anchor=east] at (e1y.west) {$\cost_1^Y :=$};

\end{tikzpicture}

%%% Local Variables:
%%% mode: latex
%%% TeX-master: "thesis"
%%% End:


We abuse notation by referring to both the pattern
and its energy cost as~$\cost_{ij}$.
The following constraints are imposed
on the energy costs:
\begin{align}
  \label{eq:ePP}
  \cost_{00},\cost_{11} & {}< \cost_{10},\cost_{01} \\
  \label{eq:eP}
  \cost_{0} & {}< \cost_{1} \\
  \label{eq:ePY}
  \cost^Y_{0} & {}> \cost^Y_{1}
\end{align}
These inequalities enact the considerations in the discussion above.
The role of \eqn{ePP} is to align
the states of neighbours on the ring
--- essentially an Ising term which spreads conformation.
\eqn{eP} makes $0$ the favoured state,
while \eqn{ePY} inverts the situation % makes $1$ the favoured state
in the presence of $Y$.
% which says that when bound to $Y$, $P$ prefers state $1$.
% \footnote{
%   So far one needs 8 parameters
%   (really 7 as energy is defined up to an additive constant);
%   symmetries of the problem can bring this number down,
%   \eg $\cost_{00} = \cost_{11}$,
%   meaning no alignment is favoured,
%   or $\cost^Y_{0} - \cost^Y_{1} = 2(\cost_{1} -\cost_{0})$
%   to exactly exchange the $0/1$ conformational distribution
%   when binding.}

Following \sct{rates}
we associate to each ring configuration $x$
the occurrence vector $\shapes(x)$
and total energy $\cost \cdot \shapes(x)$.
For example,
a ring of size $n$ uniformly in state $0$ with no bound $Y$s
has total energy $n(\cost_{00}+\cost_{0})$.

The next step is to define the set of generator rules $\generators$.
The first pair of rules that we include in this set
are $r^+_Y$ the binding of $P$ and $Y$ and its inverse $r^-_Y$.
\begin{center}
  \begin{tikzpicture}
    \node[grphnode,anchor=east] (lhs) at (0,0) {
      \tikz[ingrphdiag]{
        \e{0,0}{0,0.6};
        \e{0,1.4}{0,0.8};
        \nP{p}{0,0}{y};
        \nY{y}{0,1.4}{p};
      }};
    \path (lhs.east) +(.3,.09) edge[rule] +(1,.09)
      +(1.3,0) coordinate (r);
    \path (lhs.east) +(1,-.09) edge[rule] +(.3,-.09);
    \node[grphnode,anchor=west] (rhs) at (r) {
      \tikz[ingrphdiag]{
        \e{0,0}{0,1.2};
        \nP{rp}{0,0}{y};
        \nY{ry}{0,1.2}{p};
      }};
  \end{tikzpicture}
\end{center}
An uncoloured $P$ means
it can bind a $Y$ regardless of the state it is in.
The nature of the method presented in \sct{energy-gp}
allows us to refine each rule individually,
so we proceed to refine $r^+_Y,r^-_Y$ immediately.
We first give the rationale for the refinements informally.
The pair of rules has an ambiguous energy balance % $\Delta E$
% as it will be either $\cost^Y_0$ or $\cost^Y_1$
% depending on its instances.
because applying the forward rule $r^+_Y$ to a $P$ in state~$0$
will create an $\cost^Y_0$ pattern
% because applying the forward (backward) rule $r^+_Y$ ($r^-_Y$)
% to a $P$ in state~$0$ will create (destroy) a $\cost^Y_0$ pattern
while applying it to a $P$ in state~$1$
% while $r^+_Y$ applied to a $P$ in state~$1$
will create an $\cost^Y_1$ pattern.
Hence, we cannot assign % have no hope of assigning
rates to these rules that satisfy detailed balance ---
unless $\cost^Y_0 = \cost^Y_1$, which contradicts \eqn{ePY}.
To get $\shapes$-balanced rules one needs to refine $r^+_Y,r^-_Y$ into
\begin{center}
  \begin{tikzpicture}
    \begin{scope}
      \node[grphnode,anchor=east] (lhs1) at (0,0) {
        \tikz[ingrphdiag]{
          \e{0,0}{0,0.6};
          \e{0,1.4}{0,0.8};
          \nP[off]{p}{0,0}{y};
          \nY{y}{0,1.4}{p};
        }};
      \path (lhs1.east) +(.3,.09) edge[rule] +(1,.09)
        +(1.3,0) coordinate (r1);
      \path (lhs1.east) +(1,-.09) edge[rule] +(.3,-.09);
      \node[grphnode,anchor=west] (rhs1) at (r1) {
        \tikz[ingrphdiag]{
          \e{0,0}{0,1.2};
          \nP[off]{rp}{0,0}{y};
          \nY{ry}{0,1.2}{p};
        }};
    \end{scope}

    \begin{scope}[shift={(6,0)}]
      \node[grphnode,anchor=east] (lhs2) at (0,0) {
        \tikz[ingrphdiag]{
          \e{0,0}{0,0.6};
          \e{0,1.4}{0,0.8};
          \nP[on]{p}{0,0}{y};
          \nY{y}{0,1.4}{p};
        }};
      \path (lhs2.east) +(.3,.09) edge[rule] +(1,.09)
        +(1.3,0) coordinate (r2);
      \path (lhs2.east) +(1,-.09) edge[rule] +(.3,-.09);
      \node[grphnode,anchor=west] (rhs2) at (r2) {
        \tikz[ingrphdiag]{
          \e{0,0}{0,1.2};
          \nP[on]{rp}{0,0}{y};
          \nY{ry}{0,1.2}{p};
        }};
    \end{scope}
    \path (rhs1) -- node {and} (lhs2);
  \end{tikzpicture}
\end{center}
We call the refined rules $r^+_{Y0},r^-_{Y0}$ and $r^+_{Y1},r^-_{Y1}$.
Each rule $r^+_{Yi}$ ($i \in \set{0,1}$) specifies
enough of the context in which it applies
to have a definite energy balance $\Delta E = \cost^Y_i$.
The second pair of rules in $\generators$ flip the state of $P$:
\begin{center}
  \begin{tikzpicture}
    \node[grphnode,anchor=east] (lhs) at (0,0) {
      \tikz[ingrphdiag]{
        \nP[off]{p0}{-1,0}{};
      }};
    \path (lhs.east) +(.3,.09) edge[rule] +(1,.09)
      +(1.3,0) coordinate (r);
    \path (lhs.east) +(1,-.09) edge[rule] +(.3,-.09);
    \node[grphnode,anchor=west] (rhs) at (r) {
      \tikz[ingrphdiag]{
        \nP[on]{p0}{1,0}{};
      }};
  \end{tikzpicture}
\end{center}
This pair of rules generates many more refinements
as changing the state of $P$ will create and destroy
matches $\cost_{00},\cost_{11}$ and mismatches $\cost_{10},\cost_{01}$
between $P$ and its neighbours in the ring.
The refinements must then reveal a larger context
that includes at least the neighbourhood of $P$
and therefore account for all combinations of neighbours' states.
% Since the rule does not change the state of the neighbours,
% Since the neighbours do not change their state
Since the state of the neighbours is not changed
when the rule is applied,
% we do not need to reveal the state of \emph{their} neighbours,
we do not need to reveal the state of the neighbours' neighbours,
which saves us from an infinite recursion of revelations.%
\footnote{
  Indeed \thm{energy-gp} guarantees that
  such infinite recursions never occur.}
We must also know whether the $P$
that is subject to the action of the rule
is bound to a $Y$ as when it is % in its presence
patterns $\cost^Y_0$ and $\cost^Y_1$
would be consumed and produced.
Hence, the refinements of this second pair of rules are

% on off
\begin{flushleft}
  \hspace{.5cm}
  \begin{tikzpicture}
    \node[grphnode,anchor=east] (lhs) at (0,0) {
      \tikz[ingrphdiag]{
        \nP[on]{f2l-p1}{0,0}{r};
        \nP[off]{f2l-p2}{1.2,0}{l,r,y};
        \nP[off]{f2l-p3}{2.4,0}{l};
        \link{f2l-p1-r}{f2l-p2-l};
        \link{f2l-p2-r}{f2l-p3-l};
        \stub{f2l-p2}{y};
      }};
    \path (lhs.east) +(.3,.09) edge[rule] +(1,.09)
      +(1.3,0) coordinate (r);
    \path (lhs.east) +(1,-.09) edge[rule] +(.3,-.09);
    \node[grphnode,anchor=west] (rhs) at (r) {
      \tikz[ingrphdiag]{
        \nP[on]{f2r-p1}{0,0}{r};
        \nP[on]{f2r-p2}{1.2,0}{l,r,y};
        \nP[off]{f2r-p3}{2.4,0}{l};
        \link{f2r-p1-r}{f2r-p2-l};
        \link{f2r-p2-r}{f2r-p3-l};
        \stub{f2r-p2}{y};
      }};
    \path (rhs.east) ++(1,0) node[anchor=west] {
      $\begin{array}{@{}l@{}l@{}}
         \Delta E &{}= \cost_1 - \cost_0 \\
                  &{}+ \cost_{11} - \cost_{00}
       \end{array}$};
      % $\Delta E = \cost_{11} - \cost_{00}$};
  \end{tikzpicture}
\end{flushleft}
% off on
\begin{flushleft}
  \hspace{.5cm}
  \begin{tikzpicture}
    \node[grphnode,anchor=east] (lhs) at (0,0) {
      \tikz[ingrphdiag]{
        \nP[off]{f2l-p1}{0,0}{r};
        \nP[off]{f2l-p2}{1.2,0}{l,r,y};
        \nP[on]{f2l-p3}{2.4,0}{l};
        \link{f2l-p1-r}{f2l-p2-l};
        \link{f2l-p2-r}{f2l-p3-l};
        \stub{f2l-p2}{y};
      }};
    \path (lhs.east) +(.3,.09) edge[rule] +(1,.09)
      +(1.3,0) coordinate (r);
    \path (lhs.east) +(1,-.09) edge[rule] +(.3,-.09);
    \node[grphnode,anchor=west] (rhs) at (r) {
      \tikz[ingrphdiag]{
        \nP[off]{f2r-p1}{0,0}{r};
        \nP[on]{f2r-p2}{1.2,0}{l,r,y};
        \nP[on]{f2r-p3}{2.4,0}{l};
        \link{f2r-p1-r}{f2r-p2-l};
        \link{f2r-p2-r}{f2r-p3-l};
        \stub{f2r-p2}{y};
      }};
    \path (rhs.east) ++(1,0) node[anchor=west] {
      $\begin{array}{@{}l@{}l@{}}
         \Delta E &{}= \cost_1 - \cost_0 \\
                  &{}+ \cost_{11} - \cost_{00}
       \end{array}$};
      % $\Delta E = \cost_{11} - \cost_{00}$};
  \end{tikzpicture}
\end{flushleft}
% off off
\begin{flushleft}
  \hspace{.5cm}
  \begin{tikzpicture}
    \node[grphnode,anchor=east] (lhs) at (0,0) {
      \tikz[ingrphdiag]{
        \nP[off]{f2l-p1}{0,0}{r};
        \nP[off]{f2l-p2}{1.2,0}{l,r,y};
        \nP[off]{f2l-p3}{2.4,0}{l};
        \link{f2l-p1-r}{f2l-p2-l};
        \link{f2l-p2-r}{f2l-p3-l};
        \stub{f2l-p2}{y};
      }};
    \path (lhs.east) +(.3,.09) edge[rule] +(1,.09)
      +(1.3,0) coordinate (r);
    \path (lhs.east) +(1,-.09) edge[rule] +(.3,-.09);
    \node[grphnode,anchor=west] (rhs) at (r) {
      \tikz[ingrphdiag]{
        \nP[off]{f2r-p1}{0,0}{r};
        \nP[on]{f2r-p2}{1.2,0}{l,r,y};
        \nP[off]{f2r-p3}{2.4,0}{l};
        \link{f2r-p1-r}{f2r-p2-l};
        \link{f2r-p2-r}{f2r-p3-l};
        \stub{f2r-p2}{y};
      }};
    \path (rhs.east) ++(1,0) node[anchor=west] {
      $\begin{array}{@{}l@{}l@{}}
         \Delta E &{}= \cost_1 - \cost_0 \\
                  &{}+ \cost_{01} + \cost_{10} - 2 \cost_{00}
       \end{array}$};
      % $\Delta E = \cost_{01} + \cost_{10} - 2 \cost_{00}$};
  \end{tikzpicture}
\end{flushleft}
% on on
\begin{flushleft}
  \hspace{.5cm}
  \begin{tikzpicture}
    \node[grphnode,anchor=east] (lhs) at (0,0) {
      \tikz[ingrphdiag]{
        \nP[on]{f2l-p1}{0,0}{r};
        \nP[off]{f2l-p2}{1.2,0}{l,r,y};
        \nP[on]{f2l-p3}{2.4,0}{l};
        \link{f2l-p1-r}{f2l-p2-l};
        \link{f2l-p2-r}{f2l-p3-l};
        \stub{f2l-p2}{y};
      }};
    \path (lhs.east) +(.3,.09) edge[rule] +(1,.09)
      +(1.3,0) coordinate (r);
    \path (lhs.east) +(1,-.09) edge[rule] +(.3,-.09);
    \node[grphnode,anchor=west] (rhs) at (r) {
      \tikz[ingrphdiag]{
        \nP[on]{f2r-p1}{0,0}{r};
        \nP[on]{f2r-p2}{1.2,0}{l,r,y};
        \nP[on]{f2r-p3}{2.4,0}{l};
        \link{f2r-p1-r}{f2r-p2-l};
        \link{f2r-p2-r}{f2r-p3-l};
        \stub{f2r-p2}{y};
      }};
    \path (rhs.east) ++(1,0) node[anchor=west] {
      $\begin{array}{@{}l@{}l@{}}
         \Delta E &{}= \cost_1 - \cost_0 \\
                  &{}+ 2 \cost_{11} - \cost_{10} - \cost_{01}
       \end{array}$};
      % $\Delta E = 2 \cost_{11} - \cost_{10} - \cost_{01}$};
  \end{tikzpicture}
\end{flushleft}
% Y on off
\begin{flushleft}
  \hspace{.5cm}
  \begin{tikzpicture}
    \node[grphnode,anchor=east] (lhs) at (0,0) {
      \tikz[ingrphdiag]{
        \nP[on]{f2l-p1}{0,0}{r};
        \nP[off]{f2l-p2}{1.2,0}{l,r,y};
        \nP[off]{f2l-p3}{2.4,0}{l};
        \link{f2l-p1-r}{f2l-p2-l};
        \link{f2l-p2-r}{f2l-p3-l};
        \nY{f2l-y}{1.2,1.2}{p};
        \link{f2l-p2-y}{f2l-y-p};
      }};
    \path (lhs.east) +(.3,.09) edge[rule] +(1,.09)
      +(1.3,0) coordinate (r);
    \path (lhs.east) +(1,-.09) edge[rule] +(.3,-.09);
    \node[grphnode,anchor=west] (rhs) at (r) {
      \tikz[ingrphdiag]{
        \nP[on]{f2r-p1}{0,0}{r};
        \nP[on]{f2r-p2}{1.2,0}{l,r,y};
        \nP[off]{f2r-p3}{2.4,0}{l};
        \link{f2r-p1-r}{f2r-p2-l};
        \link{f2r-p2-r}{f2r-p3-l};
        \nY{f2r-y}{1.2,1.2}{p};
        \link{f2r-p2-y}{f2r-y-p};
      }};
    \path (rhs.east) ++(1,0) node[anchor=west] {
      $\begin{array}{@{}l@{}l@{}}
         \Delta E &{}= \cost_1 - \cost_0 \\
                  &{}+ \cost_{11} - \cost_{00} \\
                  &{}+ \cost_1^Y - \cost_0^Y
       \end{array}$};
  \end{tikzpicture}
\end{flushleft}
% Y off on
\begin{flushleft}
  \hspace{.5cm}
  \begin{tikzpicture}
    \node[grphnode,anchor=east] (lhs) at (0,0) {
      \tikz[ingrphdiag]{
        \nP[off]{f2l-p1}{0,0}{r};
        \nP[off]{f2l-p2}{1.2,0}{l,r,y};
        \nP[on]{f2l-p3}{2.4,0}{l};
        \link{f2l-p1-r}{f2l-p2-l};
        \link{f2l-p2-r}{f2l-p3-l};
        \nY{f2l-y}{1.2,1.2}{p};
        \link{f2l-p2-y}{f2l-y-p};
      }};
    \path (lhs.east) +(.3,.09) edge[rule] +(1,.09)
      +(1.3,0) coordinate (r);
    \path (lhs.east) +(1,-.09) edge[rule] +(.3,-.09);
    \node[grphnode,anchor=west] (rhs) at (r) {
      \tikz[ingrphdiag]{
        \nP[off]{f2r-p1}{0,0}{r};
        \nP[on]{f2r-p2}{1.2,0}{l,r,y};
        \nP[on]{f2r-p3}{2.4,0}{l};
        \link{f2r-p1-r}{f2r-p2-l};
        \link{f2r-p2-r}{f2r-p3-l};
        \nY{f2r-y}{1.2,1.2}{p};
        \link{f2r-p2-y}{f2r-y-p};
      }};
    \path (rhs.east) ++(1,0) node[anchor=west] {
      $\begin{array}{@{}l@{}l@{}}
         \Delta E &{}= \cost_1 - \cost_0 \\
                  &{}+ \cost_{11} - \cost_{00} \\
                  &{}+ \cost_1^Y - \cost_0^Y
       \end{array}$};
  \end{tikzpicture}
\end{flushleft}
% Y off off
\begin{flushleft}
  \hspace{.5cm}
  \begin{tikzpicture}
    \node[grphnode,anchor=east] (lhs) at (0,0) {
      \tikz[ingrphdiag]{
        \nP[off]{f2l-p1}{0,0}{r};
        \nP[off]{f2l-p2}{1.2,0}{l,r,y};
        \nP[off]{f2l-p3}{2.4,0}{l};
        \link{f2l-p1-r}{f2l-p2-l};
        \link{f2l-p2-r}{f2l-p3-l};
        \nY{f2l-y}{1.2,1.2}{p};
        \link{f2l-p2-y}{f2l-y-p};
      }};
    \path (lhs.east) +(.3,.09) edge[rule] +(1,.09)
      +(1.3,0) coordinate (r);
    \path (lhs.east) +(1,-.09) edge[rule] +(.3,-.09);
    \node[grphnode,anchor=west] (rhs) at (r) {
      \tikz[ingrphdiag]{
        \nP[off]{f2r-p1}{0,0}{r};
        \nP[on]{f2r-p2}{1.2,0}{l,r,y};
        \nP[off]{f2r-p3}{2.4,0}{l};
        \link{f2r-p1-r}{f2r-p2-l};
        \link{f2r-p2-r}{f2r-p3-l};
        \nY{f2r-y}{1.2,1.2}{p};
        \link{f2r-p2-y}{f2r-y-p};
      }};
    \path (rhs.east) ++(1,0) node[anchor=west] {
      $\begin{array}{@{}l@{}l@{}}
         \Delta E &{}= \cost_1 - \cost_0 \\
                  &{}+ \cost_{01} + \cost_{10} - 2 \cost_{00} \\
                  &{}+ \cost_1^Y - \cost_0^Y
       \end{array}$};
  \end{tikzpicture}
\end{flushleft}
% Y on on
\begin{flushleft}
  \hspace{.5cm}
  \begin{tikzpicture}
    \node[grphnode,anchor=east] (lhs) at (0,0) {
      \tikz[ingrphdiag]{
        \nP[on]{f2l-p1}{0,0}{r};
        \nP[off]{f2l-p2}{1.2,0}{l,r,y};
        \nP[on]{f2l-p3}{2.4,0}{l};
        \link{f2l-p1-r}{f2l-p2-l};
        \link{f2l-p2-r}{f2l-p3-l};
        \nY{f2l-y}{1.2,1.2}{p};
        \link{f2l-p2-y}{f2l-y-p};
      }};
    \path (lhs.east) +(.3,.09) edge[rule] +(1,.09)
      +(1.3,0) coordinate (r);
    \path (lhs.east) +(1,-.09) edge[rule] +(.3,-.09);
    \node[grphnode,anchor=west] (rhs) at (r) {
      \tikz[ingrphdiag]{
        \nP[on]{f2r-p1}{0,0}{r};
        \nP[on]{f2r-p2}{1.2,0}{l,r,y};
        \nP[on]{f2r-p3}{2.4,0}{l};
        \link{f2r-p1-r}{f2r-p2-l};
        \link{f2r-p2-r}{f2r-p3-l};
        \nY{f2r-y}{1.2,1.2}{p};
        \link{f2r-p2-y}{f2r-y-p};
      }};
    \path (rhs.east) ++(1,0) node[anchor=west] {
      $\begin{array}{@{}l@{}l@{}}
         \Delta E &{}= \cost_1 - \cost_0 \\
                  &{}+ 2 \cost_{11} - \cost_{10} - \cost_{01} \\
                  &{}+ \cost_1^Y - \cost_0^Y
       \end{array}$};
  \end{tikzpicture}
\end{flushleft}

%%% Local Variables:
%%% mode: latex
%%% TeX-master: "thesis"
%%% End:


In general, if we write $i$ for the state of the left neighbour
and $j$ for that of the right neighbour,
we have that the energy balance for the first 4 refinements is
$\cost_{i1}+\cost_{1j}-\cost_{i0}-\cost_{0j}+\cost_1-\cost_0$
and for the last 4 is
$\cost_{i1}+\cost_{1j}-\cost_{i0}-\cost_{0j}+\cost_1-\cost_0+
\cost^Y_1-\cost^Y_0$.
As there are 10 pairs of refined rules in total ($2+8$)
% (2 for the first pair of rules and 8 for the second)
and only 8 energy patterns,
there must be linear dependencies between the various balances.
Indeed, the family of vector balances has rank six
given by basis vectors $\cost^Y_1$, $\cost^Y_0$,
$\cost_{00}$, $\cost_{11}$,
$\cost_{01}+\cost_{10}$ and $\cost_1-\cost_0$.
% Thermodynamic consistency induces relationships between rates;
% a well-established fact in the case of reaction networks
% (\eg see \citet{et2}).
This example portrays how thermodynamic consistency
(\ie detailed balance)
induces relationships between the rates of the refined rules.

\begin{figure}[t]
  \centering
  \includegraphics[width=\linewidth]{flagellum-engine/y34/fe-sa}
  \caption{
    The simulation steps up the amount of $Y$ (green curve)
    at $t=100$ and down again at $t=200$.
    This sends the vast majority of the ring
    into state~$1$ (orange curve)
    and then back to state~$0$ (blue curve).
    The number of mismatches (purple curve) stays low
    even during transitions.
    The parameters for the simulation are
    $\cost_0 = \cost_{00} = \cost_{11} = -1$,
    $\cost_1 = \cost_{01} = \cost_{10} = 1$,
    $\cost^Y_0 = 2$ and $\cost^Y_1 = -2$.}
  \label{fig:PY}
\end{figure}

The final step is to choose concrete rates for our refined rules.
% This guarantees that the obtained rule set converges to the equilibrium
% specified by the choice of the energy cost vector.
% Convergence will happen whatever $\cost$ is, \ie\ symbolically.
We do so by using the symmetric linear kinetic model of \eqn{sym-lkm}.
% If, in addition, $\cost$ follows (\ref{eq:ePP}--\ref{eq:ePY})
% and one uses the symmetric linear kinetic model,
% Given that $\cost$ follows Eq.~\ref{eq:ePP}--\ref{eq:ePY},
% one can see in \fig{PY} that the ring
% (i) undergoes sharp transitions
% when $Y$ is stepped up and down again;
% and (ii) has at all times very few mismatches.
% NOTE: it is not true in general that the simulation
% will show a sharp transition when \cost follows those equations.
% When \cost_{00} = \cost_{11} = -2 and \cost_{01} = \cost_{10} = 2
% the transition is more gradual because the penalty
% of changing P's state to 1 in a ring full of 0s is bigger than
% the reward given by \cost^Y_1 (of course it depends as well on
% the time scales and the sampling frequency how sharp it looks)
In \fig{PY} one can see the result of a simulation
for the given parameter values
when $Y$ is stepped up and down again.
The model behaves as the one-dimensional cyclic Ising model
where the role of the magnetic field is played by $Y$.
% (This model and its phase transition relate to
%  a well-understood one-dimensional cyclic Ising model,
%  where the role of the magnetic field is played by $Y$.
%  In this context, our rule set plays the role of
%  the so-called Glauber dynamics.)
%
We can visualise the obtained simulations
by extracting snapshots before, during and after
the injection of $Y$s, as in \fig{snapshot}.
% Again we see few mismatches in both regimes because of
% the Ising interaction expressed by the $\cost_{ij}$ energy costs.
% We use \href{https://github.com/Kappa-Dev/KaSim}{KaSim}
% (the standard Kappa engine) for the simulation.

\begin{figure}[t!]
  \centering
  \resizebox{\linewidth}{!}{%
  \begin{tikzpicture}
    \def\radius{2cm}
    \begin{scope}
      \foreach \angle [count=\i] in {0,10.588,...,350} {%
        \begin{scope}[shift={(\angle:\radius)},rotate=\angle-90]
          \def\state{off}
          \ifnumequal{\i}{13}{\gdef\state{on}}{}
          \path[draw=gray,\state] (0,0) circle (.18cm);
        \end{scope}
      };
    \end{scope}
    \begin{scope}[shift={(5.5cm,0)}]
      \def\on{}
      \foreach \i in {2,...,9}{\listxadd{\on}{\i}}
      \foreach \i in {11,...,17}{\listxadd{\on}{\i}}
      \foreach \i in {20,...,32}{\listxadd{\on}{\i}}
      \def\ys{}
      \foreach \i in {1,2,3}{\listxadd{\ys}{\i}}
      \foreach \i in {5,...,17}{\listxadd{\ys}{\i}}
      \foreach \i in {20,21,22}{\listxadd{\ys}{\i}}
      \foreach \i in {24,...,28}{\listxadd{\ys}{\i}}
      \foreach \i in {30,31,32}{\listxadd{\ys}{\i}}
      \foreach \angle [count=\i] in {0,10.588,...,350} {%
        \begin{scope}[shift={(\angle:\radius)},rotate=\angle-90]
          \def\state{off}
          \xifinlist{\i}{\on}{\gdef\state{on}}{}
          \xifinlist{\i}{\ys}{%
            \e{0,0}{0,.5};
            \path[draw=gray,fill=white] (0,.5) circle (.18cm);
          }{}
          \path[draw=gray,\state] (0,0) circle (.18cm);
        \end{scope}
      };
    \end{scope}
    \begin{scope}[shift={(11cm,0)}]
      \def\on{}
      \foreach \i in {8,28}{\listxadd{\on}{\i}}
      \foreach \angle [count=\i] in {0,10.588,...,350} {%
        \begin{scope}[shift={(\angle:\radius)},rotate=\angle-90]
          \def\state{off}
          \xifinlist{\i}{\on}{\gdef\state{on}}{}
          \path[draw=gray,\state] (0,0) circle (.18cm);
        \end{scope}
      };
    \end{scope}
  \end{tikzpicture}}
  % \includegraphics[width=350pt]{pdf/ring}
  \caption{Snapshots of the ring configuration
    taken at times $50$, $150$, and $250$.
    % Solid (green) circles indicate conformation $1$,
    % hollow ones conformation $0$;
    % a dot in the centre indicates a bound $Y$. % (hence active) $Y$.
    At $50$ and $250$ no $Y$ is bound
    (because they have not been yet injected into the system
     or already removed)
    and the ring is globally in state $0$,
    up to tiny fluctuations.
    At time $150$, it is globally in state $1$
    as a consequence of the binding of $Y$s.}
  \label{fig:snapshot}
\end{figure}

% In this model we assumed the ring fixed
% and that there is no need for breaking/forming $PP$ bonds.
It is important to note that the refined rules shown above
are those that
assume the $P$s lie on a ring and the ring is fixed,
\ie it does not break.
The method of \sct{energy-gp},
which makes no such assumptions,
generates many more rules
as it takes into account the cases where,
for instance, the $P$ that changes state
is an end of the chain of protomers.

% Note that the example above plays out with simple energy patterns
% which only incorporate edge and agent terms.
% This corresponds to the restricted notion of energy
% developed in \citet{anc}.

% TODO: upload the model to github
% The full model is available on-line at ...
% TODO: actually, paste the model here


\section{Non-linear energy functions}
\label{sec:non-linear-energy}


\section{Conclusions}
\label{sec:direct-conclusions}




%%% Local Variables:
%%% mode: latex
%%% TeX-master: "thesis"
%%% End:


\chapter[The inverse problem: From rules to energy]{
  The inverse problem \\
  \LARGE From rules to energy}
\label{chp:inverse}
In this chapter
we would like to explore restricted versions of Kappa
for which it is possible to infer the energy function
from the rewriting rules and their associated rates.
Recall from the introduction
that in Kappa itself this problem is undecidable \citep{et1}.
One such restriction is when agents do not have sites
and thus cannot bind.
This is Petri nets.
We briefly present here the result obtained by \citet{et2}
and show the construction of the energy function
for \emph{simple} and \emph{symmetric} Petri nets
with \emph{mass action} semantics
(sisma Petri net for short).
\begin{definition}
A \emph{sisma Petri net} is a Petri net
on species $\species$ and reactions $\reactions$ for which:
\begin{enumerate}[label={(\roman*)}]
\item (simple) there are no two reactions that have
  the same stoichiometry (net change in species).
\item (symmetric) for each reaction $r \in \reactions$,
  there is a reaction $\inv{r} \in \reactions$
  that has the reverse direction,
  \ie the inputs of one are the outputs of the other.
\item (mass action) the jumping rate $q_{xy,r}$
  of going from a state $x$ to $y$
  by a reaction $r$ is proportional to
  the number of ocurrences of its left-hand side in $x$.
  In particular,
  given a rate constant $k(r)$ for reaction $r$,
  % the jumping rate of $r$ at state $x$ is
  we have
  \[ q_{xy,r} = k(r) \prod_{A \in \species}
     \frac{x(A)!}{(x(A)-\Delta_r(A))!} \]
  where $x(A)$ is the number of $A$s in $x$
  and $\Delta_r(A)$ is the net change of $A$ in reaction $r$.
\end{enumerate}
\end{definition}
Note that simple implies
no two distinct reactions can be applied
to a state $x$ to obtain state $y$.
A sisma Petri net has an energy function
\begin{equation}\label{eq:pn-energy}
  E(x) = \sum_{A \in \species} \cost(A) x(A) + \ln\bigl(x(A)!\bigr)
\end{equation}
for some function $\cost: \species \to \RR$ such that,
for all $r \in \reactions$,
\[ \sum_{A \in \species} \Delta_r(A) \cost(A) = \ln(k(r^\star)) - \ln(k(r)). \]
If there is no such function $\cost$,
the Petri net does not have detailed balance and
an energy function.
%
In the rest of the chapter
we introduce two other restrictions of Kappa
and show how to construct their energy function.


\section{Cooperative assembly systems}
\label{sec:cas}

The first restriction is when
rules can only create or destroy one edge at a time
and their rates can only depend on
how many bound sites the endpoints of the edge have.
% the type of the neighbours of the two endpoints of the edge
% but not the state of their sites
% (whether they are free or bound).
% Moreover, sites in an agent are regarded as \emph{indistinguishable}
% and thus rules can only count how many are bound.
Therefore sites are treated as \emph{indistinguishable}.
In addition, agents of the same type cannot bind.
\citet{cas} have proposed these restrictions
and a simple formalism incorporating them
to study the thermodynamics of polymer formation
when there are two types of monomers.
Here we extend their result to any number of monomer types.
% TODO: perhaps take the idea in the next sentence
% and integrate into the paragraph:
% Cooperativity means that the rate of binding (and unbinding)
% depends on the number of connections that have been established
% by the participating nodes.

In the case of two monomers,
rules are of the form
\begin{center}
  \begin{tikzpicture}
    \node[grphnode,anchor=east] (lhs) at (0,0) {
      \tikz[ingrphdiag]{
        \begin{scope}[shift={(0,0)}]
          \e{0,0}{90:1.1};
          \e{0,0}{150:1.1};
          \e{0,0}{270:1.1};
          % a
          \n[n1]{a}{0,0};
          \e{a}{.5,0};
          \site{x1}{a.north};
          \site{x2}{150:.25};
          \site{xn}{a.south};
          \site{xn+1}{a.east};
          % b1
          \n[n2]{b1}{90:1.1};
          \site{z}{b1.south};
          % b2
          \n[n2]{b2}{150:1.1};
          \site[shift={(150:1.1)}]{z}{-30:.25};
          % ellipsis
          \node at (200:.79) {\Large .};
          \node at (210:.8) {\Large .};
          \node at (220:.79) {\Large .};
          % bn
          \n[n2]{bn}{270:1.1};
          \site{z}{bn.north};
        \end{scope}
        \begin{scope}[shift={(1.2,0)}]
          \e{0,0}{90:1.1};
          \e{0,0}{30:1.1};
          \e{0,0}{270:1.1};
          % b
          \n[n2]{b}{0,0};
          \e{b}{-.5,0};
          \site{y1}{b.north};
          \site{y2}{30:7pt};
          \site{yn}{b.south};
          \site{yn+1}{b.west};
          % a1
          \n[n1]{a1}{90:1.1};
          \site{z}{a1.south};
          % a2
          \n[n1]{a2}{30:1.1};
          \site[shift={(30:1.1)}]{z}{210:.25};
          % ellipsis
          \node at (-20:.79) {\Large .};
          \node at (-30:.8) {\Large .};
          \node at (-40:.79) {\Large .};
          % an
          \n[n1]{an}{270:1.1};
          \site{z}{an.north};
        \end{scope}
      }};
    \path (lhs.east) +(.3,.09) edge[rule] +(1,.09)
      +(1.3,0) coordinate (r);
    \path (lhs.east) +(1,-.09) edge[rule] +(.3,-.09);
    \node[grphnode,anchor=west] (rhs) at (r) {
      \tikz[ingrphdiag]{
        \e{0,0}{1.1,0};
        \begin{scope}[shift={(0,0)}]
          \e{0,0}{90:1.1};
          \e{0,0}{150:1.1};
          \e{0,0}{270:1.1};
          % a
          \n[n1]{a}{0,0};
          \site{x1}{a.north};
          \site{x2}{150:.25};
          \site{xn}{a.south};
          \site{xn+1}{a.east};
          % b1
          \n[n2]{b1}{90:1.1};
          \site{z}{b1.south};
          % b2
          \n[n2]{b2}{150:1.1};
          \site[shift={(150:1.1)}]{z}{-30:.25};
          % ellipsis
          \node at (200:.79) {\Large .};
          \node at (210:.8) {\Large .};
          \node at (220:.79) {\Large .};
          % bn
          \n[n2]{bn}{270:1.1};
          \site{z}{bn.north};
        \end{scope}
        \begin{scope}[shift={(1.1,0)}]
          \e{0,0}{90:1.1};
          \e{0,0}{30:1.1};
          \e{0,0}{270:1.1};
          % b
          \n[n2]{b}{0,0};
          \site{y1}{b.north};
          \site{y2}{30:7pt};
          \site{yn}{b.south};
          \site{yn+1}{b.west};
          % a1
          \n[n1]{a1}{90:1.1};
          \site{z}{a1.south};
          % a2
          \n[n1]{a2}{30:1.1};
          \site[shift={(30:1.1)}]{z}{210:.25};
          % ellipsis
          \node at (-20:.79) {\Large .};
          \node at (-30:.8) {\Large .};
          \node at (-40:.79) {\Large .};
          % an
          \n[n1]{an}{270:1.1};
          \site{z}{an.north};
        \end{scope}
      }};
  \end{tikzpicture}
\end{center}
where the three black dots on the sides of each graph
are an ellipsis to mean that monomers can be bound
to an arbitrary number of monomers of the other type
as long as each site is bound only once
and there is a finite number of sites per monomer
fixed by the monomer type.
% TODO: is this sentence clear?
Hence, the rule schema % formula?
represents a family of rules indexed by % $i,j$
the number of bound sites in the two monomers.
% TODO: add the following?
% Note that no nodes can be created or destroyed
% by this family of rules
% and so the total number of nodes is fixed.

We formalise the ideas in the first paragraph as follows.
Let $\types$ be the set of monomer types % $A,B,\dots$
and $\valence: \types \to \NN$ the map that assigns
% for each type the number of sites they have,
to each type the number of sites a monomer of that type has,
which we refer to as their valence.
A~monomer $u$ has type $\typeof(u) \in \types$
and degree $\degree_x(u) \in \NN$ in state $x$.
We simply write $\valence(u)$ for $\valence(\typeof(u))$.
The rate constant of a rule that binds
a monomer of type $\tp \in \types$ and degree $i$ with
a monomer of type $\tp' \in \types$ and degree $j$
is $\gamma^+_{\tp,i,\tp',j}$.
The rate constant of the reverse rule (unbinding)
is $\gamma^-_{\tp,i,\tp',j}$.
%
The jumping rate $q_{xy}$ from state $x$ to $y$
% is then determined by mass action semantics,
% \ie is linear in the number of ocurrences
% of the left-hand side of the rule that operates the transition.
is then linearly determined by the number of ocurrences
of the left-hand side of the rule and the rate constant
(mass action semantics).
We assume that any two agents can be bound only once.

The binding or unbinding of any two nodes $u,v$ in $x$
can only be carried out by one rule,
namely the one that operates on degrees $\degree_x(u),\degree_x(v)$.
The binding rule has rate constant
$\alpha(u,v) := \gamma^+_{\typeof(u),\degree_x(u),\typeof(v),\degree_x(v)}$
while the unbinding rate constant is
$\beta(u,v) := \gamma^-_{\typeof(u),\degree_x(u),\typeof(v),\degree_x(v)}$.
When binding we are free to choose
one site among the $\valence(u)-\degree_x(u)$ free sites of $u$
and one among the $\valence(v)-\degree_x(v)$ free sites of $v$
in order to apply the binding rule.
On the other hand, when unbinding we have only one choice,
namely removing the only edge between $u$ and $v$.
Hence, $q_{xy}$ is equal to $\alpha(u,v) \,
(\valence(u) - \degree_x(u)) \, (\valence(v) - \degree_x(v))$
% in the forward direction (binding)
when the binding rule is applied to $x$ to obtain $y$
and to $\beta(u,v)$
% in the backward direction (unbinding).
when unbinding.

The theorem below shows under which conditions
the type of systems presented in this section
have an energy function.

\begin{proposition}
  \label{prop:cas}
  Let $\types$ be a finite set of monomer types
  and $\gamma^-_{\tp,i,\tp',j},\gamma^+_{\tp,i,\tp',j}$
  families of real values indexed by types $\tp,\tp' \in \types$,
  $0 \leqslant i < \valence(\tp)$ and
  $0 \leqslant j < \valence(\tp')$ as above.
  Given a family $\Gamma_{\tp,i}$ of non-zero real values
  the following two statements are equivalent
  \begin{enumerate}[label={(\roman*)}]
  \item The \qmatrix $\qm$ as defined above by $q_{xy}$
    has detailed balance with respect to the \pmf $\pi$
    determined by the energy function
    \[ E(x) = \sum_{\tp \in \types} \sum_{0 < i \leqslant \valence(\tp)}
              \cost_\tp(i) \, x(\tp_i) \]
    where $x(\tp_i)$ is the number of nodes of type $\tp$
    with degree $i$ in $x$ and
    % TODO: which other variable name is good for the iterator a?
    \[ \cost_\tp(i) = \sum_{0 \leqslant j < i}
       \ln\, \frac{\Gamma_{\tp,j}}{\valence(\tp) - j} \]
  % \item The ratio $\gamma^-_{\tp,i,\tp',j} / \gamma^+_{\tp,i,\tp',j}$
  %   is equal to $\Gamma_{\tp,i} \, \Gamma_{\tp',j}$
  \item For all $\tp,\tp' \in \types$,
    $0 \leqslant i < \valence(\tp)$ and
    $0 \leqslant j < \valence(\tp')$ we have
    \begin{equation}
      \label{eq:cas-ratio}
      \frac{\gamma^-_{\tp,i,\tp',j}}{\gamma^+_{\tp,i,\tp',j}} =
      \Gamma_{\tp,i} \, \Gamma_{\tp',j}
    \end{equation}
  \end{enumerate}
\end{proposition}
\begin{proof}
  ($\Rightarrow$):
  Recall the detailed balance condition
  from \defn{detailed-balance}
  which says that for all states $x,y$
  \[ \pi_x \, q_{xy} = \pi_y \, q_{yx} \]
  By substituting $\pi_x$ and $\pi_y$ as in \eqn{energy} we obtain
  \begin{equation*}
    \expn\left[
      -\sum_{\tp \in \types} \sum_{0 < i \leqslant \valence(\tp)}
      \cost_\tp(i) \, x(\tp_i)
    \right] q_{xy} =
    \expn\left[
      -\sum_{\tp \in \types} \sum_{0 < i \leqslant \valence(\tp)}
      \cost_\tp(i) \, y(\tp_i)
    \right] q_{yx}
  \end{equation*}
  and by rearranging
  \[ \prod_{\tp \in \types} \prod_{0 < i \leqslant \valence(\tp)}
     e^{\,\cost_\tp(i) \, (y(\tp_i) - x(\tp_i))} =
     \frac{q_{yx}}{q_{xy}} \]

  When $y$ is obtained from $x$ by binding nodes $u,v$,
  the difference $y(\tp_i) - x(\tp_i)$ is equal to $0$
  for all pairs $\tp, i$ except
  i) when $\tp = \typeof(u), i = \degree_x(u)$
  or $\tp = \typeof(v)$, $i = \degree_x(v)$,
  then $y(\tp_i) - x(\tp_i) = -1$; and
  ii) when $\tp = \typeof(u), i = \degree_y(u) = \degree_x(u) + 1$
  or $\tp = \typeof(v), i = \degree_y(v) = \degree_x(v) + 1$,
  then $y(\tp_i) - x(\tp_i) = 1$.
  Let $\tp_u = \typeof(u)$, $\tp_v = \typeof(v)$,
  $d_u = \degree_x(u)$ and $d_v = \degree_x(v)$.
  It follows that the last equation can be rewritten as
  \[ \expn\left[
       \cost_{\tp_u}(d_u+1) +
       \cost_{\tp_v}(d_v+1) -
       \cost_{\tp_u}(d_u) -
       \cost_{\tp_v}(d_v) \right] =
     \frac{q_{yx}}{q_{xy}} \]
  % By expanding ... % Philipp didn't like expanding
  By substituting $\cost$ we get % and $q$ we have
  % \[ \frac{
  %      \Gamma_{\tp_u,d_u} \, \Gamma_{\tp_v,d_v}}{
  %      (\valence(u) - d_u) \, (\valence(v) - d_v)} =
  \[ \frac{
     \prod_{0 \leqslant i < d_u+1}%\limits
     \frac{\Gamma_{\tp_u,i}}{\valence(u) - i} \,
     \prod_{0 \leqslant i < d_v+1}%\limits
     \frac{\Gamma_{\tp_v,i}}{\valence(v) - i}}{
     \prod_{0 \leqslant i < d_u}%\limits
     \frac{\Gamma_{\tp_u,i}}{\valence(u) - i} \,
     \prod_{0 \leqslant i < d_v}%\limits
     \frac{\Gamma_{\tp_v,i}}{\valence(v) - i}} =
     \frac{q_{yx}}{q_{xy}} \]
     % \frac{
     %   % \gamma^-_{\tp_u,d_u,\tp_v,d_v}}{
     %   % \gamma^+_{\tp_u,d_u,\tp_v,d_v} \,
     %   \beta(u,v)}{
     %   \alpha(u,v) \, (\valence(u) - d_u) \, (\valence(v) - d_v)} \]
  Products on the left cancel out and yield,
  after substituting $q$ on the right,
  % which by simplification and substitution of $q$ yields
  \[ \frac{
       \Gamma_{\tp_u,d_u} \, \Gamma_{\tp_v,d_v}}{
       (\valence(u) - d_u) \, (\valence(v) - d_v)} =
     \frac{
       \beta(u,v)}{
       \alpha(u,v) \, (\valence(u) - d_u) \, (\valence(v) - d_v)} \]
  which then simplifies to
  \[ \Gamma_{\tp_u,d_u} \, \Gamma_{\tp_v,d_v} =
    \frac{\gamma^-_{\tp_u,d_u,\tp_v,d_v}}{\gamma^+_{\tp_u,d_u,\tp_v,d_v}} \]
  This equality holds in general for nodes of any degree and type.

  ($\Leftarrow$):
  We prove that, whenever (ii) holds,
  $\pi$ verifies the detailed balance condition.
  For all $x,y$ such that $q_{xy} = 0$
  the equality $\pi_x\,q_{xy} = \pi_y\,q_{yx}$ holds
  as rules are reversible and (ii) dictates that
  a rate constant is zero if the reverse rate constant is.
  % NOTE: q_{xy} = 0 might be because there is no transition
  % between x and y or because the rate constant is zero.
  When $q_{xy} > 0$ then $y$ can be obtained from $x$
  by binding or unbinding some nodes $u,v$.
  By substituting $\tp$ for $\typeof(u)$, $\tp'$ for $\typeof(v)$,
  $i$ for $\degree_x(u)$ and $j$ for $\degree_x(v)$
  in \eqn{cas-ratio} we obtain the last equation
  in the first part of the proof.
  We can replay the transformations backwards
  to obtain $\pi_x\,q_{xy} = \pi_y\,q_{yx}$
  when $y$ is obtained by binding.
  The case of unbinding follows a similar argument.
\end{proof}


\section{Flipping and binding} % co-ANC
\label{sec:fb}

Now we look at systems whose nodes have sites
that possess an internal state.
This internal state is used to decide when to bind other nodes
or change the internal state of other sites.
For simplicity, internal states can
take one of only two possible values. %, 0 and 1.
Unlike cooperative assembly systems,
% where sites were undistinguishable from each other,
here sites are \emph{distinguishable}
% and thus have a unique name that distinguishes it from the rest
as in Kappa.
Hence, we will use contact maps as defined in \sct{kappa}
as graphs in this section.
% TODO: add the order of sites wherever it's used.
% In addition, sites are \emph{ordered}
% by a total order $<_a$ with a an agent type
% (node in the contact graph).

A site's internal state can be changed by rules
we call \emph{flips}.
The rate at which we flip a site may depend on
the type of the site and the node it belongs to,
the internal state of sites on the same node
and the type of the neighbours.
Note that it cannot depend on
the internal states of the neighbours' sites
or the nodes they are bound to.
Graphically, flips are of the form
\begin{center}
  \begin{tikzpicture}[thick]
    \node[grphnode,anchor=east] (lhs) at (0,0) {
      \tikz[ingrphdiag]{
        \nn[n4]{a}{0,0}{$u$};
        \n[n4,dotted]{b1}{10:1.1};
        \n[n4,dotted]{b2}{170:1.1};
        \n[n4,dotted]{b3}{-90:1.1};
        \e[dotted]{a}{b1};
        \e[dotted]{a}{b2};
        \e[dotted]{a}{b3};
        \site[n]{s1}{a.south};
        \site[n6]{s2}{170:.33};
        \site[n6]{s3}{10:.33};
        \site[n4,dotted,shift={(10:1.1)}]{s4}{190:.25};
        \site[n4,dotted,shift={(170:1.1)}]{s5}{-10:.25};
        \site[n4,dotted]{s6}{b3.north};
        \node at (-65:.48) {\scriptsize x};
        \node at (110:.7) {\Large .};
        \node at (90:.7) {\Large .};
        \node at (70:.7) {\Large .};
      }};
    % \begin{scope}[shift={(2,-0.5)}]
    %   \draw[rule] (0,.09) -- node[above] {
    %     $\lambda^+_{\tp,x,\sitestates,\neighbours}$} ++(1.4,0);
    %   \draw[rule] (1.4,-.09) -- node[below] {
    %     $\lambda^-_{\tp,x,\sitestates,\neighbours}$} ++(-1.4,0);
    % \end{scope}
    \path (lhs.east) +(.3,.09) edge[rule] +(1,.09)
      +(1.3,0) coordinate (r);
    \path (lhs.east) +(1,-.09) edge[rule] +(.3,-.09);
    \node[grphnode,anchor=west] (rhs) at (r) {
      \tikz[ingrphdiag]{
        \nn[n4]{a}{0,0}{$u$};
        \n[n4,dotted]{b1}{10:1.1};
        \n[n4,dotted]{b2}{170:1.1};
        \n[n4,dotted]{b3}{-90:1.1};
        \e[dotted]{a}{b1};
        \e[dotted]{a}{b2};
        \e[dotted]{a}{b3};
        \site[n5]{s1}{a.south};
        \site[n6]{s2}{170:.33};
        \site[n6]{s3}{10:.33};
        \site[n4,dotted,shift={(10:1.1)}]{s4}{190:.25};
        \site[n4,dotted,shift={(170:1.1)}]{s5}{-10:.25};
        \site[n4,dotted]{s6}{b3.north};
        \node at (-65:.48) {\scriptsize x};
        \node at (110:.7) {\Large .};
        \node at (90:.7) {\Large .};
        \node at (70:.7) {\Large .};
      }};
  \end{tikzpicture}
\end{center}
where the dotted lines % in the rule diagram
denote an optional node or edge
and site $x$ changes state from white to black.
As usual, we write $r_L$ for the left-hand side contact map of rule $r$
and $r_R$ for that of the right-hand side,
both contact maps over some fixed contact graph $C$.
The rate constants of the forward and backward rules are parametrised
by the type $\tp := r_{L,\agents}(u) = r_{R,\agents}(u)$ of $u$ in $C$,
the type $s := r_{L,\sites}(x) = r_{R,\sites}(x)$ of $x$,
the internal state vector
$\sitestates \in \set{0,1}^{\sitemap^{-1}(u)}$ of $u$,
% TODO: do we need to know site its connected to as well?
and the vector of neighbours' types
$\neighbours \in (\agents_C \union \set{\star})^{\sitemap^{-1}(u)}$
where $\star$ represents a free site.
% When all sites are free we use $\emptyvec$.
We write $\lambda^+_{\tp,s,\sitestates,\neighbours}$
for the rate constant of the forward rule,
$\lambda^-_{\tp,s,\sitestates,\neighbours}$
for that of the backward rule,
and $\Lambda_{\typeof(u),a,\sitestates,\neighbours} =
\lambda^-_{\typeof(u),a,\sitestates,\neighbours}/
\lambda^+_{\typeof(u),a,\sitestates,\neighbours}$ for the ratio
between the backward and forward rate constants % of flips.
% TODO: do we (have to) use them?
% Also, later we use $\sitestatesof_x(u)$ and $\neighboursof_x(u)$
% for the internal state vector and neighbour vector % of neighbours' types
% of node $u$ in $x$.







%%% Local Variables:
%%% mode: latex
%%% TeX-master: "thesis"
%%% End:


%% Appendix
% \appendix

%% Choose your favourite bibliography style here.
% \bibliographystyle{apacite}

%% If you want the bibliography single-spaced (which is allowed), uncomment
%% the next line.
\singlespace

%% Specify the bibliography file. Default is thesis.bib.
% \bibliography{thesis}

\printbibliography[heading=bibintoc]

%% ... that's all, folks!
\end{document}

%%% Local Variables:
%%% mode: latex
%%% TeX-master: t
%%% End:
