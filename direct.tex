

\if0
Whenever a contact map $g$ is used to compute $\matches{g}{m}$
where $m$ is a state, we say $g$ is a pattern.

In the spirit of rule based modelling languages like Kappa,
our energy function depends on the number of occurrences of a pattern.


We suppose hereafter that $\generators$ is closed under
rule inversion, \ie $\generators = \inv{\generators}$.
Hence every $(r,\psi)$-transition
has an inverse $(\inv{r},\comatch{\psi})$
and $\LTS_\generators$ is symmetric.


\section{Minimal glueings}
\label{sec:mg}

\section{Refinements} % and growth policies
\label{sec:gp}
\fi

\section{Energy-based refinement}
\label{sec:energy-gp}

\section{Linear kinetic model}
\label{sec:kinetic-model}

% is the linear kinetic model related to "Parameters for
% the description of transition states", John Leffler, Science, 1953
% https://sci-hub.ac/10.2307/1680906
% it says "we approximate the transition state as a hybrid between
% the reagent and product states"
% "whenever the plot of the logarithm of the rate constant
%  against the equilibrium constant is a straight line,
%  the approximation is justified"
% "it should then be possible to predict
%  the free energy of the transition state by a linear combination of
%  the predictions made for the reagents and for the products"
% if we then relate the free energy of the transition state to
% the rate constants using Arrhenius?



%%% Local Variables:
%%% mode: latex
%%% TeX-master: "thesis"
%%% End:
