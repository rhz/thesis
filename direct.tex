In this chapter we show how to construct a set of reversible rules
and their forward and backward rate constants from an energy function.
In the spirit of rule-based modelling languages like Kappa
where rules and observables are defined in terms of patterns,\footnote{
  Recall that a pattern is a contact map used to find subgraphs in states.}
we use a set of \emph{energy patterns} $\shapes$
for our energy function.
We assign an \emph{energy cost} $\cost(g)$ to each of them
and build the energy function as a linear combination
of their number of ocurrences. % of each energy pattern.
\[ E(m) = \sum_{g \in \shapes} \cost(g) \abs{\matches{g}{m}} \]
This is reminiscent of group contribution methods
used to estimate the standard Gibbs free energy of formation
of biomolecules \citep{group-contrib}.

As mentioned at the end of \sct{kappa},
we will derive the set of rules with detailed balance
from a set of generator rules $\generators$ (without rates).
We suppose that $\generators$ is closed under
rule inversion, \ie $\generators = \inv{\generators}$.
Given a contact graph $C$,
a simple option would be to include
every possible minimal rule in this set,
that is, include a creation and a destruction rule
for each edge in the contact graph.
Each of these rules is minimal in the sense that
it only asks for the presence of
the two participating agents and sites.
The example rule in \sct{kappa}
where agents of type $1$ and $2$ bind
% regardless of the binding state of any other site,
regardless of the context
in which these two agents happen to be,
which we denote by $r^+_{12}$,
is one such minimal rule
that can be derived from contact graph $T$.
This option is \emph{maximally permissive}
% as every possible transformation
% allowed by the contact graph
% is allowed by $\generators$.\footnote{
with respect to the contact graph.\footnote{
  Intuitively, this is analogous to the case of classical mechanics
  % where the topology of the space gives us the possible transformations
  where, a priori, movement is not constrained along in coordinate.}
Even if all transformations are possible,
many of them may be unlikely due to having a high energy.
Still one might prefer to forbid certain transformations
in some scenarios.
This is indeed the case in the example
that will be presented in \sct{alloring}.

In our previous example (\sct{kappa}),
we might want to favour the formation of
triangles over chains and other cycles.
For this we give a negative energy cost to $t$,
\ie $\cost(t) < 0$.
If $t$ is the only energy pattern,
then the energy of a state $m$ is
$E(m) = \cost(t) \abs{\matches{t}{m}}$.
In this model one might, for instance,
wonder how low the energy cost of $t$ must be
to have at least $90\%$ of all agents in a triangle
at equilibrium at least $90\%$ of the time.

We would like to find rules that have detailed balance
with respect to this energy function.
Consider the rule $r^+_{12}$ and its inverse $r^-_{12}$,
the unbinding of agents $1$ and $2$.
% Given the maximally permissive set of generator rules
% $\generators=\set{r^+_{12},r^-_{12},r^+_{23},r^-_{23},r^+_{31},r^-_{31}}$,
% we first ask ourselves if these reversible rules
We first ask ourselves if this pair of rules
could have detailed balance
for some assignment of kinetic rates.
% to the forward and backward rule.
Suppose we assign kinetic rates $k^+$ and $k^-$
to $r^+_{12}$ and $r^-_{12}$.
% and that there is at least one pair of states $m,n$
% for which only one rule, say $r^+_{12}$,
% brings us from state $m$ to $n$.
Recall from \sct{bg} that $e^{E(n)-E(m)} = q_{nm}/q_{mn}$
for systems with detailed balance.
From \eqn{kappa-ctmc}
\[ q_{mn} = \sum_{\substack{r \in \generators\\r = \tuple{r_L,r_R}}}
   k(r) \; \abs{\setof{\psi \in \matches{r_L}{m}}{m^{(r,\psi)} = n}}
\]
It is clear that at most one of the two rules
can bring us from state $m$ to $n$, say it is $r^+_{12}$.
By rule reversibility (\lem{reversibility})
$r^-_{12}$ brings us from $n$ back to $m$
and the number of matches of $r^-_{12}$ in $n$
is equal to the number of matches of $r^+_{12}$ in $m$.
% The number of matches of $r^+_{12}$
% in the numerator of $q_{nm}/q_{mn}$ will cancel out
% with the number of matches of $r^-_{12}$ in the denominator
% due to rule reversibility (\lem{reversibility}) and vice versa.
% If $r^+_{12}$ induces a rewrite of $m$ into $n$ we obtain
% $e^{E(n)-E(m)} = k^+/k^-$.
Hence, $e^{E(n)-E(m)} = k^+/k^-$.
In words, the change in energy produced by the rule application
fixes the ratio between the kinetic rates.
As a consequence,
each rule application should produce the same energy change
for there to be an assignment of kinetic rates with detailed balance.
Whenever a rule produces the same energy change
regardless of where it is applied
we say that the rule has an \emph{unambiguous energy balance}
or is $\shapes$-balanced.
The following two rule applications show that
$r^+_{12}$ is not $\shapes$-balanced.
\begin{center}
  \resizebox{.9\linewidth}{!}{%
  \begin{tikzpicture}[thick]
    % first row
    \node[grphnode,anchor=east] (lhs1) at (0,0) {
      \tikz[ingrphdiag]{
        \begin{scope}[shift={(0,0)}]
          \n[n1]{x}{0,0};
          \e{x}{.5,0};
          \site{rx}{x.east};
          \node at (26:.42) {\scriptsize $r$};
        \end{scope}
        \begin{scope}[shift={(1.2,0)}]
          \n[n2]{y}{0,0};
          \e{y}{-.5,0};
          \site{ly}{y.west};
          \node at (206:.42) {\scriptsize $l$};
        \end{scope}
      }};
    \path (lhs1.east) +(.3,0) edge[rule,dotted] +(1,0)
      +(1.3,0) coordinate (r1);
    \node[grphnode,anchor=west] (rhs1) at (r1) {
      \tikz[ingrphdiag]{
        \e{0,0}{1.1,0};
        \begin{scope}
          \n[n1]{x}{0,0};
          \site{rx}{x.east};
          \node at (26:.42) {\scriptsize $r$};
        \end{scope}
        \begin{scope}[shift={(1.1,0)}]
          \n[n2]{y}{0,0};
          \site{ly}{y.west};
          \node at (206:.42) {\scriptsize $l$};
        \end{scope}
      }};
    % second column
    \node[grphnode,anchor=east] (lhs2) at (9,0) {
      \tikz[ingrphdiag]{
        \begin{scope}[shift={(0,0)}]
          \n[n1]{x}{0,0};
          \e{x}{.5,0};
          \site{rx}{x.east};
          \node at (26:.42) {\scriptsize $r$};
        \end{scope}
        \begin{scope}[shift={(1.2,0)}]
          \n[n2]{y}{0,0};
          \e{y}{-.5,0};
          \site{ly}{y.west};
          \node at (206:.42) {\scriptsize $l$};
        \end{scope}
      }};
    \path (lhs2.east) +(.3,0) edge[rule,dotted] +(1,0)
      +(1.3,0) coordinate (r2);
    \node[grphnode,anchor=west] (rhs2) at (r2) {
      \tikz[ingrphdiag]{
        \e{0,0}{1.1,0};
        \begin{scope}
          \n[n1]{x}{0,0};
          \site{rx}{x.east};
          \node at (26:.42) {\scriptsize $r$};
        \end{scope}
        \begin{scope}[shift={(1.1,0)}]
          \n[n2]{y}{0,0};
          \site{ly}{y.west};
          \node at (206:.42) {\scriptsize $l$};
        \end{scope}
      }};
    % second row
    \path (lhs1.south) +(0,-.2) edge[rule] +(0,-.6);
    \node[grphnode,anchor=east] (lhs3) at (0,-2) {
      \tikz[ingrphdiag]{
        \begin{scope}[shift={(0,0)}]
          \n[n1]{x}{0,0};
          \e{x}{.5,0};
          \e{x}{-.5,0};
          \site{lx}{x.west};
          \site{rx}{x.east};
          \node at (206:.42) {\scriptsize $l$};
          \node at (26:.42) {\scriptsize $r$};
        \end{scope}
        \e{1.2,0}{2.3,0};
        \begin{scope}[shift={(1.2,0)}]
          \n[n2]{y}{0,0};
          \e{y}{-.5,0};
          \site{ly}{y.west};
          \site{ry}{y.east};
          \node at (206:.42) {\scriptsize $l$};
          \node at (26:.42) {\scriptsize $r$};
        \end{scope}
        \begin{scope}[shift={(2.3,0)}]
          \n[n3]{z}{0,0};
          \e{z}{.5,0};
          \site{lz}{z.west};
          \site{rz}{z.east};
          \node at (206:.42) {\scriptsize $l$};
          \node at (26:.42) {\scriptsize $r$};
        \end{scope}
      }};
    \path (lhs3.east) +(.3,0) edge[rule,dotted] +(1,0)
      +(1.3,0) coordinate (r3);
    \path (rhs1.south) +(0,-.2) edge[rule] +(0,-.6);
    \node[grphnode,anchor=west] (rhs3) at (r3) {
      \tikz[ingrphdiag]{
        \e{0,0}{2.2,0};
        \begin{scope}[shift={(0,0)}]
          \n[n1]{x}{0,0};
          \e{x}{-.5,0};
          \site{lx}{x.west};
          \site{rx}{x.east};
          \node at (206:.42) {\scriptsize $l$};
          \node at (26:.42) {\scriptsize $r$};
        \end{scope}
        \begin{scope}[shift={(1.1,0)}]
          \n[n2]{y}{0,0};
          \site{ly}{y.west};
          \site{ry}{y.east};
          \node at (206:.42) {\scriptsize $l$};
          \node at (26:.42) {\scriptsize $r$};
        \end{scope}
        \begin{scope}[shift={(2.2,0)}]
          \n[n3]{z}{0,0};
          \e{z}{.5,0};
          \site{lz}{z.west};
          \site{rz}{z.east};
          \node at (206:.42) {\scriptsize $l$};
          \node at (26:.42) {\scriptsize $r$};
        \end{scope}
      }};
    % second row, second column
    \path (lhs2.south) +(0,-.2) edge[rule] +(0,-.6);
    \node[grphnode,anchor=east] (lhs4) at (9,-2.4) {
      \tikz[ingrphdiag]{
        \e{0,0}{-56.944:1.1};
        \e{0:1.2}{-56.944:1.1};
        \begin{scope}[shift={(0,0)}]
          \n[n1]{x}{0,0};
          \e{x}{.5,0};
          \site{r1}{0:7pt};
          \site{l1}{-60:7pt};
          \node at (-86:12pt) {\scriptsize $l$};
          \node at (26:12pt) {\scriptsize $r$};
        \end{scope}
        \begin{scope}[shift={(0:1.2)}]
          \n[n2]{y}{0,0};
          \e{y}{-.5,0};
          \site{r2}{180:7pt};
          \site{l2}{-120:7pt};
          \node at (154:12pt) {\scriptsize $l$};
          \node at (-94:12pt) {\scriptsize $r$};
        \end{scope}
        \begin{scope}[shift={(-56.944:1.1)}]
          \n[n3]{z}{0,0};
          % angle is 66.111 deg
          \site{r3}{123.0555:7pt};
          \site{l3}{56.9445:7pt};
          \node at (146:12pt) {\scriptsize $r$};
          \node at (34:12pt) {\scriptsize $l$};
        \end{scope}
      }};
    \path (lhs4.east) +(.3,0) edge[rule,dotted] +(1,0)
      +(1.3,0) coordinate (r4);
    \path (rhs2.south) +(0,-.2) edge[rule] +(0,-.6);
    \node[grphnode,anchor=west] (rhs4) at (r4) {
      \tikz[ingrphdiag]{
        \e{0,0}{0:1.1};
        \e{0,0}{-60:1.1};
        \e{0:1.1}{-60:1.1};
        \begin{scope}[shift={(0,0)}]
          \n[n1]{x}{0,0};
          \site{r1}{0:7pt};
          \site{l1}{-60:7pt};
          \node at (-86:12pt) {\scriptsize $l$};
          \node at (26:12pt) {\scriptsize $r$};
        \end{scope}
        \begin{scope}[shift={(0:1.1)}]
          \n[n2]{y}{0,0};
          \site{r2}{180:7pt};
          \site{l2}{-120:7pt};
          \node at (154:12pt) {\scriptsize $l$};
          \node at (-94:12pt) {\scriptsize $r$};
        \end{scope}
        \begin{scope}[shift={(-60:1.1)}]
          \n[n3]{z}{0,0};
          \site{r3}{120:7pt};
          \site{l3}{60:7pt};
          \node at (146:12pt) {\scriptsize $r$};
          \node at (34:12pt) {\scriptsize $l$};
        \end{scope}
      }};
  \end{tikzpicture}}
\end{center}

We see that, while the application on the left
does not produce any change in energy ($\Delta E = 0$),
the one on the right creates a triangle
and thus $\Delta E = \cost(t)$.\footnote{
  We can't tolerate energetical ambiguity, heresy!}
We must then split $r^+_{12}$ into subrules that check
the surroundings of the rule application
to make sure that, for instance,
every application of such a subrule
creates one triangle or none at all.
It is important that the partition of the rule
has certain properties.
In particular, one would like that every match of the rule
can be mapped to exactly one match of one of the subrules.
Prior work by \citet{refinement} has shown how % a simple way by which
one can obtain a partition of rules with this property.
This work will be briefly presented in \sct{refinements}.










\section{Minimal glueings}
\label{sec:mg}


\section{Refinements} % and growth policies
\label{sec:refinements}


\section{Energy-based refinement}
\label{sec:energy-gp}


\section{Linear kinetic model}
\label{sec:kinetic-model}

% is the linear kinetic model related to "Parameters for
% the description of transition states", John Leffler, Science, 1953
% https://sci-hub.ac/10.2307/1680906
% it says "we approximate the transition state as a hybrid between
% the reagent and product states"
% "whenever the plot of the logarithm of the rate constant
%  against the equilibrium constant is a straight line,
%  the approximation is justified"
% "it should then be possible to predict
%  the free energy of the transition state by a linear combination of
%  the predictions made for the reagents and for the products"
%
% if we then relate the free energy of the transition state to
% the rate constants using Arrhenius?
% this has been done in transition state theory
% https://en.wikipedia.org/wiki/Eyring_equation
%
% do we get additional constraints on kinetic rates from cycles
% in the transition graph? when two cycles share an edge?


\section{Flagellum's motor}
\label{sec:alloring}




%%% Local Variables:
%%% mode: latex
%%% TeX-master: "thesis"
%%% End:
