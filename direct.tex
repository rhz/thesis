In this chapter we show how to construct a set of reversible rules
and their forward and backward rate constants from an energy function.
In the spirit of rule-based modelling languages like Kappa
where rules and observables are defined in terms of patterns,\footnote{
  A pattern is a contact map used to find subgraphs in states.}
we use a set of \emph{energy patterns} $\shapes$
for our energy function.
We assign an \emph{energy cost} $\cost(g)$ to each of them
and build the energy function as a linear combination
of their number of ocurrences. % of each energy pattern.
\[ E(m) = \sum_{g \in \shapes} \cost(g) \abs{\matches{g}{m}} \]
This is reminiscent of group contribution methods
used to estimate the standard Gibbs free energy of formation
of biomolecules \citep{group-contrib}.

As mentioned at the end of \sct{kappa},
we will derive the set of rules with detailed balance
from a set of generator rules $\generators$ (without rates).
We suppose that $\generators$ is closed under
rule inversion, \ie $\generators = \inv{\generators}$.
Given a contact graph $C$,
a simple option would be to include
every possible minimal rule in this set,
that is, include a creation and a destruction rule
for each edge in the contact graph.
Each of these rules is minimal in the sense that
it only asks for the presence of
the two participating agents and sites.
The example rule in \sct{kappa}
where agents of type $1$ and $2$ bind
% regardless of the binding state of any other site,
regardless of the context
in which these two agents happen to be,
which we denote by $r^+_{12}$,
is one such minimal rule
that can be derived from contact graph $T$.
This option is \emph{maximally permissive}
% as every possible transformation
% allowed by the contact graph
% is allowed by $\generators$.\footnote{
with respect to the contact graph.\footnote{
  Intuitively, this is analogous to the case of classical mechanics
  % where the topology of the space gives us the possible transformations
  where, a priori, movement is not constrained along in coordinate.}
Even if all transformations are possible,
many of them may be unlikely due to having a high energy.
Still one might prefer to forbid certain transformations
in some scenarios.
This is indeed the case in the example
that will be presented in \sct{alloring}.

In our previous example (\sct{kappa}),
we might want to favour the formation of
triangles over chains and other cycles.
For this we give a negative energy cost to $t$,
\ie $\cost(t) < 0$.
If $t$ is the only energy pattern,
then the energy of a state $m$ is
$E(m) = \cost(t) \abs{\matches{t}{m}}$.
In this model one might, for instance,
wonder how low the energy cost of $t$ must be
to have at least $90\%$ of all agents in a triangle
at equilibrium at least $90\%$ of the time.

We would like to find rules that have detailed balance
with respect to this energy function.
Consider the rule $r^+_{12}$ and its inverse $r^-_{12}$,
the unbinding of agents $1$ and $2$.
% Given the maximally permissive set of generator rules
% $\generators=\set{r^+_{12},r^-_{12},r^+_{23},r^-_{23},r^+_{31},r^-_{31}}$,
% we first ask ourselves if these reversible rules
We first ask ourselves if this pair of rules
could have detailed balance
for some assignment of kinetic rates.
% to the forward and backward rule.
Suppose we assign kinetic rates $k^+$ and $k^-$
to $r^+_{12}$ and $r^-_{12}$.
Recall from \sct{bg} that $e^{E(n)-E(m)} = q_{nm}/q_{mn}$
for systems with detailed balance.
From \eqn{kappa-ctmc}
\[ q_{mn} = \sum_{\substack{r \in \generators\\r = \tuple{r_L,r_R}}}
   k(r) \; \abs{\setof{\psi \in \matches{r_L}{m}}{m^{(r,\psi)} = n}}
\]
Further assume that there is only one rule, say $r^+_{12}$,
that brings you from state $m$ to $n$.
By rule reversibility (\lem{reversibility}),
$r^-_{12}$ brings you from $n$ back to $m$
and the number of matches of $r^-_{12}$ in $n$
is equal to the number of matches of $r^+_{12}$ in $m$.
% The number of matches of $r^+_{12}$
% in the numerator of $q_{nm}/q_{mn}$ will cancel out
% with the number of matches of $r^-_{12}$ in the denominator
% due to rule reversibility (\lem{reversibility}) and vice versa.
% If $r^+_{12}$ induces a rewrite of $m$ into $n$ we obtain
% $e^{E(n)-E(m)} = k^+/k^-$.
Hence, $e^{E(n)-E(m)} = k^+/k^-$.
In words, the change in energy produced by the rule application
fixes the ratio between the kinetic rates.
As a consequence,
each rule application should produce the same energy change
for there to be an assignment of kinetic rates with detailed balance.
Whenever a rule produces the same energy change
regardless of where it is applied
we say that the rule has an \emph{unambiguous energy balance}.

% we would like to find a rule that
% provides more context while performing the same action ...









\section{Minimal glueings}
\label{sec:mg}


\section{Refinements} % and growth policies
\label{sec:gp}


\section{Energy-based refinement}
\label{sec:energy-gp}


\section{Linear kinetic model}
\label{sec:kinetic-model}

% is the linear kinetic model related to "Parameters for
% the description of transition states", John Leffler, Science, 1953
% https://sci-hub.ac/10.2307/1680906
% it says "we approximate the transition state as a hybrid between
% the reagent and product states"
% "whenever the plot of the logarithm of the rate constant
%  against the equilibrium constant is a straight line,
%  the approximation is justified"
% "it should then be possible to predict
%  the free energy of the transition state by a linear combination of
%  the predictions made for the reagents and for the products"
%
% if we then relate the free energy of the transition state to
% the rate constants using Arrhenius?
% this has been done in transition state theory
% https://en.wikipedia.org/wiki/Eyring_equation
%
% do we get additional constraints on kinetic rates from cycles
% in the transition graph? when two cycles share an edge?


\section{Flagellum's motor}
\label{sec:alloring}




%%% Local Variables:
%%% mode: latex
%%% TeX-master: "thesis"
%%% End:
